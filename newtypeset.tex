11pt]{book}
%\usepackage{showframe}
\usepackage{enumitem,graphicx,charter,multicol}
\pagestyle{myheadings}
\newcommand*{\MyLabel}{}%
\newenvironment{MyDescription}[2][]{%
    \edef\MyLabel{#1}%
    \begin{description}[labelwidth=3.0cm, leftmargin=3.0cm, labelsep=0.0ex, font=\normalfont]
    \item [#2]
}{%
    \hfill\MyLabel%
    \end{description}%
}

\begin{document}
\part{First Chaper}
\newcounter{stanza}
\stepcounter{stanza}
\title{Suttanipata, a poetic translation}
\author{Laurence Khantipalo Mills}
\chapter{Uraga Sutta -- The Serpent Sutta}
\begin{MyDescription}[\arabic{stanza}]{}
   Who removes arisen anger \\
   as herbs a serpent's venom spread \\
   a bhikkhu such leaves here and there \\
   as serpent ploughs its worn-out skin.
   \stepcounter{stanza}
   \end{MyDescription}
   
   \begin{MyDescription}[\arabic{stanza}]{}
   Who lust pulls up remainderless \\
   as in water, plants and blooms of lotuses\\
   a bhikkhu such leaves here and there\\
   as serpent ploughs its worn-out skin				
\stepcounter{stanza}
   \end{MyDescription}   
   
\begin{MyDescription}[\arabic{stanza}]{}
   Who craving dams remainderless\\
   as drying of a river's fierce and rapid flow\\
   a bhikkhu such leaves here and there\\
   as serpent ploughs its worn-out skin.
   \stepcounter{stanza}
   \end{MyDescription}
   
   \begin{MyDescription}[\arabic{stanza}]{}
   Who destroys conceit entire\\
   as a great flood a bridge of reeds so frail;\\
   a bhikkhu such leaves here and there\\
   as serpent ploughs its worn-out skin.
   \stepcounter{stanza}
   \end{MyDescription}
   
   \begin{MyDescription}[\arabic{stanza}]{}
   Who in being no essence finds\\
   as seeker of flowers on Udumbara trees;\\
   a bhikkhu such leaves here and there\\
   as serpent ploughs its worn-out skin.
   \stepcounter{stanza}
   \end{MyDescription}
   
   \begin{MyDescription}[\arabic{stanza}]{}
   Who keeps no grudges inwardly\\
   but “this-being-not being” has gone beyond;\\
   a bhikkhu such leaves here and there\\
   as serpent ploughs its worn-out skin.
   \stepcounter{stanza}
   \end{MyDescription}
   
   \begin{MyDescription}[\arabic{stanza}]{}
   In who do thoughts no longer smoulder,\\
   internally curtailed, remainderless;\\
   a bhikkhu such leaves here and there\\
   as serpent ploughs its worn-out skin.
   \stepcounter{stanza}
   \end{MyDescription}
   
   \begin{MyDescription}[\arabic{stanza}]{}
   Who neither goes too far nor lags behind,\\
   all mind-proliferation gone beyond;\\
   a bhikkhu such leaves here and there\\
   as serpent ploughs its worn-out skin.
   \stepcounter{stanza}
   \end{MyDescription}
   
   \begin{MyDescription}[\arabic{stanza}]{}
   Who neither goes too far nor lags behind,\\
   who of the world has Known, “All is not thus”;\\
   a bhikkhu such leaves here and there\\
   as serpent ploughs its worn-out skin.
   \stepcounter{stanza}
   \end{MyDescription}
   
   \begin{MyDescription}[\arabic{stanza}]{}
   Who neither goes too far nor lags behind,\\
   who free of greed has Known, “All is not thus”;\\
   a bhikkhu such leaves here and there\\
   as serpent ploughs its worn-out skin.
   \stepcounter{stanza}
   \end{MyDescription}
   
   \begin{MyDescription}[\arabic{stanza}]{}
   Who neither goes too far nor lags behind,\\
   who free of lust has Known, “All is not thus”;\\
   a bhikkhu such leaves here and there\\
   as serpent ploughs its worn-out skin.
   \stepcounter{stanza}
   \end{MyDescription}
   
   \begin{MyDescription}[\arabic{stanza}]{}
   Who neither goes too far nor lags behind,\\
   who free of hate has Known, “All is not thus”;\\
   a bhikkhu such leaves here and there\\
   as serpent ploughs its worn-out skin.
   \stepcounter{stanza}
   \end{MyDescription}

\begin{MyDescription}[\arabic{stanza}]{}
Who neither goes too far nor lags behind,\\
who delusion-free has Known, “All is not thus”;\\
a bhikkhu such leaves here and there\\
as serpent ploughs its worn-out skin.
\stepcounter{stanza}
\end{MyDescription}

\begin{MyDescription}[\arabic{stanza}]{}
   		In whom are no latent tendencies at all –\\
   		whose roots of evil completely are expunged;\\
   		a bhikkhu such leaves here and there\\
   		as serpent ploughs its worn-out skin.
\stepcounter{stanza}
   \end{MyDescription}   		

\begin{MyDescription}[\arabic{stanza}]{}
   		In whom is no anxiety at all\\
   		   		to cause return to this existence here;\\
   		   		a bhikkhu such leaves here and there\\
   		   		as serpent ploughs its worn-out skin.
\stepcounter{stanza}
   \end{MyDescription}   
   		
\begin{MyDescription}[\arabic{stanza}]{}
	In whom no attachment formed at all\\
   		to cause return to all existences;\\
   		a bhikkhu such leaves here and there\\
   		as serpent ploughs its worn-out skin.		
\stepcounter{stanza}
   \end{MyDescription} 
   
   \begin{MyDescription}[\arabic{stanza}]{}
   	Who's left behind five hindrances,\\
   	   		serene then, crossed doubt, lacking inner barbs;\\
   	   		a bhikkhu such leaves here and there\\
   	   		as serpent ploughs its worn-out skin.	
   \stepcounter{stanza}
      \end{MyDescription} 
\begin{MyDescription}[(Sn. 1 - 17)]{}
\end{MyDescription}    
   
\newpage    
   \section{Notes on the Serpent Sutta}
   
A Sutta with many striking points which perhaps determined the fact that it comes first in this collection. The half-verse refrain notes that it describes a bhikkhu with various attainments. As the great commentator Buddhaghosa explains “bhikkhu” should be taken to mean `anyone who practises mindfulness', that is anyone who is devoted to maintaining mindfulness, the heart of Dharma, in their daily lives.
   
These practitioners `leave here and there': they have left attachment to `here' – this life, as well as `there', any future life. The P\=ali  uses the compound word `orapara', which is literally `this shore or bank and the further shore'.

\begin{itemize}
   
\item[5.] The fifth verse about one who finds no essence (sara) is compared with the lack of flowers upon fig-trees – Udumbara which the PTS Dictionary informs us is Ficus glomerata – needs a little explanation. None of the 800 species of Ficus or fig seems to have flowers, though they all have fruits, one of which is the well-known edible fig. But how could there be fruits without flowers first? The answer to this is that the receptacle, the small green figs, contain inside themselves the flowers which are pollinated there by small wasps, the eggs of which have been laid in there. When pollination has been accomplished the unripe receptacles swell and eventually soften, releasing a generation of small wasps which carry this process on. Trying to find identifiable flowers on Udumbara trees is a waste of time, a misleading venture as apart from the figs themselves there are no flowers. No `essential' flowers can be found.
   
\item[6.] The next verse which has the prase: `this-being-not-being' (itibhavabhavatam) includes all kinds of being or existence, even non-existence. Some people may have craving for non-existence, holding nihilistic theories, rather than the more common craving for various sorts of existence which supports the many views of eternalism. God-worshiping religions general have eternalist views.
   
\item[8-13.] These verses repeat their opening line, `Who neither goes too far nor lags behind' meaning one who does not resort to any sort of extreme. Extremes of views, speech and actions are popular in the world, now and as they were in the Buddha's time, while the Middle Way transcending all extremes is both hard to practise and requires effort and mindfulness. Slipping into extremes is not hard for the latest tendencies towards them are already embedded in our confused minds with their conception of `I am' and `I want'. `Lags behind', or in another translation, `hangs back', refers to attachment, to being and hence to the wrong views of eternalism. `Goes too far' or `overreaches' means the opposite” the extreme of non-being with its views on annihilation. (See the small Sutta at Itivuttaka, Twos, 22). The second lines of these (8-13) verses lists a number of areas in which it is possible to go too far or lag behind. In verse 8 this is `mind-proliferation', papanca, which could not be cured by extreme means. A natural Dharmic cure of papanca is Middle Way but most people, from heads of state round to nameless monks of various kinds, incline to extremes and so create for themselves and others more experience of samsara, more suffering, more conflict. `Mind-proliferation' is not just thinking too much but the indulging in extreme 'solutions' based on ego, defiled mental states and of entrenched tendencies.
   
The next five verses have in their second lines' `who of the world has known, `All is not thus''. This phrase `all is not thus' signifies: the world as it appears to most people is not as they commonly regard it. For instance, though change and impermanence are obvious in everything within the body and outside, yet generally people do not know and see this. They are blind to what they themselves are and blind to the world known through the senses. Those who are free from greed, lust, hate, delusion – they see the world as it is , or All is thus. `All' (sabbam/sarvam), is defined by the Buddha as, eye and sights, ear and sounds, nose and smells, tongue and tastes, body and touches, mind and thoughts. This is the All and no other all can be found. (See, Sn. 35.23, the All Sutta).
   
Verse 14 speaks of latent tendencies (anusaya) which all unenlightened persons have. They are like Melbourne's tram-tracks (or those of trams elsewhere). A tram must stick to its rails, it cannot turn left when the tracks go right. Its way is conditioned by the tracks and in the case of trams there is no possibility of changing that route unless the rails are re-laid. So we trundle along our tracks and never take a new way. We see and do what out latent tendencies permit us to see and do, a great limitation! But there are no limitations for those whose roots of evil (greed, hatred and delusion), `completely are expunged'.
   
Anxiety and fear are part of the ever-turning wheel of birth and death. Where there is the state of samsara there is fear. It is hard for us, enmeshed with fear and anxiety, even to imagine what the state of no-fear can be like. But to experience it we have to allow ego, the I-am view, to disappear. The same may be said for `attachment' in the next verse.
   
In the last, the five hindrances to deep meditation and spiritual experience, have gone – serenity manifests and all doubts are allayed – for one who has Seen what doubts could there be? With no `inner barbs' there is no obstacle to the heart's opening of loving-kindness and compassion.
   
For an excellent and longer commentary upon this Sutta by Ven. Nyanaponika Thera, see `The Worn-out Skin', The Wheel publication 241-242, published by Buddhist publication Society, Kandy, Sri Lanka 
\end{itemize}
   
   \chapter{Dhaniya Sutta –- With the Cattle-owner Dhaniya}
   
   \begin{MyDescription}[\arabic{stanza}]{Dhaniya:}
      	Who's left behind five hindrances,\\
      	   		Cooked is the evening rice, all milked the kine,\\
      	   		   			by Mahi's banks with friends, good cheer is mine,\\
      	   		   			my house well-thatched, my fire glows bright and still,\\
      	   		   			and so, rain on O sky, if such thy will!
      \stepcounter{stanza}
         \end{MyDescription} 

 \begin{MyDescription}[\arabic{stanza}]{The Buddha:}
Hatred and barrenness from me are gone,\\
by Mahi's banks I bide this night alone,\\
my house unroofed, my fires in ashes lie:\\
so, an it liketh thee, rain on O sky
      \stepcounter{stanza}
         \end{MyDescription} 


 \begin{MyDescription}[\arabic{stanza}]{Dhaniya:}
No stinging gnats are here to tease and fret,\\
my cattle crop the grasses lush and wet,\\
and take no hurt though floods the valley fill:\\
and so, rain on O sky, if such they will!
      \stepcounter{stanza}
         \end{MyDescription} 


 \begin{MyDescription}[\arabic{stanza}]{The Buddha:}
The raft is bound and well together cast,\\
the Further Shore attained, the flood o'erpassed.\\
of well-made raft what further need have I?\\
so, an it liketh thee, rain on O sky!
      \stepcounter{stanza}
         \end{MyDescription} 
  \begin{MyDescription}[\arabic{stanza}]{Dhaniya:}
Attentive is my wife, no wanton she,\\
long have I lived with her full happily,\\
nor ever heard of her a breath of ill:\\
and so, rain on O sky, if such thy will!
       \stepcounter{stanza}
          \end{MyDescription} 

 \begin{MyDescription}[\arabic{stanza}]{The Buddha:}
My mind attentive is, from passion freed,\\
long trained in wisdom's way, well-tamed indeed:\\
evil in me, what searcher can espy?\\
so, an it liketh me, rain on O sky!
\stepcounter{stanza}
          \end{MyDescription} 
          
 \begin{MyDescription}[\arabic{stanza}]{Dhaniya:}
My needs are met by my own body's hire,\\
my sturdy boys sit round my won house fire,\\
nor do I hear of them one word of ill:\\
and so, rain on O sky, if such they will!
          \stepcounter{stanza}
          \end{MyDescription} 
          
  \begin{MyDescription}[\arabic{stanza}]{The Buddha:}
No hireling I; to servile bonds inclined,\\
I walk all worlds with what I've earned in mind,\\
of wage or hire no smallest need have I:\\
so, an it liketh thee, rain on O sky!
\stepcounter{stanza}
            \end{MyDescription} 

 
\begin{MyDescription}[\arabic{stanza}]{Dhaniya:}  	
Cattle have I, yea, cows in milk are mine,\\
and cows with calf, and tender rising kine,\\
and lordly bulls whose ways the herds fulfil:\\
and so, rain on O sky, if such they will!
\stepcounter{stanza}
             \end{MyDescription}    
 
 \begin{MyDescription}[\arabic{stanza}]{The Buddha:}  	
Cattle I've none nor cows in milk are mine\\
nor cows with calf, nor tender rising kine,\\
nor lordly bulls to lead the herds have I:\\
so, an it liketh thee, rain on O sky!
 \stepcounter{stanza}
              \end{MyDescription}    
   
\begin{MyDescription}[\arabic{stanza}]{Dhaniya:}  	
The stakes all deeply driven, set firm and sure,\\
the newly-plaited ropes of grass secure.\\
No frenzied beast can break by any skill:\\
and so, rain on O sky, if such they will!
 \stepcounter{stanza}
              \end{MyDescription}    
              
\begin{MyDescription}[\arabic{stanza}]{The Buddha:}  	
Like bull, bursting the bond of plaited twine,\\
or elephant breaking free from stinky-vine,\\
ne'er again I'll enter in a womb to lie:\\
so, an it liketh thee, rain on O sky!
 \stepcounter{stanza}
\end{MyDescription}  
   
\begin{MyDescription}[\arabic{stanza}]{(Narrative):}  	
And now the furious showers came down amain\\
in pouring floods that covered hill and plain,\\
and, listening to the beating of the rain\\
Dhaniya, faithful, thus found voice again.
 \stepcounter{stanza}
\end{MyDescription}   

\begin{MyDescription}[\arabic{stanza}]{Dhaniya:}  	
Surely our gain is great and to be praised\\
whose eyes upon the Radiant One have gazed!\\
O Seeing One, we for refuge go to thee!\\
O Mighty Sage do Thou our Teacher be!
 \stepcounter{stanza}
\end{MyDescription}     

\begin{MyDescription}[\arabic{stanza}]{}  	
Attentive, lo! We wait my wife and I,\\
to live the goodly life, the pathway high\\
that leads beyond all birth and death to know\\
and win the final end of every woe.
 \stepcounter{stanza}
\end{MyDescription}   

\begin{MyDescription}[\arabic{stanza}]{Mara:}  	
He who has boys rejoices in his boys,\\
he who has kine, of kine are all his joys.\\
Man's assets surely are his chiefest treasure,\\
who has not assets how shall he have pleasure?\\
 \stepcounter{stanza}
\end{MyDescription}   

\begin{MyDescription}[\arabic{stanza}]{The Buddha:}  	
Whoso has boys, has sorrow of his boys,\\
whoso has kine, by kine come his annoys.\\
Man's assets, these of all his woes are chief.\\
Who has no assets, nevermore has grief.\\
 \stepcounter{stanza}
\end{MyDescription}  
\begin{MyDescription}[(Sn. 18 - 34)]{}
\end{MyDescription}   
\newpage
\section{Notes on the Dhaniya Sutta}
   This translation is based on the work of Ven. Silacara, an English bhikkhu (1871-1950). His translation, slightly amended here, reads well besides being accurate. I first read it in the library of Wat Bovoranives in Bangkok among the volumes of The Buddhist Review published by Luzac \& Co for the Buddhist Society of Great Britain and Ireland before the first World War. Ven. Silacara, at that time the Editor, lived in Burma. I have enjoyed reading at various times and places and would like to honour his memory by including it. To be a bhikkhu at that time was to brave the disapproval of the prevailing imperial culture. May this translation of his at least live on!
   
   This sutta has three characters: Dhaniya, a wealthy owner of cattle who speaks of his domestic and farming concerns, the Budsha whose verses contrast in the ways of Dharma; and Mara who raises his head only to speak one verse but a very important one. Beside this there is one narrative verse to give a picture of the monsoon rains as background to the dialogue.
   
   There are a few interesting points in these verses which call for comments. In the first verse spoken by the Buddha, his statement - 
\begin{MyDescription}[]{}
   	“my house unroofed, my fires in ashes lie”
\end{MyDescription}   
   
   is striking and memorable. As these words were spoken during the monsoon when protection from the elements is most necessary, how could the Buddha say “ my house unroofed”? This phrase puts one in mind of a verse found in the Vinaya:
\begin{MyDescription}[]{}   
   	Rain soddens when the roof is on\\
   	But never when it's opened up.n\\
   	Uncover then what is concealed\\
   	Lest it be soddened by the rain.\\
  	(Vinaya, Cullavagga, IX, 1)
\end{MyDescription}
   This vinaya quotation refers to the modern idea of a `cover-up', that certain facts about oneself should not be known and how those who practise this kind of immorality will become sodden and hence rotten. Buddhas have removed all their concealing coverings for they have nothing left to conceal. It's fine that the roof is off! Similar teachings are contained in the Sanskrit term avarana, translated by Conze as `thought-coverings' though I prefer `obscurations of mind' – the complications and weavings together of all sorts of thoughts: true and false, purified and defiled, open and concealed. These `coverings' are also related to the famous Buddhist term `papanca', multiplicity of thoughts or conceptual proliferation. Finally in this group of related matters dealing with coverings, there is the image of the Buddha Samantabhadra, aways depicted naked – nothing covered up and sky-blue in colour symbolizing space. He may or may not be accompanied by his consort , also naked and joined in bliss, the union of wisdom and compassion with nothing to hide.
   
   The second of the Buddhas's verses contains reference to the famous raft simile which appears complete at M.22. The raft made up of Dharma learned and practised and bound up with actions of body and speech which lead out of bondage is to be paddled across from this shore of samsarsa to the further shore of Nirvana. When that has been experienced the raft is no longer needed, it should be left on the nirvana shore or set adrift there – not carried around any longer. The Dharma liberates even from its own concepts. It liberates from all attachments, even from attachment to the Dharma. Of course this does not mean that one then neglects one's teachers or disowns the Dharma by which one has crossed over. Others will need the raft fro their won jounryes while a natural gratitude for the HDarma manifests in those who have seen it for themselves. The standard rendering of M.22 in “The Middle Length Discourses of the Buddha” unfortunately contains a mistranslation of the important last sentence of the simile (par. 14), which should read, “When you know the Dharma to be similar to a raft, you should abandon even the Dharma, how much more so not-dharma”. The published version is based upon the Commentaries of Buddhaghosa and is not what the P\=ali  text says. “Of well-made raft what further need have I?” – indeed!
   
   “evil in me, what searcher can espy?” The P\=ali  plainly rendered would translate as `evil in me cannot be found'. However, the translation used here reminds us of M.47, The Inquirer with its question, “are there found in the Tathagatha, or not, any defiled states cognizable through the eye or through the ear?”
   
   “I walk all the worlds with what I've earned in mind.” The venerable's translation has “I walk the world content with what I find”. The P\=ali  text has `all worlds – sabbeloke' while the verb (carati) has the meaning both of `travel, journey' and of `progress along a spiritual path'. Buddhas are able to review or investigate all states of existence though we usually have access only to the human and animal realms and even of these know little enough. `In mind' is not found in the P\=ali  though the Buddhas's `earnings' are certainly in mind, not of worldly gains.
   
   The two lines beginning, `Like bull', have been retranslated. The `stinky-vine' is some tough and malodorous liana in the forest. The third line originally reads, “No more shall I put on mortality” which is not a great rendering of the P\=ali , so I have replaced it with `Ne'er again I'll enter in a womb to lie'.
   
   Coming now to the last two verses of the Sutta, the first two lines of each of them have not been revised except to replace `hath' with `has'. The second pair of lines concern upadhi, a P\=ali  word conveying the meaning of `basis' or `support'. P\=ali  Commentaries have elaborated the meanings of this word and given it a far greater range. The original transalator used `being' as a rendering of upadhi, but this is too loose. Ven. Nyanamoli has suggested `assets' as a possible translation and this is fine so long as we remember that `assets' must include what we think we own in body, in mind as well – My body, My mind – as well as external possessions.
 \begin{MyDescription}[]{}
 “Man's assets, these of all his woes are chief,\\
   who has no assets nevermore has grief.”
 \end{MyDescription}  
  
\chapter{Kaggavis\=ana Sutta -- The Rhino Horn, Teaching for the Hermit-minded}
\begin{MyDescription}[\arabic{stanza}]{}  	
Put by the rod for all that lives,\\
tormenting not a single one;\\
long not for child – how then for friend,\\
fare singly as the rhino's horn
 \stepcounter{stanza}
\end{MyDescription}  

\begin{MyDescription}[\arabic{stanza}]{}  	
Attraction comes from meetings with\\
and from attraction dukkha's born;\\
see danger of attraction then,\\
fare singly as the rhino's horn
 \stepcounter{stanza}
\end{MyDescription}     

\begin{MyDescription}[\arabic{stanza}]{}  	
One full of ruth for friends well-loved\\
with mind unchanged, neglects the good\\
see fear in these familiar ties,\\
fare singly as the rhino's horn
\stepcounter{stanza}
\end{MyDescription} 
 
\begin{MyDescription}[\arabic{stanza}]{}  	
Tangled as the crowding bamboo boughs\\
is fond regard for partner, child:\\
as the tall tops are tangle-free\\
fare singly as the rhino's horn	
\stepcounter{stanza}
 \end{MyDescription} 
 
 \begin{MyDescription}[\arabic{stanza}]{}  	
The deer untethered roams the woods\\
going where'er it wants to graze:\\
seeing its liberty, wise one,\\
fare singly as the rhino's horn
  \stepcounter{stanza}
 \end{MyDescription}   
 
  \begin{MyDescription}[\arabic{stanza}]{}  	
`Mong friends one's asked for this or that\\
when resting, standing, going on tour,\\
seeing the liberty few desire\\
fare singly as the rhino's horn
   \stepcounter{stanza}
  \end{MyDescription}
  
   \begin{MyDescription}[\arabic{stanza}]{}  	
`Mong friends there's sexy playfulness\\ 
and love for children's very great,\\
while loath to part from those beloved\\
fare singly as the rhino's horn
    \stepcounter{stanza}
   \end{MyDescription} 
       
\begin{MyDescription}[\arabic{stanza}]{}  	
Resentment none to quarters four\\
and well-content with this and that,\\
enduring dangers undismayed\\
fare singly as the rhino's horn
\stepcounter{stanza}
\end{MyDescription} 
   
\begin{MyDescription}[\arabic{stanza}]{}  	
Some home-forsakers ill consort\\
as householders who live at home;\\
be unconcerned with others' kids!\\
fare singly as the rhino's horn
\stepcounter{stanza}
\end{MyDescription} 
   
\begin{MyDescription}[\arabic{stanza}]{}  	
Let fall the marks of householder\\
as Kovilara's parted leaves\\
a hero, having house-ties cut\\
fare singly as the rhino's horn	
\stepcounter{stanza}
\end{MyDescription} 

\begin{MyDescription}[\arabic{stanza}]{}  	
   For practice if one finds a friend – \\
   prudent, well-behaved and wise,\\
   mindful, joyful live as one\\
   all troubles overcoming
\stepcounter{stanza}
\end{MyDescription} 

\begin{MyDescription}[\arabic{stanza}]{}  	
For practice if one finds a friend – \\
prudent, well-behaved and wise,\\
mindful, joyful live as one\\
all troubles overcoming
\stepcounter{stanza}
\end{MyDescription} 

\begin{MyDescription}[\arabic{stanza}]{}  	
Surely we praise accomplished friends – \\
choose friends who're equal, or the best,\\
not finding these, live blamelessly,\\
fare singly as the rhino's horn
\stepcounter{stanza}
\end{MyDescription}
   
\begin{MyDescription}[\arabic{stanza}]{}  	
See golden bangles on an arm,\\
well-burnished by the goldsmith's art,\\
clash together, the two of them,\\
fare singly as the rhino's horn
\stepcounter{stanza}
\end{MyDescription}
   
\begin{MyDescription}[\arabic{stanza}]{}  	
When there's for me `a second one'\\
with intimate talk and curses both,\\
seeing this fear in future time\\
fare singly as the rhino's horn
\stepcounter{stanza}
\end{MyDescription}

\begin{MyDescription}[\arabic{stanza}]{}  	
Sense-things so sweet, so varied,\\
in divers forms disturb the mind;\\
seeing the bane of sense desires\\
fare singly as the rhino's horn
\stepcounter{stanza}
\end{MyDescription}

\begin{MyDescription}[\arabic{stanza}]{}  	
They are a plague, a blain, distress,\\
disease, a dart and danger too:\\
seeing this fear in sense-desires\\
fare singly as the rhino's horn
\stepcounter{stanza}
\end{MyDescription}
   
\begin{MyDescription}[\arabic{stanza}]{}  	
The heat and cold, and hunger, thirst,\\
wind, sun, mosquitoes' bites and snakes'\\
enduring one and all of these,\\
fare singly as the rhino's horn.
\stepcounter{stanza}
\end{MyDescription}
   
\begin{MyDescription}[\arabic{stanza}]{}  	
As elephant bull of noble mien,\\
full-grown, the flock forsakes and lives\\
in forests as it pleases him:\\
fare singly as the rhino's horn
\stepcounter{stanza}
\end{MyDescription}
   
\begin{MyDescription}[\arabic{stanza}]{}  	
`Who loves to live in company\\
e'en timely freedom cannot find';\\
so Kinsman of the Sun declared – \\
fare singly as the rhino's horn
\stepcounter{stanza}
\end{MyDescription}
   
\begin{MyDescription}[\arabic{stanza}]{}  	
View-contortions gone beyond,\\
right method won, the path attained,\\
`I Know! No other is my guide!'\\
fare singly as the rhino's horn
\stepcounter{stanza}
\end{MyDescription}
   
\begin{MyDescription}[\arabic{stanza}]{}  	
No greed, no guile, no thirst, no slur\\
and blown away by delusion's fault,\\
wantless in all the world's become\\
fare singly as the rhino's horn
\stepcounter{stanza}
\end{MyDescription}
   
\begin{MyDescription}[\arabic{stanza}]{}  	
Shun the evil friend who sees\\
no goal, convinced in crooked ways,\\
serve not at will the wanton one,\\
fare singly as the rhino's horn
\stepcounter{stanza}
\end{MyDescription}
   
\begin{MyDescription}[\arabic{stanza}]{}  	
Follow that friend who's deeply-learned,\\
Dharma-endowed and lucid, great,\\
knows meaning leading out of doubts\\
fare singly as the rhino's horn
\stepcounter{stanza}
\end{MyDescription}
   
\begin{MyDescription}[\arabic{stanza}]{}  	
In playful love and sensual joys\\
find no reward – no longer long;\\
embellish not but speak the truth,\\
fare singly as the rhino's horn
\stepcounter{stanza}
\end{MyDescription}
   
\begin{MyDescription}[\arabic{stanza}]{}  	
Partner, children, parents too, \\
kin and wealth – things bought with it,\\
leaving all sense-desires behind,\\
fare singly as the rhino's horn
\stepcounter{stanza}
\end{MyDescription}

\begin{MyDescription}[\arabic{stanza}]{}  	
`They are but bonds and brief their joys,\\
and few their sweets and more their ills.\\
Hooks in the throat!' This knowing well,\\
fare singly as the rhino's horn
\stepcounter{stanza}
\end{MyDescription}
   
\begin{MyDescription}[\arabic{stanza}]{}  	
Do snap the fetters, as a net\\
by river denizen is broke.\\
As fire to waste comes back no more,\\
fare singly as the rhino's horn
\stepcounter{stanza}
\end{MyDescription}

\begin{MyDescription}[\arabic{stanza}]{}  	
With downcast eyes, not loitering,\\
with guarded sense, warded thoughts,\\
with mind that festers not nor burns\\
fare singly as the rhino's horn
\stepcounter{stanza}
\end{MyDescription}
   
\begin{MyDescription}[\arabic{stanza}]{}  	
Discard householder's finery\\
as shed their leaves the Coral Trees;\\
go forth in K\=asaya robes,\\
fare singly as the rhino's horn
\stepcounter{stanza}
\end{MyDescription}
   
\begin{MyDescription}[\arabic{stanza}]{}  	
Crave not for tastes but free of greed\\
for alms food walk omitting none,\\
and unattached `mong families\\
fare singly as the rhino's horn
   \stepcounter{stanza}
\end{MyDescription}
   
\begin{MyDescription}[\arabic{stanza}]{}  	
Abandoned mind's five hindrances,\\
set aside defilements all,\\
affection-blemish having cut,\\
fare singly as the rhino's horn
\stepcounter{stanza}
\end{MyDescription}

\begin{MyDescription}[\arabic{stanza}]{}  	
Let go of pain and happiness\\
with previous joys and sorrows too,\\
gained poise and calm and purify,\\
fare singly as the rhino's horn
\stepcounter{stanza}
\end{MyDescription}
  
\begin{MyDescription}[\arabic{stanza}]{}  	
Resolved to win the Ultimate,\\
not slack in mind, nor slothful ways,\\
but steady, strong in body and mind,\\
fare singly as the rhino's horn\\
\stepcounter{stanza}
\end{MyDescription}
   
\begin{MyDescription}[\arabic{stanza}]{}  	
Seclusion, jh\=ana – do not cease\\
but what's in line with Dharma do,\\
with mastered existential fears\\
fare singly as the rhino's horn
\stepcounter{stanza}
\end{MyDescription}
   
\begin{MyDescription}[\arabic{stanza}]{}  	
Alert, aspiring craving's end,\\
clear-voiced and learned, mindful too,\\
striven, true Dharma having known,\\
fare singly as the rhino's horn
\stepcounter{stanza}
\end{MyDescription}
   
\begin{MyDescription}[\arabic{stanza}]{}  	
As lion is unafraid of sounds,\\
like wind not caught within a net,\\
as lotus not by water soiled\\
fare singly as the rhino's horn
\stepcounter{stanza}
\end{MyDescription}
      
\begin{MyDescription}[\arabic{stanza}]{}  	
As lion strong-toothed, the king of beasts,\\
subdues them all, so overcome\\
by use of practice-place remote,\\
fare singly as the rhino's horn
\stepcounter{stanza}
\end{MyDescription}

\begin{MyDescription}[\arabic{stanza}]{}  	
Frequent the mett\=a-mind, and ruth\\
at times, posed mind and joyful too –\\
unhindered mind by all the world.\\
fare singly as the rhino's horn
   \stepcounter{stanza}
\end{MyDescription}   

\begin{MyDescription}[\arabic{stanza}]{}  	
Lust, hatred and delusion gone,\\
all the fetters having snapped,\\
then at life's end, one trembles not,\\
fare singly as the rhino's horn.
\stepcounter{stanza}
\end{MyDescription} 
  
\begin{MyDescription}[\arabic{stanza}]{}  	
They serve and following having aims – \\
folk cunning, selfish-aimed and foul,\\
friends seeking nought are scarce today,\\
fare singly as the rhino's horn
\stepcounter{stanza}
\end{MyDescription} 
\begin{MyDescription}[(Sn. 35 - 75)]{}
\end{MyDescription} 
\newpage
\section{Notes for Rhinos}
   
This excellent sutta is famous throughout Buddhist traditions in various forms and deserves a good metric translation. In general few notes are needed as its meaning is clear, direct and straight to the point. This point is repeated in the fourth line of most verses, but will not appeal to those who do not esteem, even for part of their lives, its eremitic message. Though there are still monks who prefer the solitary life they are few compared to those who live in monastic settlements. And some lay scholars too may find shelter from the world's assaults for their study and practice and become `single-horned rhinos'. The rhinoceros is remarkable in India for its single horn as opposed to the twin horns of cattle, deer and so on. It is not that one should be `single as the rhinoceros' as some translators have it. In fact the animals are usually found in groups but their horns are only one.
The few notes below are preceded by their Sn. verse number.
I am very much in debt to E. M. Hare's translation of Sn. in `Woven Cadences' and have borrowed many of his good ideas though modifying them, and in his honour one whole verse, Sn. 57.
44. Kovil\=ara trees are these days called Bauhinia species. Their leaves are remarkable for their two leaflets joined at a single point. Many species open and close these two as the sun rises and sets. So their `parting' is demonstrated every day.
49. `A second one', literally `with a second' is a Pali idiom for a wife, the use of which here reminds me of `me old trouble (and strife)'!
50. Both `sense-objects' and `sense-desires' are translations of the Pali word k\=ama. This word is explained in the commentary of the K\=ama Sutta, Sn IV.1 or verses 766–771.
54. `E'en timely freedom cannot find' refers to k\=ala-vimutti, a freedom found temporarily and usually explained as the experience of jh\=ana. This contrasts with a vimutti or freedom beyond time, a liberation from all bondage. The `Kinsman of the Sun', Ādiccabandhu, is an epithet of the Buddha.
58. There seems to be, in this verse, a conflict between having a friend who is deeply learned and practised – a teacher in fact, and the refrain on faring singly. However, if one has the good fortune to meet and perhaps stay with such a teacher then when it is time to practise alone, one's retreat will be much more fruitful. With or without a teacher in our minds, we still have to `fare singly as the rhino's horn'.
64. Another tree, the Parichatta, is today known as the Erythrina indica, the species in general called Coral Trees. Most have spectacular red flowers borne on deciduous branches. They shed leaves before flowers appear. The leaves are here compared to the possessions of ordinary people (who are usually attached to them!) while the magnificent flowers which follow are hinted at by the words of the third line. `K\=asaya' robes refers to various vegetable dyes which will give robes of a yellow (many monks in Sri Lanka), reddish-brown (as in Burma) or yellowish-brown colour as used by the forest meditation monks of Thailand. These earthy colours remind practitioners of their own connection with the earth-element.
69. `What's in a line with Dharma do' tries to translate the frequently occurring `dhamm\=anudhammapatipatti', literally, `practising the Dharma according to the Dharma' and is opposed to the commonly held path of `practising the Dharma according to oneself', a very different kettle of fish. The Dharma which should be accorded with is the Dharma of what is true without any belief being necessary.
Examples:
`All conditioned things (and conditions) are impermanent
All conditioned things are dukkh\=a.
All dharmas are not self, or empty of self.'
These self-evident truths may have to be pointed out first but after deep insight (vipassan\=a) practice will be known from a practitioner's meditative experience. `Practising Dharma is the best practice of generosity, moral conduct, loving kindness and compassion with all beings near and far, human and non-human. That becomes `according to the Dharma' with awakening or breakthrough experiences. No views of any kind are held, grouped, or believed by those who have seen things as they really are, not even `Buddhist' views.
Also on 69. `and mastered existential fears' translates also as the fears of being or becoming. These may be experienced indirectly through reports in newspapers and other media of murder, wars, plagues, starvation, and all manner of inhumane conduct; or they may sometimes touch more closely on one's life. Many fears indeed! Being or becoming are also illustrated by the famous Indian painting (now Tibetan) of the Wheel of birth and death, showing the various realms of being and what one may, by making appropriate karma, experience there.
73. To pack the four Divine Abidings (Brahma-vih\=ara) into two short lines is not easy! Mett\=a or loving-kindness is usually translated has been left in the Pali for obvious reasons. Next, karun\=a or compassion has had to be expressed by the little-used `ruth'. (Maybe it is a comment on this world that we still employ `ruthless' but have forsaken `ruth'?) The third of these meditative abidings is joy but particularly the joy with others' happiness, a reflection for which we have no English word. (Does this mean we are singularly envious people?) The P\=ali word is mudit\=a. Last, upekkh\=a or equanimity is squashed into the second line as `those poised'. Still being equanimous it will not mind being so treated.
This sutta does not identify the teacher who composed and recited it. Presumably these were the Buddha's words! The first sutta in the book also records no speaker.
   
    
   \chapter{Kasībh\=aradv\=aja Sutta –- The Farmer Bharadvaja}
   
Thus have I heard:
As one time the Radiant One was dwelling among the Magadhans at South Mountain near the brahmin village of Ekan\=ala. Now at that time the brahmin Kasī-bh\=aradv\=aja had five hundred ploughts fastened to their yokes at the time of `planting'. Then in the morning the Radiant One dressed and, taking bowl and robe, went to the place where the brahmin Kasī-bh\=aradv\=aja was working.
As that time the brahmin Kasī-bh\=aradv\=aja 's food distribution was happening. Then the Radiant One approached the place for the distribution of food and stood to one side. The brahmin Kasī-bh\=aradv\=aja saw the Radiant One standing for alms as said to him:
"Samana, I plough and plant, and when I have ploughed and planted, I eat. You too, samana, ought to plough and plant; then when you have ploughed and planted, you will eat.”
“But I too, brahmin, plough and plant, and when I have ploughed and planted, I eat.”
“But we do not see Master Gotama's yoke or plough or ploughshare or goad or oxen; yet Master Gotama says, “I too brahmin, plough and plant, and when I have ploughed and planted, I eat.”
Then the brahmin Kasī-bh\=aradv\=aja addressed the Radiant One in verse:
   
\begin{MyDescription}[\arabic{stanza}]{}
`A ploughman so you claim to be\\
   but we see not your ploughmanship.\\
   If you're a ploughman, answer me,\\
   make clear your ploughmanship!”
   \stepcounter{stanza}
\end{MyDescription}

\begin{MyDescription}[\arabic{stanza}]{The Buddha:}
`With faith as seed and practice,\\
rain and learning as my yoke and plough.\\
my plough-pole, conscientiousness,\\
memory, goad and ploughshare both.
\stepcounter{stanza}
\end{MyDescription}
   
\begin{MyDescription}[\arabic{stanza}]{}
My harnessed ox is energy –\\
draws safe for yoking's end,\\
goes to where no sorrow is\\
and turns not back again.\\
\stepcounter{stanza}
\end{MyDescription}   

   
\begin{MyDescription}[\arabic{stanza}]{}
In this way is my ploughing ploughed\\
towards the crop of Deathlessness – \\
who finishes this ploughing's work\\
from all dukkh\=a will be free.
\stepcounter{stanza}
\end{MyDescription}
   
\begin{MyDescription}[\arabic{stanza}]{}
In this way is my ploughing ploughed\\
towards the crop of Deathlessness – \\
who finishes this ploughing's work\\
from all dukkh\=a will be free.
\stepcounter{stanza}
\end{MyDescription}
   
Then Kasī-bh\=aradv\=aja had a large bronze bowl filled with milk-rice and brought to the Radiant One. “May it please Master Gotama to eat the milk-rice, Master Gotama is a ploughman since he does the ploughing that has the Deathless as its crop.'
   
\begin{MyDescription}[\arabic{stanza}]{The Buddha:}
Chanting sacred verses for comestibles – not done by me,\\
for those who rightly See, Brahmin, it accords not with Dharma.\\
Chanting sacred verses thus is rejected by the Buddhas,\\
such is the Dharma, Brahmin, such is their practice.
\stepcounter{stanza}
\end{MyDescription}

\begin{MyDescription}[\arabic{stanza}]{}
A great seer with Final Knowledge, conflicts stilled.\\
one who has exhausted taints, is wholly free – \\
make offerings of food and drink to such a one:\\
the certain field for one who merit seeks.
\stepcounter{stanza}
\end{MyDescription}
   
When this was said the brahmin Kasī-bh\=aradv\=aja exclaimed to the Radiant One: `Magnificent, Master Gotama! The Dharma has been clarified by Master Gotama in many ways, as though he was righting what had been overturned, revealing what was hidden, showing the way to one who was lost, or holding a lamp in the dark so that those with eyes could see forms. I go for refuge to Master Gotama, to the Dharma and to the Sangha. May Master Gotama remember me as an up\=asaka who from today has gone for Refuge for life.'

\begin{MyDescription}[(SN76-82)]{}

\end{MyDescription}    
\newpage   
\section{Notes on the Farmer Bharadvaja Sutta}
   
This famous small sutta appears twice in the P\=ali canon, once here and again in the Samyutta-nik\=aya (see translation at the Connected Discourse of the Buddha, Book One, VII.2.1). This present translation varies from it, necessitating some notes.
First of these concerns the cultivation of rice. Most of my readers will be aware that unlike other grain crops – wheat, oats, barley, maize, etc – which are planted with dry seed straight into the prepared field, rice requires first to be planted into a flooded nursery bed and then when the seedlings are ready, re-planted in small clumps in a flooded field. Water must be maintained at the proper level throughout its growth, only beginning to dry out with the ripening of the grain.
The prose introduction of this sutta informs us that the farmer-brahmin, a wealthy landowner able to muster `five hundred' ploughmen and oxen, (`500' in Pali means `a lot, many') was that time celebrating `Vappak\=ala', the time (k\=ala) of Vappa. But what is meant by Vappa? The translation given by the PTSD and followed by all translators is `sowing' but as we have seen above, rice is not sown in the way of other grains. If a religious celebration involving many men and oxen with (probably chanting by Brahmins) and the offering of cooked milk-rice for all, is called for this seems to be the planting of the seedlings rather than the ploughing of the land. The verb for `ploughing' is `kasati' though in the case of rice this should include operations other than the initial breaking of the soil with ploughs. The mud of the field must be smooth and without lumps, so it is harrowed. This is all `kasati' – to prepare the soil. It is followed by `vapati', the verb for planting or sowing though the farmer is not mentioned in PTSD. 
   
So at this vappak\=ala what is going on? Fields had already been flooded, ploughed, with perhaps harrowing in progress. The backbreaking work of planting, traditionally women's work, though this is not mentioned here, would be in progress. The wealthy farmer dressed in his best and newest white cloth would be issuing orders and coordinating the whole operation while also superintending the rituals ensuring that there would be a good crop. A large amount of milk-rice had been prepared for the various religious `wanderers' and also for Brahmins. Part of this would also be set aside for the labourers.
In this translation `sowing' is not mentioned, its place being taken by planting'.
The Buddha's first verse of explanation begins `With faith as seed ...' The Pali word which translates as `faith' is saddh\=a (Skt Raddh\=a). The meaning of faith to most people is equivalent to `belief' but in a Buddhist sense saddh\=a does not mean belief. Belief involves accepting certain formulations of words as representing the truth. And in this world there are many such `truths' (see Sn. 884-886) underlain by belief, none of them verifiable, many of them at odds with others and hence the basis for many conflicts between believers, political or religious, even for persecution, torture and wars. Throughout the Suttas the Buddha has emphasised that holding views, attachment to them (even Buddhist ones) departs from the path of Dharma. One must know through personal experience involving wisdom, not merely believe. Saddh\=a, therefore, is a tricky word to translate and sometimes `faith' must be used. Other possible translations are `confidence' and `assurance' but neither has quite the range of meaning of saddh\=a.
Tapo, translated `practice' means to a brahmin `severe austerities' or at least a very austere mode of life. Buddhists have softened the word to mean steady Dharma practice. As the Buddha is teaching a brahmin who would not understand the higher meanings of pa\~n\~na/praj\~na I have not translated it as wisdom. But among the three steps of pa\~n\~na there is suta-masya-pa\~n\~na – the wisdom acquired by listening or learning, a feature of all Indian religions and well-known to the brahmin. In the last line of this verse `memory' is a possible translation of sati/smrti, another word well-known to brahmins. Also smrti, `that which is remembered' is used in Hindu tradition as a name of the collection of commentarial works, as opposed to the holier god-given corpus of Vedas and Upanishads known as `sruti', `that which has been heard'.
My knowledge of horticulture or perhaps agriculture has aided the translation of the second verse. I have paid little attention to the P\=ali Commentary's suggestions as these do not appear to make much sense.
The third verse is beautiful and its profound but straightforward meaning very moving. `Yokings End' is liberation.
It is not surprising that the brahmin having heard these four spontaneous verses is greatly impressed. In fact, he has been convinced that the Buddha is indeed also a cultivator who ploughs and plants but that his crop (literally `fruit') is the Deathless (amata/amrta). He wishes then to offer some of the specially prepared milk-rice to the Buddha in a large copper bowl, no doubt as a mark of his respect.
But to this invitation the Buddha replies in an unexpected way, saying that he does not chant sacred verses to gain food – presumably a thing done by many brahmins then. Actually they would be inconvenient to present day monks who if they kept to the Buddha's practice, would lose many a good morning meal. The meaning of this verse is very straightforward.
This cannot be said of the second verse, which appears to mean that Awakened persons are fit recipients of food and will be a `certain field for one who merit seeks' – in other words, donors will make good karma by offerings given to such people. Perhaps this verse is added later by monks to pad against the impact of the previous verse. In any case there is conflict between these two verses.
I have chosen the shorter version of this sutta from Samyutta-nik\=aya, rather than the text in Sn. The latter adds an incident in which the good brahmin wishes to known to whom he could offer the rice. The Buddha replies that he knows not a single person who could receive it and advises the brahmin to put it in water or in the bush where no living thing can be harmed by it. The brahmin in doing this finds that as it is poured into water it boils sizzling and hissing with much steam, so that the brahmin trembles and his hair stands on end. At the conventional end of the sutta in this version the brahmin does not ask for the Refuges and becomes an up\=asaka, but requests the leaving home with ordination as a bhikkhu.
It seems likely that the Sn. version is an expansion of a more ancient and simpler original now found in S. The magical addition of boiling milk-rice tipped into a stream seems a glorification of the Buddha done by later hands. This decided me not to translate the Sn. version.
Portions of the Commentary's explanations are translated into English in the Connected Discourses of the Buddha (Vol. 1), see notes 446 following on p. 446. There are rather astonishing remarks in these `explanations', for the P\=ali commentator has apparently had access to the Buddha's thoughts (!) no doubt a very useful trick for ascertaining the truth of any problematic statement in the Suttas. To preface any `explanation' with the words `This was his (the Buddha's) thought' is a claim to know precisely what an Awakened One was thinking. How could the truth be found in this way?
   
   
    
\chapter{Cunda Sutta -- To the Smith Cunda}

\begin{MyDescription}[\arabic{stanza}]{Cunda, Son of a smith:}
I ask of the Sage abundantly wise,\\
Buddha, Lord of Dharma, one who's craving-free,\\
Best among men, charioteer beyond compare,\\
Please do tell me what sorts of monks there are.
\stepcounter{stanza}
\end{MyDescription}
   
\begin{MyDescription}[\arabic{stanza}]{The Buddha:}
Ask by you personally I shall explain:\\
Four are the samanas, not a fifth is found – \\
Won to the Path, of the Path the Indicator,\\
Who lives upon the Path, as well the Path-polluter.
\stepcounter{stanza}
\end{MyDescription}
  
\begin{MyDescription}[\arabic{stanza}]{Cunda:}
Who do the Buddhas say is winner of the Path?\\
How will the Path-teacher be incomparable?\\
Tell about that one who lives upon the Path,\\
Also the one who is the Path-polluter?
\stepcounter{stanza}
\end{MyDescription}
   
\begin{MyDescription}[\arabic{stanza}]{The Buddha:}
Who so has passed beyond the dart of doubts\\
Nirvana-delighted, no greediness at all,\\
Leader of the world together with the gods,\\
is Such, the Path-winner, so the Buddhas say.
\stepcounter{stanza}
\end{MyDescription}
   
\begin{MyDescription}[\arabic{stanza}]{}
Who knows the Best as what is best indeed,\\
then teaches Dharma and analyses it,\\
a sage all doubt severed, one undisturbed,\\
they call bhikkhu number two indicator of the Path.
\stepcounter{stanza}
\end{MyDescription}
   
\begin{MyDescription}[\arabic{stanza}]{}
Who lives on the Way, the well-taught Dharma Path,\\
one well-trained and mindful as well,\\
whatever's unobstructing, a practitioner of that\\
they call bhikkhu number three, one who lives the Path.
\stepcounter{stanza}
\end{MyDescription}
   
\begin{MyDescription}[\arabic{stanza}]{}
Making a semblance of those with good vows,\\
deceitful one, worthless and quite unrestrained,\\
Insolent, braggart and family-defiler,\\
who goes in disguise is polluter of the Path.
\stepcounter{stanza}
\end{MyDescription}
   
\begin{MyDescription}[\arabic{stanza}]{}
a noble disciple who's recognised each and every one,\\
and knowing that among them, all are not alike,\\
this having seen, that person's faith does not decrease\\
for how with the corrupt can the uncorrupted be compared\\
or those purifies with those who are impure?
\stepcounter{stanza}
\end{MyDescription}
   
\begin{MyDescription}[(Sn. 83-90)]{}
\end{MyDescription}
   							
\newpage    
\section{A Few Words}
   
A rather strange little sutta in which the Buddha answers the questions of the Smith Cunda. He wishes to be clear about what sorts of monks there are. In the Buddha's days there were a great variety of monks, some of who wandered in groups, others who were solitary, some who had monasteries, others who dwelt in caves or hollow trees. Their doctrines varied even more and in the Discourse of the Buddha these are examined as `ditthi' or views and sometimes revealed as `wrong views'.
   
The Buddha on this occasion limits the sorts of monks to his own samanas, further limiting them to four and rather oddly adding “not a fifth is found”. This of course acknowledges the four sorts of monk, which the Buddha knows exist among his own Sangha. One may assume that what is true of the first three kinds of monks – bhikkhus, plus the fourth who is really not a bhikkhu, may also be found among the bhikkunis. Stories of recalcitrant monks and nuns may be found in plenty in the Vinaya for both bhikkhus and bhikkunis. The four sorts of monks and briefly described below.
\begin{enumerate}
\item Won to the Path. This appears to be a practitioner who has for the first time experienced what is truly the Noble Path (ariya-magga). They are the beyond doubt and have lost all greed. Also, lost is ownership for though these monks who have won to the Path they do not own it. In fact, having experienced suchness, the way all things really are they can let go of everything.
   
\item Indicator of the Path is the best of teachers of Dharma – they teach Dharma out of their deep experience of it. Who knows the best is truly awakened. He teaches with clarity, as those of his students know well and is a man who analyses accurately so that no misunderstanding can occur. They no longer have doubts about things that to ordinary persons seem either worth no understanding or are taken for granted. They point to matters most obvious, like impermanence, which in general are not noticed.
   
\item	“Who lives on the Way, the well-taught Dharma Path”. This kind of monk is learned and his actions agree with the teachings in the Suttas as well as those of those famous living Teachers. So wether from their Teachers or from Suttas the Dharma is “well taught”. This expression means that at first it is derived from the Buddha and his disciples: all of their uttered words are derived from the Enlightenment – Bodhi. They had woken up and spoke the non-basis of Awakening. Their truth is not that of speculations and `views'. As an example:

\begin{MyDescription}[]{}   
   “All conditioned thing are impermanent,\\
   All conditioned things are dukkha,\\
   All dharmas are not-self (and empty)”
\end{MyDescription}
\item Those who pretend to be bhikkhus but actually are corrupted. That this type of monk is included with the other three shows the honesty of the Buddha. If he had referred to monk number four as only to be found among the ascetics of other teachers that would have been a sort of dishonesty but the Buddha knew quite well that he had corrupt monks in his Sangha. Though it is hard to believe that corrupt monks could live near to the Buddha's presence, because of course he had the power of reading the minds of others, still some monks were (and are) shameless and thoroughly corrupt in their dealings. It is not so surprising that such monks can be found in our days. The Buddha points them out with these characteristics – 
\begin{description}
\item[a)]They pretend by copying the ways of those who keep their precepts,
\item[b)]They are deceitful, and trying to corrupt others,
\item[c)]Worthless – of receiving the gifts of honest practitioners,
\item[d)]Quite unrestrained means that they indulge even in the pleasures usually sanctioned by society
\item[e)]Insolent and braggarts, hardly needs any comments,
\item[f)]“Family-defilers”, leaves one's mind to many possibilities.
\end{description}      
\end{enumerate}   
    
\chapter{Par\=abhava Sutta -- Disaster}
   
Thus have I heard: At one time the Radiant one was dwelling at Jeta's Grove in the park of Anathapindika near Savatthi. Then as the night was ending a deva of surpassing radiance, illuminating the whole of Jeta's Grove, went up to the Radiant One and stood to one side after saluting him. Standing there that deva addressed the Radiant One with a verse:
   
\begin{MyDescription}[\arabic{stanza}]{Deva:}   
To ask the lord we come here\\
from Gotama we wish to know:\\
That one who goes disaster way\\
what's the way to disaster's woes?
\stepcounter{stanza}
\end{MyDescription}

\begin{MyDescription}[\arabic{stanza}]{The Buddha:}   
The wise one does develop well,\\
the unwise to disaster bound;\\
the lover of Dharma develops well,\\
Dharma-hater to disaster's round.
\stepcounter{stanza}
\end{MyDescription}
    
\begin{MyDescription}[\arabic{stanza}]{Deva:}   
We clearly understand this much,\\
that way's first to disaster's woe,\\
second, may the Lord advise – \\
what's the way to disaster's woe?
\stepcounter{stanza}
\end{MyDescription}

\begin{MyDescription}[\arabic{stanza}]{The Buddha:}   
The untrue, they are dear to me,\\
true persons, they're not dear, so\\
the untrue teaching one prefers:\\
that's the way to disaster's woe.
\stepcounter{stanza}
\end{MyDescription}  
 
\begin{MyDescription}[\arabic{stanza}]{Deva:}   
We clearly understand this much,\\
second that way's to disaster's woe,\\
thirdly, may the Lord advise – \\
what's the way to disaster's woe?	
\stepcounter{stanza}
\end{MyDescription} 
   
\begin{MyDescription}[\arabic{stanza}]{The Buddha:}   
Lethargic and gregarious –\\
whoever is of effort low,\\
lazy and anger marked:\\
that's the way to disaster's woe.
\stepcounter{stanza}
\end{MyDescription} 
   
\begin{MyDescription}[\arabic{stanza}]{Deva:}   
We clearly understand this much,\\
third that way's to disaster's woe,\\
fourthly, may the Lord advise – \\
what's the way to disaster's woe?
\stepcounter{stanza}
\end{MyDescription} 
   
\begin{MyDescription}[\arabic{stanza}]{The Buddha:}   
Though wealth's enough one does not help\\
mother and father who aged grow\\
though long their youth is left behind:\\
that's the way to disaster's woe.
\stepcounter{stanza}
\end{MyDescription} 
   
\begin{MyDescription}[\arabic{stanza}]{Deva:}   
We clearly understand this much,\\
that way's fourth to disaster's woe,\\
fifthly, may the Lord advise – \\
what's the way to disaster's woe?
\stepcounter{stanza}
\end{MyDescription}
   
\begin{MyDescription}[\arabic{stanza}]{The Buddha:}   
Whether with priest or monk as well
one likes to lie and cheat, also
deceiving other wanderers:
that's the way to disaster's woe.
\stepcounter{stanza}
\end{MyDescription}

\begin{MyDescription}[\arabic{stanza}]{Deva:}   
We clearly understand this much,\\
that way's fifth to disaster's woe,\\
sixthly, may the Lord advise – \\
what's the way to disaster's woe?
\stepcounter{stanza}
\end{MyDescription}

\begin{MyDescription}[\arabic{stanza}]{The Buddha:}   
A person of great property,\\
with wealth and food they overflow\\
and yet enjoy it sweets alone:\\
that's the way to disaster's woe.
\stepcounter{stanza}
\end{MyDescription}

\begin{MyDescription}[\arabic{stanza}]{Deva:}   
We clearly understand this much,\\
that way's sixth to disaster's woe,\\
seventh, may the Lord advise – \\
what's the way to disaster's woe?
\stepcounter{stanza}
\end{MyDescription}
   
\begin{MyDescription}[\arabic{stanza}]{The Buddha:}   
Proud of birth and proud of wealth'\\
so of their families they crow,\\
but meeting, slight their relatives:\\
that's the way to disaster's woe.
\stepcounter{stanza}
\end{MyDescription}  

\begin{MyDescription}[\arabic{stanza}]{Deva:}   
We clearly understand this much,\\
seventh that way's to disaster's woe,\\
eighthly, may the Lord advise – \\
what's the way to disaster's woe?
\stepcounter{stanza}
\end{MyDescription} 


\begin{MyDescription}[\arabic{stanza}]{The Buddha:}
Debauched in brink, with women too,\\
be dice debauched, -such a fellow,\\
little by little his assets waste:\\
that's the way to disaster's woe.
\stepcounter{stanza}
\end{MyDescription}    
   			
\begin{MyDescription}[\arabic{stanza}]{Deva:}
We clearly understand this much,\\
that way's eighth to disaster's woe,\\
ninthly, may the Lord advise – \\
what's the way to disaster's woe?
\stepcounter{stanza}
\end{MyDescription}			   

\begin{MyDescription}[\arabic{stanza}]{The Buddha:}
Unsatisfied with his own wife,\\
with others' wives he's seen in tow,\\
corrupted too with prostitutes:\\
that's the way to disaster's woe.
\stepcounter{stanza}
\end{MyDescription}	   

\begin{MyDescription}[\arabic{stanza}]{Deva:}
We clearly understand this much,\\
that way's ninth to disaster's woe,\\
tenthly, may the Lord advise – \\
what's the way to disaster's woe?
\stepcounter{stanza}
\end{MyDescription}      

\begin{MyDescription}[\arabic{stanza}]{The Buddha:}
A  man on longer young still weds\\
a girl with apple breasts – and lo!\\
for jealousy he cannot sleep:\\
that's the way to disaster's woe.
\stepcounter{stanza}
\end{MyDescription}     

\begin{MyDescription}[\arabic{stanza}]{Deva:}
We clearly understand this much,\\
that way's tenth to disaster's woe,\\
may the Lord eleventh advise – \\
what's the way to disaster's woe?
\stepcounter{stanza}
\end{MyDescription}     
   
\begin{MyDescription}[\arabic{stanza}]{The Buddha:}
Whoever, whether woman or man,\\
drunken, dissolute, wealth does blow,\\
then in position of power is placed:\\
that's the way to disaster's woe.
\stepcounter{stanza}
\end{MyDescription}
      
\begin{MyDescription}[\arabic{stanza}]{Deva:}
We clearly understand this much,\\
eleventh that way's to disaster's woe,\\
twelfthly, may the Lord advise – \\
what's the way to disaster's woe?
\stepcounter{stanza}
\end{MyDescription}
   	   
\begin{MyDescription}[\arabic{stanza}]{The Buddha:}
When, or noble family sprung\\
with little wealth, great craving though,\\
and still one wants to rule the realm:\\
that's the way to disaster's woe.
\stepcounter{stanza}
\end{MyDescription}
   
\begin{MyDescription}[\arabic{stanza}]{}
That's one who's wise well knows the way\\
within this world to disaster's woe,\\
and then that Noble, insight pure\\
to a blessed state one such does go.
\stepcounter{stanza}
\end{MyDescription}

\begin{MyDescription}[(Sn. 91-115)]{}
\end{MyDescription}
\newpage
\section{Commentary}
   
   This translation was first published in the newsletter, Bodhi leaf, of Wat Buddha Dhamma, Vol 2, No.4, 1986. I enjoyed translating it, partly because of its rhyme- the P\=ali original has no rhyme – and partly the contents, the truths of which are not confined to the Buddha's days! The setting of this sutta reminds me of the Sag\=atha-vagga, the Book with Verses, of the Samyutta-nik\=aya, the Connected Discourses\footnote{See Volume 1 of the Connected Discourses of the Buddha}. There, as in the present discourses, a deva asks questions of the Buddha. In this case it is within a stylised framework of the deva's acknowledging the Buddha's replies and asking further advice on, “what's the way to disasters woe?” This particular repeated line in P\=ali: kim par\=abhavato mukhaṁ has a nice swing to it which I sought to repeat. However, par\=abhavato mukhaṁ is not easily translated. `Mukha' means entrance, mouth, face, while `par\=abhava'  has the sense of decline or disaster, so the entrance or way to disaster. The whole sutta, though it contains very straightforward advice, is like a game between the playful deva and the Buddha who frames his replies in verses which end “that's the way to disasters woe”. This fits well with the what we know of deva who traditionally are taught the Dharma by singing and acting it. The `lower' devas of the Sense- realm at any rate, were used to pleasures and could only respond to Dharma teaching through singing it using such playful verbal presentations as in this sutta. Presumably, since conditions in this world are infinitely variable, the Buddha could have gone on all day with this numerical game but perhaps he thought that the deva had enough material on the causes of dukkha (not many devas, or even human beings for that matter, are much interested in this), to be getting along with. The sutta closes with the twelfth way to disaster's woe.
   The content of the sutta – from Dharma and its teachers, through family and social considerations, to the corruption of politicians, is especially wide-ranging and it would be possible to write a long commentary with suitable stories from such sources as the Suttas, Dhammapada Commentary and the J\=atakas as well as from our own times on these ways to disaster's woe.
   The rhyme which I think adds to its colour and swing, and possibly to its ease of leaning by heart, was possible given the pattern of the sutta's verses. I have not attempted rhyme elsewhere in this Sutta nip\=ata translation.
   
   \chapter{Vasala Sutta –- Who is the Outcaste?}
   
   Thus have I heard:
   At one time the Radiant One was dwelling at S\=avatthi in the Jeta Grove – Anathapindika's Park. Then in the morning the Radiant One dressed and, taking bowl and robes, entered S\=avatthi, for almsfood. At that time the sacrificial fire was burning in the house of the brahmin Aggika-bh\=aradv\=aja and the offering was held aloft. Then the Radiant One walking in almsround, house by house within S\=avatthi, came to the house of brahmin Aggika-bh\=aradv\=aja. The brahmin saw the Radiant One coming from a distance and called out this to him: `Stop there, mere shaveling, stop there, vile ascetic, stop there, foul outcaste!' When this was said, the Radiant One said to the brahmin: `Do you know, brahmin, what an outcaste is or what things make a person `outcaste'?'
   `I do not know good Gotama what an outcaste is or what things make one an outcaste. It would be good for me if the venerable Gotama were to teach me Dharma, so that I might know an outcaste and what things make an outcaste'
   `Then listen, brahmin, pay attention and I shall tell you'
   `Yes, venerable sir', replied the brahmin.

\begin{MyDescription}[\arabic{stanza}]{The Buddha:}
An angry person, rancorous,\\
with evils of hypocrisy,\\
deceitful and of fallen views,\\
as `outcaste' such a one is known
\stepcounter{stanza}
\end{MyDescription}

\begin{MyDescription}[\arabic{stanza}]{}
Whether once or twice-born then\\
if one should living beings harm,\\
compassion for them – none at all,\\
as `outcaste' such a one is known.
\stepcounter{stanza}
\end{MyDescription}
   
\begin{MyDescription}[\arabic{stanza}]{}
Who kills in towns and villages,\\
destruction brings, and then behaves\\
oppressively – well-known for that,\\
as `outcaste' such a one is known.
\stepcounter{stanza}
\end{MyDescription}

\begin{MyDescription}[\arabic{stanza}]{}
Whoso in forest or in town\\
steals whatever is not given,\\
from others to whom it's valuable,\\
as `outcaste' such a one is known.
\stepcounter{stanza}
\end{MyDescription}
   
\begin{MyDescription}[\arabic{stanza}]{}
Whoever does a debt contract\\
but urged to repay, then retorts,\\
No debt have I to you indeed',\\
as `outcaste' such a one is known.
\stepcounter{stanza}
\end{MyDescription}

\begin{MyDescription}[\arabic{stanza}]{}
Who, for a trifle that's desired\\
from traveller along the road,\\
kills, that trifle to possess,\\
as `outcaste' such a one is known.
\stepcounter{stanza}
\end{MyDescription}

\begin{MyDescription}[\arabic{stanza}]{}
Whoso for self or others' wealth,\\
or else for benefit of wealth\\
when questioned on this, falsehood speaks,\\
as `outcaste' such a one is known.
\stepcounter{stanza}
\end{MyDescription}   
   
\begin{MyDescription}[\arabic{stanza}]{}
Whoso is `seen' with others' wives,\\
of relatives and friends, those\\
consenting mutually or forced,\\
as `outcaste' such a one is known.
\stepcounter{stanza}
\end{MyDescription}   

\begin{MyDescription}[\arabic{stanza}]{}
Whoso towards their mum or dad\\
whose youth is gone and age attained,\\
though prosperous, supports them not,\\
as `outcaste' such a one is known.
\stepcounter{stanza}
\end{MyDescription}   

\begin{MyDescription}[\arabic{stanza}]{}
Whoever strikes, or utters hate\\
against mother, father, brother too,\\
sister or a mother-in law,\\
as `outcaste' such a one is known.
\stepcounter{stanza}
\end{MyDescription}   

\begin{MyDescription}[\arabic{stanza}]{}
Who, asked for good advice responds\\
by giving bad advice, and then\\
teaching in obscurities,\\
as `outcaste' such a one is known.
\stepcounter{stanza}
\end{MyDescription}   
   
\begin{MyDescription}[\arabic{stanza}]{}
Whoever evil karma makes\\
wishing others may not know,\\
and then conceals these actions bad,\\
as `outcaste' such a one is known.
\stepcounter{stanza}
\end{MyDescription}   

\begin{MyDescription}[\arabic{stanza}]{}
Who, gone to another's house,\\
enjoys fine hospitality,\\
then honours not the other back,\\
as `outcaste' such a one is known.
\stepcounter{stanza}
\end{MyDescription}   

\begin{MyDescription}[\arabic{stanza}]{}
Whoever, brahmin, samana,\\
or even indigents who beg,\\
deceives with false and lying speech,\\
as `outcaste' such a one is known.
\stepcounter{stanza}
\end{MyDescription}   

\begin{MyDescription}[\arabic{stanza}]{}
Whoso when mealtime has arrived\\
abuses brahmins, samanas\\
and then gives not a thing to them,\\
as `outcaste' such a one is known.
\stepcounter{stanza}
\end{MyDescription}   

\begin{MyDescription}[\arabic{stanza}]{}
As blanketed, delusion-wrapped,\\
who predicts untruthful things\\
desiring even trifling gain,\\
as `outcaste' such a one is known.
\stepcounter{stanza}
\end{MyDescription}  

\begin{MyDescription}[\arabic{stanza}]{}
Whoever does exalt themselves\\
while looking down on others, though\\
inferior, caused by self-conceit,\\
as `outcaste' such a one is known.
\stepcounter{stanza}
\end{MyDescription}      
   
\begin{MyDescription}[\arabic{stanza}]{}
Provocative and selfish too,\\
of evil wishes, miserly,\\
cunning, shameless, no remorse,\\
as `outcaste' such a one is known.
\stepcounter{stanza}
\end{MyDescription}      

\begin{MyDescription}[\arabic{stanza}]{}
Whoso the Buddha does revile,\\
insulting his disciples too\\
whether left home or laity,\\
as `outcaste' such a one is known.
\stepcounter{stanza}
\end{MyDescription}      

\begin{MyDescription}[\arabic{stanza}]{}
Who, though not an Arahant,\\
yet pretends to be – is Thief –\\
in this world with Brahmin gods,\\
the lowest outcaste of them all,\\
These indeed are `outcaste' called\\
as I've declared to you.
\stepcounter{stanza}
\end{MyDescription}      

\begin{MyDescription}[\arabic{stanza}]{}
One's not an outcaste caused by `birth',\\
not by `birth' a brahmin is;\\
caused by karma one's outcaste,\\
a brahmin is by karma caused.
\stepcounter{stanza}
\end{MyDescription}      

\begin{MyDescription}[\arabic{stanza}]{}
Know this is true according to\\
the example following here:\\
an outcaste boy well-known to you\\
as M\=atanga of the of the Sopakas.......
\stepcounter{stanza}
\end{MyDescription}      

\begin{MyDescription}[\arabic{stanza}]{}
M\=atanga gained the highest fame,\\
so hard a thing for him to gain;\\
then warriors, brahmins, others too,\\
many came to serve him.
\stepcounter{stanza}
\end{MyDescription}      

\begin{MyDescription}[\arabic{stanza}]{}
Upon the way to deva-worlds,\\
set forth along the spotless path\\
and cleansed of sense-desired,\\
attained to Brahma's world, they say;\\
unhindered he by `birth' at all\\
he won to Brahma-worlds.
\stepcounter{stanza}
\end{MyDescription}      
   
\begin{MyDescription}[\arabic{stanza}]{}
Though born in Veda-chanting clan,\\
brahmins with mantras as their kin,\\
frequently indeed they're seen\\
while making evil karmas,
\stepcounter{stanza}
\end{MyDescription}       

\begin{MyDescription}[\arabic{stanza}]{}
even in this world they're blamed,\\
the next for them's a painful bourn;\\
birth' hinders not a painful bourn,\\
nor from being blamed.
\stepcounter{stanza}
\end{MyDescription}    

\begin{MyDescription}[\arabic{stanza}]{}
One's not an outcaste caused by `birth',\\
not by `birth' a brahmin is;\\
caused by karma one's outcaste,\\
a brahmin is by karma caused.
\stepcounter{stanza}
\end{MyDescription}    

   
   
   
When this was said, the brahmin Aggika-bh\=aradv\=aja exclaimed to the Radiant One: `Magnificent, Master Gotama! The Dharma has been clarified by the Master Gotama in many ways, as though he was righting what was overthrown, revealing what was hidden, showing the way to one who was lost, or holding a lamp in the dark so that those with eyes can see forms. I go for refuge to Master Gotama, to the Dharma and to the Sangha. May Master Gotama remember me as an up\=asaka who from today has Gone for Refuge for life.

\begin{MyDescription}[(Sn. 116-142)]{}
\end{MyDescription}     							
\newpage   							
\section{Commentary}
`Vasala', the name of this sutta, is one of the many words in P\=ali and Sanskrit for `outcastes', those people which tradition, and in the past, high-caste laws, declared to be beneath the four basic castes or vaṇṇa/varna. These four, brahmana (priests with cattle), khattiya/ksatriya (warrior nobility, kings), vessa/vaiśa (merchants, traders) and sudda/śūdra (workers) are proclaimed by Hindu Law books to be a God-given ordering of society. Outcastes of various sorts were considered below even the śūdras. 
As to the state of these outcastes, a passage from my earlier book `Noble Friendship', outlines some of the terrible disabilities.
Many barbarous rules made by the higher castes featured in a Dalit's (outcaste's) life. They could not even pass in front of Hindu temples, much less enter them; they had to wear cast-off rags, never good clothes; they had the duty of clearing away dead animals as well as removing human excrement; in some places they were forced to wear clay pots round their necks  so that their spittle would not reach the ground, while their footsteps were obliterated by a broom tied to their waists; their women folk were compelled to wear non-precious jewellery of iron or pottery; their children had to be given `ugly' names, and finally they could not venture outdoors when the shadows were long lest their shadow fall on a high-caste person and pollute him. This fear of ritual pollution, very characteristic of Caste Hindus, underlies all these suppressive rules. It is easy to understand that people who have been so treated for hundreds of years, with no chance fro education or self-improvement, would feel that their lives were a hopeless round of degradation. Brahmins emphasized that they had been born onto such `low' births because of their sins – in other words, they were to be humble menials, do the dirty work, and say nothing. Some groups of outcastes were so polluting to the eyes of Hindus that they were not only untouchable but `unseeable' as well. There was bitter saying among the outcaste groups deriding caste hypocrisy that goes as follows: male Untouchables are always untouchables but female Untouchables, untouchable by day, became touchable by night. Perhaps the most suppressive rule of them all forbade low-caste people and Untouchables to have any religious education of practice. Religion, largely controlled by the Brahmins, involved the learning of Sanskrit, and as this reckoned to be the `root-language' spoken by the Hindu gods, it was utterly forbidden for outcastes to even to hear it, let alone learn it.' (Noble Friendship, p.39)
Even worse was the penalty prescribed for outcastes who heard brahmins chanting in Sanskrit: molten lead was to be poured into their ears. Whatever religion outcastes had was their own mixture of ritual, their own mantras (presumably not in Sanskrit) administered by their own priests who will rarely have been literate: a sort of low-grade Hinduism.
Whereas in Hinduism there is a widespread dread of ritual pollution caused by exterior factors such as low-caste persons as mentioned above, also by restrictive brahminical injunctions, such as the rule that those of high caste will lose their `purity' by crossing the ocean, the Buddha's teaching on purity emphasize that this depends on the karmas made by body, speech and mind, and ultimately by mind. So in this sutta we see the Buddha listing all the actions which make one an `outcaste', all them amoral, harmful to others and generally censured by the world. Thus, one may have high-caste brahmins who by the Buddha's standards are `outcaste', as well as those born as `outcastes' who in fact are people of great nobility of character, such as Dr. Babasaheb Ambedkar. The Buddha did occasionally say that such people were `brahmana' or `Brahmins', meaning Awake or Enlightened. ( See the verses in the Vasettha Sutta, 620-647 in Ch.III). Mere words of praise and blame have no ultimate truth. Words must be accompanied by compassionate actions.
The introductory prose section shows the Buddha going on almsround with his bowl `house by house', that is, he stopped briefly before each door to see if anyone wished to give alms of cooked food. This practice shows clearly that he had none of the Hindu prejudice that cooked by low-caste persons would be `polluted' and its consumption the cause of high-caste, such as brahmins, losing their status. The well-known brahmin Aggik\=a-bh\=aradv\=aja, as he later addressed the Buddha as Gotama, knew who he was and that he came from a high-caste warrior-noble family. At that time this brahmin was engaged in a fire-sacrifice ritual (notice the `aggika' – of fire, in his name) and had lifted the portion to be placed in the sacred fire when he saw the Buddha approaching. The latter was abused by the brahmin who feared that the benefits of the sacrifice would be lost due to the presence of a man who accepted food from low-caste and outcaste people. `Mere shaveling' (mundaka) was abusive as brahmins kept some or all of their hair and looked down upon those, like Buddhist monks and nuns, who shaved it off (Whethe the Buddha shave his head is rather a disputed matter as attested by all the images of the Buddha showing him with hair). `Vile ascetic' (samaṇaka) showed brahmin disapproval of those who left home and had a wandering religious life. Some of these will have been low-caste people as there were in the Buddha's Sangha. Last `vasalaka' as a word of abuse emphasizing low-casteness, means literally `little man', hence an insignificant person.
When the Buddha in reply asks him whether he knows who is an outcaste or what are the qualities which make a person so, the brahmin, rather surprisingly, says that he does not. Moreover, he addresses the Buddha quite politely as `venerable Gotama' though this is a familiar speech as to an equal. What could have brought about this sudden change in attitude?
We should leave aside the possibility that the Buddha used one of his powers or abilities to influence the brahmin. He preferred not to employ these unless there was a situation in which ordinary means would not serve. See for instance his remarks in Digha-nik\=aya, 11, Kevaddha Sutta. As it seems unlikely that this brahmin's attitude was changed by super-natural means we have to fall back upon what could be generally described as `the Buddha's presence'. This is well illustrated by what happened as the Buddha approached the five monks who were to be his first disciples soon after his Awakening. Though they had made a pact with each other that they would no longer treat him as a Teacher (because he had given up starvation and was again eating), as he approached, they all rose and performed the duties of pupils towards teachers. We may assume that the Buddha's presence, regardless of the brahmin's prejudices about food, turned his mind to civility. The Buddha was praised for his handsome features, radiant complexion and height (see vv.548-553 in Sela Sutta below).
Towards the end of this sutta occurs the famous verse, twice repeated:

\begin{MyDescription}[]{}
One's not an outcaste caused by `birth',\\
not by `birth' a brahmin is;\\
caused by karma one's outcaste,\\
a brahmin is by karma caused.
\stepcounter{stanza}
\end{MyDescription}      

Brahmins in the Buddha's days and some maybe even now, were wont to say that they were pure back through seven generations in both the mother's and the father's family. All of these people, they were claiming, had married only into other brahmin families and hence they were `pure'. The Buddha by no means agreed with this estimation of purity and has, in the suttas, made fun of brahmin arrogance. To know one's ancestors back through seven generations on both mother's and father's sides is quite unusual even now – unless one comes from an aristocratic family, so brahminical claims for so-called purity – ring rather hollow. This claim of superiority through birth is not confined to brahmins in the Buddha's time, as we can attain from the radio and television news of notions of superiority and the conflicts arising from it in our own time.
It is related to the claim to be an Aryr/ariya – those who are noble by race. The Buddha again does accept such claims, the likes of which have lasted in our time, notably in the causes of WWII and the slaughter of millions of people. One is noble by practising the Noble Eightfold Path, one is ennobled by Dharma, not by lineage or race.
The sutta concludes with one of the stylised passages which are characteristic of orally-transmitted works. Presumably the brahmin of this sutta was impressed and praised the Buddha's teaching. There is no reason to suppose that he spoke exactly the words exactly translated here which appear hundreds of time at the end of suttas. Whether he went to for refuge in the formula quoted here, or indeed, whether he went in any sense for refuge at all is something that we shall never know.
Three points in the sutta may be remarked upon: one a minor matter and the others weightier. Verse 117 mentions the term `twice born'. This means a man from high castes who has not only been born in the normal way but has also received a `second birth' in the ceremony of being invested with a scared thread, usually at puberty. Brahmins, for instance are `twice born'. The rest of us remain `once born' and so of lower status. Workers and outcastes (and those not touched by the Indian caste-system) are merely `once born'.
Of greater importance in the next verse which declares very straight-forwardly (for the ears of heads of state or generals0, that if they behave, as they have commonly done in history, by invasion of other lands and slaughter, they should be known as outcastes. One or two notable examples will make clear this Buddhist attitude to war: We have an Alexander who is even honoured with the epithet `Great'. By the standards of verse 118, he should be known as `Alexander the Outcaste'. As for modern times, one may choose from a long list which will no doubt include Napoleon, Hitler, Mussolini, Stalin and Moa Tsetung – outcastes all.
Also in v.126 there is the line ` teaching in obscurities', a phrase covering a multitude of errors. It may mean teaching which will confuse other while promoting the so-called teacher's ego, implying he or she has Dharma attainments which do not exist.Or this may mean teaching an obscure and useless subject from the viewpoint of Dharma. As an example there is the brahmin `science' of marks or characteristics to be found on animals or people. Such `marks', such as a mole on the cheek, are claimed to be characteristic of those who cheat and lie. Such `science' (vijj\=a/vidy\=a) is not based on any cause and effect considerations. Connected to this in modern times, there is a small P\=ali treatise in verse, on the marking upon cats, a playful and unusual subject, written by a Sanghar\=aja of what was then Siam and now Thialand!
Elephants have also been the subject s of such a `science' of markings and their meanings. These matters are not exactly essential Dharma, and may well be regarded as obscurities.

\chapter{Mett\=a Sutta –- Loving-kindness}
What should be done by one who's skilled in wholesomeness, to gain the State of Peacefulness is this:
One should be able, upright, straight and not proud,
easy to speak to, mild and well content,
easily satisfied and not caught up 
in too much bustle, and frugal in one's ways,
with senses calmed, intelligent, not bold,
not being covetous when with other folk,
not even doing little things that otherwise ones blame. (And this the thought that one should always hold):
 % % % % % %need to confirm formatting with original % % % % %
\begin{MyDescription}[]{}
\stepcounter{stanza}
\stepcounter{stanza}
\stepcounter{stanza}
\stepcounter{stanza}
\stepcounter{stanza}
\stepcounter{stanza}
\stepcounter{stanza}
\stepcounter{stanza}
\end{MyDescription}      

\begin{MyDescription}{}
`May being be all live happily and safe\\
and may their hearts rejoice within themselves.\\
Whatever there may be with breath of life,\\
whether they be frail or very strong,\\
without exception be they long or short,\\
or middle-sized, or be big or small,\\
or dense, or visible or invisible,\\
or whether they dwell far or they dwell near,\\
those that are here, those seeking to exist – \\
may beings all rejoice with themselves.'
\end{MyDescription}   
   
\begin{MyDescription}[]{}
Let no one bring about another's ruin\\
and not despise in any way or place;\\
let them not wish each other any ill\\
from provocation or from enmity.
\end{MyDescription}   

\begin{MyDescription}[]{}
Just as a mother at the risk of life\\
loves and protects her child, her only child,\\
so one should cultivate this bondless love\\
to all that live in the whole universe – \\
extending from a consciousness sublime\\
upwards and downwards and across the world,\\
untroubled, free from hate and enmity.\\
And while one stands and while one sits\\
or when one lies down still free from drowsiness\\
one should be intent on this mindfulness – \\
this is divine abiding here they say.
\end{MyDescription}      

\begin{MyDescription}[]{}
But when one lives quite free from any view,\\
is virtuous, with perfect insight won\\
and greed for selfish desires let go,\\
one surely comes no more to be reborn.
\end{MyDescription}   
 
 \begin{MyDescription}[(Sn. 143 –- 151)]{}
 \end{MyDescription}  								
\newpage    
 \section{Commentary}
   
   The State of Peacefulness, appears at the opening of this sutta and is implies in the last verse. It may in other contexts be referred to as Nirvana, liberation and so on. In between the first and last verses we find a number of conditions mentioned for attainment of the Peaceful State. There is no trace in this sutta of a `method', such as the Pali commentary presents and which is also explained in the Visuddhimagga, `The Path of Perfection', for attainment of this goal, a fact that we shall return to later.
   
   In the sutta's first line there is a clear indication of what is needed for experience of the Peaceful State: skill in wholesomeness, in other words, good conduct with body, speech and mind. This is followed by a passage listing fifteen requirements – things that one should work on and make effort with – but as they are straightforward little needs to written about them, though the following remarks may be useful. `Able' refers to a person who can do and is willing to try. `Upright' and `straight' refer, we are told, respectively to the moral behaviour with body and speech, and the same with mind. However, they may also be taken as repetitive emphasis on the importance of general honesty. With mention of `well content, easily satisfied and frugal in one's ways' we come to factors more easily practised by monks, or while on a long retreat. These three go against the current of worldliness and materialism. The ministers in charge of economic development in various countries would not be happy if large numbers of their populations began to practice them. Not being `bold' means foolishly over-estimating one's capacity and taking big steps which one cannot follow - generally based on a mixture of pride and delusion. Not wanting what others have is good for peace of mind, so not being covetous when visiting others' houses and so on is very helpful. Wise ones, like spiritual teaches, have developed their minds and can aware of the consequences of doing `little things'. Their `blame' is expressed to their unwise students often in private. This does not refer to wholesale condemnations.
   The line in brackets is not part of the text but is needed to link the fifteen requirements with the next part upon the various kinds of beings to which Mett\=a should be extended. They are defined by having `breath of life' though this does not mean only those having lungs or gills as the P\=ali word `p\=ana' (pr\=ana in Sandskrit) means not only breathing but also living. Examples of creatures fitting the following list of adjectives can easily be thought of, until we come to `dense or visible or invisible'. This could be rendered `those which are substantial, those that can be seen and those that cannot'. In the context of Dharma the last means, those that the human eye cannot see.' Rationalists might want this to mean `those that require a microscope' and though this would be a possible it is not in accordance with tradition. This definitely means `those with bodies too refined for the human eye to see', devas for instance or ghosts. Awareness of such beings increases with depth of meditation practice. `Dwelling far or near' may be taken to mean `those beings whose bodies (or non-bodies in the case of the formless-realm devas) are remote from our experience, while `near' signifies those whose existence overlaps our own experience (as with animals), or are `near' to us because of former relationships, as with some ghosts and devas which may act as guides or protectors.
   Next comes an interesting line: `those that are here, those seeking to exist'. Strict Therav\=ada philosophy upon life after death departs from the main line of Indian Buddhist teaching. The Therav\=ada Abhidhamma analyses all experience into momentary dharmas, a doctrine which rationalizes the teaching of not-self and defends the notion that nothing in terms of a being (which contradicts the idea of not-self) goes from life to life. The trouble is that often no distinction is drawn between the two kinds of truth: conditional and ultimate. Rebirth belongs to the realm of conditional truth within which we usually live. To say that `so-and-so has gone to heaven' may be quite correct according to conditional and dualistic truth. The phrase may seem to imply that this person will be the same there as he or she was here except that they have a new and less visible body. This view would be on the side of eternalism, one of two extremes which the Buddhist Dharma-in-the-Middle avoids. Moreover, such a view seems to ignore the first of the three characteristics (lakkhana) of all conditioned things: Impermence. So a phrase like `Go to Hell!' besides showing little loving-kindness and a good deal of hate, is true in its dualistic limitations: Someone should go to Hell.
   Ultimate truth in its Abhidhamma from suggest that because there are no beings, only moments of experience, then no one goes from one life to another. This theory denies that there is anyone who goes and comes, perilously near to the other extreme of nihilism. It has led Therav\=ada scholars to deny that there is any experience between lives, a teaching which is widely found in the various Indian (extinct) Buddhist sects and in Mah\=ayana.
   The concept of the `between-life' or antarabhava, usually known these days by convenient Tibetan translation: Bardo, is a agglomeration of commonsense, ESP, memory of past lives and vision of famous living teachers. The commonsense (or conditional truth) is that so-and-so died and because this person made karma  of some kind, will be reborn accordingly. The extra-sensory perception will be the visions and sensations of that being who has died by those who were close to him or her, that suggest that the deceased is still in some way present. Memory of past lives and the linking periods between them may be the experience of some people, a recollection which comes with clarity of mind, quite different from the confusions of desire and imagination.
   The visions of living Teachers may explore the processes of being reborn during the intermediate state and give disciples instructions upon how to practise, achieve liberation in that state. Teachings on the Bardo are mostly about the later. Though it is useful to have books explaining perception after human death, the transmission of that practice from living teacher is really required.
   So `those that are here' are all the beings that we can be aware of in our present life. `Those seeking to exist' are those in the Bardo seeking existence through the limitations of their karmas.
   Buddhaghosa the Pali commentator who lived about a thousand years after the Buddha, strenuously denies this obvious truth and makes inplausible suggestions as to the meaning of this line (see, Minor Reading and Illustrator, p.286-7). This is not the only case where the classical commentators try to defend  an `orthodox' Therav\=ada position and so distort the straightforward meaning of the Pali texts. The words of Pali commentators should be carefully examined.
   Why does a verse upon anger/hatred/resentment follow next? The sutta has already emphasized that Mett\=a should be extended to all kinds of beings, classified in less poetic terms elsewhere as `the footless, two-footed, four-footed and many-footed; those with perception, those with no perception, and those with neither perception nor non-perception' (A.Fours.34). But for effective practice these must not remain abstract catagories towards which one plays at extending mett\=a. Particular beings, especially `difficult' humans, or animals which evoke fear have to be involved. Real mett\=a then arises naturally through the understanding of one's fears. This verse is here so that the practitioner does not fool him or herself: `Now my practice of limitless love, unconditional love, flows to all being!' Better look at how one feels with those one fears or does not like! In mett\=a-practice beings come first, directions and direction-less practice must always follow.  This order prevents self-deception.
   There follows the famous simile of a mother's love for her child and how one's mind becoming limitless with mett\=a should resemble this. For most of us, loving all beings in that way is not going to be easy. `Boundless love” becomes possible through the experience of jh\=ana and while a few people will have spontaneous experience of this relaxed but concentrated state of meditation, most require to be diligent with regular meditation practice. Jh\=ana has no good English translation and for that reason is left untranslated. Meditation, concentration or contemplation are all Latin-based words which do not have the clear meaning of the P\=ali word jh\=ana. This word does not occur in the sutta but is implied by such expressions as `boundless', `consciousness sublime', upwards and downwards and across the world'. Other religious traditions outside the Buddha's teaching have also what is known as `saints' and mystics who experience jh\=ana. 
   The last but one verse indicates the practice of mindfulness, how one should not drift and attach to rapture of jh\=ana but rather cultivate a mental state near to wisdom (prana/prajna). This union of calm and clarity characterises the four Devine Abidings, which are frequently referred to in the suttas. The words `they say' refer in general to wise meditative persons, not specially to Buddhist teachers.
   The last verse however restricts these people to those who are `quite free from any view' that is, they are free from mere or blind belief which cannot be verified by practice with an unclouded mind. They are free even from the Buddhist assumptions which everyone will have when they start regular and sustained meditation practice. Such people are not keen to label themselves `Therav\=ada', `Vipassan\=a', Mahay\=ana', Vajray\=ana', `Zen' for these are the playthings of those who do not Know. The truly liberated are not imprisoned by such limitations though they may use such words in the instruction of others. The verse reminds us that to find this liberation we need to act in a way that neither harms others or ourselves – `virtuous' (sīlav\=a), have access to insight-wisdom (dassana), and have more moderate sense-desires and let them go, the famous or infamous k\=ama which limits the mind's freedom, so that we arrive at the state of not being ever again being driven into birth.
   No methods of meditation are offered in the P\=ali suttas. For these, an ancient times and still in present, one consulted one's teacher. Teachers in the Buddha's time would have had direct Knowledge of the Dharma and so needed to learn no methods. Teachers in later times,  if they have no direct knowledge would consult the Path of Purification, that the compendium of Therav\=ada Buddhist knowledge complied by Buddhaghosa. This is still highly revered, both teacher and book, in Therav\=ada lands. Some more recent teachers out of their own experience have taught methods that differ somewhat from this tradition.
   If one wishes to consult the traditional P\=ali sources explaining this sutta, they are available  in the translation of the Paramatthajotik\=a (The Illustrator of Ultimate Meaning, by translator Bhikkhu N\=anamoli, published by the P\=ali Text Society. The P\=ali text itself is also currently in print).
   The present translation is based, though with many corrections, upon that of David Maurice, in the long out-of-print anthology, `The Lion's Roar'.
   
   
   \chapter{Hemavata Sutta -- The Buddha teaches S\=at\=agira and Hemavata Yakkhas}
   
\begin{MyDescription}[\arabic{stanza}]{S\=at\=agira:}
Today's the lunar fifteenth day - \\
uposatha – a night divine arrived,\\
Let's go the Teacher Gotama\\
him of high repute.
\stepcounter{stanza}
\end{MyDescription}   

\begin{MyDescription}[\arabic{stanza}]{Hemavata:}
Is the mind of such a one\\
towards all beings well-disposed?\\
Within his power are his thoughts\\
towards the wished, the unwished too?
\stepcounter{stanza}
\end{MyDescription}   

\begin{MyDescription}[\arabic{stanza}]{S\=at\=agira:}
Yes, the mind of such a one\\
towards all beings well-disposed.\\
Within his power are his thoughts\\
towards the wished an unwished too.
\stepcounter{stanza}
\end{MyDescription}   

\begin{MyDescription}[\arabic{stanza}]{Hemavata:}
Is he the one who does not steal?\\
To beings he's restrained?\\
Is he far from indolence?\\
Does jh\=ana he neglect?
\stepcounter{stanza}
\end{MyDescription}   

\begin{MyDescription}[\arabic{stanza}]{S\=at\=agira:}
He is one who does not steal,\\
to beings he's restained.\\
Buddha's far from indolence\\
jh\=ana he never neglect.
\stepcounter{stanza}
\end{MyDescription}   

\begin{MyDescription}[\arabic{stanza}]{Hemavata:}
Is he not one who falsely speaks?\\
Does he use harsh or violent words\\
or employ slanderous ones?\\
Or a user of meaningless speech?
\stepcounter{stanza}
\end{MyDescription}   

\begin{MyDescription}[\arabic{stanza}]{S\=at\=agira:}
He's not one who falsely speaks\\
nor uses harsh or violent words\\
nor utters words of slander\\
but wisdom speaks which benefits.
\stepcounter{stanza}
\end{MyDescription}   

\begin{MyDescription}[\arabic{stanza}]{Hemavata:}
Does he not desire, indulge,\\
In mind he's unattached?\\
Has delusion overcome?\\
Mong Dharmas has he Eyes?
\stepcounter{stanza}
\end{MyDescription}   

\begin{MyDescription}[\arabic{stanza}]{S\=at\=agira:}
He does not desire, indulge\\
for his mind is unattached.\\
Delusions all he's overcome\\
Mong Dharmas Buddha's Eyed.
\stepcounter{stanza}
\end{MyDescription}   
   
\begin{MyDescription}[\arabic{stanza}]{Hemavata:}
Has true knowledge he attained?\\
Is his conduct perfect, pure?\\
Are his inflows now extinct?\\
Is he not again to be?
\stepcounter{stanza}
\end{MyDescription}  

\begin{MyDescription}[\arabic{stanza}]{S\=at\=agira:}
Indeed true knowledge he's attained\\
and his conduct's perfect, pure,\\
for him all inflows are extinct\\
so he'll not again become.
\stepcounter{stanza}
\end{MyDescription}  

\begin{MyDescription}[\arabic{stanza} A]{Hemavata:}
Accomplished is the Sage's mind,\\
his actions and his ways of speech,\\
of true Knowledge and conduct he's possessed\\
rightly him you praise.
\end{MyDescription}  

\begin{MyDescription}[\arabic{stanza} B]{}
Accomplished is the Sage's mind,\\
his actions and his ways of speech,\\
of true Knowledge and conduct he's possessed\\
rightly you rejoice.
\stepcounter{stanza}
\end{MyDescription}     
   
\begin{MyDescription}[\arabic{stanza}]{}
Accomplished is the Sage's mind,\\
his actions and his ways of speech,\\
of true Knowledge and conduct he's possessed\\
it's good that we see Gotama.
\stepcounter{stanza}
\end{MyDescription}     

\begin{MyDescription}[\arabic{stanza}]{}
Who limbed like antelope and lean,\\
wise, with no greed and having little food,\\
Sage in the woods who meditates alone – \\
let us go see Gotama.
\stepcounter{stanza}
\end{MyDescription}     

\begin{MyDescription}[\arabic{stanza}]{}
The Great One like a lion who lives alone,\\
among all pleasures he's expectation-free,\\
let us draw near that we may ask of him\\
how to escape from the snarefulness of death?
\stepcounter{stanza}
\end{MyDescription}     

\begin{MyDescription}[\arabic{stanza}]{}
O proclaimer of the Dharma, expounding it too,\\
one who's beyond all dharmas' Further Shore,\\
all fear and hatred you've utterly overcome\\
both of us then of Gotama inquire: -	
\stepcounter{stanza}
\end{MyDescription}     

\begin{MyDescription}[\arabic{stanza}]{}
What co-arises with the world?\\
With what's it make acquaintance?\\
The world grasps after what indeed?\\
Why's the world afflicted?
\stepcounter{stanza}
\end{MyDescription}     
   
\begin{MyDescription}[\arabic{stanza}]{The Buddha:}
Six with the world do co-arise\\
with six becomes acquainted,\\
the world's attached to six indeed,\\
so, world's by six afflicted.
\stepcounter{stanza}
\end{MyDescription}   

\begin{MyDescription}[\arabic{stanza}]{Hemavata:}
The grasping  - what is it then\\
by which the world's afflicted?\\
When asked about this, please do speak:\\
how to be free from dukkha?	
\stepcounter{stanza}
\end{MyDescription}   
   
\begin{MyDescription}[\arabic{stanza}]{The Buddha:}
The sensual pleasures five are taught\\
in the world with mind as six,\\
having let go of all desire for those\\
be thus from dukkha free.
\stepcounter{stanza}
\end{MyDescription}   

\begin{MyDescription}[\arabic{stanza}]{}
This for the world's the leading out,\\
its `as-it-is' declared to you\\
and this to you I do declare:\\
be thus from dukkha free.
\stepcounter{stanza}
\end{MyDescription}      


\begin{MyDescription}[\arabic{stanza}]{Hemavata:}
Here, who goes across the flood,\\
who goes across the sea,\\
No standpoint or support,\\
who in the deep sinks not?	
\stepcounter{stanza}
\end{MyDescription}      


\begin{MyDescription}[\arabic{stanza}]{The Buddha:}
That person ever virtuous,\\
with wisdom, concentrated well,\\
with mind turned inward, mindful – \\
crosses the flood that's hard to cross.
\stepcounter{stanza}
\end{MyDescription}      


\begin{MyDescription}[\arabic{stanza}]{}
Detached from thoughts of sense-desire,\\
all fetters overpassed,\\
delight-in being quite destroyed – \\
who in the deep sinks not.
\stepcounter{stanza}
\end{MyDescription}      


\begin{MyDescription}[\arabic{stanza}]{Hemavata:}
Behold the Great Seer of wisdom deep,
of subtle meanings Seer, one owning nought,
unattached to sensual being, free in every way,
proceeding along the pathway of the gods.
\stepcounter{stanza}
\end{MyDescription}    

\begin{MyDescription}[\arabic{stanza}]{}
Behold the Great Seer of perfect repute,\\
of subtle meanings Seer, of wisdom the imparter,\\
unattached to the senses' basis and greatly wise,\\
all-knower, treading the path of the Noble Ones.
\stepcounter{stanza}
\end{MyDescription}   

\begin{MyDescription}[\arabic{stanza}]{}
Well-viewed by us today indeed,\\
well-dawned upon us, well-arisen:\\
the Awaken One we've seen,\\
crossed the flood, from inflows free.
\stepcounter{stanza}
\end{MyDescription}   

\begin{MyDescription}[\arabic{stanza}]{}
These ten hundred Yakkhas here\\
of great power and renown,\\
all of them for refuge go – \\
You are our Teacher unexcelled!
\stepcounter{stanza}
\end{MyDescription} 

\begin{MyDescription}[\arabic{stanza}]{Both:}
Village to village we shall roam,\\
mount to mount revering him\\
the Fully Awakened One, as well\\
the Dharmaness of Perfect Dharma.
\stepcounter{stanza}
\end{MyDescription}       

 \begin{MyDescription}[(Sn. 152 –- 180)]{}
 \end{MyDescription}  								
\newpage
\section{A few Notes}

% % %Fix quotation marks   
`Yakkha': who or what are they? They appeared, in the Buddha's day to be believed semi-spiritual being who were powerful and rather easily angered, living in wild places. Possibly they were members of aboriginal tribes who had proficiency in magical matter or were believed to possess these. They were certainly feared.
   
The first of the v.153 mentions `the lunar fifteenth day' which requires a little explanation. In the Buddha's days the calendar was counted by the moon's action rather than the sun. `Months' of about 28 days, thirteen of such lunar months approximating to a solar year at 364 days needed to be augmented to complete the sun year. New Moons and Full Moons were important for these days measured the uposatha days. The uposatha day was (and is) the gathering of disciples who reconfirmed their dharma-practice twice every year on these two days by celebrating the chanting of the P\=atimokkha rules.
   
Second verse: Hemavata asks whether the Teacher Gotama's mind is well- disposed towards all beings – he is in fact asking about mett\=a/loving kindness. While in the second two lines his enquiry regards wisdom – pann\=a. He seems to be well-informed since these words cover both necessary approaches for awakening. `Within his power, are his thought towards the wished, the unwished too'. Though it is possible to talk about `thought within one's own power', this is only a way of speaking fit for the awakened. For who is this assumed person who possesses thoughts and can label them `mine'? It is not that anything is owned, not even thoughts are owned by the Awakened!
   
`Among dharmas Buddha (is) Eyed': Dharmas are the qualities, virtuous and otherwise which manifest in mind, flit though it in a constant stream always changing. Many of these mind-patterns are disregarded, neglected or repressed by ordinary people but their understanding of themselves sharpens as they begin to practice mindfulness. With the maturity of mindfulness practice they can be called `Eyed', those with insight, with deep understanding of the way things really are. More upon Eyes will be found at A. Threes. 29 where the eyeless, the one-eyed and the two-eyed are explained.
   
`Are his inflows now extinct?' Inflows (\=asava) are usually listed as the inflow of sensual desire, the inflow of being/existence, that of ignorance, to which is occasionally added, the inflow of views. These are the deepest level of confused mind and are frequently explained in the Suttas.
   
S\=atagira rightly remarks: `for him all inflows are extinct, so he will not again become'. Notice that `extinction' applies to the inflows which are conditioned and are therefore impermanent. The Buddha though will not be driven into becoming this or that sort of being again. This does not mean that a Buddha will disappear into a mysterious Nirv\=ana which is neither existence or non-existence. His condition is neither and beyond either the extremes of nihilism and eternalism.
   
Hemavata: three verses all opening with `Accomplished is the Sage's mond' have interesting final lines:
`rightly him you praise'(163A), `rightly you rejoice' (163B) and`it's good that we see Gotama' (164)
   
He's praised for his actions and speech which cause no suffering but bring benefit – they are never harmful. The true Knowledge that he is possessed of is usually described as the threefold knowledge (tevijj\=a) consisting of the knowledge of past lives, the knowledge of the future results of karmas made, and should we not praise a person like this with such virtues, though recommending him to others, or chanting his praises in pujas, if we did not what sort of practitioners would we be?
   
`Rightly you rejoice' – why would one do that? `Rejoice' here means to rejoice in the knowledge – all the good things – of the Teacher. Not to do so might mean that one was too proud to acknowledge these virtues of the Teacher. `Anumodan\=a', to rejoice in another's merits is to make very good karma indeed, while not to do so points to a narrow egocentric mind.
   
`It's good if we see Gotama': `seeing' a Teacher is a traditional Dharma practice in India. `Seeing' a Teacher may be only glimpsing him/her with the expectation that one will receive a blessing in which case it is called dharshan (dassana in P\=ali, a verb related to the pass\=ama, we see, which is found in the Sn text). More deeply committed pupils willwant to have closer connection to their teacher: to see more than a brief glimpse, to understand how to practice, or even to See or realise for themselves.
   
Hemavata, obviously a very intelligent yakkha, eventually asks the Buddha a subtle question: `What co-arises with the world?' etc. This line and others that he speaks shows his awareness, for instance, that he knows that the world's arising, in whatever way one thinks of `world', is according to Dharma, co-production – produced from many causes. The second question poses difficulties which most translations have not solved while mine is just a shot in the dark. The third is straightforward while the fourth question treats the results of the third. The verse with its four questions is partly a least a riddle.
   
On this occasion, the Buddha in answers riddle with riddle and does not really explain his reply to Hemavata's questions. All he seems to have done is add the number `six' – to each line in the verse. It would be surprising if these two Yakkhas had obtained the fullest satisfaction with this `explanatory' verse, for the Buddha explained nothing. The Sn. Comy has tried to account for this strange situation and `explains' what the various sixes are. What follows are my speculations about the meanings.
   
Verses 168-169: Buddhist understanding of creation is not that there is a creating force beyond the world which somehow brings forth in the world. Creation comes about as all necessary factors arise for a world, those factors must `arise together', or as I have translated, `co-arise'. So much for distant material creation. If `world' is understood as the five senses plus the mind as the sixth (vs.171.).
   
    
   
    
 \chapter{\=Alavaka Sutta -- With the Yakkha \=Alavaka}
    
Thus have l heard:
At one time the Lord dwelt at \=Alavi in the haunt of the yakkha \=Alavaka. Then the latter went to the Lord's dwelling and spoke to him as follows: `Monk, come out!' `Very well, friend' said the Buddha (and came out). `Monk, go in!'
   `Very well, friend' said the Buddha and entered his dwelling. he repeated these demands another twice but on the fourth demand the Buddha said: 
   
`I shall not come out to you, friend, do what you will.' 
   
`Monk, I shall ask you a question and if you cannot answer it I shall either overthrow your mind, split your heart, or seizing you by the feet, throw you to the other side of the Ganges river.'
   
`I do not see, friend, anyone in the world with its devas, maras and brahmas, in this generation with its monks and brahmins, princes and men who can either overthrow my mind, or split my heart, or seize me by the feet and throw me to the other side of the Ganges river. However, friend, ask what you will.'


\begin{MyDescription}[\arabic{stanza}]{\=Alavaka:}
For humans here what wealth is best?\\
What often done brings happiness?\\
What surely has the sweetest taste?\\
How living do they say `life': best?'
\stepcounter{stanza}
\end{MyDescription}  								


\begin{MyDescription}[\arabic{stanza}]{The Buddha:}
Faith the wealth for humans best,\\
Dharma done brings happiness,\\
Truth surely has the sweetest taste,\\
`Lived with wisdom' this life's best.
\stepcounter{stanza}
\end{MyDescription}    

\begin{MyDescription}[\arabic{stanza}]{\=Alavaka}
How can the flood be overcrossed?\\
How overcrossed the sea?\\
How dukkha can be overcome?\\
How win to purity?
\stepcounter{stanza}
\end{MyDescription}    

\begin{MyDescription}[\arabic{stanza}]{The Buddha:}
By faith the flood is overcrossed.\\
By vigilance the sea.\\
By effort dukkha's overcome.\\
By wisdom, purity.
\stepcounter{stanza}
\end{MyDescription}    

\begin{MyDescription}[\arabic{stanza}]{\=Alavaka}
How wisdom will be won\\
with riches also found?\\
How attain to fame\\
and bring together friends?\\
When passing from this world, to next,\\
how does one not grieve?
\stepcounter{stanza}
\end{MyDescription}    

\begin{MyDescription}[\arabic{stanza}]{The Buddha:}
One with faith in arahats' Dharma\\
for attainment of Nirv\`ana\\
diligent, wishing to listen,\\
and discerning, wisdom wins.
\stepcounter{stanza}
\end{MyDescription}    

\begin{MyDescription}[\arabic{stanza}]{}
One who acts appropriately,\\
who's steady and industrious,\\
finds wealth and fame, attained by truth;\\
by giving, friends are gained.
\stepcounter{stanza}
\end{MyDescription}    
   
\begin{MyDescription}[\arabic{stanza}]{}
A faithful household seeker has\\
attained these four: truthfulness,\\
virtue, courage, generosity too\\
and so grieves not when passing hence.
\stepcounter{stanza}
\end{MyDescription}  

\begin{MyDescription}[\arabic{stanza}]{}
Now if you wish, ask others too,\\
numerous monks and brahmins -- if\\
truth, generosity, taming self,\\
patience too -- what's better than these? 
\stepcounter{stanza}
\end{MyDescription}  

\begin{MyDescription}[\arabic{stanza}]{\=Alavaka:}
Why should l consult with these\\
monks and brahmins numerous\\
when now for myself l know\\
who bring my future's benefit.
\stepcounter{stanza}
\end{MyDescription}  

\begin{MyDescription}[\arabic{stanza}]{}
For my benefit truly He came here,\\
the Buddha visiting \=Alavi.\\
Now do l know where a gift\\
bestowed will bear great fruit.\\
\stepcounter{stanza}
\end{MyDescription}  

\begin{MyDescription}[\arabic{stanza}]{}
Village to village I shall roam,\\
town to town revering him -\\
the Full Awakened One, and\\
the Dharmaness of perfect Dharma.
\stepcounter{stanza}
\end{MyDescription}  

\begin{MyDescription}[(Sn. 181--192)]{}
\end{MyDescription}    
  
   
   
\chapter{Vijaya Sutta -- Victory over fascination with bodies}

\begin{MyDescription}[\arabic{stanza}]{}
Whether walking or standing still,\\
down one sits or lays it down,\\
bends it in or stretches it —\\
it's just the body's movement.
\stepcounter{stanza}
\end{MyDescription}

\begin{MyDescription}[\arabic{stanza}]{}
This body by bones and sinews bound,\\
bedaubed by membranes, flesh\\
and covered all over by skin —-\\
not seen as it really is:
\stepcounter{stanza}
\end{MyDescription}

\begin{MyDescription}[\arabic{stanza}]{}
Filled with guts, with stomach filled,\\
with bladder, liver--lump\\
with heart and lungs its filled,\\
with kidneys too and spleen.
\stepcounter{stanza}
\end{MyDescription}

\begin{MyDescription}[\arabic{stanza}]{}
Liquids like spittle and snot\\
together with sweat and fat,\\
with blood and oil for the joints,\\
with bile and grease for the skin.
\stepcounter{stanza}
\end{MyDescription}

\begin{MyDescription}[\arabic{stanza}]{}
Then by the streaming nine\\
impurity oozes out:\\
from the eye there's din: of eyes,\\
from ears, wax—dirt of ears,
\stepcounter{stanza}
\end{MyDescription}

\begin{MyDescription}[\arabic{stanza}]{}
Whether walking or standing still,\\
down one sits or lays it down,\\
bends it in or stretches it —\\
it's just the body's movement.
\stepcounter{stanza}
\end{MyDescription}

\begin{MyDescription}[\arabic{stanza}]{}
Snot-mucuses from nose,\\
vomit at times from the mouth\\
sometimes phlegm's spewed forth,\\
and from the body sweat and dirt.
\stepcounter{stanza}
\end{MyDescription}

\begin{MyDescription}[\arabic{stanza}]{}
And then within its hollow head\\
bundled brains are stuffed —\\
the fool thinks all is beautiful\\
by ignorance led on.
\stepcounter{stanza}
\end{MyDescription}

\begin{MyDescription}[\arabic{stanza}]{}
But when it lying dead,\\
bloated and livid blue,\\
cast away in the charnel-ground\\
kin care for it not.
 \stepcounter{stanza}
\end{MyDescription}

\begin{MyDescription}[\arabic{stanza}]{}
Then dogs devour, iackals too,\\
wolves and wom1s dismember it,\\
crows and vultures tear at it\\
and other creatures too.
 \stepcounter{stanza}
\end{MyDescription}

\begin{MyDescription}[\arabic{stanza}]{}
   Contemplate: this living body,\\
   that corpse was once like this\\
   and as that corpse is now\\
   so will this body be —-\\
   for body then discard desire\\
   whether within or without --
\stepcounter{stanza}
\end{MyDescription}

\begin{MyDescription}[\arabic{stanza}]{}
such a monk who's wise, desire\\
and lust discarded utterly\\
attains to Deathlessness, to peace,\\
Nirv\`ana, the unchanging state.
\stepcounter{stanza}
\end{MyDescription}

\begin{MyDescription}[\arabic{stanza}]{}
But pampered this fetid, foul\\
two-—footed thing though filled\\
with varied sorts of stench, as well\\
with oozing here and there.
\stepcounter{stanza}
\end{MyDescription}

\begin{MyDescription}[\arabic{stanza}]{}
Whoever such a body has\\
but thinks to exalt themselves\\
or to despise another —-\\
what's this but wisdom's lack?
\stepcounter{stanza}
\end{MyDescription}

\begin{MyDescription}[(Sn. 193--206)]{}
\end{MyDescription}    
\newpage
\section{Comments upon the Vijaya Sutta}

Apart from attachments to one's `own' mind, the next strongest bond is to the body, one's `own' of course but by extension to other bodies. This is a sutta fit for two sorts of persons: one who wishes to practise renunciation as a member of a monastic sangha; or second, one whose sexual desires are very powerful. The teachings of this sutta are not so appropriate to  those living a non-monastic life, or to people whose desires are of less power. Still, everyone will benefit from an occasional perusal of this sutta's teachings, a reminder of the nature of this body which we identify as `ours'. The first verse conveys the way the body really is, its movements just movements, neither refined nor gross. The movements are not `mine' or `yours' they are merely the body's. Leaving aside the embellishments, so much advertised and flaunted in our times, as well as the repulsive aspects of bodies —-   they are all just as they are, neither good nor bad, neither beautiful or ugly, neither attractive nor repulsive. Of course, this is advice to those who meditate and who wish to have some success with their practice. Others may not understand. Verse two starts to specify medicine for minds too much swayed by lust. This Dharma-medicine will seem to the attached as rather bitter in flavour. The body, one's own and others', not seen as it really is, is sketched in outline, bones, sinews, membranes, flesh, and skin and reminds one of the famous five: hair of the head, hair of the body, nails, teeth and skin. These five are first in the list of 32 parts recommended for meditation and summarize precisely what we seen when looking at another person (See, Khp.) This, a list of the Khuddaka-patha has a commentary upon the 32 parts. A few more choice ingredients of the body appear in verse 195. These are parts and liquids that in general people are not happy to see, especially when they are their own. But where should we be without bones and sinews, or how to exist lacking guts and belly? We only look on the outside of our own or others' bodies and take for granted that other more or less unpleasant hidden parts exist. Rather a one--sided view of the body!

Continuing the list with emphasis on liquids, as found also in the 32 parts: `bile, phlegm, pus, blood, sweat, fat, tears, (skin) grease, spit snot oil—-for—-the--joints, piss. The whole list may be a pre-Buddhist medical list of the body's contents though surprisingly it ignores semen. The nine streams or impurity are explained in the next verse. 

These are what flows from: 2 eyes, 2 ears, 2 nostrils, l mouth, plus the body. This adds up to eight. `Mouth' could be counted twice as two sorts of impurity are mentioned: vomit and phlegm. Or `sweat' and `dirt' may count as   two from the body. The Pali Commentary does not clarify this. ln other texts the body's openings are counted as 2 eyes, 2 ears, 2 nostrils, mouth, urinary exit  plus anus, making nine. Here, the last two are absent, rather strange if `impurities' are being counted.

While it may seem to us that many people have hollow heads without any `bundled brains', the real function of the brain was not appreciated in ancient India though many other parts of the body had functions known to Ayurveda (medical) treatises. 

The following verse turns attention from internal body bits to the body's death. `Cast away in the charnel ground' refers to a common way of recycling bodies in the Buddha's days. A portion of forest was declared both a crematory and a disposal point for bodies. Cremating bodies cost more — trees to be felled or wood to be scavenged, while taking the body to this secluded and forested area and leaving it there after due rites was more economical. It is unlikely that   the latter will recommend itself to local councils though ours are the days of recycling. Perhaps we are more sentimentally attached to the corpse of a dear   one, than were Indians of those times, This seems to have been the case as `kin care for it not' and of course they do not care for the corpse because there was no refrigeration then, so that its `bloated and livid blue' and most importantly what the text takes for granted: its stench, is unbearable.

However, it was still attractive to some creatures who were very happy to recycle it, as the next verse recalls. Sus\`ana, the charnel-ground, was not a place for the faint-hearted. Those who delivered bodies to be dealt with by fire, or by decay or by creatures, did not hang about there. Only some yogis/yoginis and  occasional bhikkhus/bhikkhunis would be bold enough to stay there, especially through the night. Most people would have found, and would find today, such serious reminders of impermanence too stark but serious Dharma practitioners lived there without fear.

This tradition from pre-Buddhist times, through the Buddha's lifetime about 2500 years ago, lasted in lndia at least another i500 years. We know this  through the gory descriptions of such `boneyards' found in Buddhist Tantras. These documents, some earlier around 500 CE and some as late as 1000 CE,  paint pictures of some fairly wild characters dwelling in these places. They were at home not only with the ghastly sights but also with various spiritual protectors as well as demonic forces that dwelt there. There is no doubt that these practitioners, for instance some of the famous 84 Siddhas, lived for long periods there to their great benefit.

A `bhikkhu' praised in the next verse, should be understood to include any devoted practitioner. Of course, there are bhikkhus and bhikkhus, a few wise but many without much practice and certainly no insight. It is sad to say this but the mere fact of a man (or woman) having shaven head and robes on does not guarantee spiritual awareness. There are so many people who assume otherwise and then lose all faith when their robed guru turns out to have worse than clay feet. 

Knowing the body `as it really is' means that most of us do not have thorough knowledge of it. somehow we muddle on with a decaying body and only wake up a little when our bodies are in their last drawn~out struggles. lt is better to see how it really is long before that time.

Verse 203 has been expanded in this translation. Literally the first two
lines read:

\begin{MyDescription}[]{}
   As this (is) so (was) that,\\
   as that (is), so (will be) this,
\end{MyDescription}

Though the Pali meaning is clear, such brevity conveys little in English. Those sus\`ana were used for such reflection and helped to discard desires whether for   one's own or others' bodies. 

Notice that the emphasis in this sutta is upon using bodily bits and corpses  to see things as they truly are. There is no trace here of stirring up hatred for the body, which is uncharacteristic of early Buddhist works. Other religious traditions and to some extent later Buddhist works do emphasize hatred against the body. This may be seen in some Pali Commentaries as well as in some Mahayana works. See for instance Chapter 8 of the Bodhicaryavatara.

\chapter{Mui Sutta -- The Sage inwardly silent}

\begin{MyDescription}[\arabic{stanza}]{}
From familiarity fear is born,\\
from household life arises dust;\\
no household, no familiar life —-\\
such is the vision for the sage. 
\stepcounter{stanza}
\end{MyDescription}

\begin{MyDescription}[\arabic{stanza}]{}
Who, cutting down what has grown up,\\
plants not again, supplies no means for growth,\\
they call that Sage who fares alone;\\
great-seeker-seen-the-place-of-peace.
\stepcounter{stanza}
\end{MyDescription}

\begin{MyDescription}[\arabic{stanza}]{}
Who has surveyed the grounds and lost the seeds\\
and supplied no means for further growth\\
is Sage seen to the end of birth and death,\\
logic abandoned and beyond reckoning.
\stepcounter{stanza}
\end{MyDescription}

\begin{MyDescription}[\arabic{stanza}]{}
Truly have been known all resting--places\\
with no desires at all for any there--\\
that sage indeed, free from crowing, greed,\\
struggles not, gone to the further shore.
\stepcounter{stanza}
\end{MyDescription}

\begin{MyDescription}[\arabic{stanza}]{}
Who is intelligent, knowing All, All overcome\\
among all the dharmas, one who cannot be sullied,\\
who All has abandoned, freed by craving's end--\\
that one do the wise proclaim as a sage.
\stepcounter{stanza}
\end{MyDescription}

\begin{MyDescription}[\arabic{stanza}]{}
In wisdom strong, in virtuous conduct stablished,\\
in concentration and enjoying jhana,\\
free from all ties, aridity and the inflows —-\\
that one do the wise proclaim as a sage.
\stepcounter{stanza}
\end{MyDescription}

\begin{MyDescription}[\arabic{stanza}]{}
The vigilant sage who practises alone,\\
who unshaken is by blame or praise,\\
is as a lion that trembles not at sounds,\\
or as wind within a net cannot be caught,\\
or like a lotus flower by water not defiled,\\
leading other people but not by others led—-\\
that one do the wise proclaim as a sage.
\stepcounter{stanza}
\end{MyDescription}

\begin{MyDescription}[\arabic{stanza}]{}
Who though oppressed, behaves unmoving as a pile-post,\\
when others about oneself use speech extreme\\
that one free from lust, sense-faculties restrained—-\\
that one do the wise proclaim as a sage.
\stepcounter{stanza}
\end{MyDescription}

\begin{MyDescription}[\arabic{stanza}]{}
Who is straight-minded as shuttle straighfly moves\\
and who conduct examines both the rough and the smooth\\
and so who turns away from evil karma-making —-\\
that one do the wise proclaim as a sage.
\stepcounter{stanza}
\end{MyDescription}

\begin{MyDescription}[\arabic{stanza}]{}
Who with a mind restrained, evil does not do\\
whether young, middle-aged or sage self-controlled,\\
who cannot be provoked nor others does provoke —-\\
that one do the wise proclaim as a sage.
\stepcounter{stanza}
\end{MyDescription}

\begin{MyDescription}[\arabic{stanza}]{}
Who lives upon almsfood, by others donated\\
receiving the first, the middle, or remainders at the end,\\
who then sings not owned praises, or hurtfully speak --\\
that one do the wise proclaim as a sage.
\stepcounter{stanza}
\end{MyDescription}

\begin{MyDescription}[\arabic{stanza}]{}
The sage not practising indulgence in sex\\
who even when youthful was not tied to anyone,\\
not indulgence in madness of wanton ways but free —-\\
that one do the wise proclaim as a sage.
\stepcounter{stanza}
\end{MyDescription}

\begin{MyDescription}[\arabic{stanza}]{}
True Knower of the universe, Seer of highest truth,\\
crossed the ocean's flood, One Thus and unattached,\\
One who's knots are cut, with no inflows left —-\\
that one do the wise proclaim as a sage.
\stepcounter{stanza}
\end{MyDescription}


\begin{MyDescription}[\arabic{stanza}]{}
The householder with wife: the `not-mine-maker'\\
of strict practicers, - their living-ways not the same:\\
house--livers not restrained from taking others' lives,\\
but the sage always guards other beings' lives.
\stepcounter{stanza}
\end{MyDescription}

\begin{MyDescription}[\arabic{stanza}]{}
In flight the crested peacock, turquoise-necked,\\
never the swiftness of the swan attains,\\
so a house-liver and a bhikkhu cannot match\\
a sage meditating in the woods.
\end{MyDescription}
  
\begin{MyDescription}[(Sn. 207--221)]{}
\end{MyDescription}      
\newpage  
\section{Un muni, mona and munitti and other matters}
There are a group related words in Pali with meanings that cannot be covered by a similar linguistic group in English. This can be seen from the table below:
\newline
\newline
\begin{tabular}{|p{4cm}|p{4cm}|p{4cm}|}
\hline
\textbf{mum}&\textbf{Mona}&\textbf{munati}\\ \hline
(noun), trans. sage, or perhaps a word not be translated& (noun), trans. silence, solitary practice.& (verb) to be wise, specifically because of solitary life \\ \hline
\end{tabular}
\newline
\newline
Originally, the person referred [to] was not a bhikkhu in the Buddhist sense though the later Pali Commentaries maintain that muni=solitary bhikkhu. Evidence in the suttas suggests that some disciples of the Buddha lived as solitaries in the forest or in caves without the burden of the monk's rules, the Vinaya. This tradition of receiving instructions from a teacher and then retiring for practice in solitude pre--dates the Buddha. What we now call Hinduism had, and still has, many holy men who practised among other things, silence to varying degrees. The most extreme would never speak and lived in solitary places so that they never had any cause to do so. There is, in Buddhist records an example of bhikkhus who decide that during their first Rains Retreat they will refrain from talking. At the end of this three or four months they return the [to?] visit the Buddha who asked them how their retreat has been been. They tell him of their silence. He rebukes them that they should practise silence like animals do. Human beings should communicate and not act as they do for lack of speech. to   this day, what they wanted to practise was and is called `mauna'. The verb `munati' is not so common in Pali but its existence demonstrates a further meaning: wisdom derived from long periods of contemplation without much conversation. 
The last verse perhaps is not entirely true of our own times when educated practitioners may be found among the laity. Many years ago in my work on Vinaya (`Moss on the Stones, unpublished), I had made another and more elegant, translation of this verse:
\begin{MyDescription}[\arabic{stanza}]{}
As the peacock, azure-necked,\\
never rivals flight of swans,\\
so householders are no match\\
for forest sage who meditates.
\stepcounter{stanza}
\end{MyDescription}

Finally, though it is not found in Pali Buddhist tradition, there is the well--known mantra for the praise of our great teacher when he is called Shakyamuni:

\begin{MyDescription}[]{}
Om Muni, Muni, Mah\=amuni, Shakyamuniye Sv\=aha.
\end{MyDescription}

\part{The Minor Chapter \\ C\={a}lavagga}

\chapter{Ratanasutta -- The Threefold Gem}
\begin{MyDescription}[\arabic{stanza}]{}
Whatever beings are assembled here,\\
creatures of earth or spirits of the sky,\\
may they be happy-minded, every one,\\
and pay good heed to what is said to them.
\stepcounter{stanza}
\end{MyDescription}

\begin{MyDescription}[\arabic{stanza}]{}
Hence, all ye spirits, hear attentively,\\
look lovingly upon the human race,\\
and, since they bring you offerings day and night,\\
keep watch and ward about them heedfully.
\stepcounter{stanza}
\end{MyDescription}

\begin{MyDescription}[\arabic{stanza}]{}
The riches of this world and of the next\\
and all precious things the heavens may hold,\\
none can compare with the Tathagata.\\
Yea, in the Buddha shines this glorious gem:\\
By virtue of this truth, may blessing be!
\stepcounter{stanza}
\end{MyDescription}

\begin{MyDescription}[\arabic{stanza}]{}
The waning out of lust, that wondrous state\\
of Deathlessness the Sakyan Sage attained\\
through calm and concentration of the mind —
\stepcounter{stanza}
\end{MyDescription}

\begin{MyDescription}[\arabic{stanza}]{}
nothing with that state can ought compare.\\
Yea, in the Dhanna shines this glorious gem\\
By virtue of this truth, may blessing be!
\stepcounter{stanza}
\end{MyDescription}

\begin{MyDescription}[\arabic{stanza}]{}
That flawless meditation praised by Him\\
who is the wisest of the wise, which brings\\
instant reward to one who practises —\\
naught with this meditation can compare.\\
Yea, in the Dhanna shines this glorious gem\\
By virtue of this truth, may blessing be!
\stepcounter{stanza}
\end{MyDescription}

\begin{MyDescription}[\arabic{stanza}]{}
Those Persons Eight who all the sages praise\\
make up four pairs. Worthy of offerings\\
are they, the followers of the Happy one\\
and offerings made bear abundant fruit.\\
Yea, in the Sangha shines this glorious gem:\\
By virtue of this truth, may blessing be!
\stepcounter{stanza}
\end{MyDescription}

\begin{MyDescription}[\arabic{stanza}]{}
Whoso, desireless, have applied themselves\\
firm-minded to the love of Gotama,\\
reached to the goal, plunged into Deathlessness\\
freely enjoy Col Peace they have attained.\\
Yea, in the Sangha shines this glorious gem:\\
By virtue of this truth, may blessing be!
\stepcounter{stanza}
\end{MyDescription}
 
\begin{MyDescription}[\arabic{stanza}]{}
Who clearly comprehend these Noble Truths\\
well-taught by him of wisdom fathomless,\\
however heedless be they afterwards\\
upon an eighth existence they'll not seize.\\
Yea, in the Sangha is this glorious gem:\\
By virtue of this truth, may blessing be!
\stepcounter{stanza}
\end{MyDescription}

\begin{MyDescription}[\arabic{stanza}]{}
As soon as one with insight is endowed\\
three things become discarded utterly:\\
wrong view of a perduring self, and doubt,\\
and clinging to vain rites and empty vows.\\
Escaped that one from all four evil states,\\
and of the six great sins incapable.\\
Yea, in the Sangha is this glorious gem:\\
By virtue of this truth, may blessing be!
\stepcounter{stanza}
\end{MyDescription}

\begin{MyDescription}[\arabic{stanza}]{}
Whatever sort of evil karma done —\\
by body even, or by speech or mind,\\
for one to hide these is not possible —\\
impossible for Seer of the State, it's said.\\
Yea, in the Sangha is this glorious gem:\\
By virtue of this truth, may blessing be!
\stepcounter{stanza}
\end{MyDescription}

\begin{MyDescription}[\arabic{stanza}]{}
Just as a forest grove puts forth its flowers\\
when the first month of summer heat has come,\\
so for the highest good of all, He taught\\
the truth sublime which to Nirvana leads.\\
Yea, in the Buddha is this glorious gem:\\
By virtue of this truth, may blessing be!
\stepcounter{stanza}
\end{MyDescription}

\begin{MyDescription}[\arabic{stanza}]{}
The Highest One, the Knower of the Highest\\
the Giver and the Bringer of the Highest\\
tis He who taught the Highest Truth of all.\\
Yea, in the Buddha is this glorious gem:\\
By virtue of this truth, may blessing be!
\stepcounter{stanza}
\end{MyDescription}

\begin{MyDescription}[\arabic{stanza}]{}
The old is withered up, new being there is not,\\
now their minds desire no future birth,\\
destroyed the seeds, no want for future growth,\\
extinguished are those wise ones as this lamp.\\
Yea, in the Sangha is this glorious gem:\\
By virtue of this truth, may blessing be!
\stepcounter{stanza}
\end{MyDescription}

\begin{MyDescription}[\arabic{stanza}]{}
Whatever beings are assembled here,
creatures of the earth or spirits of the sky,
to th' gods-and-men-adored Tath\=agata,
to the Buddha let us bow: may blessing be!
\stepcounter{stanza}
\end{MyDescription}

\begin{MyDescription}[\arabic{stanza}]{}
Whatever beings are assembled here,
creatures of earth or spirits of the sky,
to th' gods-and-men—adored Tath\=agata,
to the Dharma let us bow: may blessing be!
\stepcounter{stanza}
\end{MyDescription}

\begin{MyDescription}[\arabic{stanza}]{}
Whatever beings are assembled here,\\
creatures of earth or spirits of the sky,\\
to th' gods—and-men-adored Tath\=agatha,\\
to the Sangha let us bow: may blessing be
\stepcounter{stanza}
\end{MyDescription}

\begin{MyDescription}[(Sn. 222-238)]{}
\end{MyDescription}

\chapter{\=Amagandha Sutta\\ Food and the True Meaning of `Stench'}

\begin{MyDescription}[\arabic{stanza}]{Brahmin Tissa:}
Wild millet, grains of grass and pulse,
young shoots and roots and jungle fruits —-
Dharma--gained and by the Peaceful eaten
they who speak no lies desiring sensual pleasures.
\stepcounter{stanza}
\end{MyDescription}

\begin{MyDescription}[\arabic{stanza}]{}
But who, eating food that's well-—prepared and cooked\\
of Sali--rice, all other things to eat,\\
delicious, by others donated specially\\
that one, O Kassapa, is like a carrion-stench.
\stepcounter{stanza}
\end{MyDescription}

\begin{MyDescription}[\arabic{stanza}]{}
`No carrion--stench is mine', you say like this,\\
that it does not apply to you, O Brahma--kin —-\\
while eating sali-rice, all other things\\
with flesh of fowls so veiy well prepared;\\
the meaning of this, O Kassapa, l ask:\\
Your food, what son of carrion-stench it has?
\stepcounter{stanza}
\end{MyDescription}

\begin{MyDescription}[\arabic{stanza}]{The Buddha:}
Taking life, torture, mutilation too,\\
binding, stealing, telling lies and fraud\\
deceit, adultery and studying crooked views:\\
this is carrion-stench, not the eating of meat.
\stepcounter{stanza}
\end{MyDescription}

\begin{MyDescription}[\arabic{stanza}]{}
Those people of desires and pleasures unrestrained,\\
greedy for tastes with impurity mixed in,\\
of nihilistic views, unstable, hard to train:\\
this is carrion-stench, not the eating of meat.
\stepcounter{stanza}
\end{MyDescription}

\begin{MyDescription}[\arabic{stanza}]{}
The rough, the cruel, backbiters and betrayers,\\
those void of compassion, extremely arrogant,\\
the miserly, to others never giving anything:\\
this is carrion-stench, not the eating of meat.
\stepcounter{stanza}
\end{MyDescription}    


\begin{MyDescription}[\arabic{stanza}]{}
Who's angry, obstinate, hostile and vain,\\
deceitful, envious, a boastful person too,\\
full of oneself, with the wicked intimate:\\
this is carrion—stench, not the eating of meat.
\stepcounter{stanza}
\end{MyDescription}    

\begin{MyDescription}[\arabic{stanza}]{}
Those of evil ways: defaulters on debts,\\
imposters, slanderers, deceitful in their dealings,\\
vile men who commit evil deeds in this world:\\
this is carrion-stench, not the eating of meat.
\stepcounter{stanza}
\end{MyDescription}   
      
\begin{MyDescription}[\arabic{stanza}]{}
Those people unrestrained for living beings here,\\
taking others' property, on injury intent,\\
immoral, harsh and cruel, for others no respect:\\
this is carrion-stench, not the eating of meat.
\stepcounter{stanza}
\end{MyDescription} 

\begin{MyDescription}[\arabic{stanza}]{}
Towards others greedy or hateful — they attack them,\\
ever on misdemeanours bent, they go to darkness after death\\
such beings as this fall headlong into Hell:\\
this is carrion-stench, not the eating of meat.
\stepcounter{stanza}
\end{MyDescription} 
    
\begin{MyDescription}[\arabic{stanza}]{}
Not from fish and flesh tasting and not by nudity,\\
not by the plucking of head-hairs nor growing of matted locks,\\
not by the smearing of the ashes of the dead,\\
not wearing abrasive skins, not following sacrificial fires\\
or worldly austerities forgaining immortality,\\
nor mantras, nor offerings, oblafions, seasons' services\\
can purify a mortal still overcome by doubt.
\stepcounter{stanza}
\end{MyDescription} 

\begin{MyDescription}[\arabic{stanza}]{}
Who lives with sense-streams guarded, well-aware,\\
in the Dharma firm, enjoying gently rectitude,\\
beyond attachments gone, all dukkha left behind,\\
that wise one's unsullied by the seen and the heard.
\stepcounter{stanza}
\end{MyDescription}    

\begin{MyDescription}[\arabic{stanza}]{}
Again, again the Radiant One this topic taught\\
to that knower of the Vedas, in those mantras expert,\\
thus clarified the Sage in verses sweetly-sounding.\\
Him of no carrion-stench, free who's hard to trace.
\stepcounter{stanza}
\end{MyDescription}  

\begin{MyDescription}[\arabic{stanza}]{}
Having listened to these verses well-spoken by the Buddha,\\
free of such stench, all dukkhas dispelling,\\
he of humble heart bowed at the Tath\=agata's feet\\
and there and then requested his own Leaving--home.
\stepcounter{stanza}
\end{MyDescription}     
   
\begin{MyDescription}[(Sn. 239-252)]{}
\end{MyDescription}
\newpage
\section{The Meaning of \=amagandha and other considerations}
This short sutta provides us with a number of puzzles beginning with its title. It is critical how this compound of \=ama and gandha is translated. '\=ama' originally meant `raw' or `uncooked' while `gandha' is a general word for `smell' which can be qualified by adding the prefixes `su' — good, or `du' — bad. One might assume from this that the meaning of these two words in a compound should be `the smell of uncooked food`. ln fact, they came to mean `the smell of raw (meat)'. By extension, and considering the lack of refrigeration in tropical India in ancient times and the speed with which raw meat goes off, it implied `the stench of carrion'.
Past translators have struggled with this complex word:\\
\newline
such are flesh-savours and not eating meat' (E.M Hare),\\
\newline
this is a stench. Not the eating of meat' (Saddhatissa),\\
\newline
this is tainted fare, not the eating of flesh' (Norman/Homer),\\
\newline
the foul smell of carrion, not the eating of flesh' (jayawickrama).\\
\newline
   
These are four examples of the refrain in vv.242-248 according to the authors above. However it is translated it must apply both to eating meat and the bad `smell' of evil karma. This is because the Buddha, in these verses lists evil conduct as `carrion-stench' of great import, as opposed to `the eating of meat' which is a much lesser matter. Brahmins particularly proclaimed their purity because they adhered to a vegetarian diet while looking down upon some lower casts who were meat-eaters. To this day while travelling in India one may be asked by a high-caste person if one is a vegetarian. A positive answer to this question will gain one several points of esteem in that brahmin's mind. This is based upon a common teaching in Hinduism on what could be called the doctrine of `purity through eating'. If only purity was so easy! The Buddha's verses in this sutta point out  what is truly impure. However, this should not be taken as a rejection of vegetarian food or a denial of its benefits, specially that it involves no slaughter of animals.\\

Earlier Buddhist views about this are influenced by the example of the Buddhist monk or nun's behaviour going upon alms round with their bowls, accepting to eat this day whatever food they are offered and then eating it without the discrimination `this is good, this is not'. Such an attitude is reasonable for monks and nuns who have no money and so cannot choose what they will have. But laypeople who do have money are able to choose and may give food which has not involved killing. And many monks these days do have money. Later Buddhists, as monastic institutions grew in size, perceived that it would be better to advise their donors to adhere to vegetarian diets and to give monastics, from a concern for the animals killed, vegetarian food. Even with this consideration, the main emphasis for all Buddhists is upon the mind, with less stress upon food. That this is correct may be seen from the presence of occasional extremist vegetarians whose concern for a particular diet based upon some `view' of food is in their eyes the most important feature of practising Dharma, while to others their doctrines are a neurotic obsession.\\
  
At a few places in the suttas, including this sutta, it seems that the Buddha when he was offered food containing meat, ate it. Possibly, since references to this are few, this was a rare event. lf challenged the Buddha would give most importance to the state of mind and very much less to the content of food. And it is worth our consideration, though this cannot be an excuse for indulgence, that even the purest of vegans will not be able to eat anything without the destruction of some living beings: to live is dependent upon eating; to eat is to destroy. This is sams\=ara — the wandering through birth and death and has its dark side though compassion may limit this.\\

Another odd matter about this sutta is its participants — only two of them, once a brahmin, lissa, and the other a Buddha called Kassapa. At least this is what the Comy says. ln the sutta itself the name "fissa' does not occur while the expounder of the Dharma is just called 'Kassapa'. Now, both these names are very common in the suttas where dozens of Tissas occur and many Kassapas as well. While the interlocutor may well have been a brahmin called Tissa, that the Kassapa here should have been a Buddha is more doubtful. A Buddha by that name in D.l4 and if we take that reference literally, lived very long ago. To claim, as Comy does, that that remote Buddha and that the Kassapa of this sutta are the same makes for difficulties. The most obvious of these is found in Pali Comys that claim that a new Buddha cannot arise until all marks of a previous Buddha — teachings, robes, stupas, images and so on — have disappeared. This may be called a rather `late' doctrine and in our eyes these days a rather unimportant one. Still if the Comy is examined the question will arise: How did this supposed sutta from the mouth of the Buddha Kassapa survive the intervening aeons to appear eventually in the text of Sn? As this is such a doubtful matter and one which cannot be resolved we are faced with either its acceptance as a wonderful survival from another Budda's era, or more likely the verses of a disciple of `our' Buddha, one of a number of disciples called Kassapa. This however, will not explain references to a Buddha in the last two verses.\\
 
Two other minor matters may be mentioned here. Verse 249 gives poetically a list of wrong practises, wrong because by themselves they will not lead to liberation though some of them may have value. These austerities, mild or severe, were not praised by the Buddha s he was surrounded, outside his own disciples, by extremist doctrines and practitioners who held the wrong view that liberation was to be attained by dukkha. A similar but shorter verse is found at Dhp. 141:

\begin{MyDescription}[]{}
Neither going naked, nor matted hair, nor filth,\\
nor fasting, nor sleeping on the earth,\\
no penance on the heels, nor sweatiness, nor grime,\\
can purify a mortal still overcome by doubt.
\end{MyDescription}

Whoever the Kassapa was he was faced by a Tissa who was either stupid or prejudiced since the teacher had to repeat his teaching again and again according to verse 251!\\

A further use of this word in Sn. at verse 717 is in the negative form `nir\=amagandha' where it is translated `carrion-stench' but seems to refer not to food but to sex.

\chapter{Hiri Sutta\\ `Conscience' and so on}

\begin{MyDescription}[\arabic{stanza}]{}
Though this person says `I am your friend',\\
nothing's done for you as comrade would,\\
but bereft of conscience, e'en despising you:\\
as `not one of mine' this person should be known.
\stepcounter{stanza}
\end{MyDescription}

\begin{MyDescription}[\arabic{stanza}]{}
Who uses pleasant words to friends\\
but does not act accordingly,\\
wise people understand like this:\\
a speaker not a doer'
\stepcounter{stanza}
\end{MyDescription}

\begin{MyDescription}[\arabic{stanza}]{}
That one's no friend who diligently\\
seeks your faults, desiring strife;\\
but with whom one rests, as child on breast,\\
is friend indeed who none can part.
\stepcounter{stanza}
\end{MyDescription}

\begin{MyDescription}[\arabic{stanza}]{}
One who causes states of joy,\\
who brings praiseworthy happiness,\\
who's grown the Fruits' advantages,\\
the human burden bears.
\stepcounter{stanza}
\end{MyDescription}

\begin{MyDescription}[\arabic{stanza}]{}
Having drunk of solitude\\
and tasted Peace sublime,\\
free from sorrow, evil-free\\
one savour: Dharma's joy.
\stepcounter{stanza}
\end{MyDescription}

\begin{MyDescription}[(Sn. 253-257)]{}
\end{MyDescription}
\newpage
\section{Notes on the Hiri Sutta}
First, a few words upon the translation of `hiri' into English. Most translators have used `shame' but there are many objections to this: `Hiri' as a quality in the Suttas and Abhidhamma is wholesome, a good quality. lt features notably in the Suttas at A. Twos. I 8-9 where hiri and its companion \textit{ottappa} are translated as `shame and fear of blame' (Gradual Sayings, P.T.S) though the translator in a footnote has `conscientiousness' as an altemative for the first of these. This ungainly word is preferable to `shame', as the latter in English could not be called totally wholesome being associated, as it is, with guilt. Hiri and its companion by contrast are called bright (sukka) dharmas and praised as world's protectors. They protect the world from degenelation to greed, hatred and delusion. Protecting the world has two meanings — protecting the mind of the potential doer, and protecting others from the sufferings brought about by unrestrained evil.\\

Ny\=anamoli in his translations suggests `shamefulness' for hiri but this does not cover subsidiary meanings such as `shyness' or `bashfulness' which cannot be described as wholesome qualities. `Conscience' will be appropriate in some places and `modesty' in others, sometimes even `decency'. It is difficult to find an English word to cover all this. Ottappa is best translated as `fear of consequences' and with hiri acts as a brake for some peoples' unwholesome drives.\\

The verses of this sutta are a rag-bag — bits and pieces from here and there somehow cobbled together with little attention to coherence. Though called the Hiri Sutta it is not mostly about conscience, decency shame etc., but rather concerns the qualities of a good friend. The Pali Commentary tries to make sense of these verses by giving them an invented occasion when a brahmin asked of the Buddha four questions. However this seems an artificial `explanation' and has led to some strange translations (For the four questions see Saddhatissas translation).\\

The first three verses are straightforward but the fourth has had many and varied translations. l have not followed the Pali Commentary in interpreting this sutta. The fourth verse describes a person who has fully practised the Dharma and is fit to be a teacher of others, a bodhisattva perhaps since he/she bears the human by removing it from others who suffer. A true friend indeed! Verse 5, a favourite of mine, is from the Dhammapada (205) while verse 2 appears also at Jataka III. 193: Va\=o\=o\=aroha J\=ataka, No. 361.\\

\chapter{Mah\=amang\=ala Sutta\\ The Supreme Goad Omens}
Thus have l heard:\\
\newline
At one time the Radiant One was dwelling at ]eta's Grove in the park of An\=athapindika near S\=avatthi. Then, as night was ending, a deva of surpassing radiance, illuminating the whole of ]eta's Grove, went up to the Radiant One and stood to one side after saluting him. Standing there that deva addressed the Radiant One with a verse:

\begin{MyDescription}[\arabic{stanza}]{Deva:}
Of humans, gods, there are so many\\
who have sought to know good omens\\
which, they hope, will bring them safety:\\
tell then the supreme good omen.
\stepcounter{stanza}
\end{MyDescription}

\begin{MyDescription}[\arabic{stanza}]{The Buddha:}
Not consorting with the foolish,\\
rather with the wise consorting,\\
honouring the honourable:\\
this is a supreme good omen.
\stepcounter{stanza}
\end{MyDescription}
    
\begin{MyDescription}[\arabic{stanza}]{}
Living in befitting places,
having in the past made merit,
right direction in self-guidance:
this is a supreme good omen.
\stepcounter{stanza}
\end{MyDescription}

\begin{MyDescription}[\arabic{stanza}]{}
Ample learning, and a craft, too,\\
with a well-trained disciplining,\\
any speech fluat is well-spoken:\\
this is a supreme good omen.
\stepcounter{stanza}
\end{MyDescription} 

\begin{MyDescription}[\arabic{stanza}]{}
Aid for mother and for father,\\
and support for wife and children,\\
spheres of work that bring no conflict:\\
this is a supreme good omen.
\stepcounter{stanza}
\end{MyDescription} 

\begin{MyDescription}[\arabic{stanza}]{}
Giving, practising by Dharma,
with support for kin provided,
karmas causing no obstructions:
this is a supreme good omen.
\stepcounter{stanza}
\end{MyDescription} 

\begin{MyDescription}[\arabic{stanza}]{}
Shrinking, abstinence, from evil,\\
from besotting drink refraining,\\
diligence in Dharma-practice:
this is a supreme good omen.
\stepcounter{stanza}
\end{MyDescription} 

\begin{MyDescription}[\arabic{stanza}]{}
Respectfulness, a humble manner,\\
with content, and grateful bearing,\\
hearing Dharma when its timely:\\
this is a supreme good omen.
\stepcounter{stanza}
\end{MyDescription} 

\begin{MyDescription}[\arabic{stanza}]{}
Patience, meekness when corrected,\\
visiting too those pure in practice,\\
discussing Dharma when its timely:\\
this is a supreme good omen.
\stepcounter{stanza}
\end{MyDescription} 

\begin{MyDescription}[\arabic{stanza}]{}
Ardour, and the Good life leading,\\
insight into Truths so Noble,\\
realization of Nirvana:\\
this is a supreme good omen.
\stepcounter{stanza}
\end{MyDescription} 

\begin{MyDescription}[\arabic{stanza}]{}
Though one is touched by Worldly Dharmas\\
yet one's mind does never waver,\\
griefless, spotless and secure:\\
this is a supreme good omen.
\stepcounter{stanza}
\end{MyDescription}

\begin{MyDescription}[\arabic{stanza}]{}
Having practised all these `omens'\\
everywhere they go unvanquished,\\
they go everywhere in safety:\\
such is their supreme good omen.
\stepcounter{stanza}
\end{MyDescription}

\begin{MyDescription}[(Sn. 258-269)]{}
\end{MyDescription}
\newpage
\section{Mangala: false mangalas and Dharma-mangalas}
in the Buddha's days, as in our own, people adhered to the superstitious ideas of what is lucky/unlucky, auspicious and its opposite and even `religious' omens of fortune and misfortune. Such events, happenings and bodily marks have of course varied through the ages but the ideas and superstitions connected with it remain a part of many peoples' lives. India being a vast country with many languages and cultural differences, then as now, had differing traditions about what was lucky but no certainty could be reached about the underlying reasons why `a' was lucky and `b' unlucky. Tradition could not agree about it. This is what the Mangala Sutta's first verse is about. Devas and humans decided to ask the Buddha about this matter.\\

Before we read his list of 37 `supreme good omens', we should be clear about the usual understanding of omen. An example that I encountered years ago in Thailand will illustrate the tangled and confused nature of omens generally.\\

ln Thailand, where bhikkhus usually go out to collect their food with their bowls in the early morning, the sight of a monk or several of them, as soon as the house or shop door is opened, is reckoned to be very auspicious. This auspiciousness' does not take account of whether the monk or monks are ennobled by their Dharma practice, or whether they are guys using the robes to get an easy livelihood. This `omen' of the sight of early morning monks of whatever kind, is reckoned `good'.\\

Opposed to this illogic is the idea held by Chinese of whom there are many in Bangkok and other Thai cities. lf they behold a monk first thing in the morning, this is reckoned in their tradition as inauspicious, not a good day for the making of money. Why? Though it is hard to believe, the argument goes thus: Monks own nothing (at least they are not supposed to own much) and they teach a doctrine of nothing (a confused reference to what is called `emptiness or voidness' in English), so they are ill—omened for businessmen! Here are two cultures with quite opposed ideas upon a supposed omen! ls one more true than the other? No, both are superstitious because the reason behind these `omens' is not based upon cause and effect. The causes (seeing monks) have no real relation to the supposed effects of either auspicious (Thai idea) or inauspicious Chinese).\\

In some cultures which are supposed to be scientifically `advanced' still many may be found who adhere without thought to ancient superstitions about what is good, lucky, or an omen. The Buddha's standard of auspiciousness transcends these confused ideas and offers a clear summary of what is truly beneficial for everyone irrespective of race, language, culture and religion.\\

The sutta, which is straightforward, does not need a detailed commentary, though if one is required, the classical Pali Commentary upon this Sutta is translated in “Minor Readings and illustrator” published by the Pali Text Society. The translafion of this book and of the sutta quoted here is by Ven. Nanamoli Thera, who tirelessly devoted his life to rendering many Pali texts into English, some of them quite abstruse. In his honour, and with the permission of the PTS, I quote his translation here with one or two minor changes.\\

One note upon the line: `Though one is touched by Worldly Dharmas' may be useful. What are the worldly dharmas that one may be touched by? This refers to the famous eight lokadhamm\=a found principally in A.Egiths, 5,6 and in my old translation made in Wat Bovoranives, Bangkok. The verses of this sutta read as follows:

\begin{MyDescription}[(A. Eights. 6)]{}
Gain and loss together with honour and dishonour,\\
blame and praise, happiness, dissatisfaction\footnote{`dissatisfaction' = dukkha} too,\\
these, the impermanent conditions of mankind\\
never perpetual, perturbate are they:\\
these, the heedful one with wisdom well-endowed\\
carefully discerns as conditions perturbate.\\
Desirable conditions do not agitate the mind,\\
nor conditions undesired and can make resentment rise,\\
compliance, opposition too, are for that one no more,\\
not smouldering are they, to non-existence gone;\\
and then having Known that Stainless, Griefless State,\\
rightly one Knows becomings' Other Shore.\\
\end{MyDescription}

\begin{MyDescription}[(A. Eights. 6)]{}
\end{MyDescription}  

\chapter{S\=aciloma Sutta\\ To the Yakkha S\=aciloma}

Thus have I heard:\\
\newline

At one time the Radiant One was dwelling in Gaya at the Stone Couch in the place of the Yakkha S\=aciloma. At that time the yakkhas Khara and S\=aciloma paused nearby and the former said: `That is a monk'. `He's not a monk, he's justa `mere-monk' but wait until I find out whether he's a monk or a `mere-monk''. Then the yakkha S\=aciloma approached the Radiant One and pressed his body against him, at which the Radiant One drew back. The yakkha then said to him, `Are you afraid of me, monk?' `Friend, l am not afraid of your but your touch is evil.' `Monk, l shall ask you a question and if you do not reply to me l shall overturn your mind, split your heart and grasping you by the feet fling you to the other side of the Ganges'. `Friend, I do not see anyone indeed who in this world with its devas, maras and brahma-gods, together with its people -— monks and brahmins, rulers and ordinary persons, who could overturn my mind, split my heart and grasping me by the feet fling me to the other side of the Ganges. Still, friend, you can ask whatever you wish,'\\

Then the yakkha Saciloma addressed the Radiant One with this verse.

\begin{MyDescription}[\arabic{stanza}]{Saciloma:}
From whence the causes of both lust and hate,\\
from what are likes, dislikes and terror born,\\
what origin's there for thoughts in mind,\\
as boys harass a (captive) crow?
\stepcounter{stanza}
\end{MyDescription}

\begin{MyDescription}[\arabic{stanza}]{Buddha:}
From causes here come lusts and hate,\\
now, are. Likes, dislikes and terror's born,\\
present origin's there for the thoughts in mind\\
as boys release a (captive) crow.
\stepcounter{stanza}
\end{MyDescription}

\begin{MyDescription}[\arabic{stanza}]{}
Born of lubricity, arisen from self\\
bearing branch-born roots as the banyan figs,\\
such are they in sensuality entwined\\
as woods entangled by the stinky-vine.
\stepcounter{stanza}
\end{MyDescription}

\begin{MyDescription}[\arabic{stanza}]{}
Listen, O yakkha, for those who know\\
from where these causes come — all they dispel,\\
they cross this flood so hard to cross,\\
uncrossed before, to not become again.
\stepcounter{stanza}
\end{MyDescription}

\begin{MyDescription}[(Sn. 270-273)]{}
\end{MyDescription}
\newpage
\section{Remarks upon S\=aciloma Sutta}
Yakkhas are perhaps demonized wild non-ariyan inhabitants who lived in forests and had few possessions but some reputation for magical matters. Saciloma, a name meaning `needle-hair' was possibly a very hairy male though one may doubt that his hair was needle-like. The Buddha's remark that he was not afraid of him but that his touch was `evil' may mean that this yakkha was unwashed and malodorous. Though he seems to have been uncouth yet the question he asked is not that of an ignorant person.\\

The question-verse in its first three lines has the P\=ali interrogative `kuto' — whence, from where, from what. `Kutonidana' in the first line means `from whence the causes', `kutoia' in the second translates `bom from what', while the third line has `kuto summutthaya' is `from what origin'. This question is framed very much in Buddhist teaching: the enquiry into causality.\\
 
The simile in the last line raises a number of questions, the first of them being, what is it that boys do to a crow? We are not told in the sutta about this and most translators have resorted to the P\=ali Comy. There it is explained that boys catch a crow, tie string to its legs and let it go as far as the string permits when they jerk it and so crash the crow. Sounds like boys generally have not changed much! This may be true, or perhaps the Comy has based its tale on later behaviour of boys! As the line reads it is literally `as boys a crow...' The space here is for translation of the verb `ossaiiati' which PTSD says means `to let loose, let go, send off, giveup, dismiss, release', while ADOP adds `lets go, releases, throws, abandons'. The P\=ali Comy glosses this verb with `khipati', to throw.\\
  
Translators so far have rendered the simile: `drag down as boys will drag a crow' (E.M Hare), '(harass) as boys do a crow' (Saddhatissa), `as young boys toss up a (captive) crow' (K.R. Norman), `like (tethered) crow pulled by boys captors) earthward' (Mrs Rhys Davids in Kindred Safings, Samyutta Nikaya, Vol. I), and `(toss one around) as boys toss up a crow' (Bhikkhu Bodhi in The Connected Discourses of the Buddha, Samyutta Nilraya). While the  ossaiiati will support such translations as `drag', `toss up', `pull', etc, it cannot stretch to harass': `Harass' and even `torture' are what the boys are undoubtedly doing to the crow, an interpretation rather than a translation. As `bhipati = to throw' is a Comy. gloss this gives more latitude.\\

 The simile of the boys and the crow is interesting and unusual. The Comy explains that the boys represent the thoughts, while the crow is due mind which is either harassed or released. No doubt the P\=ali Comy identifies the boys as unwholesome states of mind and this is fitting in the first verse but even if the same meaning is given to the simile in the second verse, it will not fit. Suppose that one chooses `toss around' as the translation for `ossaiiati' in both verses this is only appropriate for the verse, not for the Buddha's reply. All translators without exception, in following the Comy, repeat the same rendering of the simile in both verses. But now, supposing that the Buddha has played with the multiple meanings of vb. `ossajiati' so that a different meaning in the second verse is appropriate in translation?\\
 
  To appreciate that this might be so, we have to consider the meanings of ito' which replaces `kuto' in the second verse. ln this verse `ito' appears three times and is rendered `here' `now' and `present'. The Buddha in this verse has emphasized that `causes', `birth' and `origin' are not so much a matter of the past, especially of such beliefs as in past lives, as the first verse suggests, but concern the present. This emphasis on the present ties in with such teachings as:
 
 \begin{MyDescription}[]{}
    One should not trace back the past\\
   or on the future build one's hopes,\\
   the past is just the left-behind,\\
   the future is the yet-unreached;\\
   rather with insight one should see\\
   each dharma as it arises now...
\end{MyDescription}

\begin{MyDescription}[M. 131 Bhaddekaratta Sutta).]{}
\end{MyDescription}
And of course with the practice of mindfulness.\\

As the second verse deals with causality in the present moment which would lead to Awakening in the present, and as the vb ossaijati can mean `release, let go, loose', it seems appropriate to translate the line as: `as boys release a captive) crow'.\\

This pair of P\=ali verses could -be corrupt, as traditions other than the P\=ali texts vary widely. The sutta is repeated in S. One.X. 10.3 and readers should consult portions of the Comy. translated in \textit{Connected Discourses of the Buddha} of the Buddha, Vol.l, for further information.

   
   
   
\chapter{Dhammacariya Sutta\\  Wrong conduct in the bhikkus life}

\begin{MyDescription}[\arabic{stanza}]{(The Buddha?):}
The Good Life living, with Dharma accordingly\\
they say that this is wealth supreme.\\
But if one leaves the household life\\
gone forth from home to homelessness
\stepcounter{stanza}
\end{MyDescription}

\begin{MyDescription}[\arabic{stanza}]{}
and then be one of those foul-mouthed,\\
beast-like, delighted doing harm,\\
such a one's of evil life\\
increasing `dust' within himself,
\stepcounter{stanza}
\end{MyDescription}

\begin{MyDescription}[\arabic{stanza}]{}
a bhikkhu delighting in quarrelling\\
while in delusion wrapped\\
knows not the Dharma even when\\
its by the Buddha pointed out;
\stepcounter{stanza}
\end{MyDescription}   

\begin{MyDescription}[\arabic{stanza}]{}
led along by ignorance so\\
that one harms those of well-grown mind,\\
and does not know defilements' path\\
that leads to hellish life.
\stepcounter{stanza}
\end{MyDescription}   

\begin{MyDescription}[\arabic{stanza}]{}
To Downfall going on and on\\
from life to life, from dark to dark,\\
a bhikkhu such as this indeed\\
hereafter to dukkha descends.
\stepcounter{stanza}
\end{MyDescription}  

\begin{MyDescription}[\arabic{stanza}]{}
One such with blemishes is like\\
a public shit-pit filled to the brim —\\
used for many years —\\
so very hard to clean..
\stepcounter{stanza}
\end{MyDescription}  

\begin{MyDescription}[\arabic{stanza}]{}
O bhikkhus, when you come to know\\
one such attached to household life ~\\
of evil desires and evil thoughts\\
and of evil ways of behaviour,
\stepcounter{stanza}
\end{MyDescription}  

\begin{MyDescription}[\arabic{stanza}]{}
all of you united then\\
should shun, avoid a person such,\\
blow away these sweepings and\\
throw away that trash,
\stepcounter{stanza}
\end{MyDescription}  
  
\begin{MyDescription}[\arabic{stanza}]{}
and suchlike chaff winnow away -\\
those sham monks, those conceited monks —\\
having blown them off, those who are\\
of evil wants and wrong resorts,
\stepcounter{stanza}
\end{MyDescription}     

\begin{MyDescription}[\arabic{stanza}]{}
then living in purity with the pure\\
with mindfulness you will abide,\\
in concord live, intelligent —\\
you will arrive at dukkha's end.
\stepcounter{stanza}
\end{MyDescription}   

\begin{MyDescription}[(Sn. 274-283)]{}
\end{MyDescription}      
  
\newpage  
   
\section{Notes on the Dhammacariya Sutta}
This sutta is unusual in that it contains no indication of who is teaching though one may assume that it is the Buddha. Of course, the Comy. offers an occasion for its teaching and makes it plain that the Buddha is exhorting the bhikkhus. Though called `Dhammacariya' it could better be known as the Adhammacariya Sutta as most of the verses concern the wrong conduct of a bhikkhu.\\

The language of condemnation of wayward monks is here quite strong and is directed at those who have major failings rather than peccadilloes. Obviously, such monks were hard to reform and the verses in the middle part of this sutta advise monks to avoid such people. Perhaps they were considered irreformable and there is certainly no suggestion here of compassionate action towards them.\\

in one sense, just as all humans can be labelled `crazy' so we are all `shit-pits' to some degree. Only those ennobled by the Dharma are free of these taints. So then the language of some of these verses seems unnecessarily severe. Perhaps these stanzas were composed by a particularly self-righteous monk who felt himself far above the failing of his brethren. How they have to be attributed to the Buddha is unknown.
  
  
\chapter{Br\=ahbmanadhammika Sutta\\ How brahmins lived by the Dharma}

Thus have l heard:\\
\newline
At one time the Radiant One dwelt at S\=avatthi, in the Jeta Grove, An\=athapindika's park. Then many decrepit old Kosalan brahmins, aged, elderly, advanced in years, attained to old age, those indeed of palatial abodes, went to the Radiant One and exchanged greeting with him. When this courteous and amiable talk was finished they sat down to one side. Sitting there these brahmins of palatial abodes said, `Master Gotama, are there now to be seen any brahmins who practise the Brahmin Dharma of the brahmins of old?' `No brahmins, there are no brahmins now to be seen who practise the Brahmin Dharma of the brahmins of old.' `It would be excellent if the good Gotama would speak to us upon the Dharma of the brahmins of old if it would not be too much trouble.' `Then brahmins,' listen well and bear in mind what l shall say'. `Indeed, venerable' said those brahmins of palatial abodes to the Radiant One. He spoke as follows:\\

\begin{MyDescription}[\arabic{stanza}]{}
ln ancient times the sages then\\
austerely lived, were self-restrained,\\
let go five bases of desire\\
to fare for their own benefit.
\stepcounter{stanza}
\end{MyDescription}  

\begin{MyDescription}[\arabic{stanza}]{}
Brahmins then no cattle had,\\
no gold, no grain they hoarded up,\\
their grain, their wealth was Vedic lore —\\
this the treasure they guarded well.
\stepcounter{stanza}
\end{MyDescription}  

\begin{MyDescription}[\arabic{stanza}]{}
For them, whatever food prepared\\
was by the doorway placed\\
from faith prepared for those who sought,\\
for (donors) thought it should be given.
\stepcounter{stanza}
\end{MyDescription}  

\begin{MyDescription}[\arabic{stanza}]{}
Then in various states and provinces\\
rich in colourful cloths well-dyed\\
with furniture and dwellings too\\
with these to brahmins they paid respect.
\stepcounter{stanza}
\end{MyDescription}  
   
\begin{MyDescription}[\arabic{stanza}]{}
Unbeaten were brahmins and inviolate —\\
guarded by Dharma-goodness then,\\
none hindered or obstructed them\\
when they arrived at household doors.
\stepcounter{stanza}
\end{MyDescription}

\begin{MyDescription}[\arabic{stanza}]{}
Until the age of eight-and-forty\\
they practised celibate student life —\\
the brahmins of those ancient times\\
fared seeking knowledge and conduct good.
\stepcounter{stanza}
\end{MyDescription}

\begin{MyDescription}[\arabic{stanza}]{}
Those brahmins went not to others' wives\\
nor bought a wife from other clans;\\
by mutual consent together they came\\
being happy with each other.
\stepcounter{stanza}
\end{MyDescription}  
 
\begin{MyDescription}[\arabic{stanza}]{}
Brahmins then did not indulge\\
in sexual intercourse out of time\\
during menstruation\\
but only when wives were free from this.
\stepcounter{stanza}
\end{MyDescription}  

\begin{MyDescription}[\arabic{stanza}]{}
The celibate life was praised by them
with virtue and uprightness,
friendliness, penance and gentleness,
harming none and patient too.
\stepcounter{stanza}
\end{MyDescription}  

\begin{MyDescription}[\arabic{stanza}]{}
Whoso `mong them strong efforts made\\
resembling Brahma, best,\\
he never did engage in sex\\
not even in a dream.
\stepcounter{stanza}
\end{MyDescription}     
  
\begin{MyDescription}[\arabic{stanza}]{}
Then some of them with wisdom blest\\
followed his practice path\\
praising the celibate life, as well\\
as virtue and as patience too.
\stepcounter{stanza}
\end{MyDescription}   

\begin{MyDescription}[\arabic{stanza}]{}
Having begged rice, butter and oil,\\
with cloths and bedding too\\
they sought and stored these righteously\\
and from them made a sacrifice:
during that sacrificial rite\\
cattle they never killed.\\
\stepcounter{stanza}
\end{MyDescription}   

\begin{MyDescription}[\arabic{stanza}]{}
Like mother (they thought), father, brother\\
or any other kind of kin,\\
cows are our kin most excellent\\
from whom come many remedies --
\stepcounter{stanza}
\end{MyDescription}   
   
\begin{MyDescription}[\arabic{stanza}]{}
givers of good and strength, of good\\
complexion and the happiness of health,\\
having seen the truth of this\\
cattle they never killed.\\
\stepcounter{stanza}
\end{MyDescription}  

\begin{MyDescription}[\arabic{stanza}]{}
Those brahmins then by Dharma did\\
what should be done, not what should not,\\
and so aware they graceful were,\\
well—built, fair-skinned, or high renown.\\
While in the world this lore was found\\
these people happily prospered.
\stepcounter{stanza}
\end{MyDescription}  

\begin{MyDescription}[\arabic{stanza}]{}
But then in them corruption came\\
for little by little they observed\\
how rajahs had to splendours won\\
with women adorned and elegant,
\stepcounter{stanza}
\end{MyDescription}    
   
\begin{MyDescription}[\arabic{stanza}]{}
and chariot, yoked to thoroughbreds,\\
caparisoned, embroideries finely sewn,\\
and houses well-designed with walls —\\
insides divided into rooms,
\stepcounter{stanza}
\end{MyDescription}    
   
\begin{MyDescription}[\arabic{stanza}]{}
filled with crowds of women fair\\
and ringed by herds of increasing cows —\\
al this the eminent wealth of men\\
the brahmins coveted in their hearts.
\stepcounter{stanza}
\end{MyDescription}    

\begin{MyDescription}[\arabic{stanza}]{}
Then they composed some Vedic hymns\\
and went chanting to Okk\=aka king:\\
`Great your wealth and great your grain,\\
make sacrifice to us with grain and wealth'.
\stepcounter{stanza}
\end{MyDescription}    

\begin{MyDescription}[\arabic{stanza}]{}
That rajah, Lord of chariots,\\
by brahmins was persuaded so\\
he offered all these sacrifices:\\
\\
of horses, men, the peg well-thrown,\\
the sacrifice of soma drink\\
the one of rich results —\\
while to the brahmins wealth he gave:
\stepcounter{stanza}
\end{MyDescription}    
   
\begin{MyDescription}[\arabic{stanza}]{}
of cattle, bedding and of cloth\\
with women adorned and elegant\\
and chariots yoked to thoroughbreds\\
caparisoned, embroideries finely sewn,
\stepcounter{stanza}
\end{MyDescription}  

\begin{MyDescription}[\arabic{stanza}]{}
dwelling in which one would delight,\\
these well-divided into rooms\\
and many different kinds of grain,\\
this wealth he to the brahmins gave.
\stepcounter{stanza}
\end{MyDescription}  

\begin{MyDescription}[\arabic{stanza}]{}
When they had all this wealth received\\
to hoard it up was their desire\\
for they were overwhelmed by greed —\\
their craving thus increased —\\
so they composed more Vedic hymns\\
and chanting went to Okk\=aka king.
\stepcounter{stanza}
\end{MyDescription}  


\begin{MyDescription}[\arabic{stanza}]{}
`As water is, and earth, as well\\
as gold, as grain as well as wealth,\\
in the same way for human beings,\\
and cattle are necessities;\\
Great your wealth and great your grain,\\
make sacrifice to us with grain and wealth'.
\stepcounter{stanza}
\end{MyDescription}  

\begin{MyDescription}[\arabic{stanza}]{}
That rajah, lord of chariots,\\
by brahmins was persuaded — so\\
in sacrifice, he caused to kill\\
cattle in hundreds, thousands too.
\stepcounter{stanza}
\end{MyDescription}  

\begin{MyDescription}[\arabic{stanza}]{}
But neither with hooves nor horns\\
do cows cause harm to anyone,\\
gentle they are as sheep\\
yielding us pails of milk;\\
in spite of this the rajah seized\\
their horns, slew them by the sword.
\stepcounter{stanza}
\end{MyDescription} 

\begin{MyDescription}[\arabic{stanza}]{}
Then devas, antigods, demons, led\\
by Indra, even the ancestors,\\
cried out `Against the Dharma is all this'\\
while fell the ..... .. upon the cows.
\stepcounter{stanza}
\end{MyDescription} 

\begin{MyDescription}[\arabic{stanza}]{}
In former times three ills were found:\\
desire and hunger and decay,\\
but due to the killing of cattle\\
ninety-eight diseases came.
\stepcounter{stanza}
\end{MyDescription} 

\begin{MyDescription}[\arabic{stanza}]{}
This adharmic wielding of weapons\\
descended from times of old:\\
in this are the innocents slain\\
while ritual priests from Dharma fell.
\stepcounter{stanza}
\end{MyDescription}     

\begin{MyDescription}[\arabic{stanza}]{}
So this ancient practice, base\\
is censured by the wise;\\
where similar things are seen\\
people blame the ritual priests.
\stepcounter{stanza}
\end{MyDescription}   

\begin{MyDescription}[\arabic{stanza}]{}
When Dharma was perverted thus
merchants and workers split apart
and warrior-nobles split as well
while wife her husband did despise.
\stepcounter{stanza}
\end{MyDescription}   

\begin{MyDescription}[\arabic{stanza}]{}
Then nobles and those of Brahma `kin'\\
and others restrained by love of caste\\
neglected then their laws on `birth'\\
and under the sway of pleasures came.
\stepcounter{stanza}
\end{MyDescription}   
 
When this was said the brahmins of palatial abodes exclaimed to the Radiant One: `Magnificent, Master Gotama! The Dhanna has been clarified by Master Gotama in many ways, as though he was fighting what had been overturned, revealing what was hidden, showing the way to one who was lost, or holding a lamp in the dark so that those with eyes could see forms. We go for refuge to Master Gotama, to the Dharma and to the Sangha. May Master Gotama remember us as upasakas who from today have Gone for Refuge for life.'

\begin{MyDescription}[(Sn. 284-315)]{}
\end{MyDescription}   

\newpage

\section{The Rich Old Brahmin's and the Buddha --}
\subsection{What happened?}
Imagine the scenario: the Buddha is seated in the grounds of the Jeta Grove outside the prosperous city of S\=avatthi (Sr\=avasti). A number of really old brahmins who are also rich come to call on the Buddha, no doubt in chariots and with servants and possibly pupils as well. That they are old is insisted upon the use in the text of no less than six adjectives to this effect, so we may picture them in the 70s, 80s or 90s even. That they are rich is described by only one word `mahasala', literally `of great halls' surrounded no doubt by great estates from which they derived much wealth. As was the custom among Indians, especially brahmins, in approaching a Teacher they first exchanged greetings and no doubt their names with the Buddha and we are told that this included `courteous and amiable talk'. This was taken to be a polite and auspicious beginning to contact with a Teacher. Sitting to one side is also polite, in this way a visitor does not presume to occupy the space immediately opposite the Teacher. Then one of these rich old brahmins speaking on behalf of the others, asked his astonishing Question.\\

At this point we may pause in our imagination of the scene and bring to the fore our examining faculties: Why did these old brahmins ask the Buddha a question which they must have known would receive a negative reply? They would know — perhaps partly from their own lives — that brahmins of their times conducted themselves far differently from the ideal brahmins of the past. And of course the Buddha denied that there were any brahmins in their times who lived according to the ancient brahmin Dharma. The old brahmins then asked him very politely, (`if it would not be too much trouble') to discourse upon this subject.\\

The puzzle posed by the sutta is why the brahmins asked this particular question. They would know the conditions of most brahmins in their society: a glimpse of this may be had from some verses of Sn. - 612-619. There we learn that so-called brahmins were farmers, craftsmen, merchants, servants, thieves, soldiers, priests and rajahs. But the Buddha's very high standards on what constituted a brahmin (620-647) went far beyond these worldly descriptions.\\

From their studies of the Vedas and auxiliary literature, they would know how brahmins were portrayed in the distant past, so why did they, according to this text, ask the Buddha about this. It could be that some of them were curious as to how the Buddha would reply for after all, he was not regarded by them as an `orthodox' brahminical teacher. Perhaps he would give a scathing account of brahmin behaviour which they could then argue with him. Or perhaps they wished an answer from a teacher they knew to be famous who would see this question in a different light. ln general though, the brahmins in the P\=ali suttas are portrayed as orthodox and conservative, having little interest in exploring others' teachings. There are notable exceptions, as with V\=settha and Bh\=aradv\=aja at Sn. 594-656, but most brahmins did not welcome other teachers examining and criticizing parts of the Brahmin Dharma. ln any case these brahmins did not ask a question which could go beyond their own tradition.\\

As an alternative to these speculation there is the possibility that the occasion' (nidana) for this discourse has nothing to with rich old brahmins asking an improbable question. Prose suttas have `occasions' describing when and where they were spoken by the Buddha and there are a few cases of this in Sn. ln the majority of the poetic suttas there is no occasion given in the Sn. text but the Sn. Comy. provides `occasions', some of which are more likely than others. Let us suppose that this `occasion' though included in the Sn. text is in fact a later addition to the verses and provides a story upon which may be build a strong Buddhist criticism of the brahmins. This supposition would dissolve away the story of old brahmins asking about their own Dharma.

\subsection{Brahmins, those then past and then present}

Sn. 284-298 verses concern how the ancient brahmins lived with their ideal conduct given in detail. Whether this is looking back upon a `golden age' of brahmins, or whether it agreed with the facts of history, is hard to tell, though it is common to all religious traditions that they begin with a flowering of true spirituality, to be followed over generations by decline. Verses 284-288 could almost apply to the lives of good bhikkhus who also are held in honour for their austere and compassionate practice and so receive gifts from their supporters. The next verse introduces a particularly brahminical practice, the long studentship of young celibate brahmins more less spent with their teachers. When this period of the Good Life was complete then brahmins married. Verse 301 throws light upon brahminical fears of `pollution', that is, of degradation from their caste, in this case caused by contact with menstrual blood. Hindu law books contain many examples of what leads to being outcasted, such as travelling overseas and partaking food not cooked by brahmin-caste people, and so on. By contrast, the Buddha emphasized pollution of the mind/emotions. However, verse 292 pictures a Good Life which would be an ornament to anyone whether in Buddhist monastic robes or within other religious traditions.\\

There follow two verses strongly advocating the celibate life in a way that could describe good bhikkhus too. From 295 onwards there are verses upon begging'. English does not have a very good word to translate `y\=acati' as understood by Buddhist teachings. A bhikkhu going on almsround must be silent, unless spoken to, and may not ask for food generally though allowances are made for sick monks. They collect in their bowls whatever people are happy to give to them. This dignified practice differs in many ways from the pictures conjured up by `begging'.\\

Verse 295 emphasizes that the original form of brahminical rites involved no bloodshed and is followed by two verses praising cattle, particularly cows, for their benefits to human society. No mention of course, is made of the so-called benefits of beef! Those brahmins then, and most of them today, were vegetarians and appreciated the many products that come from milk The results of this kindness to cattle and Dharma—practice generally are listed in 298.\\

But as is the way in this world, this happy state could not continue. The causes of brahminical decline are spelt out in the next few verses. Many of these factors would apply equally to Buddhist monastic degeneration. Just as those brahmins of old greedily stared at the wealth and luxury of the rajahs' courts and coveted what they saw, so many Buddhist monks these days want to enjoy the pleasures advertised in so many ways in `western' materialistic life.\\

Sn. 299-310 tells us precisely what those brahmins of ....[missing copy?].. The results of their karma was quite different. It is worth our reflection upon present times when so much adharma is practised and so many new diseases have arisen as dangers for us now. Could there be causes and effects — past human causes with painful results for us?\\

Weapons and killing are clearly stated (312) to be not the way of Dharma. But if we view our world now - through newspapers, radio, television and so on, what do we find every day — more and more violence. Some of it is dressed up as the lawful control of others who would bring even worse evils than we have presently. Powerful people, the present-day representatives of rajahs, speak upon this and are believed. `Well, they must know, mustn't they? They must know more than us!' So the blind follow the blind. What good can ever come from violence? This is not to recommend pacifism in its extreme forms for even the famous Buddhist king, Asoka, kept an army though he did not use it for offensive warfare after he became a Buddhist. But in general, violence begets violence while as the Dhammapada reminds us:

\begin{MyDescription}[]{}
Never here by enmity\\
are those with enmity allayed:\\
they are allayed by amity —\\
this is the Natural Law.
\end{MyDescription}

\begin{MyDescription}[(Dhp. 5)]{}
\end{MyDescription}

The sutta closes with two verses upon the destruction of traditional norms caused by the violence (in this case of the sacrifices) in their society. Even if the norms of that society (caste, discrimination and even persecution of low caste and outcaste people) does not recommend itself to our understanding of civilization, still it was an established code of behaviour. But the Buddha, whose essential teaching was (and is) karmic causation, pointed out and made clear that ritual sacrifices involving violence cannot possibly bring the good results of peace, good health and wealth. The (violent) causes do not agree with the desired peaceful results. lf we desire peace these days, this causation still applies: peace, happiness and spiritual growth can never be brought about by more and more dreadful weapons. Evil cannot be put down by more evil even when this is disguised by the powerful calls of nationalism, destruction of violent foes and our personal safety in future. 

\begin{MyDescription}[]{}
Never here by enmity.......
\end{MyDescription}

\chapter{N\=av\=a Sutta\\ Dharma as a Boat}


 \begin{MyDescription}[\arabic{stanza}]{}
 As devas do venerate their lord, Indra king\\
 So likewise to that person from whom one knows Dharma,\\
 respected clear-minded and very learned too,\\
 that teacher makes manifest the words of the Dharma
 \stepcounter{stanza}
 \end{MyDescription}
 
\begin{MyDescription}[\arabic{stanza}]{}
This having considered then the wise person\\
while practising Dharma according with Dharma\\
becomes learned, intelligent, subtle-minded too\\
by diligently dwelling with one who is Such.
\stepcounter{stanza}
\end{MyDescription}

\begin{MyDescription}[\arabic{stanza}]{}
But by following the foolish, inferior fellow\\
who's not found Dharma's goal while envious of others,\\
to death one will come before Dharma knowing\\
not having crossed over (the river of doubts).
\stepcounter{stanza}
\end{MyDescription}

\begin{MyDescription}[\arabic{stanza}]{}
Just as a person going into a river\\
swollen in flood and very swiftly flowing,\\
would be carried away by the force of the current —\\
then how can this person help others across?
\stepcounter{stanza}
\end{MyDescription}

\begin{MyDescription}[\arabic{stanza}]{}
So it's the same with the unpractised person\\
who knows not the Dharma as explained by the wise,\\
sans knowledge profound, not crossed over doubting\\
how could this person cause others to Know?
\stepcounter{stanza}
\end{MyDescription}   
 
\begin{MyDescription}[\arabic{stanza}]{}
But one who does on a strong boat embark\\
furnished with oars and rudder complete,\\
as skilled in the means, with wisdom as well,\\
that one can take so many others across.
\stepcounter{stanza}
\end{MyDescription}   

\begin{MyDescription}[\arabic{stanza}]{}
Of mind developed deeply, one who Knows truly,\\
one of great learning, or unshakeable Dharma,\\
other people can lead who possess the capacity\\
to listen attentively and penetrate deeply.
\stepcounter{stanza}
\end{MyDescription}   

\begin{MyDescription}[\arabic{stanza}]{}
Therefore be sure to frequent a True person,\\
an intelligent one who is of great learning,\\
realized in the meaning, practised on the Path,\\
a Knower of Dharma attained to the Bliss.
\stepcounter{stanza}
\end{MyDescription}   

\begin{MyDescription}[(Sn. 316-323)]{}
\end{MyDescription}

\newpage

\section{Commentary on N\=av\=a Sutta} 
316. Whether the word 'vija??\=a" ****corrupt in original***  in the second line should be translated
   `knows' (in an intellectual sense) or `Knows' (through personal experience of
   Dharma's truth) one's Dharma-teacher, or many Dharma-teachers, should be greatly respected. They have opened one's eyes to the treasure of the Dharma, a gift excelling all other gifts. How they should be revered will vary with different cultures, even simple matters of prostration and anjali vary greatly, while some
   traditional marks of reverence may not be appropriate in `western' lands. Still, these things are not so important in comparison with heartfelt devotion and helpfulness. Those who truly revere their teachers never create trouble for them nor do they stir up strife among their disciples.\\
   
   317. `Practising the Dharma according with Dharma'
   (dhamm\=anudhammapatipatti) is obviously opposed to the egocentric method of `practising Dharma according to oneself'. With the former since it is Dharma-practice according with the teachings and principles of Dharma will advance experience of the Dharma, while with the latter the only result will be to lead away from Dharma and from one's teacher's instructions. The qualities of both the teacher and the pupil are both mentioned in this verse and it is obvious that pupils, with diligent practice, will become like their teacher in virtues and wisdom, eventually to become teachers themselves. Awakened teachers are sometimes referred to as those who are Such (T\=adi) that is, they have Known the Dharma as Such (or Thus) and not otherwise.\\
   
318. The dangers of not practising with a teacher who is Such but `following a foolish inferior fellow' are made plain in this verse. This description of an unworthy teacher and the disadvantages of being a student under him/her remind us of the verse (Sn. 259):

\begin{MyDescription}[]{}
Not consorting with the foolish,\\
rather with the wise consorting,\\
honouring the honourable....
\end{MyDescription}
   
The river of doubts is the experience of all unenlightened people. What do they doubt? They think that they know through blind devotion, intellectual knowledge or by following tradition, that really they Know but they only believe. `To believe' in fact, means `not to know' although one may be sure that one's belief is pure and correct, even the only doctrine which is true. But as belief in doctrines, including Buddhist ones, means that these have not been verified, there must, somewhere in the back of the mind, be doubt. The effort to believe, for instance in six improbable things before breakfast, is a rather unskilful way to cover up doubts. Doubts make for interior conflict, while beliefs can lead to exterior conflicts, even to bloodshed, persecution and wars. What are called religious beliefs in English are included under the Pali-Sanskrit word \textit{dithi/drsti,} ***Again incorrectly formatted in original*** literally `what has been seen', hence the English translation `view'. Right views are those which accord with Dharma whether in matters of moral conduct (sala), meditation (sam\=adhi) or in wisdom (pa\~n\~n\=a/praj\~n\=a). Dharma in this sense does not equate solely with `Buddhism'. It is not only Buddhists who are good, kind and generous people and who purify their minds! Wrong views may be minor matters of belief held in the mind but not disputed with others, or they may lead to violence, killings, sectarian wars and assassinations. ln general, holding views is a block to spiritual development. Holding Buddhist views of any kind means one has not seen the Dharma for oneself. Holding wrong views — and there is a great variety of them — is destructive of Dharma both within and without. Believing in
the Dharma may be a first step but it should not be long adhered to: clinging to the Dharma is just another kind of clinging (up\=ad\=ana) and this is certainly not Dharma. Dharma is to be verified, not views to be clung to.\\

319-320. The first of these verses presents a striking simile with a question at its conclusion. So, how can this person help others across these floods? The help that they try to give others — to get them across the floods of ignorance and craving — could not succeed because they have not yet experienced `knowledge profound' and so have not gone to the Further Shore themselves. Elsewhere too the Buddha speaks about the situation where one stuck on this shore of sams\=ara will be unable to extricate another person in the same plight: `This situation does not occur, Cunda, that one person sunk in a morass will be able to pull out another in the same plight' (M. 8) but one who is no longer stuck will be able to extract another. The Suttas also speak of three persons: one sunk in quicksands, another with one foot upon dry land and last the person with both feet on firm ground. Only the last one representing Buddhas and Awakened masters, is capable of pulling out others from the sams\=ara-morass. The middle person may be able to give some help as they have some personal experience of the awakening Dharma but the first person can hardly help others. Even if one has vowed, as in Mahayana traditions, to cross over all beings to Nirvana, still one must spend many years with good masters, learning and practising, before this can be even partly accomplished.\\


321. The strong boat is the Dharma. I have been fortunate to know a number of teachers who may have crossed over in the Dharma-vessel and taken many others across. These teachers' Dharma, both their instructions and their conduct, was wonderful, the words and actions of those gone beyond self. They had hundreds or thousands of pupils who revered them as models of the Dharma manifest in a living person. So of course, they could `take so many others across'. ln another sutta (M.22) the Buddha addressing bhikkhus says: `I shall show you how Dharma resembles a raft, being for the purpose of crossing over, not for the purpose of grasping. Suppose a traveller saw a great expanse of water, whose near shore was dangerous and fearful, whose further shore was safe and free from fear, but there was no ferry or bridge. Then after considering this, he collected grass, twigs, branches and leaves and bound them together in a raft, supported by which and making efforts with his hands and feet, he got safely across. Then when he had got across, he thought `This raft has been ve|y helpful to me, since by its means I got safely across; suppose that I hoist it on my head or load it on my shoulders and go wherever I want?' `No, Lord'. `What should he do with it then? lf when he got across, he thought, `This raft has been very helpful to me since by its means I got safely across; suppose I haul it up on dry land or set it adrift on the water and go where I want to go?' `That is how he should do what should be done with a raft. So I have shown you how the Dharma resembles a raft being for the purpose of crossing over, not for the purpose of grasping. When you Know Dharma resembling a raft, then even Dharma-teaching should be abandoned, what to speak of non-Dharma.' \\

The strong boat of Dharma can also be a vast vessel as in a M\=ah\=ayana sutra:

\begin{MyDescription}[]{}
What then again is the vessel leading to Bodhi?\\
Standing upon it one guides all beings to Nirvana.\\
Great is that vessel, immense, vast like the vastness of space\\
Those who travel upon it are carried to safety, delight and to ease.
\end{MyDescription}

So going across the ocean, lake, river, floods, has been taught in many ways by the Buddha according to the persons being addressed.\\
   
322. `Unshakeable Dharma' is only found in those who Know and See. They are not unshakeable because of obstinacy, or from attachment to views, dogmas or to their own ideas. No force of any kind can shake their Knowing and Seeing the truth of the Dharma. They are unshakeable, as the traditional P\=ali Comy remarks, by any or all of the eight Worldly Dhannas: Gain and loss, honour, dishonour, blame, praise, happiness, dissatisfaction (=dukkha). See also Comy on ll.4.\\

323. `Attained to the Bliss'. Sukha in this book is usually translated `happiness' but the sukha referred to here is that of realization of Dharma beyond all views.
 
\chapter{Ki\"ys\=ala sutta\\ What is good conduct?}
\begin{MyDescription}[\arabic{stanza}]{S\=ariputta:}
With what kinds of conduct and morality,\\
growing in what sorts of karmas,\\
will a person well-established be\\
for attainment of the highest goal?
\stepcounter{stanza}
\end{MyDescription}
   
\begin{MyDescription}[\arabic{stanza}]{The Buddha:}
Let that one be an honourer of elders, never envious,\\
a knower of the right time for the teacher seeing,\\
and when Dharma's being taught, a knower of that time\\
to listen precisely to those well—spoken words.
\stepcounter{stanza}
\end{MyDescription}

\begin{MyDescription}[\arabic{stanza}]{}
And at the right time go to the teacher's presence\\
in an unassuming way, discarding obstinacy,\\
with restraint and recollection of the way to practise\\
remembering the Dharma for the life of purity.
\stepcounter{stanza}
\end{MyDescription}

\begin{MyDescription}[\arabic{stanza}]{}
Dwelling in the Dharma, delighted in Dharma,\\
in Dharma established and skilled in deciding Dharma,\\
never uttering words to the Dharma's detriment,\\
Let such a one be guided by well-spoken truths.
\stepcounter{stanza}
\end{MyDescription}

\begin{MyDescription}[\arabic{stanza}]{}
Disputatiousness, gossip, complaints and ill-will,\\
deception, hypocrisy, longing and pride,\\
aggressiveness, harshness, defilements ~ attached,\\
fare abandoning these, pride-free, of steady mind.
\stepcounter{stanza}
\end{MyDescription}

\begin{MyDescription}[\arabic{stanza}]{}
Understanding's the essence of well-spoken words\\
while that and the learnt is the essence of calmness\\
but wisdom and learning in one do not grow —\\
that person who's hasty and negligent both.
\stepcounter{stanza}
\end{MyDescription}

\begin{MyDescription}[\arabic{stanza}]{}
Delighting in Dharma by Noble Ones taught,\\
their mind, speech and body all unsurpassed —\\
in gentleness, peace, meditative-states firm,\\
attained to the essence of wisdom and learning
\stepcounter{stanza}
\end{MyDescription}
  
\begin{MyDescription}[(Sn.324-330)]{}
\end{MyDescription}   

\chapter{U\~n\~nh\=ana Sutta \\ Wake up! Make an Effort!}

\begin{MyDescription}[\arabic{stanza}]{}
Get up and sit!\\
What need of sleep!\\
For the sick what rest is there\\
pierced by the dart of pain?
\stepcounter{stanza}
\end{MyDescription}


\begin{MyDescription}[\arabic{stanza}]{}
Get up and sit!\\
Train hard for peace.\\
Let not Mara know\\
that you are negligent,\\
deluded and under his control.
\stepcounter{stanza}
\end{MyDescription}

\begin{MyDescription}[\arabic{stanza}]{}
Cross beyond this craving —\\
tied to, desiring which\\
gods and men remain.\\
Don't let this chance pass by:\\
those who do so grieve\\
sending themselves to hell!
\stepcounter{stanza}
\end{MyDescription}

\begin{MyDescription}[\arabic{stanza}]{}
From `dust' arises negligence,\\
from negligence to more:\\
by diligence and knowledge\\
pluck out the dart oneself.
\stepcounter{stanza}
\end{MyDescription}

\begin{MyDescription}[(Sn. 331-334)]{}
\end{MyDescription}
\newpage
\section{A Pithy Comment}
How long has one to live? Does an end, the old end of death, come today, tomorrow, a few weeks or months away? And here we are lying down for hours and hours. Do we sleep so long because life is so painful — too much dukkha. And when we reach bodily sufferings even the escape of sleep and drowsiness may be no longer available.\\

Get up and sit', means `do it while you can!' The obvious pleasures do not last for long and a time will come when even comforts no longer manifest. Life isn't nice for long. Even if you think of eighty years as long, think even that one hundred years passes soon and what will come after?\\

Our own Maras — our temptations and conflicts — strengthen the lazier we become. Mara is just a picturesque name for mental defilements. No Mara is `out there' to increase our dukkha — Maras are our own burdens. No need to be under Mara's control — or to be under the dominion of the daughters of Mara! So don't use Mara as an excuse of one's own weakness or failings.\\

Craving may be pleasant, evenly divinely pleasant for so are the devas attached to their delights and burdened by them. Divine attachments must seem superior to the joys of humanity but deva~happiness does not equate to liberation. In fact, tangling oneself with innumerable pleasures - where one can do this as a human — could lead as a result of that karma to a painful future. `Sending themselves to hell': no one else sends them there.\\

Dust' accumulates in the house from not cleaning it, similarly does dust in the mind which does however need moment by moment diligence to keep the dust away. Or perhaps a dart, rather than dust, will illustrate how painful is the accumulation of dirt. But it is al impermanent. Who has ever heard of permanent dirt? Though this is true one still has to open an eye and see what is to be seen. So,
\begin{MyDescription}[]{}
get up!
\end{MyDescription}

\chapter{R\=ahula Sutta\\ Teaching R\=ahula}

\begin{MyDescription}[\arabic{stanza}]{The Buddha:}
From living together constantly\\
the Teacher you don't scorn?\\
Torch-bearer to humanity\\
is he by you revered?
\stepcounter{stanza}
\end{MyDescription}

\begin{MyDescription}[\arabic{stanza}]{R\=ahula:}
From living together constantly\\
the Teacher I scorn not.\\
Torch-bearer to humanity\\
is by me revered.
\stepcounter{stanza}
\end{MyDescription}   

\begin{MyDescription}[\arabic{stanza}]{The Buddha:}
Having let go five sense-desires\\
and forms that are dear, delighting mind,\\
with faith renounce the household life,\\
be one who dukkha ends.
\stepcounter{stanza}
\end{MyDescription}

\begin{MyDescription}[\arabic{stanza}]{}
Keep company with noble friends,\\
dwell in a lonely practice-place\\
secluded, having little noise,\\
with food be moderate.
\stepcounter{stanza}
\end{MyDescription}

\begin{MyDescription}[\arabic{stanza}]{}
Robes as well as food from alms,\\
with shelter, also remedies —\\
for these things no craving form,\\
so turn not to the world again.
\stepcounter{stanza}
\end{MyDescription}

\begin{MyDescription}[\arabic{stanza}]{}
By Patimokkha stay restrained\\
and by the five sense faculties,\\
practise bodily mindfulness\\
to be dispassionate.
\stepcounter{stanza}
\end{MyDescription}

\begin{MyDescription}[\arabic{stanza}]{}
Avoid those objects beautiful\\
which may be linked with lust,\\
on the unlovely, one-pointed,\\
well-concentrated, grow the mind.
\stepcounter{stanza}
\end{MyDescription}

\begin{MyDescription}[\arabic{stanza}]{}
Develop then the signless state\\
with tendency to pride let go —\\
by fully understanding it\\
truly as peaceful you will fare.
\stepcounter{stanza}
\end{MyDescription}

ln this way the Radiant One with these verses frequently exhorted the venerable R\=ahula.

\begin{MyDescription}[Sn.335-342]{}
\end{MyDescription}
\newpage

\section{A Note upon Rahula Sutta}
This contains very ordinary Dharma advice for bhikkhus and the only unusual feature is found in verse 334. This raises the question of why the Buddha asks his son, R\=ahula, if he scorns him. Of course this happens often enough in household life, and could also occur in monastic communities. Having asked this, the Buddha continues with words praising himself. Somewhat odd! Perhaps, though this is a speculation, the first two verses are later additions for they ring rather hollow.

\chapter{Vangg\=asa Sutta\\ Vang\=asa's questions, Buddha's Answer}

Thus have l heard:\\
\newline
At one time the lord dwelt at \=al\=ava, at the Agg\=alava Shrine. Not long before the venerable Vang\=asa's preceptor, by name the elder Nigrodhakappa had become completely Cool at the Agg\=alava Shnrine. Then in Vang\=asa's mind, being at one and in solitude, arose this thought: Has my preceptor become completely Cool or has he not? Later in the day Vang\=asa left his solitude and went to the Lord and having drawn near him, paid his respects and sat to one side. While sitting there Vang\=asa said this to the Lord: While I was sitting in meditation this thought arose in my mind: Has my preceptor become completely cold or has he not? Vangasa then arose from his seat, placed his upper robe over one shoulder and, lotussing his hands respectfully, addressed the Lord with these verses.

\begin{MyDescription}[\arabic{stanza}]{Vangasa:}
We ask now the Teacher of wisdom supreme —\\
who's inseen the Dharma, who's cut off all doubts,\\
that Agg\=alava bhikkhu who died recently,\\
famous, well-known, was he truly Cool-become?
\stepcounter{stanza}
\end{MyDescription}

\begin{MyDescription}[\arabic{stanza}]{}
His name Nigrodhakappa was,\\
to that brahmin given by the Radiant One,\\
he went around revering you, and, strenuous\\
seeking Liberation, O See of the Secure.
\stepcounter{stanza}
\end{MyDescription}

\begin{MyDescription}[\arabic{stanza}]{}
O Sakya, All-Seeing, we all wish to know\\
concerning this hearkener, Kappa by name,\\
all of our ears are ready to hear —\\
the Teacher you are, the One unsurpassed.
\stepcounter{stanza}
\end{MyDescription}

\begin{MyDescription}[\arabic{stanza}]{}
Sever our doubt and tell me of this:\\
that he knew complete Cool, O Wisdom Profound:\\
tell this in our midst, O All-Seeing One,\\
as thousand-eyed Sakka by devas ringed.
\stepcounter{stanza}
\end{MyDescription}

\begin{MyDescription}[\arabic{stanza}]{}
Whatever here deluded paths bring on the bondages,\\
on ignorance's side, the bases for all doubts,\\
on reaching the Tathagatha they cease to be\\
for certainly his Eye is supreme among men.
\stepcounter{stanza}
\end{MyDescription}

\begin{MyDescription}[\arabic{stanza}]{}
lf never, no one, could defilements disperse\\
as forceful wind a piled-up mass of clouds,\\
enshrouded would be, for sure, the whole world\\
and even the illustrious would have no chance to shine.
\stepcounter{stanza}
\end{MyDescription}

\begin{MyDescription}[\arabic{stanza}]{}
But the Wise in this world are the makers of light\\
and you, a Wise One, are such I conceive,\\
we have come upon Him who Knows and who Sees —\\
to those here assembled, Kappa clearly reveal.
\stepcounter{stanza}
\end{MyDescription}

\begin{MyDescription}[\arabic{stanza}]{}
Swiftly send forth fair speech, O Fairest One,\\
as swan (its neck) stretches sounding softly forth\\
with your melodious voice so well-modulated\\
to it we listen, all of us, attentively.
\stepcounter{stanza}
\end{MyDescription}

\begin{MyDescription}[\arabic{stanza}]{}
Remainderless, you've let go of birth and death —
I'll urge the One who's Cleansed to Dharma teach;
ordinary persons cannot act out their desires
but with discrimination Tathagatas act.
\stepcounter{stanza}
\end{MyDescription}

\begin{MyDescription}[\arabic{stanza}]{}
(Your) expositions (of Dharma) so thoroughly based\\
on straightforward wisdom then thoroughly grasped\\
(and he) last lotussed his hands with greatest respect\\
so do not delude us, You of wisdom supreme.
\stepcounter{stanza}
\end{MyDescription}

\begin{MyDescription}[\arabic{stanza}]{}
Having known the Dharma noble, the basics and refined,\\
You the Energetic One who Knows, do not delude.\\
I long for your words as for water one does\\
in summer season by heat overcome. Rain down on our ears!
\stepcounter{stanza}
\end{MyDescription}

\begin{MyDescription}[\arabic{stanza}]{}
That purpose for which Kapp\=ayano led\\
the life of purity ~ surely it wasn't in vain;\\
did he become Cool or did residues remain —\\
tell of his Freedom, that we long to hear.
\stepcounter{stanza}
\end{MyDescription}

\begin{MyDescription}[\arabic{stanza}]{The Buddha:}
Craving he cut for mind and body both —\\
craving's stream that long had lain within him\\
completely he has crossed beyond all birth and death\\
So the Blessed One spoke, the Fore before the five.
\stepcounter{stanza}
\end{MyDescription}


\begin{MyDescription}[\arabic{stanza}]{Vang\=asa}
Hearing your word, O Seventh of Seers\\
I'm both pleased and truly satisfied.\\
Tuly my question's not in vain —\\
that brahmin did elude me not!
\stepcounter{stanza}
\end{MyDescription}

\begin{MyDescription}[\arabic{stanza}]{}
As he spoke, he acted so,\\
one of the Buddha's hearkeners\\
who rent the deceiver Mara's net\\
spread wide and very strong.
\stepcounter{stanza}
\end{MyDescription}

\begin{MyDescription}[\arabic{stanza}]{}
Lord, Kappa the capable\\
saw graspings', clingings' source;\\
Kapp\=ayano has gone beyond\\
death's realm so hard to cross.
\stepcounter{stanza}
\end{MyDescription}
\begin{MyDescription}[(Sn. 343-358)]{}
\end{MyDescription}

\chapter{Samm\=aparibb\=ajaniya Sulta\\ Perfection in the Wandering Life}

\begin{MyDescription}[\arabic{stanza}]{Questioner:}
Of the Sage of great wisdom, one gone across,
to the further shore gone, completely Cool, poised
who's renounced a house, sense-pleasures dispelled,
I ask: How would a bhikkhu lightly wander in the world?
\stepcounter{stanza}
\end{MyDescription}

\begin{MyDescription}[\arabic{stanza}]{The Buddha:}
Who has destroyed (belief) in omens, in luck,\\
the occurrence of dreams and other signs such,\\
who is rid of the bane of what is auspicious\\
such a one rightly would wander in the world.
\stepcounter{stanza}
\end{MyDescription}

\begin{MyDescription}[\arabic{stanza}]{}
Who sensuality is able to divert\\
both varieties, human and divine - such a bhikkhu\\
passed beyond being, knowing Dharma well,\\
such a one rightly would wander in the world.
\stepcounter{stanza}
\end{MyDescription}

\begin{MyDescription}[\arabic{stanza}]{}
Anger and avarice by the bhikkhu abandoned,\\
his back having turned upon slander as well,\\
compliance, opposition, completely disappeared,\\
such a one rightly would wander in the world.
\stepcounter{stanza}
\end{MyDescription}

\begin{MyDescription}[\arabic{stanza}]{}
Letting go the pleasant, what's unpleasant too,\\
ungrasping, unsupported by nothing at all,\\
from all the causes for the fetters — completely free,\\
such a one rightly would wander in the world.
\stepcounter{stanza}
\end{MyDescription}

\begin{MyDescription}[\arabic{stanza}]{}
Seeing no essence in mental substrata,\\
dispelled passionate desire for what can be grasped,\\
not being dependent or led by another,\\
such a one rightly would wander in the world.
\stepcounter{stanza}
\end{MyDescription}

\begin{MyDescription}[\arabic{stanza}]{}
In speech, mind and deed to others unopposed\\
and knowing very well the Dharma's full extent,\\
and one who is aspiring to the state of Nirvana\\
such a one rightly would wander in the world.
\stepcounter{stanza}
\end{MyDescription}

\begin{MyDescription}[\arabic{stanza}]{}
The bhikkhu not conceited thinking: `Me he reveres'\\
nor on being abused does he retaliate,\\
nor thrilled with others donations of food,\\
such a one rightly would wander in the world.
\stepcounter{stanza}
\end{MyDescription}

\begin{MyDescription}[\arabic{stanza}]{}
For greed and for being the bhikkhu's let go,\\
as for injury and bondage it's not done by him\\
crossed over doubts, removed is the dart,\\
such a one rightly would wander in the world.
\stepcounter{stanza}
\end{MyDescription}

\begin{MyDescription}[\arabic{stanza}]{}
A bhikkhu who knows what he himself enjoys\\
would not be one who harms others in the world\\
realizing the Dharma as it really is,\\
such a one rightly would wander in the world.
\stepcounter{stanza}
\end{MyDescription}

\begin{MyDescription}[\arabic{stanza}]{}
in whom are no hidden tendencies at all —\\
the roots of evil completely removed,\\
for them no longs left, no yearnings come anew\\
such a one rightly would wander in the world.
\stepcounter{stanza}
\end{MyDescription}

\begin{MyDescription}[\arabic{stanza}]{}
inflows eradicated and conceit let go\\
and transcended the path of sexual desire,\\
one tamed, completely Cool and imperturbable,\\
such a one rightly would wander in the world.
\stepcounter{stanza}
\end{MyDescription}

\begin{MyDescription}[\arabic{stanza}]{}
Confident and learned, one who Sees the Way,\\
one Wise who among sects is no sectarian\\
who greed has diverted, hatred, ill-will too,\\
such a one rightly would wander in the world.
\stepcounter{stanza}
\end{MyDescription}

\begin{MyDescription}[\arabic{stanza}]{}
A conqueror — purity perfected, remover of the veil\\
with majesty of dharmas, far-shorer, inturbulent,\\
skilful with knowledge of conditioned thins' cessation,\\
such a one rightly would wander in the world.
\stepcounter{stanza}
\end{MyDescription}

\begin{MyDescription}[\arabic{stanza}]{}
Of wisdom purified surmounting both\\
past and the future, gone beyond time\\
and in every way free from sense-bases\\
such a one rightly would wander in the world.
\stepcounter{stanza}
\end{MyDescription}

\begin{MyDescription}[\arabic{stanza}]{}
Final Knowledge of the State, having realized the Dharma\\
having seen openly the letting-go of inflows\\
with all the substrata completely dissolved\\
such a one rightly would wander in the world.
\stepcounter{stanza}
\end{MyDescription}

\begin{MyDescription}[\arabic{stanza}]{Questioner:}
Indeed, O Blessed One, certainly it is thus,\\
for that bhikkhu tamed, living like this —\\
one who beyond all the fetters has passed,\\
such a one rightly would wander in the world.
\stepcounter{stanza}
\end{MyDescription}

\begin{MyDescription}[(Sn. 359-375)]{}
\end{MyDescription}

\chapter{Dhammika Sutta\\  To Dhammika on the pure hearkeners' conduct}
Thus have l heard:\\
\newline
At one time the Lord dwelt at S\=avatthi, in the Jeta Grove, An\=athapindika's Park. There the up\=asaka Dhammika accompanied by five hundred upasakas went up to the Radiant One and sat to one side. Having done so and saluted the Radiant One, the upasaka Dhammika addressed him with verses.

\begin{MyDescription}[\arabic{stanza}]{Dhammika:}
i ask of Gotama — one profoundly wise:\\
Behaving in which way a hearkener is good —\\
whether from home to homelessness gone\\
or upasakas living the householder's life.
\stepcounter{stanza}
\end{MyDescription}

\begin{MyDescription}[\arabic{stanza}]{}
The birthplaces of this world together with devas\\
and final Release, you clearly understand,\\
none compare with you in seeing this profundity\\
for, as they say, you are Buddha supreme.
\stepcounter{stanza}
\end{MyDescription}

\begin{MyDescription}[\arabic{stanza}]{}
All knowledge is yours, you have perfectly revealed\\
Dharma, out of your compassion for beings all,\\
remover of the veil, one with the All-round Eye\\
and stainless do you illuminate the world.
\stepcounter{stanza}
\end{MyDescription}

\begin{MyDescription}[\arabic{stanza}]{}
Then came to your presence a n\=aga renowned,\\
Er\=avana by name, having heard you were a conqueror\\
he had secluded talk with you and then attained-\\
`S\=adhu' he exclaimed, and departed, pleased.
\stepcounter{stanza}
\end{MyDescription}

\begin{MyDescription}[\arabic{stanza}]{}
Then were there kinds, Vessavana, Kuvera,\\
who came to ask questions on Dharma from you,\\
so you, O Wise One, being asked then replied\\
and they being pleased departed from there.
\stepcounter{stanza}
\end{MyDescription}

\begin{MyDescription}[\arabic{stanza}]{}
 These theorist sectarians used to dispute -\\
 \=aj\=avakas and Niganthas all of that kind\\
 unable in wisdom they go not beyond you\\
 as a man standing still passes not the [...1] one
\stepcounter{stanza}
\end{MyDescription}

\begin{MyDescription}[\arabic{stanza}]{}
 Then there are Brahmins who're used to dispute -\\
 even old Brahmins are found among them\\
or other disputants proud of themselves\\
all, for the meaning, depend on you
\stepcounter{stanza}
\end{MyDescription}

\begin{MyDescription}[\arabic{stanza}]{}
This Dharma indeed is blissful, profound,\\
by you well-proclaimed, O Radiant One,\\
so wishing to listen are all of us here,\\
now when we asked, speak to us, Buddha the best.
\stepcounter{stanza}
\end{MyDescription}

\begin{MyDescription}[\arabic{stanza}]{}
So let all these bhikkhus well—seated here,\\
upasakas too who likewise wish to listen,\\
listen to the dharma by the stainless won,\\
as devas to Vasava's well-spoken words.

\stepcounter{stanza}
\end{MyDescription}

\begin{MyDescription}[\arabic{stanza}]{The Buddha:}
Listen, O bhikkhus, I give you chance to hear —\\
to the Dharma that's strict — all of you remember it,\\
let the intelligent seeing the benefit\\
practise the deportment of one who's left home.
\stepcounter{stanza}
\end{MyDescription}

\begin{MyDescription}[\arabic{stanza}]{}
A bhikkhu in the times prescribed should wander not\\
but seek for alms timely going round a town;\\
who goes at times prescribed, temptations do tempt\\
so the awakened go not within the wrong time.
\stepcounter{stanza}
\end{MyDescription}


\begin{MyDescription}[\arabic{stanza}]{}
Sights with sounds and tastes, smells and touches too —\\
all these with which beings are completely drunk,\\
for all of these dharmas let go desire\\
and at the right time walk for the morning meal.
\stepcounter{stanza}
\end{MyDescription}

\begin{MyDescription}[\arabic{stanza}]{}
A bhikkhu with timely almsfood gained\\
returns by himself, then seated alone,\\
contemplative with, not distracted without,\\
not exterioring, since oneself's restrained.
\stepcounter{stanza}
\end{MyDescription}

\begin{MyDescription}[\arabic{stanza}]{}
Should he with other hearkeners converse,\\
with bhikkhus, or anyone else at all,\\
of the Dharma let him speak refined,\\
not utter slander or another's blame.
\stepcounter{stanza}
\end{MyDescription}

\begin{MyDescription}[\arabic{stanza}]{}
Some, disputatious, offer warfare with words,\\
but we do not praise them, those of little wit,\\
bound by attachment to talking this and that\\
so certainly they send their minds far away.

\stepcounter{stanza}
\end{MyDescription}

\begin{MyDescription}[\arabic{stanza}]{}
The truly wise disciple having listened to the Dharma\\
pointed out by the Well farer, should carefully use\\
food-offerings, a sitting and a sleeping place\\
with water for washing the principal robes.

\stepcounter{stanza}
\end{MyDescription}

\begin{MyDescription}[\arabic{stanza}]{}
Let a bhikkhu, therefore, with almsfood and hut\\
for sitting and sleeping, for his robes washing,\\
be unsullied, quite unattached\\
as water-drop spreads not upon a lotus-leaf.
\stepcounter{stanza}
\end{MyDescription}

\begin{MyDescription}[\arabic{stanza}]{}
Now I shall tell you the household's rule\\
by practising which one's a good hearkener\\
for by one with possessions it cannot be got —\\
that dharma complete by a bhikkhu attained.
\stepcounter{stanza}
\end{MyDescription}

\begin{MyDescription}[\arabic{stanza}]{}
Kill not any beings nor cause them to be killed\\
and do not approve of them having been killed,\\
put by the rod for all that lives —\\
whether they are weak, or strong in the world.
\stepcounter{stanza}
\end{MyDescription}

\begin{MyDescription}[\arabic{stanza}]{}
What is `ungiven' — anything, anywhere,\\
that's known to be others', its theft one should avoid\\
neither order things taken, nor others' removal approve —\\
all of this `ungiven' let the hearkener avoid.
\stepcounter{stanza}
\end{MyDescription}

\begin{MyDescription}[\arabic{stanza}]{}
Le the intelligent person live a celibate life\\
as one would avoid a pit of glowing coals\\
but being unable to live the celibate life\\
go not beyond the bounds with others' partners.
\stepcounter{stanza}
\end{MyDescription}

\begin{MyDescription}[\arabic{stanza}]{}
In government assembly, or artisans' guild\\
or one to another, speak not what is false,\\
not others compel, nor approve of their lies,\\
all kinds of untruthfulness you should avoid.
\stepcounter{stanza}
\end{MyDescription}

\begin{MyDescription}[\arabic{stanza}]{}
Whatever householder this Dharma approves\\
in maddening drink should never indulge,\\
nor make others drink, nor approve if they do\\
knowing, it leads to a mind that's disturbed.
\stepcounter{stanza}
\end{MyDescription}

\begin{MyDescription}[\arabic{stanza}]{}
Fools do many evils because they are drunk\\
while causing other people to be negligent.\\
This basis of demerit should be avoided\\
but fools are delighted, confused with mind upset.
\stepcounter{stanza}
\end{MyDescription}

\begin{MyDescription}[\arabic{stanza}]{}
Kill not any being, what's not given do not take,\\
neither be a liar nor addicted to drink,\\
and, let go of sex and the non-celibate life,\\
in the `wrong-time' for food, eat not in the night.
\stepcounter{stanza}
\end{MyDescription}

\begin{MyDescription}[\arabic{stanza}]{}
Neither necklaces display nor perfumes employ,\\
use the ground as a bed or sleep upon a mat:\\
these are the uposatha eight-factored vows\\
made known by the Buddha gone to dukkha's end.
\stepcounter{stanza}
\end{MyDescription}

\begin{MyDescription}[\arabic{stanza}]{}
with devotion at heart the uposathas kept\\
completely perfected in its eight parts,\\
on the fourteenth, the fifteenth and the eight days,\\
as well the days special in the moon's half months.
\stepcounter{stanza}
\end{MyDescription}

\begin{MyDescription}[\arabic{stanza}]{}
Let that one intelligent with devoted heart
having kept uposatha, early next morning,
distribute food and drink - whatever's suitable -
to the bikkhu-sangha, rejoicing in this act.
\stepcounter{stanza}
\end{MyDescription}


\begin{MyDescription}[\arabic{stanza}]{}
Support mother and father according to Dharma,\\
do business as merchant to honesty adhering,\\
diligently practising this householder's rule--\\
then to the self-radiant devas one will arrive.
\stepcounter{stanza}
\end{MyDescription}


\begin{MyDescription}[(Sn.376-404)]{}
\end{MyDescription}



   
\part{The Great Chapter\\ Mahavagga}

\chapter{Pabbajj\=a Sutta\\ The Leaving Home of Gotama}

\begin{MyDescription}[]{Narrator:}
Now I'll tell of the Leaving Home,\\
how he, the mighty seer, went forth,\\
how he was questioned and described\\
the reason for his Leaving Home.\\
The crowded life lives in a house\\
exhales an atmosphere of dust:\\
but leaving home is open wide —\\
this he saw; chose Leaving Home.
\stepcounter{stanza}
\end{MyDescription}

\begin{MyDescription}[]{}
By doing so did he reject\\
all bodily evil acts,\\
rejected too, wrong sons of speech,\\
his livelihood he purified.
\stepcounter{stanza}
\end{MyDescription}

\begin{MyDescription}[]{}
He went to R\=ajagaha town,\\
hill-guarded fort of Magadhans;\\
there he, the Buddha, walked for alms,\\
with many a mark of excellence.
King Bimbis\=ara from within\\
his palace saw him passing by,\\
and when he saw such excellence\\
in all his marks, `Look, sins', he said,\\
How stately is that man, handsome,\\
how pure, how perfect is his gait;\\
with eyes downcast, he mindful, looks\\
only a plough—yoke's length ahead.\\
He's surely not of humble birth!\\
Send forth royal messengers at once\\
upon the path the bhikkhu takes.'
\stepcounter{stanza}
\end{MyDescription}

\begin{MyDescription}[]{}
The messengers were sent at once\\
and followed closely in his wake:\\
Now which way will the bhikkhu go?\\
Where has he chosen his abode?\\
\stepcounter{stanza}
He wanders on from house to house\\
guarding sense-doors with real restraint.\\
Fully aware and mindfully,\\
his alms bowl soon was full.\\
\stepcounter{stanza}
His almsround is now done. The Sage\\
is setting out and leaving town,\\
taking the road to Pa\=o\=oava —\\
he must be living on its hill.'
\stepcounter{stanza}
\end{MyDescription}

\begin{MyDescription}[]{}
Now when he came to his abode\\
the messengers went up to him;\\
though one of them turned back again\\
to give the King reply:\\
\stepcounter{stanza}
The bhikkhu, sire, is like a lion,\\
or like a tier, like a bull\\
and seated in a mountain-cave\\
on the eastern slope of Pa\=o\=oava!
\stepcounter{stanza}
\end{MyDescription}

\begin{MyDescription}[]{}
The Warrior hears the runner's tale,\\
then summoning up a coach of state,\\
he drove in haste from out the town,\\
out to the hill of Pa\=o\=oava.\\
\stepcounter{stanza}
He drove as far as he could go\\
and then descended from the coach;\\
the little distance that remained\\
he went on foot, drew near the Sage
\stepcounter{stanza}
\end{MyDescription}

\begin{MyDescription}[\arabic{stanza}]{}
The King sat down, and he exchanged\\
greetings, and asked about his health.\\
When this exchange of courtesy\\
was done, the king then spoke to him\\
\stepcounter{stanza}
these words: `You are indeed quite young,\\
a youth, a man in life's first phase,\\
you have the good looks of a man\\
of high-born warrior-noble stock,\\
one fit to grace a first-rate force,\\
to lead the troops of elephants,\\
wealth can I give you to enjoy;\\
please tell me of your birth.
\stepcounter{stanza}
\end{MyDescription}

\begin{MyDescription}[]{The Buddha:}
Straight over there, O king,\\
the Himalayas can be seen,\\
there, with wealth and energy,\\
living among the Kosalans\\
are the Adicca of solar race,\\
in that, the claim of Sakyas.\\
From that family I've left home\\
not desiring pleasures of sense.\\
Having seen dangers in sense—desires,\\
renunciation seen as secure,\\
I shall go on to strive\\
for there does my mind delight
\stepcounter{stanza}
\end{MyDescription}

\begin{MyDescription}[(Sn.405-424)]{}

\end{MyDescription}
\stepcounter{stanza}
\stepcounter{stanza}
\stepcounter{stanza}
\stepcounter{stanza}
\stepcounter{stanza}
\stepcounter{stanza}
\stepcounter{stanza}
\stepcounter{stanza}


\chapter{Padh\=ana Sutta - The Striving of Gotama}

\begin{MyDescription}[]{The Bodhisattva:}
As I strove to subdue myself\\
beside the broad Neranjar\=a\\
absorbed unflinchingly to gain\\
the tme surcease of bondage here,\\
Namuc\=a came and spoke to me\\
with words all garbed in pity thus:
\end{MyDescription}

\begin{MyDescription}[]{M\=ara:}
`O you are thin and you are pale\\
and you are in death's presence too:\\
a thousand parts are pledged to death\\
but life still holds one part of you.\\
Live, sir! Life's the better way;\\
you may gain merit if you live,\\
come live the life of purity, pour\\
libations on the holy fires\\
and thus a world of merit gain.\\
What can you do by struggling now?\\
The path of struggling too is rough\\
and difficult and hard to bear.'\\
Now M\=ara, as he spoke these line\\
drew near until he stood close by.
\end{MyDescription}

\begin{MyDescription}[]{The Bodhisattva:}
The Blessed One replied to him\\
as he stood thus: `O Evil One,\\
O Cousin of the Negligent,\\
you have come here for your own ends.\\
Now, merit I need not at all.\\
Let Mara talk of merit then\\
to those that stand in need of it.\\
For I have faith and energy,\\
and I have understanding, too.\\
So while I thus subdue myself\\
why do you speak to me of life?\\
There is this wind that blows, can dry\\
even the rivers' running streams;\\
so while I thus subdue myself,\\
why should it not dry up my blood?\\
And, as the blood dries up, then bile\\
and phlegm run dry, the wasting flesh\\
becalms the mind: I shall have more\\
of mindfulness and wisdom too,\\
I shall have greater concentration.\\
For living thus I come to know\\
the limits to which feeling goes.\\
My mind looks not to sense-desires\\
then sees a being's purity.\\
Your squadron's first is sense-desires\\
your second's sexual discontent,\\
Hunger and Thirst compose the third,\\
and Craving is the fourth in rank,\\
the fifth is Sloth and Accidy,\\
while fear is called the sixth in line,\\
Sceptical doubt is seventh, the eighth\\
is Sliminess, Hardheartedness;\\
Gain with Honour, Praise besides,\\
and ill-won Notoriety,\\
Self-praise and Denigrating others —\\
These are your squadrons, Namuc\=a,\\
the Black One's fighting troops.\\
None but the brave will conquer them\\
to gain bliss by the victory.\\
As though I'm weaving muja-grass\\
proclaiming no retreat: shame upon life\\
defeated here — better to die in battle now\\
than choose to lie on in defeat.\\
Ascetics and brahmins there are found\\
that have surrendered here, and they\\
are seen no more: they do not know\\
the paths the pilgrim travels by.\\
So, seeing Mara's squadrons now\\
arrayed all round, with elephants,\\
I sally forth to fight, that I\\
may not be driven from my post.\\
Your serried squadrons, which the world\\
with all its gods cannot defeat,\\
Now I'll break with wisdom sharp\\
as with a stone a raw clay pot.\\
With all mind's thoughts within the range,\\
with well-established mindfulness,\\
I'll travel on from state to state\\
many disciples leading out.\\
They, both diligent and resolute\\
carry on my S\=asana\\
and though you like it not, they'll go\\
to where they do not grieve.
\end{MyDescription}


\begin{MyDescription}[]{M\=ara:}
Though step by step for seven years\\
I've followed on the Blessed One,\\
the Fully Enlightened One, possessed\\
of mindfulness, he gave to me no chance.\\
A crow there was who walked around\\
a stone that seemed a lump of fat;\\
``Shall I find something soft in this?
And is there something tasty here?"\\
He finding nothing tasty there,\\
made off: and we from Gotama\\
depart in disappointment, too,\\
like to the crow that tried the stone.\\
Then full of sorrow he let slip\\
the lute from underneath his arm,\\
then that dejected demon\\
disappeared just there.
\end{MyDescription}

\begin{MyDescription}[(Sn. 425-449)]{}

\end{MyDescription}
\stepcounter{stanza}
\stepcounter{stanza}
\stepcounter{stanza}
\stepcounter{stanza}
\stepcounter{stanza}
\stepcounter{stanza}
\stepcounter{stanza}
\stepcounter{stanza}
\stepcounter{stanza}
\stepcounter{stanza}
\stepcounter{stanza}
\stepcounter{stanza}
\stepcounter{stanza}
\stepcounter{stanza}
\stepcounter{stanza}
\stepcounter{stanza}
\stepcounter{stanza}
\stepcounter{stanza}
\stepcounter{stanza}
\stepcounter{stanza}
\stepcounter{stanza}
\stepcounter{stanza}
\stepcounter{stanza}
\stepcounter{stanza}
\chapter{Subh\=asita Sutta The Well-spoken}

Thus have l heard:
At one time the Radiant One dwelt at Savatthi, in the ]eta Grove, Anathapindika's Park. The Radiant One spoke thus: `Bhikkhus'.\\
`Venerable Sir', those bhikkhus replied.\\
`Speech having four qualities is well-spoken, not ill-spoken; what is Dharma, not what is not-Dharma; what is kindly, not what is unkind; what is the truth, not what is false. This speech is well-spoken, not ill-spoken and blameless not blameworthy among the wise.'\\

This is what the Radiant One said, then he spoke further.

\begin{MyDescription}[\arabic{stanza}]{}
Now peaceful Ones say: first speak the well-spoken\\
and second, speak Dharma but not its opposite,\\
what's kind do speak, third, not the unkind,\\
while fourth, speak the truth but never the false.
\stepcounter{stanza}
\end{MyDescription}
 
Then the venerable Vangisa rose with robe over one shoulder and lotussed hands towards the Radiant One saying to him: `Sir, it has come to me!'\\

`Let it come to you, Vangisa.'

The venerable Vangisa then praised the Radiant One in his presence with these appropriate verses:

\begin{MyDescription}[\arabic{stanza}]{}
Only that speech should be spoken\\
from which harm does not come to oneself\\
nor torment brings upon others,\\
this truly is speech that's well-spoken.
\stepcounter{stanza}
\end{MyDescription}

\begin{MyDescription}[\arabic{stanza}]{}
Speak only those words that are kind,\\
the speech that is gladly received,\\
so whatever one speaks to the others\\
conveying no evil, is kind.
\stepcounter{stanza}
\end{MyDescription}

\begin{MyDescription}[\arabic{stanza}]{}
Truth indeed, is deathless speech —\\
this is the ancient Dharma.\\
On truth, its study and practice both\\
they say are the Peaceful firm
\stepcounter{stanza}
\end{MyDescription}

\begin{MyDescription}[\arabic{stanza}]{}
Whatever words the Buddha speaks\\
Nirvana's safety to attain,\\
bringing dukkha to an end,\\
such words they are the worthiest.
\stepcounter{stanza}
\end{MyDescription}
\newpage
\section{Notes}
This small Sutta is important for Dharma-practice. It is easier to make unwholesome karma by way of the mouth than it is through bodily action — words just slip off the tongue so easily. Think how many words one speaks every day! So the Buddha here defines what is subh\=asita, well-spoken. He does this twice, first in prose and then in a summary verse, a mnemonic aid in a world where teachings were not recorded even by writing.\\

To ensure that this teaching stuck in the mind, Venerable Vang\=asa, famous for his ability to speak inspired and spontaneous verse, then with the Buddha's approval, elaborates upon these four types of good speech. This occasion does not bring forth his best verses.\\

This sutta also occurs in S. (Vangisa-sa\.myutta 8,5) though there it has no occasion nor the Buddha s prose and verse.\\

Vs. 453 mentions `the ancient Dharma'. The word translated here as `ancient' is `sanantano' which PTSD defines as `primeval, of old, for ever, eternal'. When used as an adjective with `dharma' the meaning is that this Dharma is true: ``Whether Tath\=agatas (or Buddhas) arise or Tath\=agatas do not arise, there is this state of causality that always exists, this established order of dharmas, this natural lawfulness of Dharma, that is to say: All conditioned things are impermanent...all conditioned things are dukkha...all dharmas are not-self..." This is the ancient Dharma true of all worlds, of all beings, at all times, whether or not it is known to these beings. For the above quotation see A. Threes 134.

The Hindu understanding of the Skt. `sanatana-dharma' emphasizes time — that this Dharma is eternal and includes among other matters the society ordered into castes. This view is defended in the Bhagavad-g\=ata. [Comment in text: “a reference here!”]

\chapter{Sundar\=akabharadv\=ada Sutta\\ To Sundhar\=akabharadv\=aja on Offerings}

Thus have I heard:\\

At one time the Radiant One was dwelling among the Kosalans. At that time the brahmin Sundar\=aka-Bharadv\=aja performed on the bank of the Sundar\=aka river the fire sacrifice and offered the fire-ritual. Having completed the sacrifice and ritual the brahmin rose from his seat and surveyed the four directions, thinking, `Who will partake of the remains of this sacrifice?' It happened then that the brahmin saw the Radiant One seated at the foot of a tree not far away but with his head covered. Seeing him, the brahmin grasped the sacrificial remains in his left hand and a water-vessel in his right and approached the Radiant One who, hearing his approach, uncovered his head. Then the brahmin thought:\\
`This venerable one is shaven-headed, a mere shaveling' and desired to turn back. But it occurred to him: `Though shaven~headed there are some brahmins here like this. It would be good to inquire about his `birth'. Then Sund\=araka-Bharadv\=aja the brahmin approached the Radiant One and having done so, said this: `Of what `birth' is the venerable one?' Then the Radiant One addressed these verses to the brahmin: 

\begin{MyDescription}[\arabic{stanza}]{The Buddha:}
No brahmin am I, nor son of royalty,\\
nor of merchant stock, nor any other (caste),\\
for l know very well ordinary people's line\\
so wisely, having nothing, I fare through the world.
\stepcounter{stanza}
\end{MyDescription}

\begin{MyDescription}[\arabic{stanza}]{}
No brahmin am I, nor son of royalty,\\
nor of merchant stock, nor any other (caste),\\
for I know very well ordinary people's line\\
so wisely, having nothing, I fare through the world.
\stepcounter{stanza}
\end{MyDescription}


\begin{MyDescription}[]{Sundar\=aka:}
But brahmins, sir, of brahmins always ask\\
`Are you as well Va brahmin, friend?'
\end{MyDescription}

\begin{MyDescription}[\arabic{stanza}]{The Buddha:}
lf you say you brahmin are but call me none\\
then of you I ask the chant of S\=avitr\=a\\
consisting of three lines\\
in four and twenty syllables.
\stepcounter{stanza}
\end{MyDescription}

\begin{MyDescription}[\arabic{stanza}]{Sundar\=aka:}
On what do they rely, these seers,\\
born human, the nobles and brahmins, all of them,\\
that to the devas they sacrifices make\\
to bring about results here in this world?
\stepcounter{stanza}
\end{MyDescription}   

\begin{MyDescription}[\arabic{stanza}]{The Buddha:}
One gone to the End, one who's gone to Knowledge,\\
at the time of sacrifice receives that offering\\
and that will be a blessing, I say.
\stepcounter{stanza}
\end{MyDescription}

\begin{MyDescription}[\arabic{stanza}]{Sundar\=aka:}
Then for sure will be fruitful this my sacrifice\\
because we have seen one such as yourself —\\
one gone to Knowledge, for if seeing not\\
another would have eaten the sacrificial cake.
\stepcounter{stanza}
\end{MyDescription}  
  


\begin{MyDescription}[]{Sundar\=aka:}
I do delight in an desire to sacrifice, O Gotama,\\
but I do not know how, instruct me please, sir,\\
and how a sacrifice succeeds, do tell me of that?
\end{MyDescription} 

\begin{MyDescription}[\arabic{stanza}]{The Buddha:}
If that is so, O brahmin, lend your ears\\
and in the Dharma I shall instruct you.
\stepcounter{stanza}
\end{MyDescription}

\begin{MyDescription}[\arabic{stanza}]{}
Ask not of `birth' but of behaviour enquire —\\
truly from sticks of wood the sacred fire is born\\
so though of lowly line a sage becomes a thoroughbred\\
one both resolute, and restrained by self-respect,
\stepcounter{stanza}
\end{MyDescription}

\begin{MyDescription}[\arabic{stanza}]{}
tamed by Truth, endowed with self-restraint,\\
one gone to Knowledge's end and the Good Life living:\\
a timely offering one should give to such —\\
a brahmin seeking merit to such a one should sacrifice,
\stepcounter{stanza}
\end{MyDescription}

\begin{MyDescription}[\arabic{stanza}]{}
let go of sensuality, and homeless faring — those\\
with minds well-restrained and as a shuttle straight\\
a timely offering one should give to such —\\
a brahmin seeking merit to such a one should sacrifice,
\stepcounter{stanza}
\end{MyDescription}

\begin{MyDescription}[\arabic{stanza}]{}
free from lustfulness, sense-faculties controlled,\\
as the moon freed from old Rahn's grasp:\\
a timely offering one should give to such —\\
a brahmin seeking merit to such a one should sacrifice,
\stepcounter{stanza}
\end{MyDescription}

\begin{MyDescription}[\arabic{stanza}]{}
they who wander the world completely unattached\\
ever-mindful of mine-making, always letting go:\\
a timely offering one should give to such —\\
a brahmin seeking merit to such a one should sacrifice.
\stepcounter{stanza}
\end{MyDescription}

\begin{MyDescription}[\arabic{stanza}]{}
Whoever fares victorious, let go of sensuality,\\
who is a Knower of the end of birth and death,\\
become quite Cool as a cool-water lake;\\
such Tathagata's worthy of sacrificial cake.
\stepcounter{stanza}
\end{MyDescription}

\begin{MyDescription}[\arabic{stanza}]{}
Who's equal with equals, unequals far away,\\
a Tath\=agata — of wisdom infinite,\\
one who is unsmeared either here or hereafter:\\
such Tath\=agata's worthy of sacrificial cake.
\stepcounter{stanza}
\end{MyDescription}

\begin{MyDescription}[\arabic{stanza}]{}
In whom does not dwell deceit or conceit,\\
who's greed-free, unselfish, having no desire,\\
who anger has lost, exceeding Cool of self,\\
that Brahmin who's removed impurity of grief:
such a Tath\=agata's worthy of sacrificial cake.
\stepcounter{stanza}
\end{MyDescription}

\begin{MyDescription}[\arabic{stanza}]{}
Whoever has removed the dwellings of the mind,\\
in whom there exists no clinging any more,\\
no grasping at anything here or hereafter:\\
such Tath\=agata's worthy of sacrificial cake.
\stepcounter{stanza}
\end{MyDescription}

\begin{MyDescription}[\arabic{stanza}]{}
With a mind composed and crossed the flood,\\
a Knower of Dharma by the highest vision,\\
cleansed of pollutions, bearer of a last body:\\
such Tath\=agata's worthy of sacrificial cake.
\stepcounter{stanza}
\end{MyDescription}

\begin{MyDescription}[\arabic{stanza}]{}
No pollutions for existence, neither harsh words,\\
not smouldering are they, to non-existence gone,\\
one gone to Knowledge, completely released:\\
such Tath\=agata's worthy of sacrificial cake.
\stepcounter{stanza}
\end{MyDescription}

\begin{MyDescription}[\arabic{stanza}]{}
One gone beyond ties, no ties still exist,\\
among conceited men, one of no conceit,\\
comprehending dukkha with its range and base:\\
such Tath\=agata's worthy of sacrificial cake.
\stepcounter{stanza}
\end{MyDescription}

\begin{MyDescription}[\arabic{stanza}]{}
A seer of solitude and not depending on desire,\\
escaped from the views by other people known,\\
in whom are no conditions found at all:\\
such Tath\=agata's worthy of sacrificial cake.
\stepcounter{stanza}
\end{MyDescription}

\begin{MyDescription}[\arabic{stanza}]{}
Who's understood completely the dharmas high and low,\\
not smouldering are they, to non-existence gone,\\
by clinging's exhaustion freed and so at peace:\\
such Tath\=agata's worthy of sacrificial cake.
\stepcounter{stanza}
\end{MyDescription}

\begin{MyDescription}[\arabic{stanza}]{}
Who's a seer of exhaustion of birth and fetters all\\
and who has dispelled the sensual trail complete —\\
purified, faultless, untainted and flawless:\\
such Tath\=agata's worthy of sacrificial cake.
\stepcounter{stanza}
\end{MyDescription}

\begin{MyDescription}[\arabic{stanza}]{}
One not seeing self by means of self within,\\
firm and straightforward as well contemplative,\\
free from lust, harsh-heartedness and from all doubts:\\
such Tath\=agata's worthy of sacrificial cake.
\stepcounter{stanza}
\end{MyDescription}

\begin{MyDescription}[\arabic{stanza}]{}
In whom no conditions for delusion can be found,\\
a seer with wisdom among all the dharmas,\\
one who's the bearer of the ultimate body,\\
attained to the blissful unexcelled Awakening\\
to this extent there s purity among the powerful:\\
such Tath\=agata's worthy of sacrificial cake.
\stepcounter{stanza}
\end{MyDescription}

\begin{MyDescription}[\arabic{stanza}]{Sundar\=aka:}
In the past I sacrificed, now let my sacrifice be true\\
for now I have met such a one of wisdom's qualities\\
you're Brahma manifest indeed, accept from me O Radiant,\\
may the Radiant One eat my sacrificial cake.
\stepcounter{stanza}
\end{MyDescription}

\begin{MyDescription}[\arabic{stanza}]{The Buddha: }
Chanting sacred verses for comestibles — not done by me,\\
for those who rightly see, brahmin, it accords not with Dharma.\\
Chanting sacred verses thus is rejected by the Buddhas,\\
such is the Dharma, brahmin, such is their practice.
\stepcounter{stanza}
\end{MyDescription}

\begin{MyDescription}[\arabic{stanza}]{}
A Great Seer with Final Knowledge, conflicts stilled,\\
one who has exhausted taints, is wholly free —\\
make offerings of food and drink to such a one:\\
the certain field for one who merit seeks.
\stepcounter{stanza}
\end{MyDescription}

\begin{MyDescription}[\arabic{stanza}]{Sundaraka:}
Good indeed, sir, that I should know of this.\\
But having gained your teachings (now I ask):\\
Who should eat the gift of such as I\\
whom I'm seeking at this time of sacrifice?
\stepcounter{stanza}
\end{MyDescription}

\begin{MyDescription}[\arabic{stanza}]{The Buddha:}
Whose anger's disappeared,\\
who has unclouded mind,\\
who's free from lustfulness,\\
whose sloth is thrust aside,
\stepcounter{stanza}
\end{MyDescription}

\begin{MyDescription}[\arabic{stanza}]{The Buddha:}
Guide for what's beyond the bounds,\\
Knower of birth—and-death,\\
Sage with sagely virtues,\\
arrived at the sacrifice,
\stepcounter{stanza}
\end{MyDescription}

\begin{MyDescription}[\arabic{stanza}]{}
With super pride removed\\
revere with lotussed hands\\
honour with food and drink\\
thus prosper rightful gifts.
\stepcounter{stanza}
\end{MyDescription}

\begin{MyDescription}[\arabic{stanza}]{Sundaraka:}
The Buddha, sir, is worthy of sacrificial cake,\\
a field for merits,\\
recipient of all the world,\\
what's given to you bears great fruit.
\stepcounter{stanza}
\end{MyDescription}

When this was said, the brahmin Sundar\=aka-Bharadv\=aja said to the Radiant
   One: `Magnificent, master Gotama! The Dharma has been clarified by Master
   Gotama in many ways, as though he was righting what was overthrown, revealing
   what was hidden, showing the way to one who was lost, or holding a lamp in the
   dark so that those with eyes can see forms. I go for Refuge to Master Gotama, to
   the Dharma, and to the Bhikkhu-sangha that I may receive the Leaving home
   from the venerable Gotama with ordination.' Then the brahmin Sundar\=aka-
   Bharadv\=aja received this. Not long after his ordination the venerable, living in
   solitude, secluded, diligent and zealous by realizing from himself with Direct
   Knowledge here and now entered upon and abided in that supreme goal of the
   Good Life for the sake of which clansmen rightly leave home for homelessness.
   He Knew directly: birth is destroyed, the Good Life has been lived, what had to be done has been done, there is no more coming to any state of being. And the venerable Sundar\=aka-Bharadv\=aja became one of the arahants.

\begin{MyDescription}[(Sn. 455-486)]{}

\end{MyDescription}   


\chapter{M\=agha Sutta\\ To M\=agha an Giving}
Thus have I heard:\\

At one time the Radiant one dwelt at R\=ajagaha on the Vulture Peak Mountain. Then the young brahmin M\=agha went to the Radiant One and exchanged greetings with him. When this courteous and amiable talk was finished, he sat down to one side and spoke thus to the Radiant One: `Master Gotama, I am certainly a donor, one who is generous and glad to comply with others' requests. From wealth sought rightly, obtained rightly, acquired in  accordance with Dharma, I give to one, two, ten, twenty, a hundred or even more — so do I, Master Gotama giving and bestowing in this way accrue much merit?'\\

`Certainly young brahmin, giving and bestowing in this way you accrue much merit. lf anyone is a donor, one who is generous and glad to comply with others' requests from wealth sought rightly, obtained rightly, acquired in accordance with dharma and given to one, two, ten, twenty, a hundred or even to more, that one accrues much merit.'\\

Then the brahmin youth M\=agha addressed these verses to the Radiant One:

\begin{MyDescription}[\arabic{stanza}]{M\=agha:}
I ask the word-knower Gotama\\
who wanders homeless clad in k\=esaya cloth:\\
One glad to comply with others' requests,\\
a generous giver, one living at home,\\
a seeker of merit, desirer of merit,\\
who to other as sacrifice gives food and drink\\
how would such offerings be purified by this?
\stepcounter{stanza}
\end{MyDescription}

\begin{MyDescription}[\arabic{stanza}]{The Buddha:}
One glad to comply with others' requests,\\
a generous giver, one living at home,\\
a seeker of merit, desirer of merit,\\
who to others as sacrifice gives good and drink\\
achieves his results through those worthy of gifts.
\stepcounter{stanza}
\end{MyDescription}

\begin{MyDescription}[\arabic{stanza}]{M\=agha:}
One glad to comply with others' requests,\\
a generous giver, one living at home,\\
a seeker of merit, desirer of merit,\\
who to others as sacrifice gives good and drink —\\
Sir, who are the gift-worthy, please speak about that.
\stepcounter{stanza}
\end{MyDescription}

\begin{MyDescription}[\arabic{stanza}]{The Buddha:}
Those truly who fare unattached in the world,\\
own nothing, perfected, they're self-controlled,\\
to them would a brahmin, on merit intent,\\
sacrifice at the right time and oblations bestow.
\stepcounter{stanza}
\end{MyDescription}

\begin{MyDescription}[\arabic{stanza}]{}
Who all the fetters and bonds have cut off,\\
tamed are they, freed, with no troubles or hopes,\\
to them would a brahmin, on merit intent,\\
sacrifice at the right time and oblations bestow.
\stepcounter{stanza}
\end{MyDescription}

\begin{MyDescription}[\arabic{stanza}]{}
Who from all fetters are released\\
tamed and freed, with no troubles or hopes,\\
to them would a brahmin, on merit intent,\\
sacrifice at the right time and oblations bestow.
\stepcounter{stanza}
\end{MyDescription}

\begin{MyDescription}[\arabic{stanza}]{}
Passion and hatred, delusion — let go,\\
exhausted the inflows, lived the God Life,\\
to them would a brahmin, on merit intent,\\
sacrifice at the right time and oblations bestow.
\stepcounter{stanza}
\end{MyDescription}

\begin{MyDescription}[\arabic{stanza}]{}
In who lurks neither deceit nor conceit,\\
greed-free, unselfish, trouble-free too,\\
to them would a brahmin, on merit intent,\\
sacrifice at the right time and oblations bestow.
\stepcounter{stanza}
\end{MyDescription}

\begin{MyDescription}[\arabic{stanza}]{}
They who to cravings have not succumbed,\\
the flood overcrossed they unselfishly fare,\\
to them would a brahmin on merit intent,\\
sacrifice at the right time and oblations bestow.

\stepcounter{stanza}
\end{MyDescription}

\begin{MyDescription}[\arabic{stanza}]{}
But those with no cravings at all in the world\\
for being this, being that, now or afterwards,\\
to them would a brahmin, on merit intent,\\
sacrifice at the right time and oblations bestow.
\stepcounter{stanza}
\end{MyDescription}

\begin{MyDescription}[\arabic{stanza}]{}
They who fare homeless, sense—pleasures let go,\\
themselves well-restrained, as shuttle flies straight,\\
to them would a brahmin, on merit intent,\\
sacrifice at the right time and oblations bestow.
\stepcounter{stanza}
\end{MyDescription}

\begin{MyDescription}[\arabic{stanza}]{}
Those passion-free, their faculties restrained,\\
as the Moon from the grip of R\=ahu released,\\
to them would a brahmin, on merit intent,\\
sacrifice at the right time and oblations bestow.
\stepcounter{stanza}
\end{MyDescription}

\begin{MyDescription}[\arabic{stanza}]{}
Those who are calm, passion gone, anger-free,\\
who here have given up all places to go\\
to them would a brahmin on merit intent,\\
sacrifice at the right time and oblations bestow.
\stepcounter{stanza}
\end{MyDescription}

\begin{MyDescription}[\arabic{stanza}]{}
Who've birth and death abandoned —— nothing left\\
and all unsettling doubts have overcome,\\
to them would a brahmin, on merit intent,\\
sacrifice at the right time and oblations bestow.
\stepcounter{stanza}
\end{MyDescription}

\begin{MyDescription}[\arabic{stanza}]{}
With themselves as an island they are in the world,\\
own nothing and everywhere utterly freed\\
to them would a brahmin, on merit intent,\\
sacrifice at the right time and oblations bestow.
\stepcounter{stanza}
\end{MyDescription}

\begin{MyDescription}[\arabic{stanza}]{}
Those who Know here as really it is —\\
`this is the last, no more being to come'\\
to them would a brahmin, on merit intent,\\
sacrifice at the right time and oblations bestow.
\stepcounter{stanza}
\end{MyDescription}

\begin{MyDescription}[\arabic{stanza}]{}
The mindful in holy words learned, who in jh\=ana delight,\\
won to Awakening, the refuge of many,\\
to them would a brahmin, on merit intent,\\
sacrifice at the right time and oblations bestow.
\stepcounter{stanza}
\end{MyDescription}

\begin{MyDescription}[\arabic{stanza}]{M\=agha}
My question truly was not in vain\\
for the Radiant has spoken of gift-worthy ones.\\
This indeed you Know as it really is\\
for certainly this Dharma's Known to you.
\stepcounter{stanza}
\end{MyDescription}

\begin{MyDescription}[\arabic{stanza}]{}
One glad to comply with others' requests,\\
a generous giver, one living at home,\\
a seeker of merit, desirer of merit,\\
who to others as sacrifice gives food and drink,\\
tell me, O Radiant, the success of such sacrifice.
\stepcounter{stanza}
\end{MyDescription}

\begin{MyDescription}[\arabic{stanza}]{The Buddha:}
Do you sacrifice! But during this sacrificial act\\
make your mind happy all of the time\\
for the sacrificiant, this sacrifice is the base,\\
established in this one is rid of all faults.
\stepcounter{stanza}
\end{MyDescription}

\begin{MyDescription}[\arabic{stanza}]{}
One with passions gone would other faults restrain\\
developing boundless metta-mind, in this\\
continuously diligent by day and by night\\
suffusing all directions boundlessly.
\stepcounter{stanza}
\end{MyDescription}

\begin{MyDescription}[\arabic{stanza}]{M\=agha:}
Who can be cleansed, released and Awakened?\\
With what does the self to the Brahma-world go?\\
O Sage, when asked reply to me — one who doesn't know —\\
for the Radiant I've seen with my eyes as Brahma today\\
and its true that you're the same as Brahma for us.\\
ln the Brahma-world how does one arise, O Refulgent One?
\stepcounter{stanza}
\end{MyDescription}

\begin{MyDescription}[\arabic{stanza}]{The Buddha:}
The sacrificiant achieves triple success in sacrifice,
achieves their results through such gift-worthy ones,
so perfected in sacrifice and complying with others' requests
that one arises, I say, with in the Brahma-world.
\stepcounter{stanza}
\end{MyDescription}


When this was said the young brahmin Magha said to the Radiant One: `Magnificent, Master Gotama! The Dharma has been clarified by Master Gotama in many ways, as though he was righting what was overthrown, revealing what was hidden, showing the way to one who was lost, or holding a lamp in the dark so that those with eyes can see forms. I go for Refuge to Master Gotama, to the Dharma and to the Sangha. May Master Gotama remember me as a layman who from today has Gone for Refuge for life.

\begin{MyDescription}[(Sn. 487-509)]{}
\end{MyDescription}
   
\chapter{Sabhiya Sutta -\\ Sabbhiya's Questions}

Thus have I heard:\\

At one time the Radiant One dwelt at R\=ajagaha in the Bamboo Grove, the   Squirrels' Feeding-ground.\\
\lbrack UNFINISHED\rbrack
\stepcounter{stanza}
\stepcounter{stanza}
\stepcounter{stanza}
\stepcounter{stanza}
\stepcounter{stanza}
\stepcounter{stanza}
\stepcounter{stanza}
\stepcounter{stanza}
\stepcounter{stanza}
\stepcounter{stanza}
\stepcounter{stanza}
\stepcounter{stanza}
\stepcounter{stanza}
\stepcounter{stanza}
\stepcounter{stanza}
\stepcounter{stanza}
\stepcounter{stanza}
\stepcounter{stanza}
\stepcounter{stanza}
\stepcounter{stanza}
\stepcounter{stanza}
\stepcounter{stanza}
\stepcounter{stanza}
\stepcounter{stanza}
\stepcounter{stanza}
\stepcounter{stanza}
\stepcounter{stanza}
\stepcounter{stanza}
\stepcounter{stanza}
\stepcounter{stanza}
\stepcounter{stanza}
\stepcounter{stanza}
\stepcounter{stanza}
\stepcounter{stanza}
\stepcounter{stanza}
\chapter{Sela Sutta -\\ To Sela and his praise of the Buddha}


Thus have I heard:\\
At one time the Radiant One was journeying through the lands of the Anguttarapans, accompanied by a large sangha of bhikkhus, twelve hundred and fifty of them and arrived at a town called \=apana.\\

The ascetic Keniya of the Dreadlocked Hair heard this: `The Samana Gotama, son of the Sakiyans who left home among the Sakiyan people has been journeying among the Anguttarapans accompanied by a large sangha of bhikkhus, twelve hundred and fifty of them and he has arrived in apana. Now an excellent report has spread to this effect: The Radiant One is Arahant, fully Awakened, perfect in true knowledge and conduct, whose going is auspicious, knower of worlds, incomparable leader of people to be famed, Teacher of devas and humanity, Awakened and Radiant. He declares this world with its devas, maras, princes and people which he has realized by direct knowledge himself. He teaches Dharma good at the beginning, the middle and the end, complete with purpose and meaning, revealing the good life, that which is completely fulfilled and wholly purified. Now, it is good to see such Arahants.\\

 Then Keniya of the Dreadlocked Hair went to see the Radiant One and exchanged greetings with him and when that courteous and amiable talk was finished he sat down to one side. The Radiant One instructed, urged, roused and encouraged him with Dharma-talk after which Keniya said, `Let Master Gotama together with the Sangha of bhikkhus consent to accept tomorrow's meal from me.'\\
 
 The Radiant One replied to him, `The Sangha of bhikkhus Keniya, is large, of twelve hundred and fifty. Do remember that [?] you believe also in the brahmins.'\\
 
 This request was repeated a second and a third time by Keniya and only on the third occasion did the Buddha accept by remaining silent.\\
 
   [ ............. ..]\\
   
\begin{MyDescription}[\arabic{stanza}]{}
O you of perfect form and beauty rare,\\
proportioned well and lovely to behold,\\
in colour like fine gold, with shining teeth,\\
You, the Exalted, Energetic One,
\stepcounter{stanza}
\end{MyDescription} 
\stepcounter{stanza}
\stepcounter{stanza}
\stepcounter{stanza}
\begin{MyDescription}[\arabic{stanza}]{}
Whose body shows forth all the minor marks\\
distinguishing a well-proportioned man\\
while all upon your body can be seen\\
the signs peculiar to the Superman.
\stepcounter{stanza}
\end{MyDescription} 

\begin{MyDescription}[\arabic{stanza}]{}
You with eyes so clear, so fair your countenance\\
and you so tall, so straight, majestical\\
amidst the order of the samanas\\
do blazon forth as does the sun on high.
\stepcounter{stanza}
\end{MyDescription} 

\begin{MyDescription}[\arabic{stanza}]{}
O you a bhikkhu good to gaze upon\\
having a skin resembling finest gold,\\
what is this life of samanas to you\\
having a presence so supremely fair?
\stepcounter{stanza}
\end{MyDescription} 

\begin{MyDescription}[\arabic{stanza}]{}
You deserve to be a King who turns the wheel\\
riding in state a chariot of war,\\
lord of the earth from end to end four square,\\
a Conqueror of jambudipa chief.
\stepcounter{stanza}
\end{MyDescription} 

\begin{MyDescription}[\arabic{stanza}]{}
Nobles and wealth lords your vassals be
You Sovran Lord of lords, You King of men,
take then your power, O Gotama, and reign.
\stepcounter{stanza}
\end{MyDescription} 

   [ ......... ..]
\stepcounter{stanza}
\stepcounter{stanza}
\stepcounter{stanza}
\stepcounter{stanza}
\stepcounter{stanza}
\stepcounter{stanza}
\stepcounter{stanza}
\stepcounter{stanza}
\stepcounter{stanza}
\stepcounter{stanza}
\stepcounter{stanza}
\stepcounter{stanza}
\stepcounter{stanza}
\stepcounter{stanza}
\stepcounter{stanza}
\stepcounter{stanza}


\begin{MyDescription}[\arabic{stanza}]{}
Eight days have passed, All-seeing Sage,\\
since for refuge we have gone.\\
In seven nights, O Radiant One,\\
in your teaching we've been tamed.
\stepcounter{stanza}
\end{MyDescription} 

\begin{MyDescription}[\arabic{stanza}]{}
The Buddha you are, the Teacher you are,\\
the Sage overcomer of Mara,\\
so sheared of all evil tendencies,\\
gone across and taken all others.
\stepcounter{stanza}
\end{MyDescription}   

\begin{MyDescription}[\arabic{stanza}]{}
All attachments have been surmounted,\\
all inflows are removed\\
as a lion ungrasping\\
abandoned fear and dread.
\stepcounter{stanza}
\end{MyDescription} 

\begin{MyDescription}[\arabic{stanza}]{}
Here stand three hundred bhikkhus\\
\\
with hands held out as lotuses:\\
\\
stretch forth your feet, O Hero great\\
\\
that these now unblemished ones may bow at their Teacher's feet.
\stepcounter{stanza}
\end{MyDescription} 

\begin{MyDescription}[(Sn. 548-573)]{}
\end{MyDescription}

\chapter{Salla Sutta -\\ Dart of Death}


\begin{MyDescription}[]{}
Here's the life of mortals,\\
\\
wretched and brief,\\
\\
its end unknown,\\
\\
to dukkha joined.\\
\\
There's no means mat those\\
\\
who're born will never die.\\
\\
Reached decay, then death:\\
\\
the law for beings all.\\
\\
As with mm that's ripe\\
\\
there's always fear of falling,\\
\\
so for mortals born\\
\\
there's always fear of death.\\
\\
just as a potter's vessels\\
\\
made of clay all end\\
\\
by being broken, so\\
\\
death's the end of life.\\
\\
The young, those great in age,\\
\\
the fools, as well the wise\\
\\
all go under the sway\\
\\
of death, for death's their goal.\\
\\
Those overcome by death,\\
\\
to another world bound:\\
\\
father can't protect his son\\
\\
nor relatives their kin.\\
\\
While relatives are watching\\
\\
they weep and they lament\\
\\
See mortals one by one\\
\\
led as an ox to slaughter.\\
\\
As the world's afflicted\\
\\
by death and by decay,\\
\\
so the wise grieve not knowing world's nature well.\\
\\
Their path you do not know\\
\\
whereby they come, they go,\\
\\
neither end you see,\\
\\
useless your lament

\stepcounter{stanza}
\end{MyDescription} 
[.......... ..]
\begin{MyDescription}[]{}
should draw out the painful dart —\\
\\
lamentations and longings — the grief that is within.\\
\\
Dart withdrawn and unattached,\\
\\
the mind attains to peace,\\
\\
passed beyond all grief,\\
\\
griefless, fires put out.
\stepcounter{stanza}
\end{MyDescription} 

\begin{MyDescription}[(Sn. 574-593)]{}

\end{MyDescription}
\stepcounter{stanza}
\stepcounter{stanza}
\stepcounter{stanza}
\stepcounter{stanza}
\stepcounter{stanza}
\stepcounter{stanza}
\stepcounter{stanza}
\stepcounter{stanza}
\stepcounter{stanza}
\stepcounter{stanza}
\stepcounter{stanza}
\stepcounter{stanza}
\stepcounter{stanza}
\stepcounter{stanza}
\stepcounter{stanza}
\stepcounter{stanza}
\stepcounter{stanza}
\stepcounter{stanza}

\chapter{V\=aise\~n\~na Sutta (=M98)\\  To V\=aise\~n\~na upon who is a brahmin.}

 
 
Thus have l heard:\\
At one time the Radiant One was dwelling at lcch\=anangala. Now at that time a number of notable and prosperous brahmins were staying at lcch\=anangala, that is to say the brahmins Cank\=a, T\=arukkha, Pokkharas\=ati, J\=anussoni and Todeyya, as well as other notable and prosperous brahmins.\\

 Then, as the young brahmins D\=ase\~n\~nha and Bh\=aradvaia were walking and wandering for exercise this subject of discussion arose between them. `How is one a brahmin?' The young brahmin Bh\=aradvaja said, `When one is well-born on both sides, or pure maternal and paternal descent through seven generations in the past, then one is a brahmin'. But V\=ase\~n\~nha the young brahmin said, `When I one is virtuous and fulfils one's vows, then one is a brahmin'. Bh\=aradvaja could not convince Vase\~n\~nha while the latter failed to convince the former. Then Vase\~n\~nha said to Bh\=aradvaja, `Sir, the samana Gotama son of the Sakyas who left home from the Sakyan clan is living at lcch\=anangala, in the forest near lcch\=anangala. Now the good reputation of Master Gotama has spread in this way: That Radiant One is accomplished, completely Awakened, possessed of True Knowledge and conduct well~gone for himself and others, knower of the worlds, unexcelled trainer of those who can be tamed, teacher of devas and humanity, Awake and Radiant. Come, Bharadvaja, let us go to the samana Gotama and ask him about this. As he replies so will we bear his words in mind.\\

`Yes, sir', Bharadvaja replied.\\

So the two young brahmins approached the Radiant One and exchanged greetings with him. When this courteous and amiable talk was concluded, they sat down to one side and the young brahmin Vase\~n\~nha addressed the Radiant One in verse:

\begin{MyDescription}[\arabic{stanza}]{V\=ase\~n\~nha:}
Of Pakkharas\=ati the pupil I am\\
while student of T\=arukkha is he,\\
both of us have acknowledged mastery\\
in the threefold Veda love.
\stepcounter{stanza}
\end{MyDescription}

\begin{MyDescription}[\arabic{stanza}]{}
We have attained totality\\
over all the vedic masters teach\\
as philologists, grammarians,\\
and we chant as our masters do.
\stepcounter{stanza}
\end{MyDescription}

\begin{MyDescription}[\arabic{stanza}]{}
The subject of `birth', O Gotama,\\
is contention's cause with us:\\
he, a Bh\=aradv\=aja, does declare\\
`birth' is due to brahmin caste,\\
while I say its by karma caused:\\
know its thus, O One-with-Eyes.
\stepcounter{stanza}
\end{MyDescription}

\begin{MyDescription}[\arabic{stanza}]{}
Sir, to ask about this we have come,\\
to you acclaimed as Wide Awake,\\
each of us unable is\\
the other to convince.
\stepcounter{stanza}
\end{MyDescription}

\begin{MyDescription}[\arabic{stanza}]{}
As they raise their lotussed hands\\
towards the moon waxed full,\\
so to you, by this world revered,\\
we pay homage too.
\stepcounter{stanza}
\end{MyDescription}

\begin{MyDescription}[\arabic{stanza}]{}
So now of Gotama the Eye\\
uprisen in the world, we ask:\\
ls one by `birth' a brahmin,\\
or a brahmin karma-caused?\\
Explain to us who do not know\\
how we should `brahmin' recognize?
\stepcounter{stanza}
\end{MyDescription}

\begin{MyDescription}[\arabic{stanza}]{The Buddha:}
I shall analyse for you\\
in order due and as they are\\
the types of `birth' `mong living things\\
for many are the sorts of birth.
\stepcounter{stanza}
\end{MyDescription}

\begin{MyDescription}[\arabic{stanza}]{}
First, there's grasses and the trees\\
though of themselves they nothing know,\\
each species possessing its own marks\\
for many are the sorts of birth.
\stepcounter{stanza}
\end{MyDescription}

\begin{MyDescription}[\arabic{stanza}]{}
Next come beetles, butterflies,\\
and so on to the termites, ants,\\
each species possessing its own marks\\
for many are the sorts of birth.
\stepcounter{stanza}
\end{MyDescription}

\begin{MyDescription}[\arabic{stanza}]{}
Then, know of those four-footed kinds,\\
both the tiny and the huge\\
each species possessing its own marks\\
for many are the sorts of birth.
\stepcounter{stanza}
\end{MyDescription}

\begin{MyDescription}[\arabic{stanza}]{}
Know those whose bellies are their feet,\\
that is, the long-backed group of snakes,\\
each species possessing its own marks\\
for many are the sorts of birth.
\stepcounter{stanza}
\end{MyDescription}

\begin{MyDescription}[\arabic{stanza}]{}
Know too the many kinds of fish\\
living in their watery world,\\
each species possessing its own marks\\
for many are the sorts of birth.
\stepcounter{stanza}
\end{MyDescription}

\begin{MyDescription}[\arabic{stanza}]{}
Then know the varied winged ones,\\
the birds that range the open skies,\\
each species possessing its own marks\\
for many are the sorts of birth.
\stepcounter{stanza}
\end{MyDescription}

\begin{MyDescription}[\arabic{stanza}]{}
While in those births are differences,\\
each having their own distinctive marks,\\
among humanity such differences\\
of species — no such marks are found.
\stepcounter{stanza}
\end{MyDescription}

\begin{MyDescription}[\arabic{stanza}]{}
Neither in hair nor in the head,\\
not in the ears or eyes,\\
neither found in mouth or nose\\
not in lips or brows.
\stepcounter{stanza}
\end{MyDescription}

\begin{MyDescription}[\arabic{stanza}]{}
Neither in neck, nor shoulders found,\\
not in belly or the back,\\
neither in buttocks nor the breast,\\
not in arse or sexual parts.
\stepcounter{stanza}
\end{MyDescription}

\begin{MyDescription}[\arabic{stanza}]{}
Neither in hands nor in the feet,\\
not in fingers or the nails,\\
neither in knees nor in the thighs,\\
not in their `colour' not in sound,\\
here is no distinctive mark\\
as in the many other sorts of birth.
\stepcounter{stanza}
\end{MyDescription}

\begin{MyDescription}[\arabic{stanza}]{}
In human bodies as they are\\
such differences cannot be found:\\
the only human differences\\
are those in names alone.
\stepcounter{stanza}
\end{MyDescription}

\begin{MyDescription}[\arabic{stanza}]{}
'Mong humankind whoever lives\\
by raising cattle on a farm,\\
O V\=ase\~n\~nha you should know\\
as fanner not as Brahmin then.
\stepcounter{stanza}
\end{MyDescription}

\begin{MyDescription}[\arabic{stanza}]{}
'Mong humankind whoever lives\\
by trading wares here and there,\\
O V\=ase\~n\~nha you should know\\
as merchant not as Brahmin then.

\stepcounter{stanza}
\end{MyDescription}

\begin{MyDescription}[\arabic{stanza}]{}
'Mong humankind whoever lives\\
by work of many arts and crafts,\\
O V\=ase\~n\~nha you should know\\
as craftsman not as Brahmin then.
\stepcounter{stanza}
\end{MyDescription}

\begin{MyDescription}[\arabic{stanza}]{}
'Mong humankind whoever lives\\
by sewing others' needs and wants,\\
O V\=ase\~n\~nha you should know\\
as servant not as Brahmin then.
\stepcounter{stanza}
\end{MyDescription}

\begin{MyDescription}[\arabic{stanza}]{}
'Mong humankind whoever lives\\
 by taking things that are not given\\
 O V\=ase\~n\~nha you should know\\
 as a thief not Brahmin then.
\stepcounter{stanza}
\end{MyDescription}

\begin{MyDescription}[\arabic{stanza}]{}
'Mong humankind whoever lives\\
by the skill of archery,\\
O V\=ase\~n\~nha you should know\\
as solder not as Brahmin then.
\stepcounter{stanza}
\end{MyDescription}

\begin{MyDescription}[\arabic{stanza}]{}
'Mong humankind whoever lives\\
by performing priestly rites,\\
O V\=ase\~n\~nha you should know\\
as a priest not Brahmin then.
\stepcounter{stanza}
\end{MyDescription}

\begin{MyDescription}[\arabic{stanza}]{}
'Mong humankind whoever lives\\
through enjoying towns and lands,\\
O V\=ase\~n\~nha you should know\\
as rajah not as Brahmin then.
\stepcounter{stanza}
\end{MyDescription}

\begin{MyDescription}[\arabic{stanza}]{}
Him I call not a brahmin though\\
born from brahmin mother's line,\\
for if with sense of ownership\\
he's just supercilious:\\
owning nothing and unattached —\\
one such I say's a Brahmin then.
\stepcounter{stanza}
\end{MyDescription}

\begin{MyDescription}[\arabic{stanza}]{}
Who fetters all has severed,\\
who trembles not at all,\\
gone beyond ties, free from bonds —\\
one such I say's a Brahmin then.
\stepcounter{stanza}
\end{MyDescription}

\begin{MyDescription}[\arabic{stanza}]{}
Having cut strap and reins,\\
the rope and bridle too\\
and tipped the shafts, as one Awake —\\
one such I say's a Brahmin then.
\stepcounter{stanza}
\end{MyDescription}

\begin{MyDescription}[\arabic{stanza}]{}
Who angerless endures abuse,\\
beating and imprisonment,\\
with patience-power, an arm\=ed might -\\
one such I say's a Brahmin then.
\stepcounter{stanza}
\end{MyDescription}

\begin{MyDescription}[\arabic{stanza}]{}
Who's angerless and dutiful,\\
of virtue full and free of lust,\\
who's tamed, to final body come —\\
one such I say's a Brahmin then.
\stepcounter{stanza}
\end{MyDescription}

\begin{MyDescription}[\arabic{stanza}]{}
Like water drop on lotus leaf,\\
or mustard seed on needle point,\\
whoso clings not to sense desires,\\
one such l say's a Brahmin then.
\stepcounter{stanza}
\end{MyDescription}

\begin{MyDescription}[\arabic{stanza}]{}
here who comes to Know\\
- exhaustion of all dukkha,\\
laid down the burden, free from bonds —\\
one such I say's a Brahmin then.
\stepcounter{stanza}
\end{MyDescription}

\begin{MyDescription}[\arabic{stanza}]{}
Skilled in the Path, what's not the path,\\
in wisdom deep, sagacious one,\\
having attained the highest aim —\\
one such I say's a Brahmin then.
\stepcounter{stanza}
\end{MyDescription}

\begin{MyDescription}[\arabic{stanza}]{}
Not intimate with those gone forth,\\
nor with those who dwell at home,\\
without a shelter, wishes few —\\
one such I say's a Brahmin then.
\stepcounter{stanza}
\end{MyDescription}

\begin{MyDescription}[\arabic{stanza}]{}
Who has renounced all force\\
towards all being weak and strong,\\
who causes not to kill, nor kills —\\
one such I say's a Brahmin then.
\stepcounter{stanza}
\end{MyDescription}

\begin{MyDescription}[\arabic{stanza}]{}
Among the hostile, friendly,\\
among the violent, cool,\\
detached amid the passionate —\\
one such I say's a Brahmin then.
\stepcounter{stanza}
\end{MyDescription}

\begin{MyDescription}[\arabic{stanza}]{}
From whoever lust and hate,\\
conceit, contempt have dropped away\\
as mustard seed from needle-point —\\
one such I say's a Brahmin then.
\stepcounter{stanza}
\end{MyDescription}

\begin{MyDescription}[\arabic{stanza}]{}
Who utters speech instructive,\\
true and gentle too,\\
who gives offence to none —\\
one such I say's a Brahmin then.
\stepcounter{stanza}
\end{MyDescription}

\begin{MyDescription}[\arabic{stanza}]{}
Who in the world will never take\\
what is not given, long or short,\\
the great or small, the fair or foul —\\
one such I say's a Brahmin then.
\stepcounter{stanza}
\end{MyDescription}

\begin{MyDescription}[\arabic{stanza}]{}
In whom there are not longings found\\
for this world or the next,\\
longingless and free from bonds —\\
on such I say's a Brahmin then.
\stepcounter{stanza}
\end{MyDescription}

\begin{MyDescription}[\arabic{stanza}]{}
In whom is no dependence found,\\
with Final Knowledge, free from doubt,\\
duly wont to the Deathless deeps —\\
one such I say's a Brahmin then.
\stepcounter{stanza}
\end{MyDescription}

\begin{MyDescription}[\arabic{stanza}]{}
Here who's gone beyond both bonds:\\
to goodness and to evil too,\\
one who's sorrowless, stainless, pure —\\
one such I say's a Brahmin then.
\stepcounter{stanza}
\end{MyDescription}

\begin{MyDescription}[\arabic{stanza}]{}
Vanished is all love of being,\\
like the moon — unblemished, pure,\\
that one serene and undisturbed —\\
one such I say's a Brahmin then.
\stepcounter{stanza}
\end{MyDescription}

\begin{MyDescription}[\arabic{stanza}]{}
Who's overpassed this difficult path,\\
delusion's bond, the wandering-on,\\
who's crossed beyond, contemplative,\\
craving not, no questions left,\\
no clinging's fuel, so Cool become\\
one such I say's a Brahmin then.
\stepcounter{stanza}
\end{MyDescription}

\begin{MyDescription}[\arabic{stanza}]{}
Who has abandoned sense desires\\
as homeless one renouncing all,\\
desire for being all consumed -\\
one such I say's a Brahmin then.
\stepcounter{stanza}
\end{MyDescription}

\begin{MyDescription}[\arabic{stanza}]{}
Who has abandoned craving here\\
as homeless one renouncing all,\\
craving for being all consumed —\\
one such l say s a Brahmin then.
\stepcounter{stanza}
\end{MyDescription}

\begin{MyDescription}[\arabic{stanza}]{}
Abandoned all the human bonds\\
and gone beyond the bonds of god\\
unbound one is from every bond —\\
one such I say's a Brahmin then.
\stepcounter{stanza}
\end{MyDescription}

\begin{MyDescription}[\arabic{stanza}]{}
Abandoned boredom and delight,\\
become quite cool and assetless\\
A hero, All-worlds conqueror,\\
one such I say's a brahmin then
\stepcounter{stanza}
\end{MyDescription}

\begin{MyDescription}[\arabic{stanza}]{}
Whoever knows of beings' death,\\
their being born in every way,\\
unshackled, faring well, Awake —\\
one such I say's a Brahmin then.
\stepcounter{stanza}
\end{MyDescription}

\begin{MyDescription}[\arabic{stanza}]{}
Whose destination is unknown\\
to humans, spirits or to gods,\\
pollutions faded, Arahat —\\
one such I say's a Brahmin then.
\stepcounter{stanza}
\end{MyDescription}

\begin{MyDescription}[\arabic{stanza}]{}
For whom there is not ownership\\
before or after or midway,\\
owning nothing and unattached —\\
one such I say's a Brahmin then.
\stepcounter{stanza}
\end{MyDescription}

\begin{MyDescription}[\arabic{stanza}]{}
One noble, most excellent, heroic too,\\
the great sage and the one who conquers all,\\
who's faultless, washes, one Awake —\\
one such I say's a Brahmin then.
\stepcounter{stanza}
\end{MyDescription}

\begin{MyDescription}[\arabic{stanza}]{}
Who knows their former births\\
and sees the states of bliss and woe\\
and then who wins the waste of births -\\
one such I say's a Brahmin then.
\stepcounter{stanza}
\end{MyDescription}

\begin{MyDescription}[\arabic{stanza}]{}
Whatever's accepted and `name' and `clan'\\
is just a worldly designation —\\
by conventions handed down\\
accepted everywhere.
\stepcounter{stanza}
\end{MyDescription}

\begin{MyDescription}[\arabic{stanza}]{}
But those asleep, unquestioning,\\
who take up views, who do not Know,\\
unknowingly they've long declared:\\
one's a brahmin just by `birth'.
\stepcounter{stanza}
\end{MyDescription}

\begin{MyDescription}[\arabic{stanza}]{}
One's not a brahmin caused by `birth',\\
nor caused by `birth's' a non-brahmin;\\
a brahmin's one by karma caused,\\
by karma caused a non-brahmin.
\stepcounter{stanza}
\end{MyDescription}

\begin{MyDescription}[\arabic{stanza}]{}
By karma caused a farmer is,\\
one's a craftsman karma-caused,\\
by karma caused a merchant is,\\
one's a servant karma-caused.
\stepcounter{stanza}
\end{MyDescription}

\begin{MyDescription}[\arabic{stanza}]{}
By karma caused a robber is,\\
one's a soldier karma-caused,\\
by karma caused a priest becomes,\\
one's a ruler karma-caused.
\stepcounter{stanza}
\end{MyDescription}

\begin{MyDescription}[\arabic{stanza}]{}
Thus according as it is\\
people wise do karma see;\\
Seers of causal relatedness\\
skilled in karma, its results.
\stepcounter{stanza}
\end{MyDescription}

\begin{MyDescription}[\arabic{stanza}]{}
Karma makes the world go on\\
people by karma, circle round,\\
Seers of causal relatedness\\
skilled in karma, its results.
\stepcounter{stanza}
\end{MyDescription}

\begin{MyDescription}[\arabic{stanza}]{}
By ardour and the Good Life leading\\
with restraint and taming too:\\
by this a Brahmin one becomes,\\
one's by this a Brahmin best.
\stepcounter{stanza}
\end{MyDescription}

\begin{MyDescription}[\arabic{stanza}]{}
Possessed of Triple Knowledges,\\
at Peace, repeated being waste away —\\
know V\=ase\~n\~nha, such a one\\
is Brahma and Sakra for those who Know.
\stepcounter{stanza}
\end{MyDescription}



When this was said the young brahmins V\=ase\~n\~nha and Bh\=aradv\=aja exclaimed to the Radiant One: `Magnificent, Master Gotama! The Dharma has been clarified by Master Gotama in many ways, as though he was lighting what was overthrown revealing what was hidden, showing the way to one who was lost, or holding a lamp in the dark so that those with eyes can see forms. We go for refuge to Master Gotama, to the Dharma and to the Sangha. May Master Gotama remember us as upasakas who from today have Gone for Refuge for life.
 
\begin{MyDescription}[(Sn. 594-656)]{}

\end{MyDescription} 
\newpage

\section{Notes upon the Vasennha Sutta}
As the Buddha has defined what he means by the word `outcaste', vasala, in the Sutta of that name (Sn. 116-142), so here he discourses with two young brahmins on what makes one a true brahmin. Both outcaste and brahmin are defined according to Dharma, what one should avoid in the first case and what one should do in the second.\\

In the society of the Buddha's days (still found in present Indian attitudes), the brahmins reckoned themselves the highest among all the castes having been born of a brahmin mother and father whose families were `pure' brahminical stock back through seven generations on both sides. It was enough to have such parents and families to be born as a brahmin, they reckoned. Moreover, according to legend the brahmins came from the head, the uppermost part of primeval man, while other castes originated from me lower parts of that body, the outcastes being merely the dust upon that first man's feet. Such grounds for superiority with exclusive knowledge of Vedic rites and mysteries, endowed many brahmin men with a very inflated idea of their own importance. (Women of brahmin families though in very early times also possessed knowledge of the Three (or four) Vedas together with ritual and correct transmission through chanting, lost this eminence and came to be regarded by brahmin men as just mothers of their sons).\\


The verses of this sutta can divided into a number of sections, each one with distinct bearing upon the questions: What makes one a brahmin, `birth' as explained above, or karma — the results of intentional actions, speech and body. These sections are as follows:

\begin{enumerate}
\item Vasefifiha's laudatory and questioning verses (5 94-5 99).
\item The Buddha's analysis of Living things (600-61 l).
\item His definition of who is not a brahmin (612-619).
\item His verses upon who is a true Brahmin (620-647).
\item His concluding verses upon `caused by karma' (648~656)
\end{enumerate}

\subsection{V\=ase\~n\~nha's laudatory and questioning verses}
For words of praise in the Teacher's presence see my short essay among the introductory material of this book. The verses open with Vase\~n\~nha telling the Buddha who were their respective teachers, as well as their own attainments, a sort of verbal curriculum vitae. With this they are communicating their attainments as brahmins thereby informing him of the sort of people he will be addressing. However, this is done with a great deal of politeness by inserting such laudatory titles as `One-with-Eyes', `Wide Awake' and so on. They raise their lotussed hands, probably to their foreheads but at least to their hearts, marking their respect for the teacher they are questioning.

\subsection{The Buddha's analysis of living things}
In these verses most of them have the refrain `each species possessing its own marks/form many are the sorts of birth', except for the first one which as only the latter line. lt is important to remember that the word `birth' had (and has) a very special meaning in India. Asking another person (not uncommon even in modem India) about his or her `birth', is the equivalent to enquiring about the caste of their family. This is required before one eats or drinks with other unknown people and of course if marriage is contemplated. High caste persons (those from brahmin or noble families) can lose their status through pollution' by eating or drinking with, or having with those of low caste according to brahminical laws. The Buddha points out in the first few of these verses (600-606), that among the groups he describes there are many sorts of birth. These verses are to highlight specific differentiation as opposed to what is found among human beings as made clear in verses 607-611. in other words, all human beings whatever their caste, colour, race or language , are the same, with the same organs and characteristics. in the last verse in this section the Buddha notes that differences are merely nominal. In India this refers to `birth' or caste, in the rest of the world to class, family, wealth, education etc. Such matters are only conventional, not of the essence.\\

The Buddha is thus revealed as the first person ever to reject racism in any form. All human beings are basically the same in that they can all practise the Dharma and experience the Awakened State. People are only `high' or `low' according to Dharma by their behaviour. Elsewhere, the Buddha noted that one is noble by thought, speech and action, not by worldly ennoblement. The word for noble in P\=ali/Sanskrit is ariya/\=arya conventionally meaning a high caste person, from the clans of self-styled \=aryans invading India from around 1500 BCE onwards. But the Buddha's interpretation of this word emphasized the nobility of mind, speech and body actions. Such nobility has no boundaries of race, caste, class or language.\\

notice that in verse 610 the Buddha denies that there are essential differences in `colour'. This is va\=o\=oa in P\=ali, varna in Sanskrit, and has the general meaning in India of superior birth or inferior. High-caste people are supposed to have light colours of skin, while workers and outcasts are dark in complexion. However this myth does not always work out as many brahmins in the south of the country are very nearly black! Paired with `colour' is `sound', meaning a polished way of speech or an uneducated one. lt is possible to distinguish the caste a person belongs to by listening to their speech. Some words will point out a brahmin background, others will only be used by outcastes.\\

The last verse in this section makes it clear that the only differences among human beings are merely nominal. A man is only tradition a brahmin or an outcaste, there are no essential differences.

\subsection{The Buddha's definition of who is not a brahmin.}
Verses 612-619 define by a nominally brahmin-caste man's work how he could not be a brahmin, that is, he doesn't not live up to the high standard set by the Buddha to qualify for the title of `brahmin'. As the Buddha did not accept that one could be spiritually advanced merely by being born of brahmin-caste parents, so neither was one even nominally a brahmin when one's livelihood had nothing to do with caste labels. Nominal brahmins in the Buddha's days had departed from their ancient religious ideals and had become over time farmers, craftsmen, even rulers. They should not be counted even as nominal brahmins and certainly not as Brahmins, those who are spiritually advanced.\\

In this book I have used `brahmin' (lower-case) to mean one of brahmin parents and caste. Capitalized `Brahmin' refers to one who is purified by Dharma practice. In English we can, while not altering the spelling, change its meaning by using either a capital or lower-case letter. Note that this is not possible in Indian scripts which have no such distinction.\\

Verse 618 could surprise many brahmins who perform rituals and ceremonies and are paid for this, as the Buddha denies that one can be a Brahmin just by carrying out traditional pujas and so on. Such a person is only a priest. Priests of this sort may recite the rituals in Sanskrit but not even know the meaning of what they chant. They have learnt by heart the sounds but not the meaning. Often these ceremonies may be marked by gabbling the words at high speed. Unfortunately such behaviour can be found also among some Buddhist monks, equally bored and ignorant of their traditions. 

\subsection{IV. The Buddha's verses upon who is a true Brahmin.}
These (620-647) are found also and nearly identical in the P\=ali Dhammapada in the Brahmanavagga. All have the refrain, `one such I say's a Brahmin then'. This ideal brahmin who would be either completely Awake or far along the Dharma-path to Awakening illustrates the Buddha's interpretation of a truly honourable and marvellously spiritual person. These verses serve in India to remind ordinary brahmins how they should behave, a high ideal indeed!\\

 A few notes follow upon sundry matters in these verses:

\begin{itemize}
   
\item[620] The Buddha contrasts ordinary ideas of birth — `born from brahmin mother's line' with the real Brahmin of 'owning nothing and unattached.'

\item[621] No fetters, no ties, no bonds equals no anxiety, no trembling, nothing to fear.

\item[622] Freedom from being yoked and so a Buddha.

\item[623] Abuse, torture, imprisonment all endured without anger developing patience as one's `weapon'.

\item[624] Basics for becoming a true Brahmin, truly Awake.

\item[625] Water on a lotus-leaf or flower forms into beads like mercury and runs off, it never spreads out because of the saponitic surface of the whole plant. Likewise a mustard seed drops of a needle point.

\item[626] `Exhaustion of all dukkha' means the causes (karma, etc) producing dukkha in the past have been exhausted by good made in the present. `Laid down the burden' is freedom from all attachment to the notion `my body' and my mind' and every other sort of possessiveness.

\item[627] if one practises Dharma one should leam what is the Path and what is not. Then one should practise accordingly. The `highest aim' is then not impossible.

\item[628] A verse specially for meditative bhikkhus, munis or other solitary practitioners. (Solitary practice should only be undertaken with the advice and approval of one's Dharma teacher)

\item[629] Force used against other beings, human or other, as well as its use internally against oneself, can never achieve the good results of lasting peace. Politicians who advocate violent means to `resolve' others' violence and religious fanatics who, driven by wrong views, preach and practise violence against the followers of sects or religions differing from their own,— all of them are blinded by delusion and never learn even a little from the facts of human history. No good comes of violence and even less may be expected as a result of killing.
   MS PaI!:Z86
   
   
\item[630] Specially suitable for one of strong passions! Because one's character is fiery this does not mean that one cannot practice, in fact the energies of the passions can be turned round into wisdom and compassion.

\item[631] Mustard seeds which are small still cannot remain upon a needle-point, so for the the practitioner there is no room for lust, hate, conceit, etc.

\item[632] There are fortunately still people like this in our turbulent world. When reading this verse, l think with gratitude of Ven. Pal(????)vaddho \=acariya, my very kind and wise teacher many years ago.


\item[633] Dharma is the way of giving, generosity and letting-go, quite different to the world's way of greed, taking, hoarding and selfishness.

\item[634] Longings for this world's joys and properties are common while longings for the next world are found among the followers of most religions. The latter feel they are superior to those who merely long for worldly pleasure and power but their `spiritual' longing is still an attachment. They must be longingless and free from bonds'.

\item[635] Until Awakening occurs there is always doubt. `Final knowledge' a(???)a/a(???)a) once experienced dissolves away all doubt. The profound (note in text) knowledge of deathlessness ls another way of saying this'. 

\item[636] Some are bound by their tendencies to evil and cause themselves and others much dukkha. But others are attached to their goodness and sometimes feels themselves superior, looking down upon those who seem to them not virtuous. This is also a bondage.

\item[637] Love of being means attachment to existence, not being able to let go of life, even at the time of death. lt certainly means, or implies, more samsaric existence, on and on, round and round turns the samsara wheel. 

\item[638] `Wandering-on' is my old translation of `samsara' which means literally `wandering and wandering'. The path through samsara is always difficult because it is guided by delusion, with greed and hatred too. This is also the fuel which keeps these three fires burning. When these fires go out (where do they go to? Do fires `go' anywhere when extinguished?) there is the Cool Peace of Nirvana.

\item[639] `Homeless one renouncing all': this could be as a bhikkhu or bhikkhunis in monastic community, or as a muni living a solitary life. (One munis see vv. 207-221, 679-723, 835-847,848-861). `Renounce' means letting go from the heart, not a forced `renunciation' which will only create inner conflicts, tension and dukkha generally. lf practised correctly then all desires subside by themselves.

\item[640] Craving is consumed by itself with good Dhanna practice. lt does not require any force to be used.

\item[641] We all know something of human limitations and attachments, our bonds in this world. But then we have to know thoroughly `the bonds of gods' — spiritual bondage such as to the delights that occur through meditation practice, even bliss. lt often happens to those who devote their live to spiritual practice that they become ensnared by these extra-human experiences, especially if they follow a teaching that preaches heaven as salvation. From a Buddhist perspective this is confusion: `heaven(s)' mean the deva-worlds, some sensual and some of refined spirituality but all of them as existences in the round of birth and death. `Salvation' is something different from the deva-realms, as these verses point out.

\item[642] Without assets: this translation of an-upad/1i is Ven. Nyanamoli's term. Upadhi is a P\=ali word with a great range of meaning but generally refers to what is grasped at, what one is attached to. Later Commentators have expanded these varieties of upadhi, as any Buddhist Dictionary will make clear. Think generally of `assets' as what one assumes that one owns such as mind and body. One is an `All-worlds conqueror' not by the use of force, armies and wars, but rather by being free of any attachment to all of the possible states of existence.

\item[643] This advice is continued from the last verse. Faring well means practising the Dharma. To fare is one of only two English words which can convey the meaning of the P\=ali verb: carati, both to go on a journey, and to practise a spiritual path. The other word is `to course'.

\item[644] Those Awake, do not `go' anywhere at death, while the ordinary unenlightened persons do go — to some other existence. The state of Awakened Ones cannot be explained in words, as no language exists that has words for what is beyond all words, even all holy ones.

\item[645] `Ownership before or after or midway': this has three possible meanings. first, it has the meaning of unattachment while practising generosity, dana; one has no regrets about giving either while planning it, or while actually giving or after having given. Second, `before' means past lives, `after' refers to future, while midway is this present life. Third refers to the Awakened person of whom it can be said: `owning nothing, and unattached' — time does not apply since its limitations have been transcended.

\item[646] A list of praiseworthy qualities of those Awake. References to heroes' and `conquerors' means those who continue to practise Dharma even in difficult situations, and who `conquer' or overcome all obstructions. `Washes' refer to one purified of all mental-emotional troubles though the P\=ali word `uh\=ataka\=u' originally meant the ritual purificatory bathing in rivers by brahmins.

\item[647] This first line on `former births' is one of many references in the P\=ali suttas to lives before this one. Of course they have gone, they are past and we cannot practise in them. But they did exist and helped to shape our present existence. Those westerners who try not to take thin into account and who obviously have no experience of previous existences, distort the Buddha's teaching's. `The waste of births' can also be translation as `exhaustion', ***Comment [SW2] discuss with laurence*** their cessation because karma no longer exists to cause them to appear.
\end{itemize}

\subsection{The Buddha's concluding verses upon `caused by karma'.}
The last nine of these verses in this sutta bring together the various strands of Dharma taught here. ln the first of these verses (648) the Buddha emphasizes how names and families are only conventional descriptions having no ultimate meaning, they are the only to the extent that they are accepted within their own language, culture and convention. As these factors change so will the meaning of names. The Buddha's Awakening showed him that the accepted conventions were sometimes untruthful, even evil and then did not accord with Dharma.\\

This is why in verse 649 he refers to `those asleep, unquestioning' who take words and names as ultimate truth. `Taking up views' is a technical term for those who seize upon beliefs as true, whose beliefs (views) cannot be verified and upon a deeper level who blindly follow the basic view of `I am'. From that flow all other views. ln numerous places in the sutta the Buddha has shown the evil results of the belief in `I am', ranging from mild conflicts, round to persecutions, wars and `racial cleansing'. People, blind like this, not surprisingly by tradition uphold such views as `one's a brahmin just by birth.'\\

Verse 650 puts the Buddhist view: that it is karma not `birth' that is responsible for one's status. Were it governed by `birth' there is nothing could be changed — just grin and bear it. This is in fact the Hindu position (particularly in the Bhagavadg\=ata) that those of low castes should just accept their station in life and work unquestioningly for high caste people. But the Buddha taught that there are four types of people: one who goes from dark to dark, one from dark to light, one from light to dark, one from light to light (A. Fours.85). This has nothing to do with `birth', whether understood in the brahminical sense, or in western' society. There are many historical examples in both of those who started their lives in poor families but by diligence ended up in the light, while plenty of cases can be found of those from families of high repute who yet degenerated due to their internal tendencies to greed, hatred and delusion, as well as to external conditions.\\

The next verse, 651, may raise questions, even eyebrows, even voices. We learn from the verse that farmers, craftsmen, merchants and servants have those occupations due to karma, not due to their societal status or `birth'. This seems to be only a very short skip to the Hindu caste position of determined occupation due to past karma. This will not seem unpleasant if one's present fortunate birth is karma-caused. One has plenty of comforts and easy access to education and so on. But this takes on a quite different aspect if the present birth is among the poverty-stricken and down-trodden.\\

I suggest that the Buddha lists these commons occupations of his times (to which we would add a host of types of work found in our days) to point out that people choose their preferred work (if they have that choice) on the basis of v\=asan\=a: This term refers to the repetitive karma in a past life(s). To examples from my own life: I enjoyed learning from books and (when found) good teachers though my interests quite excluded sports and sciences. This v\=asan\=a, tendency or predilection was part of my character from childhood. So was another tendency to interest in Ethiopia. None of my family had been there yet throughout my teens and twenties I met continually people who had lived in the country — such as a girlfriend who had met the emperor Haile Selassie — and strengthened this interest. But then the great v\=asan\=a became apparent. As a British soldier in the Suez Canal Zone doing my National Service, I read the book Buddhism” by Christmas Humphreys (still in print!) during the course of one day, all 240 pages of it, and at the end knew without a doubt that I was a Buddhist.\\

So perhaps the Buddha intended to suggest that people's occupation to some extent depends upon these tendencies which draw people towards not only being farmers craftsmen, merchants and sen/ants but also (652) to being robbers, soldiers, priests and rulers. it is no coincidence that these four are found in one Verse.\\

653-4. Karma as a process, a very complex one, does really work in the mind and can by advanced practitioners be seen and known. This means that it is not a theory or mere belief. One of the six True Knowledges possessed by the Buddha was the Knowledge of others' karma and results, and since his time many famous Teachers have had this ability. `Seers of causal relatedness' refers to those who see Dependent Origination for themselves. `People by karma circle round' means that they continue to go round the Wheel of birth and death, while `beings are by karma bound' means that the vast majority of them have no choice about where they will go in their next existence. That has already been decided by the predominant sorts of karmas that they have `made. The only beings to have any choice in this are those upon the path — streamwinners or Bodhisattvas. The surety about this is reinforced by the simile: the well-secured linchpin that holds the chariot's wheel to the axle. New karma in a different direction can sometimes block out, though probably not dissolve, karma completely.\\

To encourage the young brahmins so that they did not feel that their futures were deterministically limited the Buddha spoke the next verse showing them what they had to practise. `Ardour' (tapo) was a word well-known to brahmins. To them it meant, and still means, some sort of austerity, sometimes very severe self-torture aimed at purification according to Hindus but disapproved of by the Buddha. In the Buddha's Dharma tapo means making an effort, even if this is difficult and involves renunciation. lt should not involved the harm of oneself. `Good Life” (brahmacariya) is a life based on Dharma of moral conduct
(\=sala), meditation (sam\=adhi) and wisdom (pa\~n\~n\=a, praj\~n\=a). it may or may not involve celibacy and the adoption of a monastic life, or a solitary one (muni). With this son of `restraint and taming' one becomes a true Brahmin, that is, Awakened.\\

To elaborate upon this, verse 656 mentions the triple true knowledges (vijj\=a, vidy\=a) which meant quite different things to brahmins and to Buddhists. Brahmins understood `te vijj\=a' to mean complete knowledge of the Three Vedas, their most ancient `scriptures'. (As they were learnt by heart and chanted not written down they should really be called `chantures'). Buddhists understood these words to mean: l. memory of one's previous births (lives); 2. the divine eye — ability to see distant events, people, etc.; 3. the e)exhaustion of the inflows (\=asava), the inflows of karma sensual desire, the inflow for continued existence, and the inflow of (holding) views. Sometimes a fourth inflow is added: the inflow of ignorance.\\

Possessed of Triple Knowledges', which are not `possessed' in the normal sense as there is by that time no one to possess them, goes along with `Peace' but not merely the peace experience by a good meditator, but Awakening's Peace, Repeated being' — desire for more existence — is exhausted with no desire either for it or against it.\\

The last part of the verse compares a person with these attainments to Brahma and Sakra. As the two important Hindu devas would be revered by their followers, so should an Awakened one be treated. This is rather an interesting end to these verses, an end specially spoken by the Buddha to inspire them as brahmins to practise the Dharma.


   
\chapter{Kok\=aliya Sutta - To Kok\=aliya an the results of slanderous speech}

Thus have I heard:\\

At one time the Radiant One was dwelling at Jeta's Grove in the park of An\=athapindika near S\=avatthi. Now at that time the bhikkhu Kok\=aliya approached the Radiant One and having done so saluted him and sat down to one side. Seated there Kok\=aliya bhikkhu said this to the Radiant One: Sir, S\=ariputta and Moggall\=ana are of evil desires, under the influence of evil desires. When this was said the Radiant One spoke to the bhikkhu Kok\=aliya: don't this Kok\=aliya, don't say so! Clear your mind towards S\=ariputta and Moggall\=ana for they are very friendly. A second time Kok\=aliya repeated his allegation and the Radiant One replied in the same way. An even a third time Kok\=aliya spoke his accusation and the Radiant One replied.\\

After this the bhikkhu Kok\=aliya rose from his seat, saluted the Radiant One and circumambulating him, keeping him on the right, departed. Only a short time after he left, Kok\=aliya's whole body broke out in boils the size of mustard seeds, then grew to the size of green-gram, then to chickpeas, then to jujube seeds, then to jujube fruits, then to myrobalan fruits, then to young bael fruits, then to mature bael fruits and when they had reached this size all over his body, blood and pus was discharged and Kok\=aliya died. After death he appeared in the Paduma Hell as a result of hardening his heart against S\=ariputta and Moggall\=ana.\\

Then as the night passed, Brahma Sahampati of great radiance illuming the whole of Jetavana, came to the Radiant One and after saluting him stood to one side and said this: Venerable, the bhikkhu Kok\=aliya has done his time — died and appeared subsequently in the Paduma Hell as a result of hardening his heart against S\=ariputta and Moggall\=ana. When this was said, a certain bhikkhu spoke to the Radiant One thus: How long, venerable, is life in the Paduma Hell? Bhikkhu, life in the Paduma hell is surely long, not easy to reckon in terms of years, of hundreds of years, of thousand of years, in tens of hundreds of thousands of years. But can a simile be made, sir? It can, bhikkhu. Suppose that there was a Kosalan cartload of twenty measures of sesame seed and that from this a man might take a single seed every century. That Kosalan cartload of twenty measures of sesame seeds would be more quickly used up in that way than would lifetime in the Paduma Hell. Moreover, bhikkhu, there are twenty lifetimes in the Abbuda Hell to equal one in Nirabbuda Hell...\\

twenty in Nirabbuda to equal one Ababa...\\

twenty in Ababa to equal one Ahaha...\\

twenty in Ahaha to equal one A\~na\~na...\\

twenty in A\~na\~na... one Kumuda...one Sogandhika...one Uppalaka one Pund\=arika...one Paduma. It is in Paduma that Kok\=aliya bhikkhu has arisen for hardening his heart against S\=ariputta and Moggall\=ana.\\

 The Radiant One spoke thus and having said this spoke further (these verses).\\

\begin{MyDescription}[\arabic{stanza}]{}
For every person come to birth\\
an axe is born within their mouths\\
with which these fools do chop themselves\\
when uttering evil speech.
\stepcounter{stanza}
\end{MyDescription} 

\begin{MyDescription}[\arabic{stanza}]{}
Who praises one deserving blame\\
or blames that one deserving praise\\
ill-luck does tore by means of mouth\\
and from such ill no happiness finds.
\stepcounter{stanza}
\end{MyDescription} 



\begin{MyDescription}[\arabic{stanza}]{}
Trifling the unlucky throw\\
by dice destroying wealth,\\
even all one's own, even oneself as well\\
compared to that greater `throw' —\\
the thinking ill of Sugatas.
\stepcounter{stanza}
\end{MyDescription} 


\begin{MyDescription}[\arabic{stanza}]{}
Having maligned the Noble Ones\\
with voice and mind directing ill\\
one then arrives at (self-made) hell\\
of millions of aeons (slow to end).
\stepcounter{stanza}
\end{MyDescription} 

\begin{MyDescription}[\arabic{stanza}]{}
With one denying truth there goes to hell\\
that one who having done, says “I did not'.\\
Humans having made such karmas base\\
equal are they in the other world.
\stepcounter{stanza}
\end{MyDescription} 

\begin{MyDescription}[\arabic{stanza}]{}
Whoso offends the inoffensive one\\
who's innocent and blameless, both,\\
upon that fool does evil fall\\
as fine dust flung against the wind.
\stepcounter{stanza}
\end{MyDescription} 

\begin{MyDescription}[\arabic{stanza}]{}
With one denying truth there goes to hell\\
that one who having done, says “I did not'.\\
Humans having made such karmas base\\
equal are they in the other world.
\stepcounter{stanza}
\end{MyDescription} 

\begin{MyDescription}[\arabic{stanza}]{}
That person prone to coveting\\
will speak of others in dispraise -\\
one faithless and ill-mannered too,\\
jealous, set on slandering.
\stepcounter{stanza}
\end{MyDescription} 

\begin{MyDescription}[\arabic{stanza}]{}
One foul mouthed, of baseless talk\\
ignoble, treacherous, evil, doing\\
wrong deeds, luckless, ill—begotten human scum -\\
Speak little here! Or else hell-dweller be!
\stepcounter{stanza}
\end{MyDescription} 

\begin{MyDescription}[\arabic{stanza}]{}
Dirt do you scatter for your own happiness\\
whenever you revile those who are good,\\
faring through the world many evil you have done\\
in the long night falling down a precipice.
\stepcounter{stanza}
\end{MyDescription} 

\begin{MyDescription}[\arabic{stanza}]{}
No one's karma is destroyed,\\
truly as Master it returns\\
so the foolish misery bring\\
upon themselves in future time.
\stepcounter{stanza}
\end{MyDescription} 

\begin{MyDescription}[\arabic{stanza}]{}
Bashed by bars of iron,\\
iron spikes' edges bite\\
and the food appropriately is\\
like white—hot balls of iron.
\stepcounter{stanza}
\end{MyDescription} 

\begin{MyDescription}[\arabic{stanza}]{}
And softly speak no speakers there\\
they hasten not to help nor to safety lead,\\
they enter all-directions fire,\\
on burning ember-mats they lie.
\stepcounter{stanza}
\end{MyDescription} 

\begin{MyDescription}[\arabic{stanza}]{}
Tangled they are in fiery nets\\
and pounded there with hammers of iron\\
and led, immersed, through darkness blind\\
spreading in all directions.
\stepcounter{stanza}
\end{MyDescription} 

\begin{MyDescription}[\arabic{stanza}]{}
And enter they in iron cauldrons afire\\
in which for long they're stowed\\
rising up and sinking down,\\
bubbling in masses of fire.
\stepcounter{stanza}
\end{MyDescription} 

\begin{MyDescription}[\arabic{stanza}]{}
There the evil-doers cook\\
in a mixed stew of blood and pus,\\
to whatever direction they turn\\
there they fester at the touch.
\stepcounter{stanza}
\end{MyDescription} 

\begin{MyDescription}[\arabic{stanza}]{}
Then the evil-doers cook\\
in worm-infested waters\\
and cannot flee for there are sides,\\
vast vessels with all surfaces concavities.
\stepcounter{stanza}
\end{MyDescription} 

\begin{MyDescription}[\arabic{stanza}]{}
There looms the sharp—edged Swordleaf scrub —\\
they enter and their limbs are slashed\\
and there with hooks their tongues are seized,\\
pulled to and fro, they're beaten up.
\stepcounter{stanza}
\end{MyDescription} 

\begin{MyDescription}[\arabic{stanza}]{}
They draw near Vetaran\=a Creek,\\
biting and bladed hard to cross,\\
there headlong down the foolish fall —\\
these evil-doers evil done.
\stepcounter{stanza}
\end{MyDescription} 

\begin{MyDescription}[\arabic{stanza}]{}
Then while they wail, the mottled flocks\\
of ebon ravens them devour;\\
jackals, hounds, great vultures, hawks\\
and crows rend them and ravage there.
\stepcounter{stanza}
\end{MyDescription} 

\begin{MyDescription}[\arabic{stanza}]{}
Misery unmitigated, this mode of life\\
which evil-doers get to see,\\
therefore let one in life's remainder be\\
not careless, one who does what should be done.
\stepcounter{stanza}
\end{MyDescription} 

\begin{MyDescription}[\arabic{stanza}]{}
those who know reckon the term\\
of these in the Paduma Hell in loads\\
of sesame, five myriad lakhs of seeds\\
and the, twelve hundred lakhs beside.
\stepcounter{stanza}
\end{MyDescription} 

\begin{MyDescription}[\arabic{stanza}]{}
Thus are Hell's many ills here told\\
and term that must be spent there too;\\
towards, therefore, those praiseworthy,\\
the friendly, pure — guard both words and thoughts.
\stepcounter{stanza}
\end{MyDescription} 
   

\section{Some Reflections on `Hell'}
There are a few P\=ali texts centred upon this subject but the present Sutta is the only one of these found in Sn. the subject of the hell-realms (niraya in P\=ali,naraka in Skt.) is rare when compared to the vast recorded teachings of the Buddha. Mention of it occurs always in the context of some serious wrongdoing, for peccadilloes do not result in the experience of hell. Another point to consider is that as `all conditioned things are impermanent' (sabbe sa\=ikh\=ar\=a anicc\=a, Dhp.), hell certainly is conditioned by the causes that have led to its suffering, so it is impermanent, as are all other forms of existence. In this it differs from the hell of theistic religions which they sometimes have proclaimed as eternal. Buddhist teachings could not agree that impermanent cause, even the slaughter of millions of men, could have permanent results, so even such monsters as Hitler, Stalin and Ma-tse-tun could not suffer hell forever. However, all texts are agreed that it continues for a very long time. `Hell' is not a translation of viraya favoured by some because of the eternalistic overtone which does not apply to the Buddhist meaning and so prefer `purgatory'. I have used `hell' here as it is a direct and brief term fitting well into verse, and applicable to both the great sufferings experienced by evil-doers whether in the eternalist religions or in the Buddhadharma.\\

The wrong-doer in this Sutta is a bhikkhu called Kok\=aliya (in Sn.) or Kok\=alika (in other Pali texts). In any case his name means `one from the town of Kok\=ali'. His story, the essence of which is his slander of the Buddha's chief disciples, S\=ariputta and Moggall\=ana, in the presence of the Buddha saying that these two enlightened monks were of evil desires, and they are influenced by evil desires'. Presumably he repeated this allegation to others as well. According to the texts this resulted in a plague of boils all over his body and we could speculate whether modern medicine could or could not find a cure for these. The P\=ali commentators presumably would deny that any cure was possible as the affliction was brought about by evil karma. Then Kok\=aliya died of his illness and the text relates that he found himself immediately in hell. As this frightful experience is usually connected with great anger and violence and though in this case it seems to be motivated by envy and because this touches upon a doctrine of the heavy results of doing evil to the Nobles ones (ariya), it should be examined in brief here.\\

This teaching, found in its fullest development in the P\=ali Comys, assigns degrees of painful karmic results according to the attainment of the recipients. Thus, injuring a Buddha causes in the wrongdoer the worst and longest sufferings while attacking ordinary persons, including monks and nuns will not have such bad consequences, with all shades of karma results between. This is a questionable doctrine as explained below and my lead us to doubt whether we should accept this Sutta as the Buddha-word (Buddhavacana) or not.\\

The Buddha's teaching on Karma is clear and straightforward. A decision in the mind is mental karma, a decision to speak and one has made vocal karma and if this is followed by deliberate action this is called bodily karma, the last two are of course guided by mind. These three actions or karmas may be motivated by greed, hatred and delusion in which case they are called `unwholesome karma' the results of which will be painful, or by non-greed (=generosity), non-hatred (loving-kindness and compassion), or by non-delusion (=wisdom) in which case such karma is called `wholesome' and the results of it to be experienced sooner or later will be pleasant. The initial decision to think, say and do is taken in the mind and the following results are experienced, for happiness or suffering in both mind and body. This sutta concerns unwholesome karma with painful results.\\

Then, there is also this teaching, albeit rather uncommon in the P\=ali Suttas, that good karma made towards Enlightened persons has much greater results for the doer, than good expressed towards ordinary people who of course may include beggars or other poverty stricken or ill people. Such emphasis, found in the P\=ali Comys and much in evidence in Buddhist countries means that those who are supposed to be enlightened or on their way to Enlightenment, receive very generous donations while care of the poor, the diseased, the criminal and the mad receive little, as the doer's merit will be insignificant because these people are not pu(??)avanta — those possessing merit. This cannot be correct! Suppose a good-hearted person wishes to alleviate the sufferings of those poor people, doing this out of loving-kindness and compassion, one cannot say that his or her merit will be little. Why? Because this person is compassionate, with good intentions.\\

In the case of Kok\=aliya he is said to have uttered envious and untruthful words about the two enlightened disciples in the Buddha's presence. Result; a frightful disease followed by a long stay in exceedingly uncomfortable surroundings. Had he uttered the same words about an ordinary fellow-monk or lay-supporter, presumably his kan"nic result would have been quite insignificant. Does this sound like the rational teaching of karma and its results? No, because there is an unexplained and probably inexplicable cause or condition involving the noble attainments of the great disciples, which condition has seriously lengthened his terrible karmic results.\\

It is proper to mention here that the other side of this doctrine, doing good towards noble disciples, such as great generosity by donors, has also a deleterious result. As they are reputed to be Awakened, such Teachers attract lay-donors who wish to make merit, sure that their gifts will bring them good results in the future. These Teachers are then showered by gifts which have no place in their lives and which may be difficult for them to dispose of.\\

Summing up this point; the doctrine of variable results of karma according the recipient's spiritual status is at least questionable though widely believed in. It may be objected that the Dharma's working are not mechanistic or totally rational. This is true for the Dharma in some aspects transcends rationalism, as for example the presence of supernormal powers in some living teachers or the bodily relics in the ashes of some great teacher's cremation.\\

Hell's depictions in many religious traditions have remarkable similarities. Such murals and paintings on cloth and in books are usually explained as efforts to evoke fear of retribution among evildoers. Whether such pictures are successful or not the writer does not know though controlling the impulses to evil by evoking fear cannot be the best way of teaching people Dharma. But he does know that Teacher, Acharn Singthong, in N.E. Thailand, controlled an outbreak of rustling water-buffalo in the locality by such means. This Teacher, highly respected by the villagers, gave to them on a Full Moon night a sermon lasting about three hours on the hell-realms. The audience of monks, nuns and laypeople, were spellbound and no one even changed position on the hard boards of the floor of Wat Pa Geow's hall. One could have heard a pin drop. Buffalo-rustling ceased immediately. perhaps it is true then that some people will only practise the Dharma after being scared of the results of evil-doing.\\

As a matter related to this Sutta it should be remembered that without the operation of the senses and the assembling of knowledge by mind there are no worlds, indeed, no existences. So when people raise the question, `Hells? Where are they?' the answer is that they exist where all worlds exist, including the one that we are aware of now, in the mind. One does not have to go anywhere to reach hell, just make the appropriate karma and hell is here. lt can continue to be here (where else would it be?) when an evildoer dies. Just as we make our own karmas to produce and continue with human life, so it is with other possibilities for existence, including hell. `The mind goes before all dharmas' as the famous first verse of the Dhammapada says.\\

Moving on now to examine the Sutta's structure, its composite nature soon becomes obvious. Kok\=aliya's story and some of the verses here (Sn. 657-660) have been popular and are preserved also at S.l.6.l0 and at A.Tens.89. This seems to be the earliest version around which other material has been added. For instance, Sn.661-662 are also Dhp.306\&125, while the remaining verses, Sn. 663-678 are an addition found only in Sn. These sixteen verses may be the work of one author who has written them in an unusual metre. 

\subsection{Particular points raised in the Sutta.}
In the introductory prose there is an example of a verbal convention in the Buddha's days: that of repeating a question or statement three times and receiving an answer also thrice repeated. ln this case Kok\=aliya repeats his allegations against the chief disciples three times while the Buddha warns him not to accuse them as they are `very friendly'. Indians in those days, not only Buddhists, according to the P\=ali Suttas seemed to regard this thrice-repeated, statement or question as bringing to a head or finally resolving the matter in hand. This could be illustrated in this way:

\begin{itemize}
\item   Statement/accusation
\item   Reply/warning
\item   x 3 = karmic result for speaker.
\item   Other examples: Sela questioning Keniya (Sn. 111.7)
\item   Did you say `Buddha?'
\item   Yes, l said `Buddha'
\item   x 3 = result: Sela and disciples went to the Buddha and were awakened.
\end{itemize}

Many more may be found in other Sutta collections. After Kok\=aliya has spoken three times against the pair of chief disciples and been reprimanded for this by the Buddha, he is recorded to have done a rather astonishing thing. Not only does he pay his respects to the Buddha (by lotussing his hands and bowing down probably from the standing position) but also as a greater mark of respect still he departs keeping the Buddha on his right side, that is, circumambulating him in a clockwise direction. Of course we shall never know if this is merely a commonly repeated phrase in P\=ali, or whether the miscreant Kok\=aliya actually did so. Though he could have done this out of mockery of the usual conventions of reverence, it could also be that his enmity was not at all directed at the Buddha but an obsession directed to the chief disciples.\\

Kok\=aliya's death through increasingly large `boils' needs a note upon the seeds and fruit they are compared to. Mustard seeds are very small, less than a thirty-secondth of an inch across. Green-grain is one of a family of similar pulses. which can be used for making Dahl. Chickpeas are also a familiar ingredient of Indian cookery to this day but being covered with boils of this size — up to one half-inch diameter, would already be very serious. Jujube fruit kernels would be larger assuming that this means what is now known as `Chinese dates' are in fact \textit{Zizyphus jujuba} and the whole fruit larger still. Myrobalan (\textit{Terminalia} spp. from India) produces an astringent fruit widely used in Ayurveda, and by Buddhist monks, the green fruit exceeding at least two jujubes. Bael (\textit{bilva} in P\=ali) is a tree in the Citrus family producing a good-tasting digestive flesh which is reached only by cracking a hard shell. Fruits vary in size from small ones easily held in one hand to `two-handers'. One shudders to think of `boils' of this size.\\

Another matter worth noting is the use of Brahma Sahampati as messenger of Kok\=aliya's fate. He serves to reinforce the Buddha's authority as the supposed utterer of this sutta. He is pictured as appearing to the Buddha in the last hours of the night and telling him what had happened to Kok\=aliya. After his disappearance the Buddha then relays this information to the bhikkhus and explains, in answer to a question, about the length of life in the Paduma Hell.\\

This is then explained, supposedly by the Buddha, in a complex piece of Buddhist arithmetic, all of which may be summed up by a more convenient expression in English as `a very, very long time'. The combination of Brahma Sahampati's appearance and the Buddha's `mathematical' calculations and the subsequent verses upon the horrors of hell may give us pause for thought when it comes to this Sutta's authenticity.\\

Various translators introduce into these verses the term `warders of hell' for which there is no word in the P\=ali text. These are perhaps assumed to exist and if this is so — certainly later Buddhist texts assume that such `warders' exist - then it raises a question. The inhabitants of the various states of existence including this human one, have appeared there (or here) due to the karma that they have created — so much is clear. But what are we to understand about `warders' who are said to intensify the sufferings of the inmates, as though they were commercially frying fish and chips, flipping them over and stirring them round in boiling oil? Are these supposed 'warders' present because they too have made much evil karma? But why should they have power over other inmates? One assumption is that they indeed have made such karma. Another, more subtle explanation is that the perception of `warders' by the inmates like the rest of the hellish landscape, is mind—made by the latter. This properly solves the presence of hellish `devils' as depicted not only in Buddhist art but in Christian murals also.\\

From verse 663 to the sutta's conclusion I have based my translation upon E.M. Hare's verses and occasionally used whole lines of his as he has often struck an appropriately colourful or gory note.

\chapter{N\=alaka Sutta\\ The sages Asita and N\=alaka and the  Buddhas advice}

\section{prologue - Telling the story}

\begin{MyDescription}[]{Narrator:}
\end{MyDescription}
\begin{tabbing}
In midday meditation \hspace{1.5cm} \=      The sage Asita saw\\
brilliantly arrayed \> the thrice-ten deva troop\\
happy and joyful \> waving flags the while\\
with Sakka their superior \> all highly elated\\
Then when he had seen \> the devas so delighted\\
respectfully he greeted them \> and questioned them like this\\
\end{tabbing}


\begin{MyDescription}[]{Asita:}
\end{MyDescription}   
\begin{tabbing}
Why this deva-sangha is \hspace{1.5cm} \= so exceedingly joyful\\
they've brought along banners \> for brandishing about?\\
Even when the devas \> battled anti-gods\\
with a win for deva-hosts \> and loss for demon-hordes\\
then was no such celebration \> so what have devas seen?\\
what wonder have they heard? \> Why devas are delighted?\\
They whistle and they sing \> clap hands and strum sitars\\
with dancing and with music \> so they celebrate\\
O you deva-dwellers \> on Meru's airy peaks,\\
I beg you, good sirs, \> soon dispel my doubts.
\end{tabbing}

\begin{MyDescription}[]{Devas:}
\end{MyDescription}   
\begin{tabbing}
A Bodhisattva has been born \= at the sakyans' city\\
in lands along Lumb\a=ani- \> precious gem beyond compare-\\
for the weal and welfare \> of those in the human realm.\\
He, best beings of all \> foremost among mankind\\
mighty bull among men, \> of creatures all supreme,\\
will revolve the wheel \> in ancient seers woods\\
Likened to a roaring lion \> mightiest of beasts.\\
\end{tabbing}
\begin{MyDescription}[]{Narrator:}
\end{MyDescription}   
\begin{tabbing}
Having known this news \= then the sage in haste\\
in mind descended to \> Suddhodhana's abode\\
sat he down and said\\
\end{tabbing}

\begin{MyDescription}[]{Asita:}
\end{MyDescription}   
\begin{tabbing}
Where then is this prince\= I Wish to see him now\\
\end{tabbing}

\begin{MyDescription}[]{Narrator}
\end{MyDescription}   
\begin{tabbing}
So Sakyans he beseeched\\
then to him Asita named \hspace{1.5cm} \= did sakyans show their son\\
the prince in colour clear \> as rays from shining gold\\
burnished and illustrious both \> of supernal hue\\
Joy with rapture great \> filled Asita's heart\\
on perceiving this young prince \> bright as fiery flame,\\
purity like lunar lord \> stars herding through the sky\\
dazzling as the sun \> on cloudless autumn days.\\
Sky beings all above \> carried canopy of state\\
of many-tiered parasols \> as well as gold handled whisks\\
but no one saw the bearers \> of whisks and parasols.\\
The sage with dreadlocked hair \> also kanhasiri called\\
seeing then the prince \> golden jewel upon brocade,\\
white parasols of state \> held above his head-\\
received him in his arms \> with gladdened mind and joy\\
As soon as he received \> the foremost sakyan man\\
he, skilled in lore of signs \> and mastery of mantras,\\
exclaimed\\
\end{tabbing}

\begin{MyDescription}[]{Asita}
\end{MyDescription}   
\begin{tabbing}
Highest unexcelled \hspace{1cm}\= among the race of men\\
but recollected then \> that soon, so soon he'd die\\
\end{tabbing}

\begin{MyDescription}[]{Sakyans}
\end{MyDescription}  
\begin{tabbing}
Seeing the sobbing prince \hspace{0.5cm} \=  the sakyans asked of him\\
`Surely for this prince \> no peril will befall?'
\end{tabbing}

\begin{MyDescription}[]{Asita}
\end{MyDescription}  
\begin{tabbing}
The sage in answer said \hspace{0.5cm}\= to anxious sakyans throng\\
`No fears do I foresee \> to come upon the prince\\
nor any harm at all \> in future will befall\\
nor he's unfortunate \> so do not be depressed,\\
for he will touch upon \> Enlightenment divine\\
and turn the Dharma wheel- \> Seer of perfect purity\\
with compassion for the many \> set forth the goodly life.\\
But with only brief \> time left within my life\\
while in this time I'll die \> having no chance to hear\\
the dharma of that one \> of power incomparable\\
this saddens me so \> such loss distresses me'.\\
\end{tabbing}

\begin{MyDescription}[]{Narrator}
\end{MyDescription}  
\begin{tabbing}
having roused in Sakyans \hspace{0.5cm} \= this joy profound, the sage\\
keeper of pure precepts \> left inner palace suites.\\
Then of his compassion \> to his sister's son set out\\
arousing in him interest \> in the Dharma deep:\\
\end{tabbing}

\begin{MyDescription}[]{Asita}
\end{MyDescription}  
\begin{tabbing}
`From persons having heard \hspace{0.5cm}\= the sound of `Buddha' word\\
who sambodhi attained \> practising the dharma-path\\
go there, then question him, \> as his disciples live with him\\
practice with that radiant lord \> precepts of purity'
\end{tabbing}

Narrator

\begin{tabbing}
So,instructed by him, \hspace{1cm} \= whose mind set on benefit,\\
who foresaw in future time \> perfect purity complete\\
that N\a=alaka, his nephew \> much merit stored away,\\
with guarded senses waited \> in expectation of the victor.\\
Having heard of the victor's \> revolution of the noble wheel\\
he went to him and saw him \> that prime among the saviours\\
and trust arose in him \> in the greatest sage\\
then he enquired upon \> the Silentness supreme,\\
thus coming to fulfill \> the sages wish.\\
\end{tabbing}

\stepcounter{stanza}
\stepcounter{stanza}
\stepcounter{stanza}
\stepcounter{stanza}
\stepcounter{stanza}
\stepcounter{stanza}
\stepcounter{stanza}
\stepcounter{stanza}
\stepcounter{stanza}
\stepcounter{stanza}
\stepcounter{stanza}
\stepcounter{stanza}
\stepcounter{stanza}
\stepcounter{stanza}
\stepcounter{stanza}
\stepcounter{stanza}
\stepcounter{stanza}
\stepcounter{stanza}
\stepcounter{stanza}

\begin{MyDescription}[\arabic{stanza}]{N\=alaka}
Having understood Asita's speech-\\
that it accords with truthfulness,\\
Gotama, we question you\\
on dharmas gone to the further shore.
\stepcounter{stanza}
\end{MyDescription} 

\begin{MyDescription}[\arabic{stanza}]{}
I came to homelessness but now I wish\\
as a bhikkhu to behave,\\
speak to me, Sage, as l request\\
on the highest state of Silentness.
\stepcounter{stanza}
\end{MyDescription} 

\begin{MyDescription}[\arabic{stanza}]{The Buddha:}
Knowledge of Silence I'll convey,\\
hard to do , to master difficult,\\
so be both firm and resolute\\
and I'll speak upon this thing.
\stepcounter{stanza}
\end{MyDescription} 

\begin{MyDescription}[\arabic{stanza}]{}
In town there's always praise and blame\\
so practise even-mindedness,\\
guard against faults of mind -\\
fare calm and free from arrogance.
\stepcounter{stanza}
\end{MyDescription} 

\begin{MyDescription}[\arabic{stanza}]{}
As crown-fire crests\\
and forest-fuel flies up,\\
so do women tempt the sage\\
but be not by them tempted.
\stepcounter{stanza}
\end{MyDescription} 

\begin{MyDescription}[\arabic{stanza}]{}
Refrain from sexual dharmas\\
whether pleasures fine or coarse,\\
be not attached, repelled\\
for beings weak or strong.
\stepcounter{stanza}
\end{MyDescription} 

\begin{MyDescription}[\arabic{stanza}]{}
Comparing others with oneself —\\
`As I am so are they' and\\
`As they are so am I' —\\
kill not nor cause to kill.
\stepcounter{stanza}
\end{MyDescription} 

\begin{MyDescription}[\arabic{stanza}]{}
Wishes and greed give up to which\\
ordinary persons are attached,\\
be one-with-vision and set out\\
to go across this hellish state.
\stepcounter{stanza}
\end{MyDescription} 

\begin{MyDescription}[\arabic{stanza}]{}
Empty-bellied, with little food,\\
few in wishes, greedless too,\\
the wishless he and hungerless\\
the wishless come quite Cool.
\stepcounter{stanza}
\end{MyDescription} 


\begin{MyDescription}[\arabic{stanza}]{}
The sage on almsround having walked\\
going then to lonely woods\\
and drawing near the roots of a tree\\
takes a seat just there.
\stepcounter{stanza}
\end{MyDescription} 


\begin{MyDescription}[\arabic{stanza}]{}
Firmly intent on jh\=ana\\
and delighting in the woods\\
who at the tree-roots meditates\\
satisfies himself
\stepcounter{stanza}
\end{MyDescription} 


\begin{MyDescription}[\arabic{stanza}]{}
until the end of night\\
when to a village he goes,\\
there, by gifts not pleased\\
nor by invitations.
\stepcounter{stanza}
\end{MyDescription} 


\begin{MyDescription}[\arabic{stanza}]{}
The sage to village come\\
hastens not among the houses,\\
but cuts off talk while seeking food\\
and refrains from hints.
\stepcounter{stanza}
\end{MyDescription} 


\begin{MyDescription}[\arabic{stanza}]{}
`Good it is that l have gained',\\
`good that I have not as well'.\\
 One such thinks both alike\\
 returning to his tree.
\stepcounter{stanza}
\end{MyDescription} 

\begin{MyDescription}[\arabic{stanza}]{}
Going about with bowl in hand\\
not dumb but others think him so,\\
he does not scorn a trifling gift\\
nor despise its donor.
\stepcounter{stanza}
\end{MyDescription} 

\begin{MyDescription}[\arabic{stanza}]{}
Refined and basic practices\\
the Samana's made clear,\\
but Beyond with both they not go\\
nor through one only experience.
\stepcounter{stanza}
\end{MyDescription} 

\begin{MyDescription}[\arabic{stanza}]{}
in who no craving's left —\\
that bhikkhu cut across the stream\\
`should do, should not do', given up,\\
in him no fever's found.
\stepcounter{stanza}
\end{MyDescription} 

\begin{MyDescription}[\arabic{stanza}]{}
Further than this the Radiant said\\
Wisdom Still I teach:\\
Be like a razor's edge,\\
tongue-tip upon the palate,\\
thus be restrained in belly.
\stepcounter{stanza}
\end{MyDescription} 

\begin{MyDescription}[\arabic{stanza}]{}

Be not indolent in mind\\
but neither think too much\\
and be free from all carrion-stench:\\
aim at life of purity.
\stepcounter{stanza}
\end{MyDescription} 

\begin{MyDescription}[\arabic{stanza}]{}
Train yourself in solitary life,\\
the way of life of samanas,\\
take high delight in being one\\
its called the Singleness.
\stepcounter{stanza}
\end{MyDescription} 

\begin{MyDescription}[\arabic{stanza}]{}
With this you will shine forth\\
in all directions ten\\
then the praises of the wise, those skilled\\
in meditation — sensuality let go,\\
as one loving me, you'll all the more\\
grow in faith and modesty.
\stepcounter{stanza}
\end{MyDescription} 

\begin{MyDescription}[\arabic{stanza}]{}
Know this from waters' flow —\\
those by rocks and pools\\
such rills and becks gush noisily,\\
great waterways flow quiet.
\stepcounter{stanza}
\end{MyDescription} 

\begin{MyDescription}[\arabic{stanza}]{}
What is unfilled makes noise\\
but silent is what's full,\\
the fool is like the pot half-filled,\\
the wise one's like a lake that's full.
\stepcounter{stanza}
\end{MyDescription} 

\begin{MyDescription}[\arabic{stanza}]{}
when a samana speaks much\\
full of goodness and meaning:\\
Knowing Dharma he speaks,\\
Knowing he speaks so much.
\stepcounter{stanza}
\end{MyDescription} 

\begin{MyDescription}[\arabic{stanza}]{}
Who Knows is self-restrained,\\
Knowing he speaks not much:\\
Worth a Sage to Silence,\\
a Sage to Silence reached.
\stepcounter{stanza}
\end{MyDescription} 

\begin{MyDescription}[(Sn. 679-723)]{}
\end{MyDescription} 



   

\section{Notes upon Sundry Words}

Subjects commented upon are italicized.\\

The Prologue\\

This visionary beginning to the Sutta has been translated rather into the form of alliterative Anglo-Saxon verse, perhaps not very successfully. Whether that is so or not a few matters deserve to be explained to the reader or chanter.\\

The Sage Asita appears rarely in P\=ali Suttas but as an important and well known person of those days. Obviously well-practised in jhana he had access through his meditation to visions and other worlds. On this occasion he saw the `Thirty-three' (sometimes called the 30 — for the Pali legend about them, see Dhammapada Commentary translated as Buddhist Legends, volume 3, p.315 ff). This realm of devas had a leader usually called \textit{Sakka} (skt. Sakra also Inda/Indra) who according to the Suttas became a devoted disciple of the Buddha. Naturally, Asita wanted to know why these devas were jubilant and so enquired.\\
 
He mentions a war with the anti-gods fought by these devas at some past time. Yes, as this is still the K\=ama-world, there is war even though it is a heaven! K\=ama, even when it is subtle, still produces conflict. The any'-gods is a translation of the P\=ali `a + sura', which literally means `not + god'. There are many stories about the asuras and their quarrelsome nature.\\


The Thirty-three devas are celebrating the birth of our Bodhisatta, the prince Siddhattha who later became the Buddha of our times. After his enlightenment beneath the Bodhi tree, he \textit{`revolved the Wheel' of the Dharma} (Dhamma-cakka-pavattana). This phrase has a definite meaning beyond the fact that he `taught Dharma'. The Wheel, seen on so many temples, images, sculptures, flags and so on throughout the Buddhist world, is a symbol of movement, something that is not static. Originally a symbol used by the greater Indian monarchs as a sign of their power and authority, it was tamed by the Buddha to represent the power for good of the Dharma which would increase both in its exterior presence in this world through such things as temples, monks and nuns, but increase also in the student's mind for his or her comfort. Unlike the usual run of royalty (and these days, presidents) Dharma conquers not by violence but through its innate truthfulness, through its advocation of loving- kindness and compassion. The Wheel of Dharma always revolves because that truthfulness is always true though the present Buddha's teaching of it may in time be forgotten. Hence it is known as the `Saccadhamma' — the truthful Dharma which applies everywhere and at all times. A Buddha may be said to the wheel a heave to keep it running!\\

The phrase under discussion here occurs three times in this Sutta and has been rendered as \textit{`revolve the wheel', `turn the Dharma-wheel} and \textit{revolution of the noble wheel'}\\

\textit{Lunar Lord} a few verses further on is a fancy name for the moon and his herding of the stars through the sky - charming poetry.\\

The panoply of state, the \textit{parasols} held above rulers and the yak-tail fly-whisks and so on that traditionally accompany a rajah — possibly now found only in Thailand- are pictured in these verses, an ancient tradition indeed. The author was a bhikkhu in Thailand at the time of King Bhumibol Adulyadej's Fourth Cycle 4 x 12 = 48 years old) when he processed round the capital with right royal traditional splendour. All shops were closed and no one gazed out from upstairs windows or balconies but pavements were packed and silent with no cheering or clapping. The procession of elephants with the king mounted upon the first accompanied by royal retainers in traditional Thai costume was very impressive. Such precious articles as parasols and whisks, as described in the Sutta, were much in evidence and solemnity of the occasion emphasized by the crowd's silence. It was broken only by a single walking official play a tiny instrument, the high notes of which were punctuated every few steps by the single beat together of twenty—four drums. Altogether an awe-inspiring event with the monarch dismounting from his massive elephant to visit the various temples upon his route and pay homage there.\\

Returning more strictly to the Sutta, note that the Buddha throughout Sn. is referred to as a \textit{Jina — Victor or Conqueror —} of whatever it is in ourselves which prevents the Seeing of things-as-they-really-are. This occurs here in a passage of narrative as the Buddha generally is not shown as calling himself by this title. (Jina is also used by Jains, that is followers of the Jinas, as a title for their supreme teachers). The Buddha however does refer to himself as a Tath\=agata One who has come from/gone to Thusness), or a Samana (One who's at peace/equanimous).\\

In the same narrative passage Asita enquires about the \textit{Silentness supreme}. This one translator's effort to render \textit{moneyya\"y}, an abstract noun connected with \textit{muni}, a sage. Other renderings are `best of sagehoods', `highest wisdom' and `still wisdom's crown', while the Pali has `moneyya\"y uttama\"y padam'. \textit{`Muni'} is a tricky word to translate into English. Its best known context in English is as an -epithet of the Buddha: Sakyamuni or Shakyamuni. In P\=ali it is Sakiyamuni and usually translated as Sage of the Sakiyas. It is also well-known in a Sanskrit mantra: Om muni muni mah\=amuni sakyamuniye svaha. In this book muni is translated, `Sage'.\\

But to say this does not exhaust the meaning of muni which cannot in fact be translated by one word. This is because the root of this word is also connected to silence — the deliberate Hindu practice of not-speaking, and hence to the development of wisdom. In the Vinaya-pitaka an incident is recorded in which some bhikkhus make a pact not to speak to each other during their three months of Rains-retreat. When this was finished and they journeyed to meet the Buddha he enquired how they had practised and they told him of their deliberate silence: he reproved them for this, calling their practice acting like animals. They should not behave so but should communicate, he said, gladdening each other with Dharma. This silent practice in Skt. is called mauna or in P\=ali mona. It seems that the Buddha disapproved of this common Indian practice counting it as an extreme action not fitting those who practised the Middle Way. It is obviously different if solitary meditators practise silence. This sort of silence, \textit{moneyya\"y} or \textit{mona}, is really inner silence, not struggling with oneself not to talk, naturally not talking if there is no good reason to do so and it arises from a kind of wisdom gone beyond words. This kind of wisdom is expressed through the verb `mun\=ati' - to be one-with-silent-wisdom, or a mum, a silent sage. All this should be borne in mind (!) when the word `sage' is seen in this translation.\\

Coming to vs. 75, note its last line, \textit{`kill not nor cause to kill.'} Similar theme and the same refrain is found in Dhp. 129-130.\\

The next verse in P\=ali contains the word `cakkhum\=a'or `one~with-vision' which is literally one-with-eyes. See the A. Sutta, Threes, 29.\\

Shortly we arrive at vs. 714 which is accompanied in my manuscript translation with an exclamatory, Note! This is a warning that this difficult verse is translated in different ways by E.M Hare, H. Saddhatissa, K.R. Norman with I.B Homer, as well as W. Rahula, and N.A. jayawickrama. Some of these versions though they have been printed make no sense or are mere literal versions word by word. One or two of them resort to the P\=ali Comy's involved `explanation'. The present translation tries to make sense of the P\=ali, (could there be an ancient corruption?) just as it stands.\\

The first line contains the word `ucc\=ava\"y' which most of the above translators have rendered `high and low'. However, as this refers to Dharma practices, while `high' ones does not sound out of place, surely it is inappropriate to talk of `low' ones. hence my translation of `refined and basic'.\\

The Samana in the second line is the Buddha talking modestly of himself, since samana was a common term for wandering monks. But this samana did not just teach these two types of practices but made them clear, clarified them or illuminated them. The P\=ali vb. `pak\=asita' has this meaning of shining, clarifying.\\

The second two liens have confounded all the above translators and the only advantage of this one is that it does make sense, though whether it represents the original meaning will be for future scholars to decide. The words `diguna\"y' and `ekaguna\"y' have, in the previous translations, been rendered `twice' and `once' but I have used `both' and `one' which makes better sense. Now, I assume that the verse concerns the two sorts of practice mentioned in the first two lines. The second two seem to mean that practitioners, only referred to with `they', do not go to the Beyond, that is Liberation or Nirvana, by means of either, both sorts of practices, nor with only one sort. What does this mean? This explanation seems best: some people think that only one sort of practice is needed — either the refined or the basic; others assume that both are required, the refined and the basic. Both these people hold a view about practice and while they do so neither will successfully find the Beyond. They will not find it as they have not let go of their views.\\

But even this explanation still begs the question: what is basic practice and what, refined? Asking this question assumes that `ucc\=avaca' has this meaning of high/low, or basic/refined. It can also mean `various'. Basic (or `low') might refer to the various ways of making merit ***???(puiwa)***??? — for these see Tens Ways of Making Merit (dasa-puiwa-kirya-vatthu) in the Comy to the Ud\=ana, or in the fifth chapter of the Up\=asakajan\=ala\"ykara. Both works can be obtained from the P\=ali Text Society but neither has been translated. ln any case, no practitioner of Dhra1a can dispense with making good karma which of course involves kindness and compassion.\\

If making good karma, (puiwa) or merit, counts as basic practice what then would be the refined? Many would think that the more subtle types of practice would qualify, such as vipassan\=a or insight meditation or, outside Therav\=ada, refined samadhi experience in Ch'an/Zen, or tantric methods mostly found in Tibetan Buddhist sects, or of course Dzogchen. But then if these are designated as `refined' practices and thought of in that light with respect to oneself will it not sound like conceit? Selfless practice would be best.\\

`in all directions ten' (dasa-disa) is often used in verse to mean everywhere, all round, in every direction (vs. 719). In the same verse there is `growing in faith and modesty'. Modesty (hiri) has been commented on in the Notes following the Hiri Sutta (Sn. 253-Z57). Upon saddh\=a it needs only be said that the English translation of this as `faith' is not very accurate. The English word is, after all, used in Christian contexts where one is expected to believe, in some cases without question, whatever the church asserts to be true. But Buddhists call such beliefs `views' which unqualified by any adjectives always means `wrong views' — simple because they are not or cannot be questioned. So saddha in Buddhist teachings means more like `confidence' which deepens with practice and is balanced by wisdom (pa\~n\~na/praj\~na). Saddha would be ill-translated by `belief'.\\

In the line before this there is the word m\=amaka, literally `one who make mine' but meaning `one who loves me' or is `devoted to me'. The person with saddh\=a makes the Buddha his or her own and cannot be shaken from his teaching.   
  	

  
\chapter{Dvayat\=anupassan\=a Sutta\\ Contemplation of Pairs}


\part{The Chapter of Eights\\  A\~n\~nakavagga}
   
   
\chapter{K\=ama Sutta\\ Objects, Desires and Pleasures}

\stepcounter{stanza}
\stepcounter{stanza}
\stepcounter{stanza}
\stepcounter{stanza}
\stepcounter{stanza}\stepcounter{stanza}
\stepcounter{stanza}
\stepcounter{stanza}
\stepcounter{stanza}
\stepcounter{stanza}\stepcounter{stanza}
\stepcounter{stanza}
\stepcounter{stanza}
\stepcounter{stanza}
\stepcounter{stanza}\stepcounter{stanza}
\stepcounter{stanza}
\stepcounter{stanza}
\stepcounter{stanza}
\stepcounter{stanza}\stepcounter{stanza}
\stepcounter{stanza}
\stepcounter{stanza}
\stepcounter{stanza}
\stepcounter{stanza}\stepcounter{stanza}
\stepcounter{stanza}
\stepcounter{stanza}
\stepcounter{stanza}
\stepcounter{stanza}\stepcounter{stanza}
\stepcounter{stanza}
\stepcounter{stanza}
\stepcounter{stanza}
\stepcounter{stanza}\stepcounter{stanza}
\stepcounter{stanza}
\stepcounter{stanza}
\stepcounter{stanza}
\stepcounter{stanza}
\stepcounter{stanza}
\stepcounter{stanza}

\begin{MyDescription}[\arabic{stanza}]{}
If one with a desiring mind\\
Succeeds in ???? sensual pleasure,\\
A mortal such is pleased in mind\\
With wishes all fulfilled.
\stepcounter{stanza}
\end{MyDescription}


\begin{MyDescription}[\arabic{stanza}]{}
But if from this person passionate\\
All of these pleasures disappear\\
Then does this pleasure-addict feel\\
As though by arrows pierced
\stepcounter{stanza}
\end{MyDescription}


\begin{MyDescription}[\arabic{stanza}]{}
The one who shuns these pleasures of sense\\
Like treading not on a serpent's head,\\
Such a one with mindfulness\\
This tangled world transcends.
\stepcounter{stanza}
\end{MyDescription}

\begin{MyDescription}[\arabic{stanza}]{}
Obsessed with fields and property\\
With money, estates and those employed,\\
With many pleasures, women and kin,\\
Such a person greedily --
\stepcounter{stanza}
\end{MyDescription}

\begin{MyDescription}[\arabic{stanza}]{}
Do weaknesses bring down indeed,\\
By dangers is that person crushed\\
And then by dukkhas stuck against —\\
As water into broken boat.
\stepcounter{stanza}
\end{MyDescription}

\begin{MyDescription}[\arabic{stanza}]{}
So let a mindful one avoid\\
At every tum these sense-desires,\\
With them abandoned, cross the flood\\
As boat is baled for the Further Shore.
\stepcounter{stanza}
\end{MyDescription}

\begin{MyDescription}[(Sn. 766-771)]{}

\end{MyDescription}



\section{The famous word `K\=ama'}

First, we can consider its range of meaning, a range which no one word in English can cover. As it is a very important word, used in so many different ways, there are only two choices to make in its translation. We may choose not to translate it letting the context of its use bring out its sense but this has the disadvantage that its meaning may not be revealed in full. ln this Sutta-nip\=ata translation I have not chosen this alternative. The second of these is to use appropriate and different English words to render the various meanings of `K\=ama'. This also has a disadvantage as the full range of this word is not apparent if one reads only English. This note, then is to bring out these varied meanings of `K\=ama' as well as drawing attention to compound words in P\=ali of which `K\=ama' is part. \\

\underline{K\=ama meaning `desire'.} This means `desire based on the senses'. A sense-object is seen (heard, etc.), by way of the organ of the eye or ear (no problem so far!), it registers in the mind and is identified with a name (this is the operation of sauna which includes memory as part of perception (still no trouble!). Then desire — k\=ama — may arise wanting that which has been perceived. This is where we may have difficulties — maybe we get what we want but still are not satisfied (though all sense-objects are impermanent), or we don't get it and so suffer in another way. K\=ama as sense-desire is very much tied up with my idea of my self, so even its fulfilment is a limitation. Desires of this kind, as we learn from so many places in the P\=ali Suttas, are compared to a blazing fire. Stoke up the fires of desire and suffer even more!. Our materialistic culture with its unending advertisements stokes these fires and in doing so feeds the fires, so that the sufferings of no peace, satisfaction of true happiness can be had in the long run.\\

Of course, as with the English word `desire' — it may be used in good beneficial contexts — so the word `k\=ama' also stretches to cover beneficial matters. One may desire the Dharma and in P\=ali one would be spoken of as `dhammak\=amo'. One may also desire the benefit of others, even their Liberation — this desire also falls under the word `k\=ama'. People sometimes ask, `But can I say that I desire Nirv\=ana?'. A reply to this might be that at the beginning of Buddhist study and practice one may desire to experience Nirv\=ana but as practice progresses the desire for that fades away as the Path to it and Nirv\=ana meld together.\\

Sense-desires are very varied, some very refined (as desire for sublimely peaceful states of mind), and some much grosser (as with eating delicious food, or of course for sex). In the latter contexts it is appropriate to use the words `sensual' and `sensuality' as translations of k\=ama, and when k\=ama is part of the compound `k\=amar\=aga', sex and sexual are definitely indicated. Raga by itself and combined with k\=ama- indicates `lust'.\\

\underline{K\=ama as sense-objects, the world of k\=ama.} So far k\=ama as a component of mind has been mentioned, but the word also stretches to include the variegated nature of sense objects. In this translation of Sn. 50 in The Rhino's Horn you will find —

\begin{MyDescription}[]{}
Sense-desires so varied, sweet,\\
in divers forms disturb the mind,\\
Seeing the bane of sense—desires\\
Fare singly as the rhino's horn.
\end{MyDescription}
   
This translation emphasizes sense-desires as an aspect of k\=ama. But it can also be translated —
   
\begin{MyDescription}[]{}
Things of sense so varied, sweet\\
in divers forms disturb the mind,\\
When danger's seen in things of sense\\
Fare singly as the rhino's horn.
\end{MyDescription}

`Things of sense' (Vatthuk\=ama) emphasizes the array of objects known by the way of eye, ear, nose, tongue and sense of touch. For a person with no sense-restraint they disturb the mind. English has no one word which will stretch over both the interior sense-desire and the exterior sense-objects. There are many examples of translators using the wrong meaning of the work k\=ama in their works.\\

Beyond desires and things of sense there is also k\=ama as enjoyment, sense-pleasure, sensuality/sexuality. In general, Theravada Buddhist teachings counsel that one should restrain one's senses and not indulge in this aspect of k\=ama. However, the Suttas have been preserved by bhikkus and emphasize their attitude to sense-restraint. Lay people in traditional Theravada countries tend to disregard this and enjoy life, unless there are secluding themselves for Dharma-practice as on the Uposatha days (full moon or new moon) or on a longer meditation retreat.\\

Turning away from k\=ama as in a monastic life, is very different from its natural enjoyment. The first is emphasized by the P\=ali K\=ama Sutta, the second by the Hindu (Sanskrit) K\=ama Sutra. In this book and other Hindu works on k\=ama the meaning of this word is not confined as some think, to sexual enjoyment. In fact they have treatises on civilized and refined enjoyment of all the senses: music fit for the time of day and the persons present, gardens and flower-arrangements, food and different sorts of incense, and so on. None of these things from a Buddhist viewpoint are in any way wrong or evil — they are just beautiful parts of this world — `things of sense so varied, sweet.' Whether they disturb the mind or not depends on how much Dharma-practice one has done. One who has gone far on the Path according to Theravada sources seems to be a person no longer interested in sense-objects, having few desires and little or no enjoyment in the different aspects of k\=ama. This suggests a rather dour character, serious and unsmiling. But the monastic Teachers that I have me usually had an excellent sense of humour, and some of them with their very earthy stories, had their audience roaring with laughter.\\

A rather different approach to k\=ama is found in some Tibetan Buddhist teachings such as Dzogchen. Here the array of sense-objects are looked upon as the ornaments of the world we live in. These ornaments are to be offered (which implies we let go of them, thus not controlling them but not renounce them), along with everything else: form (=body), sound, smell, taste, touch, the whole range of dharmas (=mind). The five great nectars, blood, that which arises from the fusion of the last two, the Awakened Heart, the `wheel' of practice — all offered infinitely. ln this tradition, all aspect of k\=ama, within the mind as desires outside in the world as sense-objects, and the enjoyments depending on them are to be integrated without the usual judgements — of this is good or that is bad. This allows what is repressed in mind to be liberated along with all the Qualities praised in the Dharma. Actions which would bring about the suffering of others through this process of integration are avoided by the samaya or relationship one has with Teachers and fellow-practitioners.

\chapter{Guha\~n\~naka Sutta\\ The Eight on the body as a cave}


\begin{MyDescription}[\arabic{stanza}]{}
The person who's to their body-cave\\
Clouded by many moods, and in delusion sunk,\\
Hard it is for that one, far from detachment\\
To abandon sensual pleasures in the world.
\stepcounter{stanza}
\end{MyDescription}

\begin{MyDescription}[\arabic{stanza}]{}
Bound the worldly pleasures of the past,\\
And hard to liberate are they in future time,\\
From others they're not free, not liberated.-\\
They're attached to past and the future too.
\stepcounter{stanza}
\end{MyDescription}

\begin{MyDescription}[\arabic{stanza}]{}
Those who are niggardly, who hank after pleasures,\\
infatuated they are, all their things — losses all!\\
But subjected to pain they lament their losses —\\
For how can all this be taken away, they wail?
\stepcounter{stanza}
\end{MyDescription}

\begin{MyDescription}[\arabic{stanza}]{}
Therefore should a person train,\\
Seeing the roughness of the world,\\
To take not to a wicked way,\\
For the wise say, life is short!
\stepcounter{stanza}
\end{MyDescription}

\begin{MyDescription}[\arabic{stanza}]{}
I see here trembling, fearful in the world,\\
That these people variously desiring different being —\\
Base people floundering in the jaws of death,\\
Not free from craving for repeated birth.
\stepcounter{stanza}
\end{MyDescription}

\begin{MyDescription}[\arabic{stanza}]{}
Look at them trembling with their egotistic selfishness,\\
Like fish in a stream fast drying—up,\\
Seeing it so, fare unselfish in this life\\
And cease worrying on different states of being.
\stepcounter{stanza}
\end{MyDescription}

\begin{MyDescription}[\arabic{stanza}]{}
No longer longing towards either extreme\\
Having understood touch, together with letting go,\\
One should do what others will praise and not blame,\\
A wise one is not stained by what is seen and heard.
\stepcounter{stanza}
\end{MyDescription}

\begin{MyDescription}[\arabic{stanza}]{}
The sage has known perception and crossed the flood\\
So with nothing tainted, nothing wrapped around,\\
Fares on in diligence with the arrow drawn\\
Neither longing for this world nor for another.
\stepcounter{stanza}
\end{MyDescription}

\begin{MyDescription}[(Sn. 772-779)]{}

\end{MyDescription}

\chapter{Du\~n\~na\~n\~naka Sutta\\ The Eight on the corruptions of the mind}

\begin{MyDescription}[\arabic{stanza}]{}
Some speak with wicked intent\\
while others are convinced their words are due\\
but whatever talk there is the sage enters no debate\\
and yet nowhere barren is the silent sage.
\stepcounter{stanza}
\end{MyDescription}

\begin{MyDescription}[\arabic{stanza}]{}
But a person led by his own desires\\
and then continuing accordingly finds it hard\\
to let them go, accepting his own thoughts as true\\
becomes one who speaks as a believer.
\stepcounter{stanza}
\end{MyDescription}

\begin{MyDescription}[\arabic{stanza}]{}
So if a person without being asked,\\
having practiced and praised virtues,\\
even those of himself, invented by himself,\\
the good say this is an ignoble act indeed.
\stepcounter{stanza}
\end{MyDescription}

\begin{MyDescription}[\arabic{stanza}]{}
But that bhikkhu who's serene at heart\\
and praises neither his own practices or virtue,\\
not labelling himself `I' in `this', the good praise him\\
`no arrogance has he for anything in the world'.
\stepcounter{stanza}
\end{MyDescription}

\begin{MyDescription}[\arabic{stanza}]{}
Who's thoughts, imagined and put together, then prefer\\
even though their source is not purified,\\
seeing advantage for himself, relies upon this,\\
depending on what is imagined, constructed and conventional.
\stepcounter{stanza}
\end{MyDescription}

\begin{MyDescription}[\arabic{stanza}]{}
When one has grasped from among many Dharma-doctrines\\
after due considerations one clings to a View\\
or condemns those of others\\
hence it's not easy to transcend those Dharmas.
\stepcounter{stanza}
\end{MyDescription}

\begin{MyDescription}[\arabic{stanza}]{}
There is not in the world such a purified person\\
who continues in these views about existential states\\
for this person of purity, let go of illusion and conceit,\\
how can he be in any way reckoned?
\stepcounter{stanza}
\end{MyDescription}

\begin{MyDescription}[\arabic{stanza}]{}
Who is not attached still enters into doctrinal debates\\
but one unattached how could he take sides?\\
For him is neither the view of self, or that not-self,\\
Will all views shaken off, relying on one.
\stepcounter{stanza}
\end{MyDescription}


\begin{MyDescription}[(Sn. 780-787)]{}

\end{MyDescription}
   
   
 
\chapter{Suddha\~n\~naka Sutta\\ The Eight on Purity}

\begin{MyDescription}[\arabic{stanza}]{}
`A pure one I see', free completely from disease\\
so by `seeing' such (it is said) one attains to purity.\\
Convinced about this and holding it highest\\
that one relies on this knowledge while contemplating purity.
\stepcounter{stanza}
\end{MyDescription}

\begin{MyDescription}[\arabic{stanza}]{}
But if a person by `seeing's' purified\\
or if through such knowledge could leave dukkha aside\\
then one with assets still by another could be purified:\\
this view betrays one who speaks in this way.
\stepcounter{stanza}
\end{MyDescription}

\begin{MyDescription}[\arabic{stanza}]{}
The Brahmin says not `that by another, one is purified' —\\
not by sights or by sounds, rites and vows and what's sensed.\\
Such person's not stuck upon merit or evil??\\
with selfishness renounced, constructing nothing here.
\stepcounter{stanza}
\end{MyDescription}

\begin{MyDescription}[\arabic{stanza}]{}
Former (things \footnote{`things': teacher, lover, view, objects etc.}) let go them to other (things) attached,\\
following craving, their bondage, they do not overcross,\\
so they (continue) with grasping and discarding,\\
as monkey letting go a branch to seize upon another.
\stepcounter{stanza}
\end{MyDescription}

\begin{MyDescription}[\arabic{stanza}]{}
A person undertaking (holy) vows goes high and low —\\
they waver, fettered by conditional perceptions.\\
But one who has learnt well and the Dharma penetrated\\
goes not up and down — that one of wisdom profound.
\stepcounter{stanza}
\end{MyDescription}

\begin{MyDescription}[\arabic{stanza}]{}
within all the dharmas whether seen or they're heard,\\
or otherwise sensed, this one fights not at all,\\
that one who sees them nakedly While faring to the end,\\
by whom in the world could he be described?
\stepcounter{stanza}
\end{MyDescription}

\begin{MyDescription}[\arabic{stanza}]{}
They neither form views, show nothing's preferred\\
nor do they claim a purity supreme,\\
having loosened craving's knot with which they were bound\\
no longer they have longing for what's in the world
\stepcounter{stanza}
\end{MyDescription}

\begin{MyDescription}[\arabic{stanza}]{}
Having Known, having Seen, there': nothing to be grasped\\
by a Brahmin gone beyond all limitations,\\
neither lustful with lusts nor to listlessness attached —\\
in this there is nothing that's grasped as the highest.
\stepcounter{stanza}
\end{MyDescription}

\begin{MyDescription}[(Sn 788-795)]{}

\end{MyDescription}

\chapter{Parama\~n\~nhada Sutta\\ The Eight on the Ultimate}

\begin{MyDescription}[\arabic{stanza}]{}
Whoever should take to himself certain views\footnote{Marginal note: as thin is about a bhikku},\\
thinking them best, supreme in the world,\\
and hence he proclaims all others as low —\\
by this he does not become free from disputes.
\stepcounter{stanza}
\end{MyDescription}

\begin{MyDescription}[\arabic{stanza}]{}
In whatever is seen by him, heard and cognized,\\
vows and rites done — he sees profit in these\\
and so from his grasping at that very view\\
all others he sees as worthless, as low.
\stepcounter{stanza}
\end{MyDescription}

\begin{MyDescription}[\arabic{stanza}]{}
Intelligent people declare it a bond\\
if relying on one he sees others as low,\\
therefore should a bhikkhu rely not on rites,\\
on vows, on the seen, the heard and cognized.
\stepcounter{stanza}
\end{MyDescription}

\begin{MyDescription}[\arabic{stanza}]{}
And so in this world let him fashion no views\\
relying on knowledge\footnote{Marginal note: trad. knowledge}, rites and vows done,\\
nor let him conceive that he's on a par,\\
nor think himself low, nor higher than them.
\stepcounter{stanza}
\end{MyDescription}

\begin{MyDescription}[\arabic{stanza}]{}
Abandoning own views, not grasping (at more)\\
and even in knowledge not seeking support,\\
'mong those who dispute he never takes sides,\\
to the various views he does not recourse.
\stepcounter{stanza}
\end{MyDescription}

\begin{MyDescription}[\arabic{stanza}]{}
Having no bias for either extreme —\\
for being, or not, here, the next world,\\
for a bhikkhu like this there's no settling down\\
'mong dharmas seized and decided (by them).
\stepcounter{stanza}
\end{MyDescription}

\begin{MyDescription}[\arabic{stanza}]{}
Concerning the seen, the heard and cognized,\\
not the least notion is fashioned by him,\\
that one who's perfected grasps at no view,\\
by whom in the world could he be described?
\stepcounter{stanza}
\end{MyDescription}

\begin{MyDescription}[\arabic{stanza}]{}
Neither they're fashioned nor honoured at all —\\
those doctrines, they're never accepted by him:\\
Perfected, not guided by rites or by vows,\\
One Thus, not returning, beyond has he gone.
\stepcounter{stanza}
\end{MyDescription}

\begin{MyDescription}[(Sn. 796-803)]{}

\end{MyDescription}

   

\section{Commentary}

\chapter{Jar\=a Sutta\\ Ageing and decay}

\begin{MyDescription}[\arabic{stanza}]{}
Short indeed is this life —\\
within a hundred years one dies,\\	
and, if any live longer\\
then they die of decay.
\stepcounter{stanza}
\end{MyDescription}

\begin{MyDescription}[\arabic{stanza}]{}
People grieve for what is `mine':\\
though possessions are not permanent\\
and subject to destruction —\\
see this and homeless dwell.
\stepcounter{stanza}
\end{MyDescription}

\begin{MyDescription}[\arabic{stanza}]{}
In death it's all abandoned\\
yet still some think `it's mine';\\
knowing this, the wise to me devoted\\
should stoop not making it `owned'.
\stepcounter{stanza}
\end{MyDescription}

\begin{MyDescription}[\arabic{stanza}]{}
As one who's waking then sees not\\
the things that happened in sleep,\\
so the beloved are not seen —
departed and done their time.
\stepcounter{stanza}
\end{MyDescription}

\begin{MyDescription}[\arabic{stanza}]{}
People now are seen and heard\\
and this are called by name,\\
but alone will the name remain\\
in speaking of those gone.
\stepcounter{stanza}
\end{MyDescription}

\begin{MyDescription}[\arabic{stanza}]{}
In `mine-making' greedy, they do not let go\\
of sorrow, lamenting and avarice,\\
therefore sages leaving possessions\\
freely wander, seers of security
\stepcounter{stanza}
\end{MyDescription}

\begin{MyDescription}[\arabic{stanza}]{}
For a bhikkhu practicing in solitude,\\
keeping company with secluded mind,\\
of such a one are all agreed:\\
In being he'll not be seen again'.
\stepcounter{stanza}
\end{MyDescription}

\begin{MyDescription}[\arabic{stanza}]{}
In all matters the sage is unsupported\\
nothing that makes dear, nor un-dear,\\
sorrow and avarice do not stain that one,\\
As water does not stay upon a leaf\footnote{marginal note: `any leaf?'}
\stepcounter{stanza}
\end{MyDescription}

\begin{MyDescription}[\arabic{stanza}]{}
As a water-drop on lotus plant,\\
as water does not stain a lotus flower,\\
even so the sage is never stained\\
by seen, heard or whatever's sensed.
\stepcounter{stanza}
\end{MyDescription}

\begin{MyDescription}[\arabic{stanza}]{}
Certainly the wise do not conceive
upon the seen, the heard,` and sensed,
nor wish for purity through another
for they are not attached nor yet displeased.
\stepcounter{stanza}
\end{MyDescription}

\begin{MyDescription}[(Sn. 804-813)]{}

\end{MyDescription}

\newpage

\section{Reflections on the Jar\=a Sutta}

\begin{enumerate}


\item `Short indeed is this life.' When young the days are long and life has infinite possibilities —-. we think. Death then is something that happens to others, not to us. As we grow older live passes by more rapidly filled with many pleasures, pains and responsibilities. But when really old we remark that `l don't know where that week (or month or year) has gone'. So even with modern medical facilities `within a hundred years one dies'. Some do live longer but the Buddhist emphasis, contrary to the medical view supports only that the body should be kept alive as long as possible, is that A yu (long-life) should be accompanied and guided by j\~nana (clarity of mind or wisdom).

\item `Seeing' impermanence is very important, not just occasionally through the loss of dear people or possessions but deeply and thoroughly in one's heart through the arising and passing of thoughts whether they are holy ones or those based on greed, hatred and delusion — all should be known as impermanent. Whether this experience results in dwelling homeless or not depends upon one's circumstances. And the homelessness of the bhikkhu or bhikkhuni will not suit everybody. A kind of homelessness may be lived in a household life when there is little or no `mine-making' (mama\=ukara). This however presupposes that there is no longer the tendency to `l-making' (ah\=ink\=ara).

\item People often think that their possessions, in which they may include their bodies and some aspects of their minds which they are glad to own, are really theirs in spite of the message of the death of uncountable trillions of human beings in the distant and recent past. `You can't take it with you when you go' is a piece of valuable folk-wisdom though many try to exercise control of their wealth from beyond the grave. Buddhist tradition is to give away wealth and possessions before one dies — at least one makes some good karma by such generosity, while `western' traditions generally emphasize making a will which leaves to executors the task of allotting bequests to friends and relatives of the deceased. This is not the best way of disposing of so called `possessions'.

\item Seeing life as a dream, not as real and substantial, is a very helpful practice. Even when it is regarded as real, solid and so on grasping is possible even through one is grasping at more illusions:

\begin{MyDescription}[(The Diamond-cutter-Vajracchedika Sutra)]{}
As stars, a fault of vision, as a lamp,\\
A magic show, as rain-cloud, as a bubble,\\
as dream, a lightning-strike, as drops of dew,\\
like this should be viewed all that is conditioned.
\end{MyDescription}
The illusive is less easily grasped and letting-go becomes easier. `Departed and done their time': those departed (peta) have literally `done their time' (K\=alakatain), so it is not only those in gaol but all of us captured by the desires and pleases of sams\=ara\\
- we are still doing our time.\\

\item This well~known verse used in obituaries and the like in Buddhist countries underlines how frail our self-importance is. As we live now we have so many connections with others and perhaps our names are well-known, even famous. After death our fame fades away and as generation succeeds generation others' knowledge of us grows less and less till even mighty rulers are little more than half-forgotten names. Who now knows what sort of person King Asoka was and how he conducted his court and treated his wives? We know him mostly from his famous Edicts carved upon rocks, while even Buddhist legends about him are less reliable and open to question in many ways. If such a mighty ruler's fame will fade in only two thousand years or so, what remembrance will there be of our own small doings in a tenth of that time?

\item No one is truly secure because of many and expensive possessions. Security comes from letting-go, both of persons and possessions `out there', and to grasping 'one's own' body and mind.

\item `In being he'll not be seen again' means that as a bhikkhu practicing Dharma in solitude with a non-roaming mind, one which is secluded from distractions, he (but this applies equally to female practitioners) will not reappear in birth and death free from the Wheel of being or becoming.\\

\item `The sage is unsupported' — he or she has no need of supports, no need to lean on anything, not even persons, on institutions, upon dogmas or sectarian commentaries, not even upon the wise and Enlightened. And why? No props are necessary for those who have reached the Further Shore and no raft either. Such unstained sages are compared to the leaves of tropical plants which shed the rain falling on them immediately and so are not ravelled in sorrow and avarice.\\

\item Lotus plants, both leaves and flowers, have a soapy covering so water does not lay upon them at all. This is the basis for many references in the Buddha's teachings to lotuses and their purity. Hence they are never stained, not by the mud in which they grow nor by any pollution in the rain or atmosphere. All that rolls off and does not adhere to the surface. Sages are like that.\\

\item Not conceiving upon the seen, heard and sense for this is a common human activity, relying on no senses at all must seem strange. Even stranger is the fact that they do not `conceive', meaning that they have no conceit. (This play upon related words occurs also in P\=ali with the verb malwati (****???***) and the noun m\=ana). So in the sage there is no measuring of him or herself against either people — there is no `I am superior', `I am equal' or `I am inferior' for this is what conceit means in the Buddhadharma. The sage knows that purity comes from the heart and so could be neither attached, on the side of greed, nor displeased, on hatred's side.
\end{enumerate}   


\chapter{Tissametteyya Sutta\\ To Tissametteyya on the disadvantages of sex}

\begin{MyDescription}[\arabic{stanza}]{Tissamettayya:}
Attached to sexual intercourse:\\
Sir, tell its disadvantages,\\
having heard your Teaching then,\\
secluded we will train ourselves.
\end{MyDescription}
   
\begin{MyDescription}[\arabic{stanza}]{The Buddha:}
Attached to sexual intercourse,\\
forgetful of the Teaching then,\\
wrong things that person practices,\\
and does what is not Noble.
\end{MyDescription}

\begin{MyDescription}[\arabic{stanza}]{}
Who formerly fared on alone\\
but now in sex indulges,\\
`Low' they say's that common worldly one,\\
like vehicle swerving off the track.
\end{MyDescription}

\begin{MyDescription}[\arabic{stanza}]{}
That one who had renown and fame —\\
that, for sure, diminishes,\\
having seen this, train yourself\\
renouncing sexual intercourse.
\end{MyDescription}

\begin{MyDescription}[\arabic{stanza}]{}
Overcome by (lustful) thoughts\\
that one broods as a beggar does\\
and hearing reproach of others then\\
Such a person is depressed.
\end{MyDescription}

\begin{MyDescription}[\arabic{stanza}]{}
For yourself creating `arms'\\
of others reprimanding words,\\
so with great entanglement\\
sinks down into untruthfulness.
\end{MyDescription}

\begin{MyDescription}[\arabic{stanza}]{}
Well-known as `one who's wise'\\
when vowing to the single life\\
but later then engaged in sex\\
will be `a fool defiled'.
\end{MyDescription}

\begin{MyDescription}[\arabic{stanza}]{}
The disadvantage having known,\\
the sage, at start and afterwards\\
should stablish fast the single life\\
having no recourse to sex.
\end{MyDescription}

\begin{MyDescription}[\arabic{stanza}]{}
So train yourself in solitude\\
for that's the life of Noble Ones,\\
but not conceive oneself as ' best' —\\
them near indeed to Nibbana.
\end{MyDescription}

\begin{MyDescription}[\arabic{stanza}]{}
The sage who's rid of sense-desires,
who to them's indifferent,
who's crossed the flood, is envied then
by those enmeshed with pleasures of sense.
\end{MyDescription}

\begin{MyDescription}[(Sn. 814-823)]{}

\end{MyDescription}

\section{Verse by verse Commentary on this Sutta.}

\begin{enumerate}


l. This Sutta's `disadvantages' of sex for a Dharma—practitioner makes a rather strange list:
\begin{MyDescription}[]{}
one becomes forgetful of the teachings\\
others blame a celibate who later turns or returns to sex\\
less of fame and reputations due to the last\\
fantasies and brooding increase (= more moha-delusion).
\end{MyDescription}

These are dealt with below.

\begin{MyDescription}[]{}
My list of disadvantages is rather more practical:\\
possible entanglements which are difficult to get out of\\
less opportunity for (meditation) practice in a relationship\\
exhaustion from work and family.
\end{MyDescription}

It is much better to emphasize the advantages of the Good Life as a celibate rather than listing supposed or real disadvantages: 

\begin{MyDescription}[]{}
time and place available for practice if a monk/nun\\
livelihood comparatively easy\\
celibate practitioners and honoured and supported\\
mind may be un burdened from many worldly problems.
\end{MyDescription}

The P\=ali word used in this verse: \textit{methuna} means both sexual intercourse and sexuality generally.

\item Why would a person in a relationship necessarily be `forgetful of the S\=asana' (teaching)? This is a bit similar to present day Thai ideas of a bhikkhu who is believed to have lost or cast aside his Dharma knowledge at the time of his disrobing, a sort of `lose robes, lose Dharma'. Such a person is said to practice wrongly and does what is not Ariyan. This word presumably means `what is not of the Noble Ones', and is not a racial reference. Still, these Noble Ones include all who have true insight into the Dharma from Stream-winners to Arahats. In the later list of the ten fetters (sa\=yyojana) the first two of these, stream winner and once-returner, still (have lust and so can have sex, while the Non-returner cannot due to lack of sexual desire, while Arahats are well beyond such worldly matters. This scheme of listing which fetters disappear with each attainment seems very artificial and inadequate. Now in the present verse since sex is labelled as ignoble and no reference made to the (later?) four stages of Noble insight, it seems that any sexual relationship must, by anyone, be looked down upon.

\item That Dharma cannot include the love of a partner is emphasized in this verse. `Fared on (the verb `carati'— see introductory section) alone' means the celibate life either as a lay person, or as a monk/nun. Judgement by others that one is now `low' having given this up, is still very much alive in Sri Lanka. `Swerving off the track' might be true for some: a young American bhikkhu who disrobed after some years as a forest monk comes to mind. He plunged into the varied fleshpots of Bangkok. But this would not be the pattern for most people whose progress in the Dharma may need a partner. Rather than denigrating sex, as this Sutta tries to do, it would be an improvement to admit that the path of many great and noble people in this world has been made possible through the support that they receive from their partners. After all, love must be an ingredient, a very important one, upon every spiritual path. Certainly there can be love without sex but the combination of the two is even more powerful. Not all Buddhist traditions involve celibacy, notably of course tantric varieties of the Dharma. Their approach is conceivably saner and lacks the rather shrill tone of this Sutta's denial of sexuality. This verse is an appeal based on love of status: having reached `renown and fame' as a celibate practitioners, perhaps as a Chao Khum (Thailand), Sayadaw (Bunna), Mahathera Sri Lanka), suddenly by disrobing one becomes ordinary. The argument seems to be: remain celibate, have no sex, so that `renown and fame' are preserved. What kind of argument is this!

\item This is a verse of warning: think lustful thoughts and as a result brood upon the conflict of having these within a life of celibacy, leading to guilt and depression. But this practitioner seems not to know much Dharma. There are all the contemplations of the impermanence of the body and its inevitable decay, eventually becoming a fearsome sight with an indescribable stench (see l, l I, the Vijaya Sutta) as well as reflections on non-self and emptiness. All thoughts whether wholesome or unwholesome are empty of any essence, they have no owner, so who is getting depressed or feeling guilty? They arise and pass away due to conditioning and there is no one who can force them to disappear. Obsession with thoughts of sex and guilt for thinking them are signs that one needs to practise more the methods mentioned above. As for others' reproach, well, some even people always criticize the most virtuous, an even slander them. Remember! the Buddha said that there is no one who cannot be blamed for even he himself was an object of blame (See Dhp. Verse 22 7). If one listened to every slur and took it to heart, one would never practice Dharma.

\item Creating `arms' or weapons for punishing oneself on the basis of \underline{others'} reprimands continues the topic of the last verse. These reprimanding words uttered y other people, instead of letting them go, are used by self-hatred as `weapons' to beat oneself up, to lower one's self-esteem. In this case, one's conceit of oneself, the way one conceives of oneself, is 'I am inferior' and my inferiority compared with others is increasing. Others are viewed as `superior to myself' or perhaps as `equal to myself'. Having low self esteem makes it easier to do things which as they multiply drag my self-conceit even lower. `Sinking down into untruthfulness' means that one's actions (karmas) of mind, speech and body depart increasingly from the truth of the Dharma. Cure: an effort to make all sorts of good karma beginning with simple things: offerings of food to teachers and to the poor, speaking kind words to those who suffer, being helpful to those who need it, etc. Then pages of chanting Dharma every day, and eventually begin to practice meditation. Do not try to do the difficult meditation practices first.

\item{Another warning verse about losing reputation. Same person, with robes or other marks of celibacy and one is praised as wide, without them and sexually active and one's a fool. Like the last verse this one is concerned with the Eight Worldly Dharmas, principally the dark sides of the pairs: loss, disrepute, blame and suffering (dukkha). For these see, the Mangala Sutta (11.4) Commentary. The author of this verse assuming that it is not the Buddha, has not considered that the subject here is `well-known as wise' so he or she will not be at all upset by others derision. A truly wise person has equanimity (upekh\=a) so that his or her mind could not be shaken.

\item This verse continues from the last and is another appeal to self-pride and cherishing one's image — not the most Buddhist attitude surely!

\item The training of oneself in solitude is good for some people at some times. In Buddhist traditions it has never been compulsory and it is nonsense to assert as this verse does, that the Noble Ones' life is solitary or always spent in the woods. Some who are ennobled by the Dharma may chose to spend their lives in the forest but others may dwell in cities to help those who have difficulties there. The second two lines of the verse are very true indeed: the Noble Ones have no conceit of themselves as the `best' but then they have no conceit at all, hence the mention of Nirvana.

\item The last verse is another appeal for celibacy: that on will be envied `by those enmeshed with pleasures of sense'. That a practitioner should stand firm in celibacy for this reasons strikes on as very peculiar. 
\end{enumerate}

This odd Sutta could only have originated from the Buddha if one allows that he could have `off' days. But this would mean that he was only Buddha sometimes, while at others he would have been unenlightened! Not a Buddhist consideration! It is better to regard this Sutta as the work of some rather unenlightened monks, defending their own status but despising those Dharma-followers who led a household life. How it got included in the Sn. ls a problem now insoluble.\\

 We are told by the P\=ali Comy. That this Tissa Metteyya and the young Brahmin of the same name who appears at Ns. 1040 are not the same person. The Comy. Does relate a story as the background for this Sutta thought its details do not sound very convincing.
   
  
   
\chapter{Pasura Sutta Being Overbold, the disadvantages of debate.}

\begin{MyDescription}[\arabic{stanza}]{}
They say: `In our Dharma purity's found'\\
but deny that it is found in the Dharma of others.\\
On what they depend they say `it's the best',\\
and so settle down in their individual truths.
\stepcounter{stanza}
\end{MyDescription}

\begin{MyDescription}[\arabic{stanza}]{}
Those disputants into the assembly rush\\
and perceive opposedly `the other' as a fool.\\
But in disputes, on others they rely —\\
these so-called experts ever-loving praise
\stepcounter{stanza}
\end{MyDescription}

\begin{MyDescription}[\arabic{stanza}]{}
Engrossed in conflict midst the assembly,\\
fearing defeat, they wish only for praise,\\
having been refuted, that one's truly confused,\\
angry at blame seeks weakness in the other.
\stepcounter{stanza}
\end{MyDescription}
   
\begin{MyDescription}[\arabic{stanza}]{}
Through investigation is your argument\\
refuted and destroyed' — so they say.\\
That one grieves and laments — that mere arguer,\\
`Oh! I am overcome' that person wails.
\stepcounter{stanza}
\end{MyDescription}   

\begin{MyDescription}[\arabic{stanza}]{}
Arisen among monks — those controversies\\
among them cause both elation and depression.\\
Refrain therefore, from disputation!\\
No meaning's in it save the prize of praise
\stepcounter{stanza}
\end{MyDescription}   


\begin{MyDescription}[\arabic{stanza}]{}
Praised in the midst of the assembly\\
for the presentation of arguments,\\
then that one laughs, or else is haughty.\\
So they say, `Conceited by winning debate'.
\stepcounter{stanza}
\end{MyDescription}

\begin{MyDescription}[\arabic{stanza}]{}
Though haughtiness will be ground for a downfall,\\
still proudly that one speaks, and with arrogance:\\
this having seen, refrain from disputations —\\
not by that is there purity, so the skilled say.
\stepcounter{stanza}
\end{MyDescription}

\begin{MyDescription}[\arabic{stanza}]{}
Just as a strongman fed\\
upon royal food might roar forth\\
wishing for a champion rival\\
find from the first there's nought to fight.
\stepcounter{stanza}
\end{MyDescription}

\begin{MyDescription}[\arabic{stanza}]{}
Those holding a view and disputing, say thus:\\
`This alone is the truth', so they aver;\\
then reply to them: `But no one's here\\
to retaliate through disputation'.
\stepcounter{stanza}
\end{MyDescription}

\begin{MyDescription}[\arabic{stanza}]{}
They continue with their practice, offering no opposition\\
against others, othering (?) no view opposed to view.\\
But then, Pas\=ara, what would you obtain?\\
For them there is nothing to be grasped as the highest.
\stepcounter{stanza}
\end{MyDescription}

\begin{MyDescription}[\arabic{stanza}]{}
As you've come here, in your mind\\
thinking and speculating on various views,\\
you have met with a Washen One\\
But will not be able to make progress with him.
\stepcounter{stanza}
\end{MyDescription}

\begin{MyDescription}[\arabic{stanza}]{}
As you've come here, in your mind\\
thinking and speculating on various views,\\
you have met with a Washen One\\
But will not be able to make progress with him.
\stepcounter{stanza}
\end{MyDescription}
   
   
   
Verse 831 maybe corrupt.\\
Lines 872-834. Many different translations which\\
are evidence of doubts in meaning.\\
Line 833. Pas\=ara was a wanderer who was a debater.


\chapter{M\=agandiya Sutta\\ M\=agandiya learns the muni's life.}

\begin{MyDescription}[\arabic{stanza}]{The Buddha:}
As Craving with Longing and Lust had been Seen\\
no spark of desire existed for sex,\\
What then about this filled with piss and with shit\\
That even with foot I'd not wish to touch!
\stepcounter{stanza}
\end{MyDescription}   

\begin{MyDescription}[\arabic{stanza}]{M\=agandiya:}
If you don't wish for a jewel such as this,\\
a woman desired by many lords of men,\\
what view do you hold, living by what rite,\\
by what vows to arise in what kind of life?
\stepcounter{stanza}
\end{MyDescription}   

\begin{MyDescription}[\arabic{stanza}]{The Buddha:}
As nothing is grasped among various Dharmas\\
so for me there is not any `This I proclaim',\\
having seen but not grasped among many views\\
through discernment among them I saw inner peace.
\stepcounter{stanza}
\end{MyDescription}  

\begin{MyDescription}[\arabic{stanza}]{M\=agandiya:}
Among what's constructed thoroughly knowing,\\
Ungrasping, O Sage, do you speak upon these,\\
`inward peacefulness' — what meaning has that,\\
how will the wise declare it to be?
\stepcounter{stanza}
\end{MyDescription}   

\begin{MyDescription}[\arabic{stanza}]{The Buddha:}
Neither from views, not from learning or knowledge,\\
not from rites, or from vows, does purity come I say\\
nor from no views, no learning, no knowledge acquired\\
no rites and no vows - none of them at all,\\
Neither by grasping nor giving them up\\
is their peace unsupported, and no hunger `to be'.
\stepcounter{stanza}
\end{MyDescription}  

\begin{MyDescription}[\arabic{stanza}]{M\=agandiya:}
If you speak then not of purity by views,\\
not by learning, not by knowledge, not rites and not vows\\
nor from no views, no learning, no knowledge acquired\\
by no rites and no vows — none of them at all,\\
then I think that this is very deluded Dharma,\\
for some depend on views as the source of purity.
\stepcounter{stanza}
\end{MyDescription} 

\begin{MyDescription}[\arabic{stanza}]{The Buddha:}
Questioning repeatedly dependent on views\\
grasped at again, you've arrived at delusion,\\
not having experienced even a tiny perception of peace\\
so therefore you see this as very deluded.
\stepcounter{stanza}
\end{MyDescription} 

\begin{MyDescription}[\arabic{stanza}]{}
Who as `equal' considers, `greater' or `less'\\
conceiving others thus would dispute because of this\\
but who by these three never is swayed\\
`equal, superior does not exist.
\stepcounter{stanza}
\end{MyDescription} 
   
\begin{MyDescription}[\arabic{stanza}]{}
Why would this Brahmin declare `this is the true',\\
with whom would he argue that `this is false',\\
in whom there is not `equal', `unequal'\\
with whom would he join another in dispute?
\stepcounter{stanza}
\end{MyDescription} 

\begin{MyDescription}[\arabic{stanza}]{}
With home let go, faring on in homelessness,\\
in villages the Sage having no intimates,\\
rid of sensual desires, having no preference,\\
would not with any arguments people engage
\stepcounter{stanza}
\end{MyDescription} 

\begin{MyDescription}[\arabic{stanza}]{}
Unattached, one wanders forth in the world,\\
a N\=aga, ungrasping, would not dispute those,\\
just as the water lily, thorny-stemmed species,\\
sullied is not by water or mud,\\
even so is the ungreedy Sage proclaiming Peace\\
unsullied by desires and pleasures in the world.
\stepcounter{stanza}
\end{MyDescription} 

\begin{MyDescription}[\arabic{stanza}]{}
The Wise One's not conceited by view or by intelligence\\
for that one there is no `making-it-mine'\\
and cannot be led by good works or by learning,\\
cannot be led away by mind~shelters of view.
\stepcounter{stanza}
\end{MyDescription} 

\begin{MyDescription}[\arabic{stanza}]{}
For one detached from perception, there exist no ties,\\
for one by wisdom freed, no delusions are there,\\
but those who have grasped perceptions and views\\
they wander the world stirring up strife.
\stepcounter{stanza}
\end{MyDescription} 

\begin{MyDescription}[(Sn. 835-847)]{}

\end{MyDescription} 

\newpage

\section{The two opening verses of this Sutta}
In the first line of the first verse we are presented with a statement that sex and lust with longing had been seen — but by who? And in what way had they been seen? This Sutta does not identify who has spoken these words. The verse continues with some very scornful words about someone's body though we are not told` whose.\\

The second verse is obviously spoken by another person who concludes with an interesting question, or rather, a series of them. The P\=ali still fails to identify either of these persons. Only in the third verse do we discover that the first verse is supposed to be spoken by the Buddha and the second by his supposed questioner, M\=agandiya.\\

These two verses are worthy of closer examination as they present a number of puzzling questions. The first of these concerns the three nouns: craving, longing and lust which as aspects of mind must always concern those living a celibate life. So are they just that — three troublesome mind-states? The answer to this is that in a few Suttas and P\=ali Comys these three have become a potent aspect of M\=ara's assault upon the potential Buddha while he was seated under the Bodhi tree just before his Awakening. This assault is mentioned in Sn. 425-440 where such mind-aspects and sense-desires, fear and hard heartedness are personalized into soldiers in Mara's army. ln the same way, craving, longing and lust are transformed into the famous Daughters of Mara. `Famous' because generations of Buddhist artists have delighted in portraying their seductive forms and alluring gestures upon walls and in manuscripts, while monks have also enjoyed elaborating  upon this story. Of course, in the various accounts of this incident in both P\=ali and Sanskrit, the Daughters of Mara are defeated because the Buddha cannot be seduced by them. lf we understand this line to refer to the three gorgeous girls, we must capitalize their names, but not do the same for the verb `seen' which would mean that the Buddha had only seen them been aware of them as sight—objects but taken no interest. On the other hand if they are personalized mind-states then they do not merit capitalized names but the verb `Seen' should have a capital letter to indicate that insight or vipassan\=a regarding lust and so on. It is worth noting that though the Daughters of Mara legend occurs in the classic P\=ali Comys, it is rarely found in the Suttas. So much for the first line!\\

According to the Suttas supported by the Comys, sexual desire is eliminated with the attainment of refined aspects of the paths and fruits. The Buddha and his Arahat disciples are depicted in the Vinaya and Sutta as having gone beyond sex and so having no longer to struggle to maintain celibacy, having in fact none of me problems that most people have with sexuality. This systematized view, slowly becoming known as `Theravada', promoted the growth of celibate sanghas of monks and nuns some of whom emphasized that only those in robes could reach the more refined stages of liberation. Ordinary practitioners could not become Arahats and if by some strange collection of causes they did, either they would have to be ordained on that very day, or they would die! Though this seems most unlikely, it is opposed by the presence at A.Sixes 131-151 of a list of lay practitioners `who have Gone to the End, Seen the Deathless'. Some of them are familiar and others more obscure but in any case these present-day Buddhists who are not ordained should take heart and remember these heroes from so long ago. Their names `Tapussa, Bhallika, Sudatta An\=athap\=o\=oa, Citta Macci-K\=asandika, Hatthaka Alavaka, Mah\=an\=ama Sakka, Ugga Ves\=alika, S\=ara Ambattha, J\=avaka Kom\=arabhacca, Nakulapit\=a, Tavaka\=o\=oika, Pur\=ana, Isidatta, Sandh\=ana, Vijaya, Vajiiyamahita, Me\=o\=oaka, V\=asettha, Arittha, S\=aragga'. There are no women in this list. The survival of these men's names among P\=ali Suttas very full of teachings to and about the monastic sanghas with Liberation limited to only ordained people, is a small indication that in the Buddha's days liberation was available to all.\\

The last two lines of this verse contain words of scorn said to be uttered by the Buddha upon being presented with the beautiful M\=agandiy\=a, daughter of the brahmin M\=agandiy\=a. The Dhp. Comy provides details of this story which does not appear in any Sutta. The essence of this is as follows: M\=agandiy\=a rejected many offers of marriage for his daughter made by wealthy and powerful princes. However, upon seeing the footprints of the Buddha (note the connection with the Signs of the Superman — D30 and the remarks following Sn. 1031), was sure that he would be a suitable husband. After meeting the Buddha and offering him his daughter well-adorned the story continues with the popular account of the Buddha's Awakening in which Mara and his three daughters, Tanh\=a (craving), Arat\=a (longing) and R\=ag\=a (lust) — whose bodies are rumoured to surpass all human beauty try to upset the Bodhisattva's intention. This encounter is made the excuse for the future Buddha to scorn a mere human girl — M\=agandiya, with these words —

\begin{MyDescription}
`what then about this filled with piss and shit,\\
that even with food I'd not wish to touch!'
\end{MyDescription}

Now, all Buddhists hold that their Teacher was remarkable for his Great Compassion (Mah\=akarun\=a) towards every human being and the accounts of this life confirm this. The scornful words quoted above are said to have been spoken in the presence of M\=agandiya herself and hardly sound like compassionate talk! As it turned out in the Dhp Comy story these insulting words caused M\=agandiya — not surprisingly - to hate the Buddha and to seek her revenge on him by burning alive many of the ladies of the local king who were his disciples. So M\=agandiya showed herself as a very nasty piece of work who came to a grisly end. But could a Buddha act in such a way as to bring this about?\\

As there are a number of M\=adandiy\=as in the Suttas and Comys it may be that these have been confused so that fragments of their legends have been patched together by a misogynist monk who as put the above words in the Buddha's mouth. After this tangle, M\=agandiy\=a the Brahmin addresses to the Buddha a number of questions quite unrelated to what has gone before.\\

The first verse of this Sutta, Sn 835, is also found in the story-cycle of King Udena in Dhp Comy vol 1 P. 199ff of the English translation, Buddhist Legends.\\

Vs. 836: `in what kind of life'. `Life' translates `bhavaj' literally `being', `existence'.\\

Vs. 845: `N\=aga', literally a serpent or serpent-spirit connected with water. Worshipped to bring rain. Also an elephant, but here means a mighty Teacher.

\chapter{Pur\=abhada Sutta\\ `Before Breaking-up' and so on - a muni's qualities}\\


\begin{MyDescription}{\arabic{stanza}}[Questioner:]
Please Gotama, do you speak to me
upon the person perfected:
how's their insight and their conduct
so that they can be called `Peaceful One'?
\stepcounter{stanza}
\end{MyDescription}

\begin{MyDescription}{\arabic{stanza}}[The Buddha:]
One who is craving-free\\
before the body's breaking-up\\
not dependent on the past\\
in the present is prepared\\
(and in future) has nought preferred,
\stepcounter{stanza}
\end{MyDescription}

\begin{MyDescription}{\arabic{stanza}}[]
gone anger and gone fear as well,\\
gone boasting, gone remorse,\\
wise-speaker with no arrogance,\\
a Sage restrained in speech,
\stepcounter{stanza}
\end{MyDescription}

\begin{MyDescription}{\arabic{stanza}}[]
no hopes for what's to come\\
no mourning for the past\\
not led astray by views,\\
the singled seer `mid senses' touch
\stepcounter{stanza}
\end{MyDescription}

\begin{MyDescription}{\arabic{stanza}}[]
one not concealing, not deceitful,\\
not hankering and neither mean,\\
not rough with others, not causing disgust\footnote{verse-line too long? And???}\\
and not to slander given,
\stepcounter{stanza}
\end{MyDescription}

\begin{MyDescription}{\arabic{stanza}}[]
to pleasures not addicted\\
and not to pride inclined,\\
gentle, ready witted, not\\
credulous and not attached,
\stepcounter{stanza}
\end{MyDescription}

\begin{MyDescription}{\arabic{stanza}}[]
training not in hope of gain\\
nor disturbed by getting none,\\
by cravings unobstructed,\\
hankering not for tastes,
\stepcounter{stanza}
\end{MyDescription}

\begin{MyDescription}{\arabic{stanza}}[]
ever mindful and equanimous,\\
so, who as `equal' thinks not of themselves\\
nor as better nor as worse\\
has no of inflation any sense
\stepcounter{stanza}
\end{MyDescription}

\begin{MyDescription}{\arabic{stanza}}[]
and for whom there's no dependence'\footnote{on \underline{craving} of view},\\
not dependent Dharma having known\\
for such exists no craving for\\
existence, non~existence \footnote{Being, non-being: 2 extreme views}
\stepcounter{stanza}
\end{MyDescription}

\begin{MyDescription}{\arabic{stanza}}[]
that one I call the Peaceful
who no sensual pleasures seeks
who therefore has no ties,
crossed entanglement,
\stepcounter{stanza}
\end{MyDescription}

\begin{MyDescription}{\arabic{stanza}}[]
ever mindful and equanimous,\\
so, who as `equal' thinks not ofsense themselves\\
nor as better nor as worse\\
has no of inflation any 
\stepcounter{stanza}
\end{MyDescription}

\begin{MyDescription}{\arabic{stanza}}[]
on account of which, the people\\
with monks and Brahmins might accuse\\
that one is undisturbed,\\
and by such words unmoved,
\stepcounter{stanza}
\end{MyDescription}

\begin{MyDescription}{\arabic{stanza}}[]
gone greediness and never mean,\\
not speaking of themselves as `high'\\
not `equal', nor `inferior'\\
so the unfittable does not fit,
\stepcounter{stanza}
\end{MyDescription}

\begin{MyDescription}{\arabic{stanza}}[]
for whom is nothing owned in the world\\
and having nothing does not grieve,\\
who `mong Dharmas ventures not\\
is truly called a Peaceful One!
\stepcounter{stanza}
\end{MyDescription}

\begin{MyDescription}{(Sn. 848-861)}[]

\end{MyDescription}

\chapter{Katahaviv\=ada Sutta\\ Arguments and Disputes.}

\begin{MyDescription}{\arabic{stanza}}[Questioner:\footnote{Who or what is this questioner?}]
Whence so many arguments, disputes\\
and sorrow, lamentation, selfishness,\\
arrogance, pride and slander too?\\
whence come all these? Please upon them speak.
\stepcounter{stanza}
\end{MyDescription}

\begin{MyDescription}{\arabic{stanza}}[The Buddha:]
Much love of arguments, disputes\\
means sorrow, lamentation, selfishness\\
with arrogance, pride and slander too.\\
Inclined to selfishness, arguments, disputes;\\
quarrels, slander also come to birth.
\stepcounter{stanza}
\end{MyDescription}

\begin{MyDescription}{\arabic{stanza}}[Questioner:\footnote{Ends: fulfilments}]
From what causes in the world there's dearness, love,\\
these various greeds that wander in the world,\\
from these causes, hopes and their ends as well,\\
these bring about a human being' s future.
\stepcounter{stanza}
\end{MyDescription}

\begin{MyDescription}{\arabic{stanza}}[The Buddha:]
From desires in the world as causes of the dear,\\
these various greeds that wander in the world,\\
from these causes, hopes and their ends as well,\\
these bring about a human being's future.
\stepcounter{stanza}
\end{MyDescription}

\begin{MyDescription}{\arabic{stanza}}[Questioner:]
From what causes in the world is there desire\\
and much deliberation on this — whence it comes?\\
And anger too, false-speaking also doubtfulness\\
and dharmas such as these by the Samana declared
\stepcounter{stanza}
\end{MyDescription}

\begin{MyDescription}{\arabic{stanza}}[The Buddha:]
`It's pleasant, unpleasant', so in the world they say\\
and depending on these arises desire\\
but having seen forms, their arising and decay\\
then a person in this world certainly deliberates.
\stepcounter{stanza}
\end{MyDescription}

\begin{MyDescription}{\arabic{stanza}}[]
With\footnote{Pleasant/unpleasant=duality} anger, false-speaking, also doubtfulness\\
and all such dharmas, this quality exists.\\
the doubting person in the knowledge-path should train\\
for the Samana dharmas has declared after having Known
\stepcounter{stanza}
\end{MyDescription}

\begin{MyDescription}{\arabic{stanza}}[Questioner:]
The pleasant, the unpleasant, originate from what?\\
ln the absence of what do these cease to be?\\
That which is being\footnote{Being(bhava)=existence}, non-being as well,\\
what their origination, do tell me of this?
\stepcounter{stanza}
\end{MyDescription}

\begin{MyDescription}{\arabic{stanza}}[The Buddha:]
`Touch\footnote{Phass\=a=(roughly) `touch'}', the origination of pleasant, unpleasant,\\
`Touch' being absent these cease to be.\\
That which is being, non~being as well,\\
its origin's thus, I tell you of this.
\stepcounter{stanza}
\end{MyDescription}

\begin{MyDescription}{\arabic{stanza}}[Questioner:]
From what causes in the world does touch come to be\\
And whence does possessiveness also arise?\\
in the absence of what is `mine' making not?\\
When what exists not are no `touches' touched?
\stepcounter{stanza}
\end{MyDescription}

\begin{MyDescription}{\arabic{stanza}}[The Buddha:]
`Touches' depend upon mind, upon form\\
possessiveness caused by longing repeated,\\
when longing's not found, possessiveness's gone,\\
When form\footnote{N\=ama r\=apa: name and form} is no longer no `touches' are `touched'.
\stepcounter{stanza}
\end{MyDescription}

\begin{MyDescription}{\arabic{stanza}}[Questioner:]
For one in what state does form cease to be\\
how bliss and dukkha come to cease as well,\\
please do you tell me how these come to cease\\
For this we would know — such is my intent.
\stepcounter{stanza}
\end{MyDescription}

\begin{MyDescription}{\arabic{stanza}}[The Buddha\footnote{Requires Comy, Jaya}:]
Neither one of normal perception nor yet abnormal,
neither unperceiving no cessation of perception,\\
but form ceases for one who (has known) it thus:\\
Conceptual proliferation has perception as its cause.
\stepcounter{stanza}
\end{MyDescription}

\begin{MyDescription}{\arabic{stanza}}[Questioner:]
Whatever we've asked of you, to us you've explained,\\
another query we'd ask, please speak upon this,\\
those reckoned as wise here, do they say that\\
purity of soul” is just for this (life)\\
or do some of them state there's another beyond?
\stepcounter{stanza}
\end{MyDescription}

\begin{MyDescription}{\arabic{stanza}}[The Buddha:]
Here some reckoned as wise do certainly say:\\
Purity of soul is just for this life';\\
but others who claim to be clever aver\\
[line missing]
\stepcounter{stanza}
\end{MyDescription}

\begin{MyDescription}{\arabic{stanza}}[]
And Knowing that these are dependent on views\\
having Known their dependence the investigative Sage\\
since Liberated Knows, so no longer disputes,\\
the wise one goes not from being to being\footnote{Existence to existence}.
\stepcounter{stanza}
\end{MyDescription}


\chapter{C\=alaviy\=aha Sutta\\ Smaller discourse on quarrelling}

\begin{MyDescription}{}[]
Indeed the truth is one, there's not another,\\
about this the One who Knows does not dispute with another,\\
but the Samanas proclaim their varied `truths'\\
and so they speak not in the same way.
\stepcounter{stanza}
\end{MyDescription}

\begin{MyDescription}{}[]
Why do they speak such varied truths,\\
these so-called experts disputatious -\\
Are there really many and various truths\\
Or do they just rehearse their logic?
\stepcounter{stanza}
\end{MyDescription}

\begin{MyDescription}{}[The Buddha:]
Indeed, there are not many and varied truths\\
differing from perception of the ever-true in the world\\
but they work upon their views with logic:\\
Truth! Falsehood!' So they speak in dualities
\stepcounter{stanza}
\end{MyDescription}

\begin{MyDescription}{}[]
And since perfected in some extreme view,
puffed with pride and maddened by conceit,
he anoints himself as though the master-mind
likewise thinking his view's perfected too.
\stepcounter{stanza}
\end{MyDescription}

\chapter{Mah\=aviy\=aha Sutta\\ Greater discourse an Quarrelling}

\begin{MyDescription}{}[]
One who speaks dogmatically, who's settled down in view,\\
Will not be deferent, one not easily trained.\\
To that attached, his own views `pure',\\
`pure path' according to what he's seen.
\stepcounter{stanza}
\end{MyDescription}

\begin{MyDescription}{}[]
The paragon with wisdom comes not near\\
To following views, by partial knowledge bound.\\
Having known opinions of common people,\\
He's equanimous, though others study them.
\stepcounter{stanza}
\end{MyDescription}


\chapter{Tuvataka Sutta The Quick Way}

\chapter{Attanda\=o\=oa Sutta Assuming Forcefulness' and so on\footnote{Taking/grasping weapon}}
   Fear's born assuming forcefulness —
   see how the people fight!
   I'll tell you how l'm deeply moved,
   how l have felt so stirred.
   935.
   Seeing how people flounder
   as fish in little water
   attacking one the other
   its fearfulness appeared.
   936.
   Once l wished a place to stay,
   but all the world is essenceless,
   gTaking/grasping weapon
   
   
   
   V.lS
   935-95 4
   turmoil in every quaner,
   l saw no place secure.
   937.
   Folks' never-ending enmity”
   l saw, took no delight,
   but then l saw the hard-to—see,
   the dart within the heart.
   938.
   Affected y this dan
   one runs in all directions
   but with the dart pulled out
   one neither runs nor sinks.
   939.
   On this, the training's chanted thus:
    _
   “Reading n'0sane??, reading drati??
   MS Page;409
   
   
   
   V-15 935-954
   Zl
   N§Page41U
   Whatever bonds within the world
   they should not be pursued
   knowing in depth all sense-desires
   for Nin/ana train.
   940.
   Truthful and not arrogant,
   deceit none, slander, hate,
   rid of greed's evil, avarice
   beyond them all's the sage.
   941.
   Not sleepy, drowsy, slothful not,
   living not with negligence,
   taking no stand on arrogance:
   ______________-—
   2' Kama: pleasureldesirel
   
   
   
   V. I 5
   MS F08!-411
   that mind inclines to Nibbana.
   942.
   Be not into lying led,
   for forms have no affection,
   know thoroughly conceit, '
   violence avoid fare thus.
   943.
   Delight not in the past
   nor be content with nev\mess,
   sad not with disappearance
   nor crave for the attractive.
   944.
   Greed I say's `the great flood',
   its torrent the rush of lust,
   lust's objects an imagining,
   935-954
   
   
   
   V. 1 5 935-954
   the swamp of lust is hard to cross.
   945.
   The sage on firm ground stands,
   not swayed from truth, a paragon,
   having relinquished All,
   `peaceful' that one's called.
   946.
   The wise indeed, all wisdom won,
   on Dhamma not dependent,
   wanders perfected in this world
   and envies none herein.
   947.
   Who sense-desires has crossed beyond,
   undone worldly ties
   and bondless, cut across the stream
   NS P:ge:4l2
   
   
   
   V.l5
   MS Pi§B:4l3
   no longer grieves or broods.
   948.
   Let what's `before' just wither up,
   `after' for you be not a thing,
   if then `between' you will not grasp
   You will fare at peace.
   949.
   For whom with mind-and-bodily forms
   there is no `making-mine' at all
   grieves not when they are not
   and suffers here no loss.
   950.
   For whom there is no `this is mine'
   nor no `To others it belongs',
   in whom `myself' cannot be found
   
   
   
   V-15 935-954
   Grieves not that `l have none'.
   951.
   Asked upon one unshakeable
   l tell of this one's goodness:
   Not harsh, not covetous at all,
   Steadfast, impartial everywhere.
   952.
   For one who's steadfast, Knows”,
   That one does not accumulate“,
   Unattached to making effort,
   Sees security everywhere.
   953.
   A sage speaks not as though“
   _______,_._-—-——
   “Capital `K' = enlightened.
   “Good/back???
   Z` No conceit: or dialects of caste.
   MS |>1ge=al4
   a._=-..a:_.., “twa-
   
   
   
   V. I5 935-954
   `mung equai, low or high,
   Serene, devoid of avarice
   Does not accept, reject.
   954.
   So said the Lord.
   (Sn 935-954).
   MS Pam 415
      

\end{document}  

   
   
  
   
   
   
   V. I 6
   MS Pag'e:4\6
   San'putta:
   Sariputta Sutta
   Sariputta asks the Buddha
   Not seen before by me
   nor heard by anyone:
   such sweetly-spoken Teacher
   from Tusita came to lead a group.
   955.
   One by himself attained to blis,
   all darkness he dispelled,
   so that the One-wlth—Eyes be seen
   by world together with the gods.
   956.
   One's who's `Thus', the unattached,
   that Buddha undeceptive,
   955-975
   
   
   
   V.l6
   with many disciples, devotees,
   for them l ask a question“.
   957.
   For a monk avoiding society“,
   seeking out a lonely place —
   bone yards, at the base of trees
   or caves within the mountain wastes —
   958.
   Living-places high or low,
   hHow many are the terrors there
   that a monk in his silent place
   trembles not at all?
   959.
   How many are the troubles here
    
   ' 25Although Sariputta is asking a question `forthem', it seems to be a question for monks
   26L0athe and such like — don't use as looks like 'Dosa'
   MS Pa2e:4l7
   
   
   
   V.i6
   for a monk to overcome
   while living in a place remote,
   or going to the Ungone-Point”?
   960.
   What ways of speaking would be his?
   What place should he frequent?
   What sorts of rules, kinds of vows
   For the monk with mind intent?
   961.
   What is the training he adopts,
   one-pointed, mindful, wise
   to blow away all blemishes
   as does a smith with silver?
   962.
   _______'___.__»—-———
   27? Again distain with amatain disdain (Comy)
   MS Pag::4l 8
   955-975
   
   
   
   QI>"il1lh"¢_`v_\`A1A  ~...s,A~  '...1,..;_/-_a».~.NM .a ~  `A&44>1%m;P._4_,<>A41.<< t.-,~_@ at WXA@¢ ...% .0 A.» 3 , ~.fl_._
   V.l6
   The Buddha“:
   As One who Knows I'll explain to you
   what's pleasant for you practicing avoidance,
   who live and who rest in a lonely abode
   wishing Awakening in keeping with Dharma."
   within limits the mindful monk practices
   then of five fears is this wise one not afraid:
   March flies and mosquitoes, of slithering snakes,
   of men's assaults and fierce four-footed beasts.
   964.
   Nor be disturbed by those with differing Dharma
   even having seen their many perils,
   further then this seeker of the good
   will overcome all fearfulness too.
   965.
   Afflicted by sicknesses, hunger as well,
   zslt is difficult to find a verb to trans j-lgBCChBtl/Vll')3gU,..
   29? Kamana-nudhan ?
   M5 PIHE=4l9
   
   
   
   V. 1 6
   MS Pa§:`rZO
   side”.
   the cold and strong heat he should endure,
   by these many touches should he be unmoved
   having energy stirred and striving with strength.
   966.
   Neither should he steal, nor should he tell lies,
   but let love suffuse the fearful and the unafraid
   and when his mind is agitated let him know
   `This should be removed' - it's on the Dark One's
   967.
   lnto the power of anger and of arrogance
   he shouldn't fall, but firm, eradicate their roots,
   all being attached he overcomes complete,
   all that is dear to him, all that repels.
   968.
   __'___,____._-»———'—'
   3° Kanha = Krishna!
   955-975
   
   
   
   V. 16
   MS P.1ge'4Zl
   With wisdom esteemed, with joy punfied,
   removing supports for all fearfulnesses,
   let him conquer dislike for his lone lodging-place
   and conquer the four that cause him to lament:
   969.
   `Alas, what shall l eat' and `where indeed eat it',
   `last night l slept badly' and `where sleep today' —
   one-in-training, a wanderer, of no flag the follower
   should such thoughts let go, leading to lamentation.
   970.
   Satisfied, receiving timely food and clothes
   knowing moderation in them and
   protected by them, in a village the restrained
   though roughly he's addressed speaks no harsh word
   971.
   
   
   
   V.16 955-975
   M5 l>agE;4Z2
   With eyes cast down, feet not longing-guided,
   to ihana devoted, very watchful he should be,
   let him grow in equanimity with mind composed,
   check his scruples, how he inclines to doubt.
   972.
   With words of reproof let the mindful one rejoice
   and shatter his scom for his fellow-celibates
   and utter skilful words at the proper time
   and think not upon views and beliefs of common folk.
   973.
   And then in the world, there are the dusty five
   in which the mindful one guided, trains himself well,
   lust overcoming to bodies and to sounds
   to tastes, to perfumes and touches too.
   974.
   
   
   
   .. .__,_\.tt~/_.~`i_..~,,Q.ue~t.=~»_ rm mt &m.   M. Q   h_,.t='._»_.M¢~=.  -_\M_., A4;v~&¢~_ ,_,_.._.,~_._._. .=.. 14¢ _.
   V. 16
   MS PzID~423
   And when in these things he has guided“ desire,
   mindful, that mind of a well-liberated mind
   then he in due time thoroughly examining Dharma
   with mind become one he shall the darkness rend.
   975.
   Thus the Master spoke.
   (Sn 955-975)
   _  t
   3' Guided (v|aCyya): better trans than `dispel `disciplined' `subdued' etc.
   
   
   
   V.16
   NS Page 424
   The Chapter
   on the Way to the Beyond
   955-975
   
   
   
   V.l
   Vatthugcitha
   . The Prologue telling the story
   A brahmin Wh0'd mastered all mantras
   Desiring the state of no—tbingness
   From Kosalans' fair city he left then
   Towards the southern parts. 976
   By Godhivari river he sojourned
   In Assaka's realm near A1aka's border
   Surviving on gleanings and fmit. 977
   Close by to him a village large,
   With revenue derived from there
   Great the sacrifice he performed. 978
   With ritual offerings made
   For the sacrifice, he returned
   To his hennitage again
   And there another brahmin came. 979
   Footsore and thirsty, he
   With teeth unclean, dust-covered head
   ...then approached him begging for
   at least five hundred coins.. 980.
   976-1031 \/
   
   
   
   
   
   
   V.1
   Bavari:
   . Brahmin:
   Having seen him, Bavari
   invited him to take a seat
   and asked about his comfort, health —
   then to the stranger spoke these words:
   Whatever was given for me to give,
   All this I've given away,
   So brahmin please forgive me,
   I've not five hundred coins..
   If your honour will not give _
   To me Who begs from him
   Then let your head be split apart
   In seven days from now.
   Having done preparatory rites
   That charlatan a fearful curse pronounced
   So that having heard his words
   `one-with-dukkha' did Bavari become.
   He took no food and withered up
   afflicted with the dart of grief
   and then With mind of such a kind
   his heart enjoyed no jhana .
   Seeing him suffering, terrified,
   981
   982
   983
   984
   985
   976-1031
   
   
   
   
   
   
   __-.@~..e M11 ~i.i...¢~...M_ia.=\,._ *AL'\;@Y“4m/“<\`*3J aw-t__.@A».`,.t.mit~i_W_i.a@~i\_~,»_,A -L~vn_-~~__<»¥-..k'._._ >,\_“»..._._»ai<_....L'._,»_.i»&.n__ .1.-a.i>=..K<.Qk~-;»__ _ - ~\>rv_& n
   V.l
   Devi:
   Bavari:
   Devi:
   Bavari.
   Devi:
   MS Page:377
   Seeing him suffezing, terrified,
   a deva there who Wished his good
   on drawing near to Bavari ~
   to him she spoke these words: 986
   He doesn't know about the head,
   that charlatau desiring wealth,
   of heads, andsplitting heads apart
   in him no kn0wing's found. 987
   Ifmy lady knows of this,
   when asked, please tell rne too,
   let me hear your words on tis,
   on heads and splitting heads apart. 988
   I do not know about this thing,
   in me no knowing's found,
   on heads and splitting heads apart
   but by Victors it has been Seen. 989
   Who, then knows about this thing?
   Who on this sphere of earth?
   On heads and splitting heads apart,
   O deva, tell me this. 990
   From out of Kapilavatthu town
   
   
   
   M '\  £5 \ 
   \ !~§
   . ILL:/4—
   /a r v e
   v.1 / [A/\ `H/A,.,q ,976-1031
   came lately, Leader of the world,
   / a-son bringing light
   Narrative:
   Bavan:
   MS Page:373
   a scion of Okkaka king. 991.
   He is indeed a Wakened One
   all dharmfls gone across,
   all straightly—knowing' s power won,
   in all dharmas, Seer,
   to exhaustion of all dharmas won
   freed by all assets' wearing out - 992.
   One Awakened, lord of the world,
   the Seer who teaches Dharma,
   go to him and then enquire —
   that matter he'll explain. 993.
   On hearing `Sambuddho'_- that word,
   Bavan was overjoyed
   and vrief diminished too
   c
   while rapture then arose in him. 994.
   Glad at heart, overjoyed, in awe
   spoke Bavari to that devati:
   In which village, in which town,
   in which state is the worid's lord found`?
   Where should we go to honour him,
   
   
   
   V.l
   Bavaria:
   Deva:
   Bavara:
   MS Page 419
   996.
   while rapture then arose in him. 994
   Glad at heart, overjoyed, in awe
   spoke Bavara to that devata:
   In which village, in which town,
   in which state is the world's lord found?
   Where should we go to honour him,
   the All-awakened, best of men? 995
   ln Kosa|a's kingdom the  dwells,
   the greatly wise truly of Knowledge profound,
   of Sakyas the scion, burdenless, from inflows free,
   the eminent among men knows splitting the head.
   Addressing then his brahmin pupils,
   those who had mastered the mantras:
   Come here, young brahmins, listen well
   for l shall speak to you.
   Whose rare appearance in the world
   is hard then to experience,
   has appeared for us today,
   acclaimed as All-awakened One,
   quickly now go to Savatthi
   997.
   976-1031
   
   
   
   V. I
   Pupils:
   Bavara:
   MS Pal!-430
   l O00.
   976-1031 ""
   to see this Best of men. 998.
   How, O brahmin, shall we know
   on seeing him that he's Awake?
   Tell us, who are so ignorant
   that him we'|l recognize? 999.
   In mantra-hymns come down to us
   the signs of Superman —
   two and thirty there complete
   in order are described.
   Upon whose body these appear —
   these signs of the Superman —
   two possibilities are there for binh,
   a third bourn is not found: l0Ol.
   So should he choose the household life
   this world he'll conquer weaponless,
   non-violently, without a sword,
   by Dham1a rule it righteously. I002.
   But if he go forth from home
   to the state of homelessness,
   
   
   
   V.1
   1003.
   l O04.
   Narrative:
   MS Paw 43!
   he'll be Awake, removed the veils,
   one of worth, the unexcelled.
   Question in your mind alone
   my birth, my caste, how l appear,
   my mantras, pupils and so on,
   with heads and splitting heads apart.
   lf he's indeed the One Awake,
   who, lacking obscurations, Sees;
   to Questions asked in mind alone
   he will reply with words.
   The voice of Bavari they heard,
   those brahmin pupils — all sixteen:
   Ajita, Tissamettayya,
   Puooaka, then there's Mettagu,
   Dhotaka, Upasiva then
   Nanda, also Hemaka,
   Todeyya, Kappa —- just those two,
   ]atuka66i the leamed one,
   i005.
   I006.
   1007.
   976-1031
   
   
   
   V. I
   MS Pafie 432
   I009.
   I011
   976-1 O31
   Bhadravudha, udaya and
   as well the brahmin Posala,
   Moghavaja, the very wise
   and Pingiya, the greatest sage - I008.
   Ail of them with their pupils' groups
   in all the world they're famed —
   enioyers of jhana, meditators Wise,
   pattemed by past good karmas made.
   Having bowed down to Bavara
   and circumambulated him,
   then in deer-skins clad, with dreadlocks all,
   they headed for the north: 1010.
   From Patififihana in Alaka's land,
   then to the city, Mahissati,
   from there to Uiieni and Gonaddhai,
   to Vedisa and to Vana town,
   Next to Kosambi and Saketa
   and Savatthi of cities best
   on to Setavya, Kapilavatthu,
   
   
   
   V. I
   MS figs 433
   Kusinara and surrounding lands,
   To Pava and to Bhoga town,
   to the Magadhans' city of Vesali,
   to the rocky Pasanaka Shrine —
   delightful, mind-delighting place.
   As a person thirsty for water
   or merchant for profit great
   or a sunbumt person seeks for shade,
   so they hastily climbed the Rock.
   The Lord on that occasion was
   in honour seated with the bhikkhu-sangha,
   teaching Dhanna to all the monks
   as lion roaring in the iungly woods.
   Ajita saw then the Sambuddha
   as the sun's brilliance devoid of rays,
   or as the moon completely full
   an'ived at its fifteenth day.
   Then standing to one side he saw
   the set of signs complete
   976-I031
   lOl2.
   I013.
   1014.
   1015.
   1016.
   
   
   
   V. 1
   Aiita:
   The Buddha:
   Ajita:
   The Buddha:
   M5 Paw 434
   976-1 O31
   upon the Buddha's body, so
   joyful, in his mind he asked: 1017.
   Speak now about my Master's age,
   tell of his clan and body-marks,
   say how far he's mastered the mantras
   and how many the brahmins he instructs. 1018.
   His age is a hundred and twenty years,
   by clan he is a Bavara,
   upon his body appear three signs,
   Three Vedas he has mastered all. 1019.
   In lore of signs and legends in tradition —
   in the glossaries and the ritual treatises —
   in his own Dhanna to perfection he's arrived
   and five hundred students he instructs. 1020.
   O highest of men, with craving cut,
   describe in detail all the signs
   upon the body of Bavara,
   so there may be no doubt in us. 1021.
   Cover his face with his tongue he can,
   
   
   
   V.l
   I O22.
   Nan'ative:
   1 024.
   Aiita:
   The Buddha:
   I 026.
   NS Page-435
   hair grows between his brows,
   ensheathed is the cloth-concealed:
   Know this, O brahmin youth.
   Now none there heard the questions asked
   but all the answers heard
   then the people, overjoyed,
   with lotused hands they thought:
   What deva indeed, whether Brahma
   or lndra called Suiampati —
   these questions asked in mind
   to whom are they addressed?
   Bavara has questioned you
   on heads and splitting heads apart.
   O Lord, do you explain this,
   dispel our doubt, O Sage.
   Know ignorance as `head',
   gnosis as that which `splits the head',
   with mindfulness, meditation, faith
   by determination, effort too.
   1023.
   I025.
   976-1031
   
   
   
   V. 1
   Narrative:
   102 7.
   Aiita:
   I028.
   The Buddha:
   Nan'ative:
   MS Rum 436
   Then the young brahmin overawed,
   with great emotion overcome,
   (respectfully) with his deerskin (cloak)
   over one shoulder (placed),
   put his head at (the Buddha's) feet.
   Sir, the brahmin Bavara
   with all his pupils too,
   overjoyed, glad-minded
   to the great Seer's feet bowed down.
   May all be well with Bavara
   with his brahmin pupils too
   and you as well be happy,
   live long O brahmin youth!
   Bavara, yourself as well
   and all the rest have many doubts
   ask now whatever's in your minds —
   you have the opportunity.
   So permitted by the All-awake
   1029.
   1030.
   976-I031
   
   
   
   V. 1
   MS P388 `B7
   976-I031
   Ajita sat, and with lotused hands,
   asked the initial question
   addressed to the Tathagata. 1031.
   (Sn. 976 51031).
   
   
   
   V.1 976-I031
   The Signs of a Superman:
   A Commentary on Sn. 1000,1017, 1022
   Before this strange subject is examined, its cultural background needs
   reflection. Brahmins of the Aryan peoples who settled at first in N.W lndia, had
   a great opinion of themselves. Though in times more ancient than that there had
   been women among them who were experts in rituals, knowing all the mantras,
   by the times of the Buddha all brahmin priest were men. As many of these
   priests after listening to the words of the Buddha became his disciples and many
   ordained as bhikkhus, they brought with them their underlying sense of male
   superiority. This has been transferred by them through chanting, to the Suttas.
   Among these brahminical relics are the strange legends of the Superman which
   within the Buddhadhama apply to only perfectly Awakened Buddhas and
   Dharma-wheel tuming emperors. Within Sn. It is interesting to notice that the
   verses we are concemed with appear in connection with brahmins. We should
   also be aware that there are other Suttas, for instance M.1 15.12 or
   A.0nes.XV.1-28, which raise the related subject of the lmpossibles, that lay
   down the law that women cannot be either Supennen as a Buddha, or as Dharma-
   emperor, ruling the entire world. The reasons why this is said to be is not made
   very clear. The Dhanna is said to be `just like this and not otherwise'.
   The list of 32 signs upon the bodies of a SLlp9lTI13l'l (ma/13-pun'sa~
   lakkhaaa) were, for those who knew how to read them, clear evidence of their
   MS Page:43B
   
   
   
   V.l 976-1031
   spiritual attainment — a vew sharp contrast to the Buddha's actual teachings.
   That they have survived for such a long time even into the present, is shown by
   newly-made images of the Buddha whose feet often have lotus-flowers upon their
   soles. The prime source of the list of the 32 is found in the Long Discourses
   (Dagha (D)) 30 para 1.2). This sutta tries to explain in tenns of cause (kanna),
   effect (the physical `signs') the various and sometimes curious marks of the
   Supennan. The order of these signs in the following list differs from their
   explanation later in the D. Sutta. There appears to be little reason in this so-
   called cause and effect. Scholars have suggested that the Lakkhana Sutta (D.3O)
   is a later production by bhikkhus after the passing of the Buddha.
   The list of 32 signs follows with some comments.
   l. He has feet with level tread (This sounds like flat footedness).
   2. On the soles of his feet are wheels with the thousand spokes, complete with
   felloe and hub (Seen on Buddha-images seated in Vaira-positions/full-lotus).
   3. He has projecting heels. (Occasional standing Buddha-images in Thailand
   have this).
   4. He has long fingers and toes. (Sometimes in Buddha-images this is taken to
   mean that fingers are of the same length, the same with toes).
   5. He has soft and tender hands and feet. (Thus maiking him of high caste.).
   6. His hands and feet are netlike. (Reticulation of veins under the skin).
   MS PIge:439
   
   
   
   V.1 976-1031
   7. He has high—raised ankles. (Short legs? Long feet?)
   8. His legs are like and antelope's. (Slender and well-shaped).
   9. Standing and without bending he can touch and rub his knees with either
   hand. (Results in ill-proportioned ape-like images).
   10. His male organs are enclosed in a sheath (The first of Bavari's signs
   commented on below. Literally the Pali says: `ensheathed is the cloth-
   concealed' — devoid of meaning unless one knows what this euphemism
   hides).
   1 1. His complexion is bright, the colour of gold.
   12. His skin is delicate and so smooth that no dust can adhere to his body.
   13. His body-hairs are separate, one to each pore (is this not usual for
   humans?).
   14. His body-hairs grow upwards, each blue-black like collyrium (used as a
   cosmetic).
   15. His body is divinely straight.
   16. He has the seven convex surfaces.
   17. The front of his body is like a lion's.
   18. There is no hollow between his shoulders —— blades).
   MS Page 440
   
   
   
   V.l 976-I O31
   19. He is proportioned like a banyan tree (the height of the body is the same
   as the span of his outstretched am1s).
   20. His chest is evenly rounded.
   21. He has a perfect sense of taste (but how did others know this?).
   22. He has jaws like a lion's (but such jaws are for piercing and tearing!)
   23. He has 40 teeth. (But how could a normal human jaw accommodate
   them?)
   24. His teeth are even.
   25. There are no spaces between his teeth.
   26. His canine teeth are very bright.
   Z7. His tongue is very long (The second of Bavari's marks).
   28. He has a Brahma-like voice (like that of a karavika-bird).
   29. His eyes are deep blue.
   30. He has eyelashes like a cow's.
   3 l . The hair between his eyes is white and soft like cotton-down (the third of
   Bavari's marks).
   32. His head is like a royal turban. (That is, his head rises to a protuberance at
   the top of his head — see many Buddha-images).
   MS Page:`1-4|
   
   
   
   V.l 976-1031
   A supennan with even a few of these marks would be freakish to our eyes,
   even if only a male child. When grown it would hardly impress by its strange
   appearance, even though this was supposed to signify supen'or spirituality. The
   meanings of some signs are obscure though the Pali Comys try to provide
   convincing explanations. An example of this, which is rather important, is
   No.10: `ensheathed is the cloth~concealed.' Even when Pali Dictionaries have
   been consulted and `cloth-concealed' is revealed as male genitalia, we are no
   further towards understanding what `ensheathed' refers to. Saddhatissa's
   translation is rather coy with `the foreskin completely covers the phallus'. But
   such a rationalist rendering fails to make clear what could be wonderful about
   this! As marks of a Supennan, all of these thirty-two should be exceptional in
   some way, while to say that the male organ is sheathed by the foreskin is no more
   than indicating the ordinary man's equipment. Presumably both penis and testes
   are meant to be covered by a flap of skin but then how could the fonner be used,
   even for urination? Do they have to be `ensheathed' because the supposed
   brahmin authors of this list were shy about making their meaning clearer? Or
   were the authors, Buddhist monks, similarly shy? The literal meaning sounds as
   though the Supemian had some sort of sexual abnormality which was
   undiscussable, puzzling when we remember that prior to Buddhahood, he
   procreated a son, Rahula, apparently nonnally. lt could be that celibate monks
   who suffered conflicts between a code of rules (the Patimokkha) and their 0Wl'l
   M5 |"zge:44Z
   
   
   
   V.l 976-1031
   sexuality thought it best to mystify the Buddha's body. But then, this effort to
   see as though through fog — 'ensheathed is the cloth-concealed' — hardly agrees
   with the Buddha's occasional revealing of his sexual organs to convince doubtful
   brahmins that he possessed all of the sign of a superman (see for example M.
   91.7). Whichever way this is regarded it is an unsolvable problem.
   Bavari's two other marks or signs present lesser problems. In the above
   translation of Sn. 1022 the first line reads `Cover his face with his tongue he
   can,' certainly not an ordinary feat. If an explanation is needed then it may be
   noticed that some practitioners of yoga as part of a practice for limiting the
   ordinary way of breathing restrain this by cutting the ligament under the tongue,
   a frenum, so that the tongue can be tumed back into the throat. This would
   enable also the tongue to be extended further over the face.
   As for the second of these, `hair grows between his brows' this seems to
   be a hairy male. Buddha-images are found with this mark between the eyebrows.
   This is certainly not as special as the other two signs.
   l have followed the translation of this list by Maurice Walshe in his Dagha-
   nikaya, “Thus Have I heard", later re-issued as ”Long Discourses of the Buddha".
   My OWI1 explanations and exclamations follow in brackets.
   The list remains the only sutta passage to give an all-round account of what
   the Buddha may have looked like, apart from occasional references, mostly in
   verse, or his handsome features. See for instance the Sn. Verses 548-553 in this
   book which also mention `the signs peculiar to the Superman'.
   MS P1gl5=443
   
   
   
   MS PAgg:M4
   976-1031
   
   
   
   V.2
   Ajita;
   The Buddha:
   Ajita:
   The Buddha:
   1035
   MS Pig:-1445
   A_11tamal1aVapuacba
   Alzta s Quesflons
   The world by what its Wrapped
   and why it shines not forth
   say too with what its smeared
   and what's its greatest fear' 1032
   The world is wrapped in ignorance,
   it shines not forth from sloth and greed,
   by longing: it IS soiled l say,
   dukkha, its greatest fear I033
   The streams are flowing everywhere,
   how can the streams be blocked,
   say how the streams may be restrained,
   by what the streams are dammed' I034
   Whatever streams are in the world,
   they may be blocked by mindfulness —
   that l say is streams' restraint,
   by wisdom they are dammed
   
   
   
   V.2
   Ajita:
   The Buddha:
   Ajita:
   The Buddha:
   MS Pagerflb
   So wisdom it is and mindfulness!
   Now, sir, I ask you, tell me this:
   the namer-mind, the bodily fonn —
   where does it cease to be?
   That question asked by you
   l tell about it now,
   the namer-mind and bodily fonn
   where they cease to be:
   by cessation of the consciousness,
   they wholly cease to be.
   Who have the Dhanna measured up,
   who train themselves, the multitude,
   how, sir, do they behave themselves,
   please answer what I speak.
   No greediness in pleasures of sense
   having a tranquil mind and clar,
   skilled in all the Dhan"na's ways —
   that mindful bhikkhu who's left home.
   lO32-1039
   l036.
   1037.
   lO3B.
   lO39.
   (Sn. lO32 — i039).
   
   
   
   V.3
   Tissametteyya:
   The Buddha:
   MS Page:447
   1042.
   fissametteyyamzinavapucchai
   Zissametteyi/a is tluesfiozzs.
   Who has contentment in the world?
   Who is not agitated?
   Who has experienced both extremes
   but wise, in the middle does not stick?
   Who do you say's a person great?
   Who, seamstress-craving's gone beyond? 1040
   A pure life leading `mid pleasures of sense,
   ever mindful and craving-free,
   a bhikkhu cool, after reflection deep,
   agitation's none in such a one. 1041
   Who has experienced both extremes,
   who wise, in the middle does not stick,
   he, l say, is a person great,
   who the seamstress-craving's gone beyond.
   (Sn. 1040- 1042)
   1 040- 1 O42
   
   
   
   VA-
   PHOTOCOPY POOR —
   Pufifiakamdna Vapucché
   Pufifiaka is' lluesfians
   LAURENCE PLEASE SEND ORIGINAL AGAIN
   MS P2321448
   1043-1 042
   
   
   
   V.5
   Mettaga:
   1049
   The Buddha:
   1050.
   Mettaga:
   HS PiE9:4`19
   Mettagziménavapuccbzi
   Mettagzi is Questions
   Reached Vedas' end, l deem, developed yourself,
   so l ask the Lord thus, please tell me of this:
   how then have resulted these various dukld1as
   of various fonns found in the world?
   On dukkhas' arising you've asked me indeed,
   so as l have Knonm, l impart it to you:
   The divers—fom-led dukkhas come to exist
   from attachments to asset of whatever kind. -
   That ignorant one attached indeed to assets
   stupidly reaches to dukkha repeatedly,
   therefore that one inseeing dukkha's birth and arising
   such attached-to assets should not be created. 1051
   Whatever we asked, to us you explained,
   another thing l ask, please speak upon that:
   how do the wise ones cross over the flood
   of birth and decay, lamentation and grief?
   I O49-1 060
   
   
   
   V.5
   1052.
   The Buddha:
   1053
   Mettaga:
   i054.
   The Budda;
   MS Page 450
   l O49-l O60
   O Sage, do well declare this to me now
   for certainly this dham1a has been known by you.
   This Dham1a l'll explain to you,
   Seen-now and not traditional law,
   knowing which the mindful fare
   and cross the world's entanglements.
   l am delighted, Seeker Great,
   with this Dhamaa ultimate
   knowing which the mindful fare
   and cross the world's entanglements.
   Whatever you cognize above,
   below, across and in between,
   consciously dispel delight in them
   and settling-down — in `being' you'll not stand. 1055.
   Mindfully who live like this, aware,
   such bhikkhus, slet go of `making-mine'
   with birth and decay, lamenting and grief:
   iust here do the wise all dukkha let go. 1056.
   
   
   
   V.5
   Mettaga:
   I057.
   The Buddha:
   MS P1ge:451
   With the Great Seeker's speech I am overjoyed,
   well explained, O Gotama, is `assets unattached'
   for surely the Lord all dukld1a has let go
   for certainly this Dham1a has been Known by you.
   Those who the Sage always advises
   surely they're able to let go of dukkha,
   to the Naga drawn near, to you l bow dovm,
   maybe the Lord will advise me as well. 1058
   Knowing well that brahmin true to Love's end reached
   having nought and unattached to sensual being
   for certain that one this flood has overcrossed,
   crossed to the Further Shore all doubts and harshness.
   ` 1059
   This is one who Knows, reached Love's end just here,
   cut bondage to any being, either high or low,
   free from craving and desire, free from distress,
   I049-l O60
   I say such one's crossed over birth and decay. 1060.
   (Sn. IO49 — 1060)
   
   
   
   V.6
   Dhotaka:
   The Buddha:
   Dhotaka:
   I063.
   The Buddha:
   MS P-`|B!:45Z
   Dlrotalramzina Vapllfltihzi
   Illmtaka is Questians
   O Seeker Great, l long to hear your word
   the having heard your speech
   for Nirvana train myself —
   I ask the Lord thus, please tell me of this. 1061
   Ascetically exert yourself
   just wise and mindful here
   then having heard my worcls
   for Nin/ana train yourself. i062
   ln the world of devas and humanity l see
   the with-nothing brahmin who wanders about,
   to the All—seeing Seer, to you l bow down:
   free me, O Sakya, from consuming doubt.
   To liberate, O Dhotaka, from consuming doubt
   l do not go about the world,
   but, having Known this Dharma best
   beyond this flood you go across. 1064
   I061-1068
   
   
   
   V.6
   Dhotaka:
   The Buddha;
   1066.
   Dhotaka:
   I06 7.
   The Buddha:
   NS Page 453
   Teach me, O Brahma, out of your compassion,
   the State of Seclusion that l may know it well
   so that l may live just like the sky,
   kind-minded, peaceful, also clinging-free. 1065
   This Peace l shall explain to you,
   Seen-Now and not traditional lore,
   knowing which the mindful fare
   and cross the world's entanglements.
   l am delighted, Seeker Great
   with this Peace that's ultimate,
   knowing which the mindful fare
   and cross the world's entanglements.
   Whatever you cognize above,
   below, across and in between
   Here having Known clinging just to this
   form no craving for high or low. I068.
   (Sn. 1061 — I068)
   
   
   
   V.7
   Upasiva:
   The Buddha:
   Upasiva:
   1071
   The Buddha:
   lO72.
   MS P1g!.454
   I/pasa vamaina vapzmcizai
   Upasiva is Questions
   Alone, O Sakya, unsupported too
   the might flood l do not dare to cross,
   All-seeing One, please tell me of the means
   using which l may overpass the flood. 1069
   Mindfully do you no-thingness regard,
   rely on `there—is-not' to go across the flood,
   abandon conversation, let go of sense-desires,
   See craving's exhaustion by night and by day. 1070
   That one who's unattached to sense-desires,
   relying on no-thingness, left others aside,
   freed in the highest of consciousness's freedom,
   will that one be `stablished or fall away from this?
   That one who's unattached to sense-desires,
   relying on no—thingness, left others aside,
   freed in the highest of consciousness's freedom
   that one will be established not fall away from this.
   1069-1076
   
   
   
   V. 7
   Upasiva:
   The Buddha:
   I074.
   Upasiva:
   The Buddha:
   1076.
   MS Page'45S
   1069-1076
   Should that one remain even for a heap of years,
   O All—Seeing One, but still' not fall away?
   Would that one liberated cool-become iust there,
   consciousness ceased in that very state? 1073.
   As flame blown out by force of wind
   has gone to its `goal', cannot be described,
   likewise the Sage `in mind-and-body' freed:
   gone to the Goal and cannot be described.
   Does one not exist who's reached the Goal?
   Or does one dwell forever free
   O Sage, do well declare this to me now
   for certainly this dham1a's known by you. 1075.
   Of one who's reached the Goal no measure's found
   there is not that by which one could be named,
   when dhannas for that one are emptied out,
   emptied are the ways of telling too.
   (Sn. 1069- 1076).
   
   
   
   V. 8
   Nanda:
   The Buddha:
   Nanda:
   The Buddha:
   M5 P152 456
   Nandaméavapucclzei
   Nanda is Questions
   People say that in the world
   there are sages - how is this?
   Do they say `sage' for knowledge won
   or for a certain way of life?
   The lntelligent ones say not `a sage',
   for view, tradition, knowledge won;
   those foeless, desireless and free from distress
   1077-1 O83
   1077.
   who so fare along are sages, l say. I078.
   Some of these monks and brahmins they say
   that purity comes from the seen and the heard,
   from rites and from vows and from other things too,
   have they, O Lord, while practising thus
   crossed over birth and crossed decay, sir?
   I ask the Lord thus, please tell me of this. 1079.
   Some of these monks and brahmins they say
   that purity comes from the seen and the heard,
   from rites and from vows and from other things too
   
   
   
   V. 8
   Nanda:
   The Buddha:
   Nanda:
   MS Page 457
   I077-1083
   and even though they have practised thus
   I say they've not crossed over birth and decay. I080.
   Some of these monks and brahmins they say
   that purity comes from the seen and the heard,
   from rites and from vows and from other things too,
   if you say, O Sage, they've not crossed the flood,
   who fares through the world of devas, mankind,
   crossed over birth and crossed decay, sir?
   l ask the Lord thus, please tell me of this. I081.
   Of these monks and these brahmins I do not say
   that all are shrouded by birth and decay:
   those who've let go of the seen heard and known,
   of rites and of vows and others — all,
   completely craving Known and from the inflows free —
   those persons I say have crossed over the flood. 1082.
   By the Great Seeker's words l'm truly delighted
   well-explained, O Gotama, is `to assets unattached',
   those who've let go of the seen, heard and known,
   of rites and of vows and of other things — all,
   completely craving Known and from the inflows free —
   also l say they've crossed over the flood. 1083.
   
   
   
   V8
   MS Page 458
   (Sn. I077-1083)
   1077-I083
   
   
   
   V.9
   Hemaka:
   1084.
   I085
   The Buddha:
   MS Page 459
   Hemalraména vapucclzai
   Ilelzzalra is Questions
   Those eldem outside Gotama's teaching-path
   explained their doctrines in this way:
   `so it was and so will be' —
   all that was but traditional ore,
   all that increased uncertainty.
   And l took no delight in it.
   Please, O Sage, explain to me
   Dhamma destroying craving now
   knowing which the mindful fare
   and cross the world's entanglements.
   Here among all those pleasing things —
   the seen and the heard, the sensed and thought —
   whose wants removed and passion too —
   this the unchanging Nirvana-state. 1086
   With Final Knowledge those mindful ones,
   now-seen the Dham1a, cool-become
   and ever they at peace remain,
   lO84-1087
   
   
   
   V 9
   I087.
   MS HR:46O
   1084-1087
   they've crossed the world's entanglements.
   (Sn. 1084 - 1087)
   
   
   
   V. 1 O
   Todeyya:
   I088.
   The Buddha:
   Todeyya:
   The Buddha:
   MS Pa3e:46l
   T0l1€J'jV3H1éil13V3pI1001I2
   Todeyya is Questions
   ln whom do sense-desires dwell not
   and craving can't be found,
   crossed beyond consuming doubt —
   what's freedom for this one?
   ln whom do sense-desires dwell not
   and craving can't be found,
   crossed beyond consuming doubt —
   than this, is freedom not apart. 1089.
   No inclinations or with longings still?
   One with wisdom wanting, or with wisdom won?
   Explain to me please, O All-seeing Sakya,
   how l may well discem a sage. 1090.
   No inclinations at all and no longings left,
   not one with wisdom wanting but one with wisdom won
   know as a sage, Todeyya, one who nothing owns,
   one who is to sensual being unattached. 1091.
   1088-IO!
   
   
   
   V. 1 O
   MS Page;46Z
   (Sn. 1088-1091)
   1088-10$
   
   
   
   V. 1 1
   Kappa:
   The Buddha;
   MS PJR:463
   Kappamanavapuccllai
   Kappa is Question
   ln midstream standing there,
   in flood's fearsome surge —
   oppressed by age and death,
   of the island tell me, sir,
   do tell me of this isle
   where this will be no more.
   ln midstream standing there,
   in flood's fearsome surge
   oppressed by age and death, Kappa
   of the island l shall tell.
   'Ov\ming nothing and unattached'
   that's the isle of no-beyond,
   Nin/ana do l call it
   death-decay destroyed.
   With Final Knowledge these mindful ones,
   Now-seen the Dhanna, cool-become,
   I092.
   1093.
   1094.
   1092-10!
   
   
   
   V.
   11
   I095.
   M5 l`age:464
   they're neither under Mara': sway
   nor Mara's followers.
   (Sn. I092 — I095)
   1092-10$
   
   
   
   V.l2
   ]atukaooa:
   1 O96
   IO97.
   The Buddha;
   1098
   MS FWK465
   Jatukafififiména Vapucclzai
   ~...~,
   Jatukaoaa s Questions
   l've heard of the heroic one
   from desires free for pleasures sensuous,
   with Question l come for the flood-crosser, from desire free
   O Lord with Eye innate, tell me the Peaceful State —
   how tmly it is, tell me of this.
   The Lord comports himself as the senses' conqueror
   as the hot sun the earth by its heat,
   of small wit am l, Thou of wisdom great,
   explain to me Dhanna that l may understand
   the abandonment here of birth and old age.
   Greed for sensual pleasures curb
   and safety see in letting-go,
   let there not be found in you
   grasping or rejection.
   lO96—ll(
   
   
   
   V.l2 lO96—1I(
   Dry up whatever's gone `before'
   while `after' have no thing at all,
   if in the `middle' you don't grasp
   you'll fare in Perfect Peace. 1099.
   O brahmin, from greediness free
   for every sort of name-and-form,
   the inflows are not found by which
   t one'd go beneath death': sway. I 100.
   (Sn. 1096 - I100)
   MSPWM66
   
   
   
   V. l 3 1
   Bhadravudha:
   The Buddha:
   MS Pzgme 7
   llO4.
   Bbadrévudlzamzinavapucobé
   Bbadzzeivudlra is Question
   With home let go, craving cut and undisturbed,
   delight let go, crossed the flood and free,
   time let go and truly wise — l beg of you —
   various people here from provinces assembled, l 101
   O Hero, they're longing for your words
   do well declare these to them now
   for certainly this Dham1a has been Known by you,
   then having heard the Naga they'll depart from here.
   l 102
   Beyond all acquisitive craving you should train
   above, below, across and in between,
   whatever is grasped at in the world,
   with that does Mara that person pursue. l 103
   Knowing this, therefore the mindful bhikkhu
   in all worlds grasps not anything,
   understanding well those acquisitive beings,
   those people adhering to the realm of death.
   
   
   
   V. I 3
   MS Pazx:46B
   (Sn. 1101-1104)
   1101-1 H
   
   
   
   V. I 4
   Udaya:
   1105.
   The Buddha:
   I I07.
   Udaya:
   MS P:ge:469
   Udayamaina I/apucclzzi
   Udayais Questions
   To the contemplative seated fre of dust,
   done what should be done, with inflows none,
   to the one beyond all dhannas gone,
   to him with a question have l come:
   Please say by shattering ignorance
   how Final Knowledge's free.
   Let go desires for pleasures of sense
   with all bad-mindedness as well,
   sloth dispel and check remorse,
   poised mindfulness completely pure
   forerun by Dharma - distinguishing:
   l say by shattering ignorance
   there's Final Knowledge free.
   What's the fetter of the world
   and what the world's wandering? `
   By abandonment of what
   1106-
   llO5-ll
   
   
   
   V. 1 4
   The Buddha:
   Udaya:
   H10.
   The Buddha;
   MS PaSec470
   1 105-I I '
   is it Nirvana named? 1 I08.
   Enjoyment: the fetter of the world,
   while thinking, the world's wandering,
   the letting-go of craving —
   it's nin/ana named. 1 109.
   How is consciousness broken up
   in one who practises mindfully?
   To the Lord we come with this question,
   we wish to hear your words.
   Feeling both within, without -
   in that do not delight,
   thus consciousness is broken up
   in one who practises mindfuily. I I 1 1.
   (Sn. I105 -llll)
   
   
   
   V. 1 5
   Posala:
   The Buddha:
   MS Page:47l
   Posélamaiua Vapzmclzai
   Poseila is Question."
   To One illuming the past,
   the Llndistuibed who's cut off doubt,
   Gone across the dhannas all,
   with this question have l come:
   ln whom are no perceptions of form,
   abandoned it in bodies all,
   who sees within, without `there's nought',
   of such a one, what knowledge known
   how will such a one be led?
   The Tathagata knows thoroughly
   all supports of consciousness:
   knows this one as stuck,
   that one freed, or bound for the beyond.
   Having known the existence of no-thingness
   together with its bond of indulgence,
   then investigating this and inseeing that,
   lll2.
   1113
   H14.
   1112-ll
   
   
   
   V.l5 H12-ll'
   - just this is the knowledge of a brahmin perfected.
   1 1 15.
   (Sn. H12 -1115)
   M5Page-472
   
   
   
   V.l6 16-
   Mogharéia:
   The Buddha:
   MS P1ge:473
   Moglzaraijamziuavapuccllzi
   Moglzaraija is question
   The Sakyan I've requested twice
   but the Seer has not replied,
   I have heard tell that questioned thrice
   a Sage divine explains.
   ~.....t_'.=...,,.sA._.,,~_~.`.>  .='__,...%.,..@a.<...._ v.u'a...W~.....@_<.~a.W,m.¢  -»  m_*&.&<  ~_e_.-_..._. '~_eM._t___. L..._...@=e_._ ~__.. i-$5. VA“.
   I116
   I do not understand the view of the renowned Gotama
   upon this world, the other world,
   on Brahma's realm, the deva's world.
   So to the One of vision supreme
   with this quely have i come:
   how should one regard the world
   that the Death—king sees one not?
   Be ever mindful, Mogharaja,
   and as empty, view the woild,
   with view of self pulled up
   and thus crossed over death —
   who sees the world like this
   III7
   lll8
   
   
   
   V.16
   MS Page 474
   the King of Death sees not. 1 1 19.
   (Sn. 1116-1119)
   1116-11
   ,“._..(,__,..T»-<
   
   
   
   V.l7 H2011
   Pfigiyaména Vapzmclzzi
   H731]/a is Question
   Piigiya: Aged am l, feeble and my beauty fled,_
   1120
   The Buddha:
   H21.
   my eyes no longer clear, my hearing weak,
   let me not die confused along the way,
   teach me the Dharma so that l shall know
   how to let go of birth and of decay.
   Having seen them smitten by (the sight of) forms,
   by bodily fom1s those careless folk are beaten down,
   therefore, Piigiya, you should be aware:
   let go these forms for not again becoming.
   Piigiya: There's nothing in the world: four directions chief,
   MS Pay 475
   1122.
   four intermediates, above, below, ten in all —
   unseen, unheard, not felt or known by you,
   teach to me Dharma so that when I've known it
   birth and ageing are both abandoned here.
   
   
   
   V.17
   The Buddha: Seeing humanity by cravings afflicted —
   M5 Page 476
   bumt by being bom, overcome by age,
   therefore, Piigiya, you should be aware:
   let go of craving for not again~becoming. 1 I23.
   (Sn. 1120 -1 123)
   1120-11}
   
   
   
   Notes on Sn. 1032 -I123
   1032. The world (citra) in first line. The world as exterior is more often
   loka in Pali. But loka and citta are closely interrelated in Buddhist Suttas.
   1036. 1.3. Namer-mind: in Pali this is nama, what (that which?) gives .
   names to things. These two words, from Pali, and from English, are
   etymologically related.
   1040. The Tissametteyya who asks this question is unlikely to be the one
   who is addressed in lV. 7, or Sn. 814-823.
   1040. 1.3. Extremes (in views and practices) never approved by the
   Buddha. ln Pali anta (close to English 'end') is usually translated `extreme'.
   1040. A `great person' in the Buddhist sense, rather than the Hindu. See
   the Commentarial notes to Sn. 1000, 1017, 1022.
   1040. 1.3. Sibbaniji — seamstress. The seamstress who sews together
   saysara, the round of birth and death.
   1046. Some brahmins made offerings to gods in order to procure the
   results of sense/sex enjoyment.
   1062. Live a frugal life without indulgence or bodily self-toiture — either
   extreme.
   1061-1068. The Buddhas does not teach Dhotaka what he asks, so there
   are two possibilities; The text is lost or comipted, or, the Buddha taught with
   these few words which were accompanies by a mind-transmission.
   1070. Nothingness: the nature of all that can be conceived — no
   essence.
   Nothingness: one has no views about pennanence.
   Craving's exhaustion by night and by day, not `destruction'.
   1074. `Mind and body' in Theravada translated as `body-group' by some
   authors.
   1076. `Emptied' is better translation than `removed, destroyed or
   abolished' as used by Ven. Nyanamoli.
   MS Page 477
   
   
   
   l077. Muni was used by all practitioners of Dharma, not only Buddha.
   1078. `Fare' is the only English word meaning both joumey/travel as well
   as practice Dharma which is regarded as a ioumey. The verb `cavati' also has the
   two meanings of ordinary and spiritual joumeys.
   1079. The seen and the heard is placing reliance upon the ways that
   unimportant things are done (such as putting on robes) and not paying attention
   to really important matters (such as ageing and death).
   1083. The last line of this verse: is this Nanda's agreement with the
   Buddha or does this contain his conceit that he too has the same view?
   1084. Teaching or path or generally meaning religion. It is hard to find
   an English word to fit all the meanings of Sasana.
   1 100. Name and fonn: nama-rapa. Nama is mind generally, while rapa is
   body.
   l 105. `seated free of dust' —- refers to mental-emotional dust.
   l l l5. no-thingness: does not mean the arapa (fom1less) attainment in
   meditation. `brahmin' here in the Buddhist sense.
   l l 16. Twice the Buddha did not reply to Mogharaia. But the third time
   his mind had settled down. Suflna means emptiness of persons, ego, selfhood,
   ownership etc, it does not mean `nothing'.
   MS Page-:478
   
   
   
   ___ ,. ._._...;Y“_~.D. ~._..t._—,..~d.m~“L,__i,_ n._h.,<.at__._i. `.A»_~__.Wn~_...w-_1=...4t~a..m .~.-=.._1-,;l...>t_ ,- ~#.i_~ tau.“ ,,,._=,» ___£ Wine“. -Mm  km.` u;_»~~"~»a;_~~~1~\1.v_v *..“.¢'_4Y
   V.l3 1124-1149
   Epilogue: Pingzya's Praises of the Way to the Beyond
   The Radiant One said this while he stayed among the Magadhese at the
   Pasanika Shrine. There he was asked, then he questioned, the sixteen brahmin pupils,
   answering their questions. If one should understand the meaning of even one question
   and then practise the Dharrna according to the Dharma, then one Would go to the
   farther shore beyond ageing and death. As this Dhanna leads to the farther shore it is
   known as the Teaching leading to the Farther Shore (Parayana Sutta).
   Ajita, Tissametteyya,
   Punriaka, then there's Mettagfi
   Dhotaka, Upasiva then
   Nanda also Hemaka, 1124.
   The pair Todeyya, Kappa then
   J atukanni the learned one,
   Bhadrfivudha, Udaya and
   the brahmin Who's called Posala,
   Mogharaja the very wise
   and Pingiya the seer so great — 1125.
   All these approached the perfect Seer,
   the Buddha Who practised perfectly,
   to question him on subtle points,
   to the Buddha-best they journeyed. 1126.
   Having been questioned the Buddha replied
   with Dhanna according to how it really is.
   MS P21181421
   
   
   
   l079. The seen and the heard is placing reliance upon the ways that
   unimportant things are done (such as putting on robes) and not paying attention to
   really important matters (such as ageing and death).
   1083. The last line of this verse: is this Nanda's agreement with the Buddha or
   does this contain his conceit that he too has the same view?
   1084. Teaching or path or generally meaning religion. It is hard to find an
   English word to fit all the meanings of Sasana. i
   1100. Name and form: nama rupa. Nama is mind generally, while rupa is
   body.
   ll05. `seated free of dust' ~ refers to mental-emotional dust.
   1115. no-thingness: does not mean the arupa (formless) attainment in
   meditation. `brahmin' here in the Buddhist sense. '
   ll l6. Twice the Buddha did not reply to Mogharaja. But the third time his
   mind had settled down. Sufifia means emptiness of persons, ego, selfhood, ownership
   etc, it does not mean `nothing'.
   \/[S P328421
   
   
   
   __-m,.,..~» ~._.,MY.»u., .m,.`.~fl4..44 ua_.`M_..,M__i,.¢..,. ,»_".t.`_-,..¢@,,.£_,`,~t»._...e   ta W iJ1'~5l%|A_i` Q54
   V.l8
   1124-I149
   In answering their questions the Sage
   delighted those brahmins. 1127.
   [GAP IN TRANSLATION UNTlL....]
   MS Page 423
   `Going to the Further Shore', I'll chant —
   As He had Seen so did He teach,
   that pure One of wisdom profound
   rid of desire and (interior) jungle —
   falsehood ~ for what reason would he speak? 1131.
   Now therefore I shall eulogize
   those words profoundly beautiful
   of One who has renounced the stains
   of delusion, pride, hypocrisy. 1132.
   All-seer, Awake, Dispeller of dark,
   beyond all being gone, gone to world's end,
   with no inflows left, let go of all dukkha-causes,
   that one is `brahmin' rightly named: him do I serve.
   1133.
   As bird that leaves behind a copse
   might then in fruit-filled forest live,
   so have I left those of wisdom lean
   as swan to a great lake arrived. 1134.
   
   
   
   V.18
   Those who explained to me before
   I heard the Buddha's utterance,
   said `thus it was', `thus will be' -
   all that was only oral lore,
   all that promoted more disputes.
   Dispeller of darkness, seated alone,
   the well-born, him the maker of Light,
   Gotarna, him profoundly wise,
   Gotama, greatly intelligent —
   the Dharma to me he pointed out
   which can be seen right here and now,
   by time, not limitecl at all
   craving's exhaustion, troubles' end,
   with which naught can compare.
   [GAP 1N TRANSLATION UNTIL]
   [ENDS]
   MS Pag::4'1/1
   w
   The mind of faith and rapture, mindfulness
   from G0tama's teachings never does depart,
   for in whoever way the Deeply-Wise One goes
   to that very direction do I bow down.
   1135
   1136.
   1137
   1143.
   1124-1149
   X
   
   
